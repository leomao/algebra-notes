%1 TEX root=../main.tex
\subsection{Week 10}
\subsubsection{Free groups}
A free group generate by a non-empty set $X$ is that
there are no relations satisfied by any of elements in $X$.

\begin{definition}
  A free group on $X$ is a group $F$ with an inclusion map $i: X \to F$ satisfying the
  following universal property: For any group $G$ and any map $f: X \to G$,
  exists a unique group homo $\phi : F \to G$ that the following diagram commutes.
  \[
    \begin{tikzcd}
    X \arrow{r}{i} \arrow[swap]{dr}{f} & F \arrow{d}{\phi} \\
    & G
    \end{tikzcd}
  \]
\end{definition}

\begin{theorem}
  $F$ exists and is unique up to isomorphism. (Denote it as $F(X) = F$).
\end{theorem}

\begin{proof}
  For $X$, we create a new disjoint set $X^{-1} = \{ x^{-1} : x \in X\}$
  and an element $1 \notin X \cup X^{-1}$.

  Define $F(X) = \{ 1 \} \cup \left\{ x_1^{\delta_1} x_2^{\delta_2} \cdots x_m^{\delta_m} :
   m \in \mathbb{N} , x_i \in X, \delta_i = \pm 1, x_{i+1}^{\delta_{i+1}} \neq
   \left( x_i^{\delta_i} \right)^{-1}\right\}$, and
  \[ x_1^{\delta_1} x_2^{\delta_2} \cdots x_m^{\delta_m} =
    y_1^{\epsilon_1} y_2^{\epsilon_2} \cdots y_m^{\epsilon_m}  \iff n = m \text{ and }
    \delta_i = \epsilon_i \text{ and } x_i = y_i , \forall i \]

  For each $y \in X \cup X^{-1}$, we define $\sigma_y : F(X) \to F(X)$ by
  \[
    \sigma_y (x_1^{\delta_1} x_2^{\delta_2} \cdots x_m^{\delta_m})
    =
    \begin{cases}
      y x_1^{\delta_1} x_2^{\delta_2} \cdots x_m^{\delta_m} & \text{if } x_1^{\delta_1} \neq y^{-1} \\
      \begin{cases}
        x_1^{\delta_1} x_2^{\delta_2} \cdots x_m^{\delta_m} & (m \geq 2) \\
        1 & (m = 1)
      \end{cases} & \text{if } x_1^{\delta_1} = y^{-1}
    \end{cases}
  \]

  Then $\sigma_y$ is a permutation of $F(X)$, since if
  $ \sigma_y(x_1^{\delta_1} x_2^{\delta_2} \cdots x_m^{\delta_m}) =
  \sigma_y(y_1^{\epsilon_1} y_2^{\epsilon_2} \cdots y_m^{\epsilon_m}) $.
  \begin{itemize}
    \item[m = n:] either $x_1^{\delta_1} = y_1^{\epsilon_1} = y^{-1}$ or not,
      then either $x_2^{\delta_1} x_3^{\delta_2} \cdots x_m^{\delta_m} =
  y_2^{\epsilon_1} y_3^{\epsilon_2} \cdots y_m^{\epsilon_m}$ or
  $y x_1^{\delta_1} x_2^{\delta_2} \cdots x_m^{\delta_m} =
  y y_1^{\epsilon_1} y_2^{\epsilon_2} \cdots y_m^{\epsilon_m}$. Both of them leads to
  $x_1^{\delta_1} x_2^{\delta_2} \cdots x_m^{\delta_m} =
  y_1^{\epsilon_1} y_2^{\epsilon_2} \cdots y_m^{\epsilon_m}$.
  \item[m = n+2:] Omimi
  \end{itemize}
  Also $\sigma_y$ is onto since omimi. And notice that $\sigma_{y^{-1}} \circ \sigma_y = id_{F(X)}$

  Define $A = \langle \sigma_x : x \in X \rangle \leq S_{F(X)}$. and define $\phi: F(X) \to A$ by
  $\phi(1) = id_{F(X)}$ and \\ $x_1^{\delta_1} \cdots x_m^{\delta_m} \mapsto \sigma_{x_1}^{\delta_1}
  \cdots \sigma_{x_m}^{\delta_m}$. The it is omimi that $\phi$ is a bijection. So we define
  $x::X \cdot y::X = \phi^{-1}( \phi(x) \circ \phi(y) )$.

  The $\phi$  in the universal property could be defined as
  $\phi(x_1^{\delta_1} x_2^{\delta_2} \cdots x_m^{\delta_m}) = f(x_1)^{\delta_1} \cdots f(x_m)^{\delta_m}$.
\end{proof}

\begin{prop}
  Let $G = \langle a_1, \cdots , a_n \rangle$ and $X = \{ x_1, \cdots, x_m \}$. Then
  $G \cong \quot{F(X)}{K}$ for some normal subgroup $K$. $K$ is called the subgroup of relations
  connecting the generators.

  Define $f = x_i :: X_i \to a_i :: G$. By universal property, $\exists \phi
  = x_i :: F(X) \mapsto a_i :: G$. Then
  $\quot{F(x)}{\ker \phi} \cong G$.
\end{prop}

\begin{definition}
  Let $X = \{x_1, x_2, \cdots, x_n\}$ and $R \subset F(X)$.
  Let $N(R)$ be the smallest normal subgroup of $F(X)$ containing $R$,
  Then $G = \quot{F(X)}{N(R)}$ is written as $\langle x_1, \cdots, x_n \mid \text{elements of } R \rangle$,
  which is called a presentation of $G$. If $\abs{R} < \infty$, then $G$ is said to be finitely
  presented.
\end{definition}

\begin{example}
  \[ D_n = \gen*{
    \begin{bmatrix}
      \cos \frac{2\pi}{n} & -\sin \frac{2\pi}{n} \\
      \sin \frac{2\pi}{n} &  \cos \frac{2\pi}{n}
    \end{bmatrix},
    \begin{bmatrix}
      1 & 0 \\
      0 & -1
    \end{bmatrix}
    }
  \]

  We find that $x^n , y^2 , xyxy \in \ker \phi$. Then $R = \{ x^n , y^2 , xyxy \} \subseteq \ker \phi
  \implies N(R) \leq \ker \phi$.  By factor theorem, $\exists \bar{\phi} :: \quot{F(X)}{N(R)} \to D_n$.
  But notice that
  \[ \abs*{\quot{F(x)}{N(R)}}  \leq 2n \]
  since $xyxy = 1 \implies xy = yx^{-1}$, so every element could be turn into
  $x^i y^j$. Hence $\bar{\phi}$ is an isomorphism.
\end{example}

\begin{prop}
  Let $X = \{x_1, x_2, \cdots, x_n\}$. Then $\quot{F(X)}{[F(X), F(X)]} \cong \mathbb{Z}^n$.
\end{prop}

\begin{proof}
  Define $f = x_i :: X \mapsto e_i :: \mathbb{Z}^n$. Then $\phi = x_i :: F(X) \mapsto e_i :: \mathbb{Z}^n$.
  By 1st isomorphism theorem $\quot{F(X)}{\ker \phi} \cong \mathbb{Z}^n$ which is abelian,
  so $[F(X), F(X)] \leq \ker \phi$.
  By factor theorem, 一個ㄛ圖.

  Claim that $\bar{\phi}$ is 1-1.
  \begin{proof}
    Since $\quot{F(X)}{[F(X), F(X)]}$ is abelian, $\forall a \in \quot{F(X)}{[F(X), F(X)]}$, we can write
    $a = \bar{x}_1^{n_1} \bar{x}_2^{n_2} \cdots \bar{x}_m^{n_m}$.
    If $\bar{\phi}(\bar{a}) = (m_1, \cdots, m_n) = 0$ in $\mathbb{Z}^n$, then $m_i = 0,\, \forall i
    \implies a = 1$
  \end{proof}
\end{proof}
