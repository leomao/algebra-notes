%! TEX root=../main.tex
\subsection{Week 2}
\subsubsection{Permutation groups and Dihedral groups}
\begin{definition}
  A permutation of a set $B$ is a 1-1 and onto function from $B$ to $B$.

  Let $S_B \defeq \text{the set of permutations of $B$}$. Then
  $(S_B, \cdot, {\rm Id}_B)$ forms a group.

  If $B = \{a_1, \dots, a_n\}$, then $S_B \cong S_{\{1,\dots,n\}}$ and write
  $S_n = S_{\{1,\dots,n\}}$, called the symmetric group of degree $n$.
\end{definition}

\begin{theorem}[Cayley theorem]
  Any group is isomorphic to a subgroup of some permutation group.
  
  (Hint): Let $G$ be a group. Set $B = G$. Consider $a \in G$ as
  $\sigma_a: G \to G, x \mapsto ax$.
  Then $\sigma_a \in S_G \implies G \le S_G$.
\end{theorem}

\begin{fact}
  $S_n$ is a finite group of order $n!$, i.e. $\abs{S_n} = n!$.
  \begin{proof}
    Obviously. Just count the possibilities of all permutations.
  \end{proof}
\end{fact}

\underline{Cyclic notation}: $\sigma \in S_5$, say $\sigma = \begin{pmatrix}
  1 & 2 & 3 & 4 & 5 \\
  4 & 3 & 5 & 1 & 2
\end{pmatrix}$.
Write $\sigma = \cycle{1, 4}\cycle{2, 3, 5}$.

$\Rightarrow$ Any permutation can be written as a product of disjoint cycles.

\begin{example}
  In $S_7$, $\sigma_1 = \cycle{1,2,3}\cycle{4,5,6}\cycle{7},
  \sigma_2 = \cycle{1,3,5,6}\cycle{2,4,7}$.

  Then $\sigma_1\sigma_2 = \cycle{2,5,4,7,3,6},
  \sigma_1^{-1} = \cycle{1,3,2}\cycle{4,6,5}$.
\end{example}

\begin{definition}
  A 2 cycle is called a {\bf transposition}.
\end{definition}

\begin{example}
  $\cycle{1,2,3} = \cycle{1,3}\cycle{1,2},
  \cycle{1,2,3,4,5} = \cycle{1,5}\cycle{1,4}\cycle{1,3}\cycle{1,2}$.

  Any permutation is a product of 2 cycles.
\end{example}

\underline{Useful formula}: $\sigma \in S_n$,
$\sigma \cycle{j_1, \dots, j_m} \sigma^{-1} =
\cycle{\sigma(j_1),\dots,\sigma(j_m)}$.

\begin{example}
  Let $\sigma = \cycle{1,2,3}\cycle{4,5,6,7}$,
  $\sigma \cycle{2,3,4} \sigma^{-1} = \cycle{3,1,5}$.
\end{example}

\begin{proof}
  Note that both sides are functions. Let $j_{m+1} = j_1$ for convenience.
  For $i \in \{1,\dots,n\}$,
  \begin{enumerate}[\underline{Case \arabic*}:]
    \item $\exists k$ s.t.  $\sigma(j_k) = i$,
      $\sigma\cycle{j_1,\dots,j_m}\sigma^{-1}(i) = \sigma\cycle{j_1,\dots,j_m}(j_k)
      = \sigma(j_{k+1})$. (So $\sigma(j_k) \mapsto \sigma(j_{k+1})$)
    \item Otherwise, $\sigma\cycle{j_1,\dots,j_m}\sigma^{-1}(i) = \sigma\sigma^{-1}(i) = i$.
  \end{enumerate}
  Then we can conclude that 
  $\sigma \cycle{j_1, \dots, j_m} \sigma^{-1} = \cycle{\sigma(j_1),\dots,\sigma(j_m)}$.
\end{proof}

\begin{fact}
  $S_n = \gen{\cycle{1,2},\dots,\cycle{1,n}}$.
  \begin{proof}
    $\cycle{1,i}^{-1} = \cycle{1,i}$ and
    $\cycle{i,j} = \cycle{1,i}\cycle{1,j}\cycle{1,i}^{-1}$.
  \end{proof}
\end{fact}

\begin{definition}
  Let $G$ be a group and $S \subset G$. The subgroup generated by $S$ defined
  to be the smallest subgroup of $G$ which contains $S$, denoted by
  $\gen{S}$.
\end{definition}

\begin{exercise} \mbox{}
  \begin{enumerate}
    \item $S_n = \gen{\cycle{1,2},\cycle{2,3},\dots,\cycle{n-1,n}}, \quad n \ge 2$.
    \item $S_n = \gen{\cycle{1,2},\cycle{1,2,\dots,n}}, \quad n \ge 2$.
  \end{enumerate}
\end{exercise}

\begin{definition}
  $A_n = \{ \text{even permutations of $S_n$} \} \le S_n,
  \abs{A_n} = \frac{n!}{2}$.
\end{definition}

\begin{exercise} \mbox{}
  \begin{enumerate}
    \item $A_n = \gen{\cycle{1,2,3},\cycle{1,2,4},\dots,\cycle{1,2,n}}, n \ge 3$.
    \item $A_n = \gen{\cycle{1,2,3},\cycle{2,3,4},\dots,\cycle{n-2,n-1,n}}, n \ge 3$.
  \end{enumerate}
\end{exercise}

\begin{remark}
  $\gen{S} = \bigcap\limits_{S \subseteq H \le G} H =
  \{ a_1a_2\dots a_k \mid k \in \Nb, a_i \in S \cup S^{-1}\} \cup \{1\}$
\end{remark}

The orthogonal transformations on $\Rb^2$: $\text{O}(2)$.

Let $A = \begin{pmatrix}a_1 & a_2 \\ b_1 & b_2\end{pmatrix} \in \text{O}(2)$.

略... (這邊討論旋轉和反射的矩陣)

\begin{enumerate}[\underline{Case \arabic*}:]
  \item $A = \begin{pmatrix}
      \cos\alpha & -\sin\alpha \\
      \sin\alpha & \cos\alpha
    \end{pmatrix}$ is counterclockwise roration w.r.t. $\alpha$.
  \item $A = \begin{pmatrix}
      \cos\alpha & \sin\alpha \\
      \sin\alpha & -\cos\alpha
    \end{pmatrix}$ is the reflection.
    $A^2 = I_2 \implies$ eigenvalues are $\pm 1$.

    Easy to show that $\text{L}_A(v) = v - 2 \inpd{v, v_2} v_2$.
\end{enumerate}

$\text{O}(2) = \{\text{rotations}\} \cup \{\text{reflections}\}$.

\begin{definition}
  The dihedral group $D_n$ is the group of symmetries of a regular $n$-gon.

  In general, $D_n = \gen{ {\rm T, R} \mid
  {\rm T}^n = 1, {\rm R}^2 = 1, {\rm TR} = {\rm RT}^{-1} } \le O(2)
  \le S_n, \abs{D_n} = 2n$.
\end{definition}

\begin{definition}
  Let $\rm T$ be a linear transformation from $\Rb^n \to \Rb^n$.
  \begin{itemize}
    \item $\rm T$ is called a rotation if $\exists$ a $\rm T$-invariant
      subspace $W \subseteq \Rb^n$ with $\dim W = 2$ s.t.
      $\begin{cases}
        {\rm T} \big|_W \text{~is a rotation} \\
        {\rm T} \big|_{W^\bot} = {\rm id}_{W^\bot}
      \end{cases}$
    \item $\rm T$ is called a reflection if $\exists$ a $\rm T$-invariant
      subspace $W \subseteq \Rb^n$ with $\dim W = 1$ s.t.
      $\begin{cases}
        {\rm T} \big|_W = -{\rm id}_W \\
        {\rm T} \big|_{W^\bot} = {\rm id}_{W^\bot}
      \end{cases}$
  \end{itemize}
\end{definition}

\underline{Main result}: the group of orthogonal transformations
$= \gen{ \text{rotations}, \text{reflections} }$.

\begin{prop}
  For ${\rm T}: \Rb^n \to \Rb^n$, $\exists$ a $\rm T$-invariant
  subspace $W \subseteq \Rb^n$ with $1 \le \dim W \le 2$.
  \begin{proof}
    Let $A=[{\rm T}]_\alpha \in M_{n\times n}(\Rb) \subseteq M_{n\times n}(\Cb)$.
    Consider $\widetilde{{\rm L}_A}: \Cb^n \to \Cb^n, v \mapsto Av$.

    Then $\exists$ an eigenvalue $\lambda \in \Cb$ and an eigenvector
    $v \in \Cb^n$ for $\widetilde{{\rm L}_A}$.
    Let $\lambda = \lambda_1 + \lambda_2 i, v = v_1 + v_2 i$. By definition,
    we have
    \[
      Av = \widetilde{{\rm L}_A}(v) = \lambda v =
      (\lambda_1 + \lambda_2 i)(v_1 + v_2 i)
      \implies \begin{cases}
        Av_1 = \lambda_1 v_1 - \lambda_2 v_2 \\
        Av_1 = \lambda_2 v_1 + \lambda_1 v_2
      \end{cases},
    \]
    so $W = \gen{ v_1, v_2 }$.
  \end{proof}
\end{prop}

\begin{exercise} \mbox{}
  \begin{enumerate}
    \item If $\rm T$ is orthogonal, then $W^\bot$ is also $\rm T$-invariant.
    \item Use induction on $n$ to show the main result.
  \end{enumerate}
\end{exercise}

For $n = 3, A \in \text{O}(3)$, we have $A \sim \begin{pmatrix}
  \cos\alpha & -\sin\alpha & \\
  \sin\alpha & \cos\alpha & \\
   & & \pm 1
\end{pmatrix}$.

\subsubsection{Cyclic groups and internal direct product}

\begin{definition}
  If $G = \gen{a} = \{ \dots, a^{-2}, a^{-1}, a, 1, a, a^2, \dots \}
  = \{\, a^n \mid n \in \Zb \,\}$, then $G$ is a cyclic group generated by $a$.
\end{definition}

\begin{example}
  $\Zb = \gen{1} = \gen{-1}$.
\end{example}

\begin{example}
  Let $A = \begin{pmatrix}
    \cos \frac{2\pi}{n} & -\sin \frac{2\pi}{n} \\ 
    \sin \frac{2\pi}{n} & \cos \frac{2\pi}{n}
  \end{pmatrix} \in \text{SO}(2)$. Then $\gen{A} =
  \{ I_2, A, A^2, \dots, A^{n-1} \}$ and $A^n = I_2, A^m = A^r$ where
  $m \equiv r \pmod n$.
\end{example}

\begin{example}
  $\quot{\Zb}{n\Zb} = \{ \ob{0}, \ob{1}, \dots, \ob{(n-1)} \}$ with
  $\ob{j} = \{\, m \in \Zb \mid m \equiv j \pmod n \,\}$.

  Define $\ob{i} + \ob{j} = \begin{cases}
    \ob{i+j} & \text{~if~} 0 \le i + j \le n \\
    \ob{i+j-n} & \text{~otherwise}
  \end{cases} \implies (\quot{\Zb}{n\Zb}, +, \ob{0})$ forms a group.
\end{example}

\begin{remark}
  $\ob{i} \times \ob{j} = \ob{i \times j}$.
  \begin{itemize}
    \item 略
    \item If $\gcd(j, n) = d, \exists h, k \in \Zb$ s.t. $hj + kn = d$.
  \end{itemize}
\end{remark}

\begin{definition}
  $\left( \quot{\Zb}{n\Zb} \right)^\times = \{\, j \in \quot{\Zb}{n\Zb} \mid
  \gcd(j,n) = 1 \,\} \implies \left(\left( \quot{\Zb}{n\Zb} \right)^\times,
  \times, \ob{1} \right)$ forms a group.
\end{definition}

\begin{example}
  略... 簡化剩餘系, 原根 (generator) ($1, 2, 4, p^k, 2p^k$, $p$ is an odd prime)
\end{example}

\begin{definition} \mbox{}
  \begin{itemize}
    \item The {\bf order} of a finite gorup $G$ is the number of elements in
      $G$, denoted by $\abs{G}$.
    \item Let $a \in G$, the order of $a$ is defined to be the least positive
      integer $n$ s.t. $a^n = 1$, denoted by $\ord(a) = n$.
    \item If $a^n \ne 1 \quad \forall n \in \Nb$, then we call
      ``$a$ has infinte order''.
  \end{itemize}
\end{definition}

\begin{prop}
  Let $G = \gen{a}$ with $\ord(a) = n$. Then
  \begin{enumerate}
    \item $a^m = 1 \iff n \mid m$.
      \begin{proof} \mbox{}
        \begin{description}
          \item[$\Leftarrow:$] Let $m = dn$, then $a^m = (a^n)^d = 1$.
          \item[$\Rightarrow:$] Let $m = qn + r, 0 \le r < n$.
            If $r \ne 0$, then $a^r = a^{m - qn} = (a^m)(a^n)^{-q} = 1$.
            But $r < n$, which is a contradiction.
            Hence $r = 0 \implies n \mid m$. \qedhere
        \end{description}
      \end{proof}
    \item $\ord(a^r) = n / \gcd(r, n)$.
      \begin{proof}
        Let $\gcd(r, n) = d, n = dn', r = dr'$ with $\gcd(n', r') = 1$.
        Plan to show ``$\ord(a^r) = n'$.''
        \begin{itemize}
          \item $(a^r)^{n'} = a^{r'dn'} = (a^n)^{r'} = 1 \implies \ord(a^r) \mid n'$.
          \item $1 = (a^r)^{\ord(a^r)} = a^{r \ord(a^r)} \implies
            n \mid r \ord(a^r) \implies n' \mid r' \ord(a^r) \implies
            n' \mid \ord(a^r)$.
        \end{itemize}
      \end{proof}
  \end{enumerate}
\end{prop}

\begin{prop}
  Any subgroup of a cyclic group is cyclic.
  \begin{proof}
    Let $G = \gen{a}$ and $H \le G$. If $H = \{1\}$, then
    $H = \gen{1}$, done!

    Otherwise, $d = \min \{ m \in \Nb \mid a^m \in H \}$, by well-ordering
    axiom. Claim $H = \gen{a^d}$.
    \begin{description}
      \item[$\supset:$] $a^d \in H$ by the definition of $d$.
      \item[$\subset:$] $\forall a^m \in H$, write $m = qd + r, 0 \le r < d$.
        If $r \ne 0$, then $a^r = a^{m - qd} = a^m (a^d)^{-q} \in H$, which
        is a contradiction. Hence $r = 0 \implies d \mid m$.
    \end{description}
  \end{proof}
\end{prop}

\begin{exercise} \mbox{}
  \begin{enumerate}
    \item $\ord(a) = \ord(a^{-1}) = n$.
    \item $\gen{a^r} = \gen{a^{\gcd(n, r)}}$.
    \item $\gen{a^{r_1}} = \gen{a^{r_2}} \iff \gcd(n, r_1) = \gcd(n, r_2)$.
    \item $\forall m \mid n, \exists! H \le \gen{a}$ s.t.
      $\abs{H} = m$. Conversely, if $H \le \gen{a}$, then $\abs{H} \mid n$.
  \end{enumerate}
\end{exercise}

\begin{prop}
  Let $G = \gen{a}$. Then
  \begin{enumerate}
    \item $\ord(a) = n \implies G \cong \quot{\Zb}{n\Zb}$
    \item $\ord(a) = \infty \implies G \cong \Zb$
  \end{enumerate}
  \label{prop:eqzg}
  \begin{exercise}
    Show Prop \ref{prop:eqzg}.
  \end{exercise}
\end{prop}

\begin{definition}
  Let $G_1, G_2 \le G$. $G$ is the internal direct product of $G_1, G_2$ if
  $G_1 \times G_2 \to G, (g_1, g_2) \mapsto g_1g_2$ is an isom.
\end{definition}

\begin{remark}
  In this case, we find that
  \begin{itemize}
    \item $G = G_1 G_2 = \{\, g_1 g_2 \mid g_1 \in G_1, g_2 \in G_2 \,\}$.
    \item $G_1 \cap G_2 = \{ 1 \}$. (consider $a \ne 1 \in G_1 \cap G_2$, then
      $(1, a) \mapsto a, (a, 1) \mapsto a$, but the function is 1-1, which
      is a contradiction.)
    \item If $a \in G$ with $a = g_1g_2 = g_1'g_2'$, then
      $(g_1')^{-1}g_1 = (g_2')g_2^{-1} \in G_1 \cap G_2 = \{ 1 \} \implies
      \begin{cases} g_1 = g_1' \\ g_2 = g_2'\end{cases}$.
    \item For $g_1 \in G_1, g_2 \in G_2, (g_1, g_2) = (g_1, 1)(1, g_2) =
      (1, g_2)(g_1, 1) \implies g_1g_2 = g_2g_1$.
  \end{itemize}
\end{remark}

\begin{exercise} TFAE
  \begin{enumerate}
    \item $G$ is the internal direct product of $G_1, G_2$.
    \item $\forall a \in G, \exists! g_1 \in G_1, g_2 \in G_2$ s.t.
      $a = g_1g_2$ ; $\forall g_1 \in G_1, g_2 \in G_2, g_1g_2 = g_2g_1$.
    \item $G_1 \cap G_2 = \{ 1 \}$ ; $G = G_1G_2$ ;
      $\forall g_1 \in G_1, g_2 \in G_2, g_1g_2 = g_2g_1$.
  \end{enumerate}
\end{exercise}

\begin{example} \mbox{}
  \begin{enumerate}
    \item $G = \quot{\Zb}{6\Zb} = \{ \ob{0}, \ob{1}, \ob{2}, \ob{3}, \ob{4}, \ob{5} \},
      G_1 = \{ \ob{0}, \ob{3} \}, G_2 = \{ \ob{0}, \ob{2}, \ob{4} \}$.
      We have $G \cong G_1 \times G_2$.
    \item $G = S_3, G_1 = \gen{\cycle{1,2}}, G_2 = \gen{\cycle{1,2,3}}$.
      We have $G_1 \times G_2 \not\cong G$ since $\cycle{1,2}\cycle{1,2,3} \ne
      \cycle{1,2,3}\cycle{1,2}$.
  \end{enumerate}
\end{example}

\begin{example}
  $G = S_3, G_1 = \gen{\cycle{1,2}}, G_2 = \gen{\cycle{2,3}},
  G_1G_2 = \{ 1, \cycle{1,2}, \cycle{2,3}, \cycle{1,2,3} \} \not\le G$
  since $\cycle{1,3,2} = \cycle{1,2,3}^{-1} \notin G_1G_2$.
\end{example}

\begin{prop}
  Let $H, K \le G$. Then $HK \le G \iff HK = KH$.
  \begin{proof} \mbox{}
    \begin{description}
      \item[$\Rightarrow:$] $\begin{cases} H \le HK \\ K \le HK \end{cases}
          \implies KH \subseteq HK$ ;
          $\forall hk \in HK, \exists h'k' \in HK$ s.t. $(hk)(h'k') = 1 \implies
          hk = (k')^{-1}(h')^{-1} \in KH \implies HK \subseteq KH$.
      \item[$\Leftarrow:$] For $h_1k_1, h_2k_2 \in HK$, $(h_1k_1)(h_2k_2)^{-1}
        = h_1k_1k_2^{-1}h_2^{-1} = h_1h'k' \in HK$.
    \end{description}
  \end{proof}
\end{prop}

