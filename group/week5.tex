%! TEX root=../main.tex
\subsection{Week 5}
\subsubsection{Group actions \RNum{1}}

\begin{definition}
  A group $G$ is said to act on a nonempty set $X$ if $\exists$ a map
  $G \times X \to X$ with $(g, x) \mapsto gx$ s.t.
  \begin{enumerate}
    \item $1x = x$
    \item $(g_1g_2)x = g_1(g_2 x) \quad \forall g_1, g_2 \in G$
  \end{enumerate}
\end{definition}

\begin{prop}
  $\{ \text{actions of $G$} \} \leftrightarrow
  \{ \text{group homo.~} G \to S_X \}$
  \begin{proof}
    Given an action $(g, x) \mapsto gx$, consider $\varphi: G \to S_X$ s.t.
    $\varphi: g \mapsto (\tau_g: x \mapsto gx)$.
    \begin{itemize}
      \item 1-1: $gx = gy \implies g^{-1}(gx) = y \implies x = y$.
      \item onto: $\forall y \in X$, let $x = g^{-1}y$, then $y = gx$.
      \item group homo.: $\varphi(gg') = (\tau_{gg'}: x \mapsto gg'x)
        = \tau_g \circ \tau_g' = \varphi(g)\varphi(g')$.
    \end{itemize}
    Conversely, given a group homo. $\varphi: G \to S_X$, consider
    $(g, x) \mapsto \varphi(g)(x)$.
    \begin{itemize}
      \item $1x = \varphi(1)(x) = \text{Id}(x) = x$.
      \item $g_1g_2x = \varphi(g_1g_2)(x) = \varphi(g_1) \circ \varphi(g_2)(x)
        = g_1(g_2x)$. \qedhere
    \end{itemize}
  \end{proof}
\end{prop}

\begin{definition}
  A representation of $G$ on a vector space $V$ is a group action of $G$ on
  $V$ linearly. i.e. $\exists$ group homo. $\varphi: G \to \text{GL}(V)$.
\end{definition}

\begin{example}
  \[
    \quot{\Zb}{m\Zb} \to \text{SO}(2), \quad
    \ob{k} \mapsto \begin{pmatrix}
      \cos \frac{2k\pi}{m} & -\sin \frac{2k\pi}{m} \\
      \sin \frac{2k\pi}{m} & \cos \frac{2k\pi}{m}
    \end{pmatrix}
  \]
\end{example}

\begin{example}
  \[
    S_n \to \text{GL}(n, \Rb), \quad
    \sigma \mapsto (\tau_\sigma: e_i \mapsto e_{\sigma(i)})
  \]
\end{example}

\begin{remark} \mbox{}
  \begin{enumerate}
    \item An action $G \times X \to X$ is said to be faithful if the
      corresponding group homo. $\varphi: G \toone S_X$, denoted by
      $G \acts X$.
    \item In general, $\ker \varphi = \{\, g \in G \mid gx = x \quad
      \forall x \in X \,\} = \bigcap_{x \in X} \{\, g \mid gx = x \,\}$.

      Define $G_x = \{\, g \mid gx = x \,\} \le G$ is the isotropy subgroup
      of $G$ at $x$. (the stabilizer of $G$ at $x$)
    \item $\varphi: G \to S_X \implies \quot{G}{\Ker \varphi} \toone S_X$.
      So $\quot{G}{\Ker \varphi} \times X \to X$ is faithful.
    \item Let $\mathcal{C}(X) = \{\, f: X \to \Cb \,\}$. If $G \acts X$, then
      $G \acts \mathcal{C}(X)$ by
      $G \times \mathcal{C}(X) \to \mathcal{C}(X)$ with
      $(g, f) \mapsto g f(x) = f(g^{-1}x)$.

      The reason: $(g_1g_2)f(x) = f((g_1g_2)^{-1}x) = f(g_2^{-1}g_1^{-1}x)
      = g_1(g_2 f)(x)$.
  \end{enumerate}
\end{remark}

\begin{definition}
  Let $G \acts X$ and $x \in X$.
  \begin{itemize}
    \item The {\bf orbit} of $x$ is defined to be
      $Gx = \{\, gx \mid g \in G \,\}$.
    \item $G \acts X$ is said to be transitive if $\exists$ only one orbit.
      i.e. $\forall x, y \in X, \exists g \in G$ s.t. $y = gx$.
  \end{itemize}
  The set of orbits forms a partition: $x \sim y \iff \exists g \in G
  \text{~s.t.~} y = gx$.
\end{definition}

\begin{prop}
  Let $G \acts X$ and $x \in X$. Then $\abs{Gx} = [G : G_x]$.

  In particular, $\abs{G} < \infty \implies \abs{G} = \abs{Gx}\abs{G_x}
  \quad \forall x \in X$.
  \begin{proof}
    Define $\psi: Gx \to \{ \text{left coset of~} G_x \}$ as
    $\psi: gx \mapsto g G_x$.
    \begin{itemize}
      \item well-defined and 1-1:
        $g_1x = g_2x \iff g_2^{-1}g_1x = x \iff g_2^{-1}g_1 \in G_x \iff
        g_2^{-1}g_1G_x = G_x \iff g_1G_x = g_2G_x$.
      \item onto: $\forall g \in G, \psi(gx) = gG_x$. \qedhere
    \end{itemize}
  \end{proof}
\end{prop}

\subsubsection{Action by left multiplication}
\begin{itemize}
  \item The action $G \times G \to G, \quad (g, x) \mapsto gx$ is associated
    with $\varphi: G \toone S_G$.
    It is faithful (Cayley theorem) and transitive.
  \item Let $H \le G$ and $X \defeq \{ \text{left coset of~} H \}$.
    The group action $(g, xH) \mapsto gxH \leadsto \varphi: G \to S_X$.
    \[
      \ker \varphi = \bigcap_{x \in G} \tikz[anchor=base, baseline]{
      \node (conj-of-H) {$\underbrace{xHx^{-1}}$}; } \le H
    \]
    \begin{tikzpicture}[overlay]
      \node[inner sep=0pt,outer sep=0pt] (t) at ($(conj-of-H) + (2, -0.5)$) {
         \footnotesize a conjugate of $H$};
       \path[->,shorten >= 6pt] ($(conj-of-H.base) + (0, -0.3)$) edge
         [bend right=10] (t.west) ;
    \end{tikzpicture}
    which is the largest normal subgroup in $G$ contained in $H$.
    \begin{proof}
      If $\begin{cases} N \lhd G \\ N \le H\end{cases}, \forall x \in G,
        xNx^{-1} \le xHx^{-1} \implies N = N(xx^{-1}) = xNx^{-1} \le xHx^{-1}$.
    \end{proof}
\end{itemize}

\begin{prop}
  Let $H \le G$ with $[G:H] = p$ being the smallest prime dividing $\abs{G}$.
  Then $H \lhd G$.
  \begin{proof}
    Let $X = \{ a_1H, \dots, a_pH \}$ (all left coests of $H$) and
    $\varphi: G \to S_p$ be the associated group homo. for the group action
    $(g, a_iH) \mapsto ga_iH$.

    By the 1st isom. thm., $\quot{G}{\Ker\varphi} \toone S_p$.

    By Lagrange thm. $\abs*{\quot{G}{\Ker\varphi}} \Div \abs{S_p} = p!$ and
    $\abs*{\quot{G}{\Ker\varphi}} \Div \abs{G} \implies
    \abs*{\quot{G}{\Ker\varphi}} \Div p$.

    So $\abs*{\quot{G}{\Ker\varphi}} = 1 \text{~or~} p$.

    If $\abs*{\quot{G}{\Ker\varphi}} = 1 \implies G = \Ker\varphi \le H \lneq G$,
    which is a contradiction.

    So $\abs*{\quot{G}{\Ker\varphi}} = p \implies [G:\Ker\varphi] = p
    \implies [G:H][H:\Ker\varphi] = p \implies [H:\Ker\varphi] = 1
    \implies H = \Ker\varphi \lhd G$.
  \end{proof}
\end{prop}

\subsubsection{Action by conjugation}
\begin{itemize}
  \item The action $G \times G \to G \quad (g,x) \mapsto gxg^{-1}$ is
    associated with the group homo. $\varphi: G \to S_G \quad g \mapsto
    (\tau_g: x \mapsto gxg^{-1})$.
    \[
      \Inn(G) \defeq \{\, \tau_g \mid g \in G \,\}
    \]

    \begin{fact} 
      $\tau_g$ is an automorphism. (isom. $G \to G$)
    \end{fact}

    So $\varphi: G \onto \Inn(G) \le \Aut(G) \le S_G$.

    $\Ker\varphi = \{\, g\in G \mid gxg^{-1} = x \quad \forall x \in G \,\}
    = Z_G$.

    By the 1st isom. thm., $\quot{G}{\Ker\varphi} \cong \Inn(G)$.
    \begin{itemize}
      \item The conjugacy class:
        $Gx = \{\, gxg^{-1} \mid g \in G \,\} = \text{Cl}(x)$.
      \item The centralizer of $x$ in $G$:
        $G_x = \{\, g \in G \mid gxg^{-1} = x \,\} = Z_G(x)$.
    \end{itemize}
    \[
      \abs{\text{Cl}(x)} = [G:Z_G(x)], \text{~if~} \abs{G} < \infty, 
      \abs{G} = \abs{\text{Cl}(x)}\abs{Z_G(x)}
    \]
  \item For $H \lhd G$, define $G \times H \to H \quad (g, h) \mapsto ghg^{-1}$
    with the group homo. $\varphi: G \to \Aut(H)$.
    \[
      \Ker\varphi = \{\, g\in G \mid gxg^{-1} = x \quad \forall x \in H \,\}
      = Z_G(H)
      \implies \quot{G}{Z_G(H)} \le \Aut(H)
    \]
  \item The normalizer of $H$ in $G$:
    $N_G(H) = \{\, g\in G \mid gHg^{-1} \le H \,\}$
\end{itemize}
