%! TEX root=../main.tex
\subsection{Week 4}
\subsubsection{Universal property and direct sum \& product}
In general, let $f_1: G_1 \to G, f_2: G_2 \to G$ are group homo.
$f_1 \times f_2: G_1 \times G_2 \to G, (a, b) \mapsto f_1(a)f_2(b)$.
But we have $(a, b) = (a, 1)(1, b) = (1, b)(a, 1)$, so
$f_1(a)f_2(b) = f_2(b)f_1(a) \implies $ need $G$ to be abelian.

So we intend to define the direct sum in the category of abelian group.

\underline{Notation}: For abelian groups, we use ``$+$'' to denote the group
operation and ``$0$'' to denote the identity.

\begin{definition}
  Given a non-empty family of abelian groups $\{\, G_s \mid s \in \Lambda \,\}$,
  a (external) direct sum of $\{\, G_s \mid s \in \Lambda \,\}$ is an
  abelian group $\bigoplus_{s\in \Lambda} G_s$ with the embedding mappings
  $i_{s_0}: G_{s_0} \to \bigoplus_{s\in \Lambda} G_s,
  \forall s_0 \in \Lambda$ satisfying the universal property:

  for any abelian group $H$ and group homo. $\varphi_s: G_s \to H
  \forall s \in \Lambda, \quad \exists!$ group homo. $\varphi:
  \bigoplus_{s\in \Lambda} G_s \to H$ s.t. 又一個ㄛ圖
\end{definition}

\begin{theorem}
  $\bigoplus_{s\in \Lambda} G_s$ exists and is unique up to isomorphisms.

  \begin{proof}
    Existence: $\bigoplus_{s\in \Lambda} G_s = \{\, (g_s)_{s\in \Lambda}
      \mid g_s \in G_s, \text{~almost all of the $g_s$' are $0$} \,\}$ and
      \[ i_{s_0}: G_{s_0} \to \bigoplus_{s\in \Lambda} G_s,
        a_{s_0} \mapsto (g_{s_0})_{s\in \Lambda} \text{~with~}
        g_{s_0} = a_{s_0}, g_s = 0, \forall s \ne s_0. \]
        group operaion: $(g_s)_{s \in \Lambda} + (g_s')_{s \in \Lambda}
        \defeq (g_s + g_s')_{s \in \Lambda} \in
        \bigoplus_{s\in \Lambda} G_s$.
        這邊也一個ㄛ圖

    Uniqueness: Assume $\exists$ another $G$ satisfies the universal property,
    一個大ㄛ圖 ($G, \bigoplus_{s\in \Lambda} G_s$ 互相有唯一個映射可以
    keep $i_{s_0}$, $\varphi \circ \psi = \text{id}_{G}, \psi \circ \varphi
    = \text{id}_{\bigoplus_{s\in \Lambda} G_s}$)
  \end{proof}
\end{theorem}

\begin{definition}
  Given a non-empty family of groups $\{\, G_s \mid s \in \Lambda \,\}$,
  a direct product of $\{\, G_s \mid s \in \Lambda \,\}$ is a group
  $\prod_{s\in \Lambda} G_s$ with projections
  $p_{s_0}: \prod_{s\in \Lambda} G_s \to G_{s_0}, \forall s_0 \in \Lambda$
  satifsfying the following universal property:

  for any group $H$ with group homo.
  $\varphi_s: H \to G_s, \forall s \in \Lambda$, $\exists! \varphi:
  H \to \prod_{s\in \Lambda} G_s$ s.t. 又一個ㄛ圖
\end{definition}

\begin{theorem}
  $\prod_{s\in \Lambda} G_s$ exists and is unique up to isomorphisms.

  \begin{proof}
    Existence: $\prod_{s\in \Lambda} G_s = \{\, (g_s)_{s\in \Lambda}
      \mid g_s \in G_s \,\}$ and
      \[ p_{s_0}: \prod_{s\in \Lambda} G_s \to G_{s_0},
        (g_{s_0})_{s\in \Lambda} \mapsto g_{s_0}, \forall s_0 \in \Lambda \]
      \begin{itemize}
        \item group operaion: $(g_s)_{s \in \Lambda} \cdot (g_s')_{s \in \Lambda}
          \defeq (g_s g_s')_{s \in \Lambda} \in \prod_{s\in \Lambda} G_s$.
        \item Define $\varphi$:
          這邊也一個ㄛ圖
          which is uniquely defined.
      \end{itemize}

    Uniqueness: Assume $\exists$ another $G$ satisfies the universal property,
    一個大ㄛ圖 ($G, \prod_{s\in \Lambda} G_s$ 互相有唯一個映射可以
    keep $i_{s_0}$, $\varphi \circ \psi = \text{id}_{G}, \psi \circ \varphi
    = \text{id}_{\prod_{s\in \Lambda} G_s}$)
  \end{proof}
\end{theorem}

\begin{exercise}
  Google the definition of the {\bf direct limit} and show the existence and
  uniqueness.
\end{exercise}

\begin{exercise}
  Google the definition of the {\bf inverse limit} and show the existence and
  uniqueness.
\end{exercise}

\underline{Motivation}: $\zeta_m$ is called an $m$-th root of unity if
$\zeta_m^m = 1$.
\[ \varinjlim\limits_n \quot{\Zb}{2^n\Zb} \cong
\{\, \text{$2^n$-th roots of unity} : n \in \Nb \,\} \]


\[ \varinjlim\limits_n \quot{\Zb}{2^n\Zb} =
  \quot{(\bigoplus_{n\in\Nb} \quot{\Zb}{2^n\Zb})}{
  \gen{ i_k(a) - i_j(f_{kj}(a)) \mid k \le j, a \in \quot{\Zb}{2^k\Zb} }}
\]
where $f_{kj}: \quot{\Zb}{2^k\Zb} \to \quot{\Zb}{2^j\Zb}$.

Inverse limit:
\[
  \varprojlim \quot{\Zb}{2^n\Zb} = \left\{\,
    (n_1, n_2, \dots ) \in \prod_n \quot{\Zb}{2^n\Zb} \middle|
    \forall i < j, n_i \equiv n_j \pmod 2^{i+1}   \,\right\}
\]

\subsubsection{Rings and fields}

\begin{definition}
  A {\bf ring} is sa non-empty set $R$ with two operations $R\times R \to R$
  \[
    (a, b) \mapsto a + b \quad \text{and} \quad (a, b) \mapsto ab
  \]
  satisfying
  \begin{enumerate}
    \item $(R, +, 0)$ is an abelian group.
    \item $(R, \cdot)$ is a semigroup. (if it is a monoid, then it is called
      ``a ring with 1.'')
    \item (Distributive laws) $\forall a, b, c \in \Rb, \begin{cases}
      a(b + c) = ab + ac\\ (b + c)a = ba + ca\end{cases}$
  \end{enumerate}
\end{definition}

\begin{example}
  $\Zb, \Rb, \Cb, \quot{\Zb}{n\Zb}, M_{n\times n}(\Fb)$
\end{example}

\begin{example}
  Let $G$ be an abelian group.
  Define (endomorphism, automorphism)
  \[
    \text{End}(G) \defeq \{\, \text{group homo.~} G \to G \,\} \quad
    \text{Aut}(G) \defeq \{\, \text{group isom.~} G \to G \,\}
  \]
  A natural ring structure on $\text{End}(G)$ is:
  \[
    \forall a \in G, \begin{cases}
      (f+g)(a) \defeq f(a)g(a) \\
      (f\cdot g)(a) \defeq f(g(a))
    \end{cases}
  \]
\end{example}

\begin{example}
  $\Zb\left[\sqrt{2}\right] = \left\{\,
  a + b\sqrt{2} \relmiddle| a, b \in \Zb \,\right\} \subset \Rb$.
\end{example}

\begin{definition}
  Let $R$ be a ring with $1$.
  \begin{enumerate}[(a)]
    \item $\forall a \in R, a \ne 0$, a in called a unit if
      $\exists a^{-1} \in R$.
  \item $\left(R^\times = \{\text{units in $R$}\}, \cdot, 1)\right)$ forms
    a group.
  \item $R$ is called a division ring if $R \setminus \{0\} = R^\times$.
  \item $R$ is said to be commutative if $ab = ba, \forall a, b \in R$.
  \item $R$ is a field if $R$ is a commutative division ring.
  \item $a \ne 0$ is called a left zero divisor if $\exists b \in R, b \ne 0$
    s.t. $ab = 0$.
  \item $a$ is called a zero divisor if $a$ is either a left or right zero
    divisor.
  \item $R$ is called an integral domain if $R$ is a commutative ring without
    zero divisors.
  \end{enumerate}
\end{definition}

\underline{Fact}:
\begin{enumerate}
  \item fields $\implies$ integral domains.
  \item finite + integral domain $\implies$ fields.
    \begin{proof}
      Let $R = \{ 0, a_1, \dots, a_n \}$, for $a \in R, a \ne 0$,
      $aa_i = aa_j \implies a(a_i - a_j) = 0 \implies i = j$.
      So $\{0, aa_1, \dots, aa_n \} = R \implies \exists a_i$ s.t. $aa_i = 1$.
    \end{proof}
\end{enumerate}

\begin{prop}
  TFAE
  \begin{enumerate}
    \item $\quot{\Zb}{n\Zb}$ is an integral domain.
    \item $\quot{\Zb}{n\Zb}$ is a field.
    \item $n = p$ is a prime.
  \end{enumerate}
  easy to prove.
\end{prop}

\begin{definition} \mbox{}
  \begin{itemize}
    \item $f: R_1 \to R_2$ is called a ring homomorphism if
      $\forall a, b \in R, \begin{cases}
        f(a+b) &= f(a) + f(b) \\
        f(ab) &= f(a)f(b)
      \end{cases}$.
    \item $\Image f$ is a subring of $R_2$.
    \item $\Ker f = \{\, x \in R_1 \mid f(x) = 0 \,\}$ is an additive group of
      $R_1$ and $\forall r \in R_1, x \in \Ker f, f(rx) = f(r)f(x) = f(r)0 = 0
      \implies rx \in \Ker f, xr \in \Ker f$.
    \item $\quot{R_1}{\Ker f}$ is an additive group and
      $\quot{R_1}{\Ker f} \cong \Image f$ (additive isomorphism).
  \end{itemize}
\end{definition}

\begin{definition}
  Let $I$ be an additive subgroup of $R$.
  $I$ is called an ideal if $\forall r \in R, x \in I, rx \in I, xr \in I$.

  $\left(\quot{R}{I}, +, \cdot \right)$ forms a quotient ring under
  \[ \forall r_1, r_2 \in R, (r_1+I)(r_2+I) = r_1r_2 + I \]
  well-defined: easy to show.
\end{definition}

\begin{exercise}
  State and show the isomorphism theorems and the factor theorem.
\end{exercise}

\begin{prop}
  If $R$ is a ring with $1$, then $\exists!$ ring homo. $\varphi: \Zb \to R$
  s.t. $\varphi(1) = 1$.
  \begin{proof}
    Let $\varphi: \Zb \to R$ is a ring homo. s.t. $\varphi(1) = 1$. Then
    $\forall n \in \Zb, \varphi(n) = \varphi(1) + \dots + \varphi(1) = n1$.
    Now $\forall n, m \in \Zb, \varphi(n)\varphi(m) = (n1)(m1) = n(m1) = (nm)1$
    by the distributive law. So $\varphi$ is well-defined and unique.
  \end{proof}
  \label{prop:phi1e1}
\end{prop}

\begin{definition}
  In Prop \ref{prop:phi1e1}, $\Ker \varphi = m\Zb$ for some $m > 0$.
  We call $m$ the characteristic of $R$, denoted by $\Char R = m$.
\end{definition}

\begin{prop} \mbox{}
  \begin{enumerate}
    \item If $R$ is an integral domain, then $\Char R = 0 \text{~or~} p$,
      where $p$ is a prime. (try to prove this)
    \item In the case of $\Char R = p$,
      $\forall a, b \in R, (a + b)^p = a^p + b^p$.
      \begin{proof}
        \[ (a+b)^p = a^p + \binom{p}{1}a^{p-1}b + \dots + b^p = a^p + b^p \]
        because $p \mid \binom{p}{1} \implies \binom{p}{i}a^{p-i}b^{i} = 0$.
      \end{proof}
  \end{enumerate}
\end{prop}

\begin{exercise}
  Let $F$ be a field. Show that
  \begin{enumerate}
    \item if $\Char F = 0$, then $\Qb \toone \text{subfield of~} F$.
    \item if $\Char F = p$, then
      $\quot{\Zb}{p\Zb} \toone \text{subfield of~} F$.
  \end{enumerate}
  \label{ex:4-4}
\end{exercise}

\begin{theorem}
  If $F$ is a finite field, then $\abs{F} = p^n$ for some $n \in \Nb$ and
  $p$ is a prime.
  \begin{proof}
    By Ex. \ref{ex:4-4}, $\Char F = p$, $p$ is a prime and $\quot{\Zb}{p\Zb}
    \toone F$.

    We have $\quot{\Zb}{p\Zb} \times F \to F, (r, v) \mapsto rv$.
    $F$ can be rearded as a vector space over $\quot{\Zb}{p\Zb}$.

    Let $\dim_{\quot{\Zb}{p\Zb}} F = n$, then $F \cong
    \left(\quot{\Zb}{p\Zb}\right)^n \implies \abs{F} = n$.
  \end{proof}
\end{theorem}

\begin{theorem}
  Let $F$ be a field. Then any finite subgroup $G$ of $(F^\times, \cdot, 1)$
  is cyclic.

  \begin{proof}
    Let $\abs{G} = n$. Define $h$ to be the max order of an element in $G$,
    say $a^h = 1$.

    If $h = n$, then $\abs{\gen{a}} = h = n = \abs{G}$ and $\gen{a} \subseteq G$,
    so $G = \gen{a}$.

    Otherwise, $h < n$. We know that $x^h - 1$ has at most $h$ roots.
    So $\exists b \in G$ is not a root of $x^h - 1$.
    Let $\ord(b) = h'$, so $h' \mid n$ and $h' \not\mid h$.
    So $\exists$ a prime $p$ s.t. $p^r \mid h'$ but $p^r \not\mid h$.

    Write $h = mp^s, s < r$ and $\gcd(m, p) = 1 \implies
    \ord \left( a^{p^s} \right) = m$.

    Write $h' = qp^r \implies \ord \left( b^q \right) = p^r$.

    Since $\gcd(m, p^r) = 1, \ord\left(a^{p^s} b^q \right) = mp^r > mp^s = h$,
    which is a contradiction.
  \end{proof}
\end{theorem}

\begin{exercise} \mbox{}
  \begin{enumerate}
    \item Let $a, b \in G$ with $ab = ba$ and $\ord(a) = m, \ord(b) = n$.
      If $\gcd(m, n) = 1$, then $\ord(ab) = mn$.
      In general, is the order of $ab$ equal to $\lcm(m, n)$?
    \item Let $G$ be a finite group and $H, K \le G$. Then
      $\abs{HK} = \frac{\abs{H}\abs{K}}{\abs{H \cap K}}$.
  \end{enumerate}
\end{exercise}
