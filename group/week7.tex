%1 TEX root=../main.tex
\subsection{Week 7}
\subsubsection{Composition series}
\underline{Ques}: How to simplify a finite group $G$?

\underline{Strategy}:
\begin{itemize}
  \item If $G = \{1\}$, then done.
  \item Otherwise, check whether $G$ has a nontrivial proper normal subgroup.
  \item If no, then $G$ is said to be a simple group.
  \item Otherwise, find a normal subgroup $G_1$ as large as possible s.t.
    $\quot{G}{G_1}$ is simple.
  \item If $G_1$ is simple, then done.
  \item Otherwse, repeat above on $G_1$ and get $G_2, \dots, G_n$ s.t.
    \[ G_n = \{1\} \lhd G_1 \lhd \dots \lhd G_1 \lhd G_0 = G \quad
      \tikz[anchor=base, baseline]{
    \node(n-comp-fac) {$\quot{G_i}{G_{i+1}}$}; } \text{~is simple} \]
    \begin{tikzpicture}[overlay]
      \node[inner sep=0pt,outer sep=0pt] (t) at ($(n-comp-fac) + (2, -0.5)$) {
         \footnotesize composition factors};
       \path[->,shorten >= 6pt] ($(n-comp-fac.base) + (0, -0.15)$) edge
         [bend right=10] (t.west) ;
    \end{tikzpicture}
    Say ``it is a composition series'' with $\text{length}(G) = n$.
\end{itemize}

Hence simple groups can be regarded as basic building blocks of groups.

The classification of all finite simple groups is given as follows:
\begin{enumerate}
  \item $\quot{\Zb}{p\Zb}$, $p$ is a prime.
  \item $A_n, n \ge 5$.
  \item simple groups of Lie type.
  \item $26$ sporadic simple groups.
\end{enumerate}

\begin{example}
  $G = S_4, G_1 = A_4, G_2 = V_4, G_3 = \gen{\cycle{1,2}\cycle{3,4}},
  G_4 = \{1\} \leadsto \text{length}(S_4) = 4$.

  factors: $C_2, C_3, C_2, C_2$.
\end{example}

\begin{example}
  $G = \quot{\Zb}{12\Zb} = \gen{\bar{1}}$.
  \begin{itemize}
    \item $G_1 = \gen{\bar{2}}, G_2 = \gen{\bar{4}}, G_3 = \gen{\bar{0}}
      \leadsto \text{length}(3)$, factors: $C_2, C_2, C_3$.
    \item $G_1' = \gen{\bar{2}}, G_2' = \gen{\bar{6}}, G_3' = \gen{\bar{0}}
      \leadsto \text{length}(3)$, factors: $C_2, C_3, C_2$.
    \item $G_1'' = \gen{\bar{3}}, G_2'' = \gen{\bar{6}}, G_3'' = \gen{\bar{0}}
      \leadsto \text{length}(3)$, factors: $C_3, C_2, C_2$.
  \end{itemize}
\end{example}

\begin{example}
  Let $\abs{G} = p^n$. We know $\forall 0 \le k \le n$, $\exists G_k \lhd G$
  with $\abs{G_k} = p^k$ and $G_i \lneq G_{i+1}$.

  $\text{length}(G) = n$, factors: $C_p, \dots, C_p$. ($n$ times)
\end{example}

\begin{theorem}[Jorden-H\"older theorem]
  If $G$ has a composition series, then any two composition series have the
  same length and the same factors up to permutation.
\end{theorem}

\begin{lemma}[Zassenhaus lemma]
  Let $H' \lhd H \le G, K' \lhd K \le G$. Then 
  $(H\cap K')H' \lhd (H\cap K)H', (H'\cap K)K' \lhd (H\cap K)K'$ and
  \[
    \quot{(H\cap K)H'}{(H\cap K')H'} \cong \quot{(H\cap K)K'}{(H'\cap K)K'}.
  \]
\end{lemma}

\begin{theorem}[Schreier theorem]
  Any two normal series of $G$ have equivalent refinements.

  refinements: inserting a finite number of subgroups into the normal series.
  \begin{proof}
    For two normal series:
    \begin{align*}
      & \{1\} = H_r \lhd H_{r-1} \lhd \dots \lhd H_1 \lhd H_0 = G \\
      & \{1\} = K_s \lhd K_{r-1} \lhd \dots \lhd K_1 \lhd K_0 = G
    \end{align*}
    We define
    \begin{align*}
      H_{ij} = (H_i \cap K_j)H_{i+1} \\
      K_{ji} = (H_i \cap K_j)K_{j+1}.
    \end{align*}
    Then we have
    \begin{align*}
      & \{1\} = H_{(r-1)s} \lhd H_{(r-1)(s-1)} \lhd \dots \lhd
      H_{(r-1)0} = H_{r-1} = H_{(r-2)s} \lhd \dots \lhd
      H_{10} = H_1 = H_{0s} \lhd \dots \lhd H_{00} = G \\
      & \{1\} = K_{(s-1)r} \lhd K_{(s-1)(r-1)} \lhd \dots \lhd
      K_{(s-1)0} = K_{s-1} = K_{(s-2)r} \lhd \dots \lhd
      K_{10} = K_1 = K_{0r} \lhd \dots \lhd K_{00} = G
    \end{align*}
    Both have size $= rs$. By lemma,
    $\quot{H_{ij}}{H_{i(j+1)}} \cong \quot{K_{ji}}{K_{j(i+1)}}$.
    Note that if $H_{ij} = H_{i(j+1)}$, then $K_{ji} = K_{j(i+1)}$.
  \end{proof}
\end{theorem}

\begin{proof}[proof of Jorden-H\"older theorem]
  Let 
  \[
  \begin{cases}
    \{1\} = G_n \lhd \dots \lhd G_1 \lhd G_0 = G  & (*)\\
    \{1\} = G_m' \lhd \dots \lhd G_1' \lhd G_0' = G & (**)
  \end{cases}
  \]
  be two composition series.

  By Schreier theorem, we get two refined equivalent series $(*)', (**)'$.
  Since $(*), (**)$ are already composition series, $(*)=(*)', (**)=(**)'$
  So $(*), (**)$ are equivalent.
\end{proof}

\begin{proof}[proof of lemma]
  First prove $(H\cap K')H' \lhd (H\cap K)H'$.
  \begin{itemize}
    \item $\forall g \in H \cap K, g K'g^{-1} = K' \leadsto
      (gHg^{-1}) \cap (gK'g^{-1} = H \cap K'$ and $gH'g^{-1} = H'$. So
      \[ g(H\cap K')H'g^{-1} = (H\cap K')H' \]
    \item $\forall g \in H', ab \in (H\cap K')H'$, 
  \end{itemize}

  To prove
  \[
    \quot{(H\cap K)H'}{(H\cap K')H'} \cong \quot{(H\cap K)K'}{(H'\cap K)K'}.
  \]
  \begin{align*}
    \quot{(H\cap K)H'}{(H\cap K')H'} &\cong
    \quot{(H\cap K)(H\cap K')H'}{(H\cap K')H'} \\
    &\cong \quot{(H\cap K)}{(H\cap K)\cap(H\cap K')H'} \\
    &\cong \quot{(H\cap K)}{K\cap(H\cap K')H'} \\
    &\cong \quot{(H\cap K)}{(H'\cap K)(H\cap K')}
  \end{align*}
  ($K\cap(H\cap K')H' = (H'\cap K)(H\cap K')$, tricky)
  By symmetry, 
  \[
    \quot{(H\cap K)K'}{(H'\cap K')K'} \cong
    \quot{(H\cap K)}{(H'\cap K)(H\cap K')}
  \]
\end{proof}

\begin{prop}
  Let $\abs{G} < \infty$. Then $G$ is solvable $\iff$ all composition factors
  are cyclic of prime order.
  \begin{proof}
    ``$\Leftarrow$'': by def.

    ``$\Rightarrow$'': If $\quot{G_i}{G_{i+1}} \cong C_n$ with
    $n = p_1^{m_1} p_2^{m_2} \dots p_r^{m_r}$.
  \end{proof}
\end{prop}

\begin{observation*}
  Let $K \lhd G$. 把 $K, \quot{G}{K}$ 拆成兩個 composition series 的話,
  就可以把兩串接起來,長度就是加起來。
\end{observation*}

\begin{exercise}
  Let $\{1\} = G_n \lhd G_{n-1} \lhd \dots \lhd G_1 \lhd G_0 = G$ be a
  composition series of $G$ and $K \lhd G$.

  Then after we eliminate equalities,
  \begin{enumerate}
    \item $\{1\} = (K \cap G_n) \lhd (K \cap G_{n-1}) \lhd \dots \lhd
      (K \cap G_1) \lhd (K \cap G_0) = K$ is a composition series of $K$.
    \item $\{\bar{1}\} = \quot{KG_n}{K} \lhd \quot{KG_{n-1}}{K} \lhd \dots \lhd
      \quot{KG_1}{K} \lhd \quot{KG_0}{K} = \quot{G}{K}$ is a composition
      series of $\quot{G}{K}$.
  \end{enumerate}
\end{exercise}

\begin{exercise}
  Let $\begin{cases}
    H \lhd G \\
    K \lhd G
  \end{cases}$ with $H \ne K$ s.t. $\quot{G}{H}, \quot{G}{K}$ are simple.
  Then $\quot{H}{H\cap K}, \quot{K}{K \cap H}$ are simple too.
\end{exercise}

\begin{exercise}
  Let $\{1\} = G_n \lhd G_{n-1} \lhd \dots \lhd G_1 \lhd G_0 = G$ be a
  composition series of length $n$.

  Show by induction on $n$ that for every composition series of $G$:
  \[
    \{1\} = H_m \lhd H_{n-1} \lhd \dots \lhd H_1 \lhd H_0 = G,
  \]
  we have $m = n$ and
  \[
    \left\{
      \quot{H_{n-1}}{H_n}, \dots, \quot{H_0}{H_1}
    \right\} =
    \left\{
      \quot{G_{n-1}}{G_n}, \dots, \quot{G_0}{G_1}
    \right\}
  \]

\end{exercise}

\begin{exercise}
  Exhibit all composition series for
  $Q_8, D_4, \quot{\Zb}{8\Zb}, \quot{\Zb}{4\Zb} \oplus \quot{\Zb}{2\Zb},
  \quot{\Zb}{2\Zb} \oplus \quot{\Zb}{2\Zb} \oplus \quot{\Zb}{2\Zb}$
  respectively.
\end{exercise}

\subsubsection{Modules over a PID}

\begin{definition}
  Let $R$ be a ring with $1$. A $R$-moduule is an abelian group $M$
  (written additively) on which $R$ acts linearly.
  $R \times M \to M \quad (r, x) \mapsto rx$
  \begin{enumerate}
    \item $r(x + y) = rx + ry \quad r \in R, x, y \in M$
    \item $(r_1 + r_2)x = r_1x + r_2x \quad r_1, r_2 \in R, x \in M$
    \item $(r_1r_2)x = r_1(r_2x) \quad r_1, r_2 \in R, x \in M$
    \item $1x = x \quad x \in M$
  \end{enumerate}
\end{definition}

\begin{example}
  A $k$-vector space is a $k$-module.
\end{example}

\begin{example}
  An abelian group $G$ can be regarded as a $\Zb$-module.
  \[
    \begin{aligned}
      \Zb \times G &\to G\\
      (n, a) &\mapsto na
    \end{aligned}
    \quad \text{by} \quad
    na = \begin{cases}
      \underbrace{a + \dots + a}_{n\text{~times}} & \text{if~} n \ge 0 \\
      \underbrace{(-a) + \dots + (-a)}_{n\text{~times}} & \text{if~} n < 0
    \end{cases}
  \]
\end{example}

\begin{example}
  Let $I$ be an ideal of $R$. Then $I$ can be regarded as an $R$-module
  since $\forall r \in R, a \in I, \quad ra \in I$.
\end{example}

\begin{definition}
  A submodule $N$ of $M$ is an additive subgroup of $M$ s.t.
  $\forall r \in R, a \in N, \quad ra \in N$.
\end{definition}

\begin{prop}
  Let $\phi \ne S \subseteq M$. The submodule generated by $S$ is defined to be
  \begin{align*}
    \gen{S}_R = \left\{
      \sum_{\text{finite}} r_i x_i \middle| x_i \in S, r_i \in R
    \right\} &= \text{the least submodule containg $S$} \\
             &= \bigcap_{S \subset N \subset M} N
  \end{align*}
\end{prop}

\begin{definition}
  An $R$-module $M$ is said to be finitely generated if
  $\exists x_1, \dots, x_n \in M$ s.t.
  $M = \gen{x_1, \dots, x_n}_R = Rx_1 + Rx_2 + \dots Rx_n$
\end{definition}

\begin{example}
  $R$ is generated by $1$ as an $R$-module.
\end{example}

\begin{definition}
  An additive group homo. $\varphi: M_1 \to M_2$ is called an $R$-module
  homo. if
  \[ \varphi(rx) = r\varphi(x) \quad \forall r \in R, x \in M_1 \]
\end{definition}

\begin{definition}
  An integral domain $R$ is called a principal ideal domain (PID)
  if $\forall I$ ideal in $R$, $\exists a \in R$ s.t. $I = \gen{a}_R$.
\end{definition}

\begin{example}
  $\Zb$ is a PID.

  For $I \subseteq \Zb$, $I$ is an additive subgroup, so
  $I = m\Zb = \gen{m}_\Zb$.
\end{example}

\begin{definition}
  $M$ is said to be a free module of rank $n$ if
  $M \cong R^n = R \oplus \dots \oplus R$ (or $R \times \dots \times R$)
\end{definition}

\begin{theorem}
  If $R$ is a PID, then any submodule of $R^n$ is free of rank $\le n$.
  \begin{proof}
    By induction on $n$, $n = 1$, $\forall I \subseteq R, \exists a \in R$ s.t.
    $I = \gen{a}_R = Ra \cong R$ ({\bf as a $R$-module}).

    Let $n > 1$ and $N$ be a submodule of $R^n$.
    Consider
    \[\arraycolsep=1pt
      \pi_1:
      \begin{array}{ccl}
        R^n & \to & R \\
        (r_1, \dots, r_n) & \mapsto & r_1
      \end{array}
      \quad \text{and} \quad
      \pi = \pi_1\Big|_N: N \to R
    \]
    \begin{description}
      \item[case 1:] $\Image \pi = \{0\}$. In this case,
        $N \subseteq \ker\pi_1 \cong R^{n-1}$.
        By induction hypothesis, $N$ is free of rank $\le n-1 < n$.
      \item[case 2:] $\Image \pi = \gen{a}$, say $\pi(x) = a$.
        Claim: $N = Rx \oplus \ker\pi,
        \ker \pi \subseteq \ker \pi_1 \cong R^{n-1}$.
        \begin{itemize}
          \item $Rx \cap \ker\pi = \{0\}$:
            $rx \in Rx \cap \ker\pi \implies \pi(rx) = 0 \implies r = 0
            \implies rx = 0$
          \item $\forall y \in N, \pi(y) = r_0a $ for some $r_0 \in R$,
            $\pi(y - r_0x) = 0 \implies y - r_0x \in \ker\pi$.
            So $N \subseteq Rx \oplus \ker\pi$. \qedhere
        \end{itemize}
    \end{description}
  \end{proof}
\end{theorem}

Recall that the elementary matrices are
\begin{itemize}
  \item $D_i(u) = \diag(1,\dots,1,u,1, \dots,1)$.
    $D_i(u) \in \text{GL}(n, R)$ if $u$ is a unit.
  \item $B_{ij}(a) = I_n + ae_{ij}, a\in R, i \ne j$.
    $B_{ij}(a)^{-1} = B_{ij}(-a) \implies B_{ij}(a) \in \text{GL}(n, R)$.
  \item $P_{ij} = I_n - e_{ii} - e_{jj} + e_{ij} + e_{ji}$.
\end{itemize}

\begin{fact}
  If $R$ is a PID and $\gen{a, b}_R = \gen{d}_R$, then $d = \gcd(a, b)$.
  \begin{proof} \mbox{}
    \begin{itemize}
      \item
        $a \in \gen{d}_R \implies a = rd$ for some $r \in R \implies d \Div a$.
        $v \in \gen{d}_R \implies d \Div b$.
      \item Let $c \Div a, c \Div b$, say $a = k_1c, b = k_2c$.
      $d \in \gen{a, b}_R \implies d = x_1a + x_2b$ for some $x_1, x_2 \in R$.
      So $d = x_1k_1c + x_2k_2c = (x_1k_1 + x_2k_2)c \implies c \Div d$.
      \qedhere
    \end{itemize}
  \end{proof}
\end{fact}

\begin{theorem}
  Let $R$ be a PID and $A \in M_{n\times m}(R)$. Then
  $\exists P \in \text{GL}_n(R)$ and $Q \in \text{GL}_m(R)$ s.t.
  \[
    PAQ = \begin{pmatrix}
    d_1 \\
    & d_2 \\
    & & \ddots \\
    & & & d_r \\
    & & & & 0 \\
    & & & & & \ddots \\
    & & & & & & 0
    \end{pmatrix}
    \quad \text{with} \quad
    d_i \Div d_{i+1} \quad \forall i = 1, \dots, r-1
  \]
  \begin{proof}
    Define the length $l(a)$ of $a \ne 0$ to be $r$ if $a = p_1p_2 \dots p_r$
    where $p_1, \dots, p_r$ are prime elements.

    prime elements: $p \Div ab \implies p \Div a \text{~or~} p \Div b$.

    \begin{enumerate}
      \item We may assume $a_{11} \ne 0$ and $l(a_{11}) \le l(a_{ij})
        \forall a_{ij} \ne 0$. (換一換就上去了...XD)
      \item We may assume $\begin{cases}
          a_{11} \Div a_{1k} & \forall k = 2, \dots, m \\
          a_{11} \Div a_{k1} & \forall k = 2, \dots, n
        \end{cases}$.
        If $a_{11} \nmid a_{1k}$, then we can interchange 2nd and $k$th
        columns to assume $a = a_{11} \nmid a_{12} = b$.

        Let $d = \gcd(a, b) \implies \begin{cases}
          l(d) < l(a) \\
          d = ax + by \text{~for some~} x, y \in R
        \end{cases} \implies 1 = \frac{a}{d}x + \frac{b}{d}y$.
        Write $b' = \frac{b}{d}, a' = -\frac{a}{d}$.
        Then
        \[
          \begin{pmatrix} -a' & b' \\ y & -x \end{pmatrix}
          \begin{pmatrix} x & b' \\ y & a' \end{pmatrix}
          = I_2
        \]
        反正就是移一下減掉, length 會一直變小 $\implies$ 這個操作會停.
      \item 有這個 $\begin{cases}
          a_{11} \Div a_{1k} & \forall k = 2, \dots, m \\
          a_{11} \Div a_{k1} & \forall k = 2, \dots, n
        \end{cases}$ 就可以全部消掉變成
        \[
          \begin{pmatrix}
            a_{11} & 0 & \dots & 0 \\
            0 & b_{22} & \dots & b_{2m}\\
            \vdots & \vdots & \ddots & \vdots \\
            0 & b_{n2} & \dots & b_{nm}
          \end{pmatrix}
        \]
      \item May assume $a_{11} \Div b_{kl} \quad \forall k, l$.
        不是的話就把該 row 往第一 row 加上去,重複前面的操作,
        $l(a_{11})$ 總是變小,因此會停.
      \item 遞迴下去...
    \end{enumerate}
    最後就弄出想要的矩陣了.
  \end{proof}
\end{theorem}

