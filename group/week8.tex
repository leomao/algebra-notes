%1 TEX root=../main.tex
\subsection{Week 8}
\subsubsection{Fundamental theorem of finitely generated abelian groups}

\begin{theorem}[Structure theorem of finitely generated module over a PID]
  Let $R$ be a PID and $M$ be a finitely generated $R$-module.
  Then $M \cong \quot{R}{d_1 R} \oplus \dots \oplus \quot{R}{d_l R} \oplus R^s,
  d_i \in R$ with $d_i \Div d_{i+1} \quad \forall i = 1, \dots, l-1$
  for some $s \in \Zb^{\ge 0}$.

  \begin{proof}
    Let $M = \gen{x_1, \dots, x_n}_R$ and consider
    \[
      \arraycolsep=1pt
      \begin{array}{rcl}
        \varphi: & R^n & \onto M \\
                 & e_i & \to x_i
      \end{array}
    \]
    By 1st isom. thm., $\quot{R^n}{\ker \varphi} \cong M$.

    We know $\ker \varphi \cong R^m$ ($e_i' \mapsto f_i, e_i' \in R^m$)
    for some $m \le n$ and
    $\forall x \in \ker\varphi \quad \exists! x_1, \dots, x_m \in R$ s.t.
    $x = \sum_{i=1}^m x_i f_i$.

    Note that $\ker \varphi \subseteq R^n$. So we can write
    $f_i = \sum_{j=1}^n a_{ji}e_j \quad \forall i = 1,\dots, m$.
    Then $x = \sum x_i \sum a_{ji}e_j =
    \sum \left(\sum a_{ji}x_i\right) e_j$.

    $R$ is a PID $\implies \exists P \in \text{GL}_n(R), Q \in \text{GL}_m(R)$
    s.t.
    \[
      PAQ = \begin{pmatrix}
      d_1 \\
      & \ddots \\
      & & d_r \\
      & & & 0 \\
      & & & & \ddots
      \end{pmatrix}
      \quad \text{with} \quad
      d_i \Div d_{i+1} \quad \forall i = 1, \dots, r-1
    \]
    So consider $[w_i] = Q e_i$. Since $P, Q$ invertible, $R^n = \bigoplus R w_i, \ker \varphi = \bigoplus d_i R w_i$
    Hence
    \[ M \simeq R / ker \varphi = \bigoplus R w_i / \bigoplus d_i R w_i = \bigoplus R / d_i R \]
  \end{proof}

  $\arraycolsep=3pt
  \begin{array}{rcl}
    R & \onto & \quot{Rw_i}{Rd_i'w_i} \\
    1 & \to & \ob{w_i} \\
    r & \to & \ob{rw_i}
  \end{array}
    $
\end{theorem}

\begin{remark}
  If $R$ is commutative, then ``$R^n \cong R^m \implies n = m$.''
\end{remark}

\begin{theorem}
  Let $G$ be a finitely generated abelian group. Then
  Then $G \cong \quot{\Zb}{d_1 \Zb} \oplus \dots \oplus \quot{\Zb}{d_l \Zb}
  \oplus R^s, d_i \in \Zb$ with $d_i \Div d_{i+1} \quad
  \forall i = 1, \dots, l-1$ for some $s \in \Zb^{\ge 0}$.

  Since $G$ can be regarded as a f.g. $\Zb$-module and $\Zb$ is a PID,
  it follows from the main theorem.

  $\Tor(G) = \quot{\Zb}{d_1 \Zb} \oplus \dots \oplus \quot{\Zb}{d_l \Zb}
  \le G$ and $\quot{G}{\Tor(G)} \cong \Zb^s$ (free part of $G$).
\end{theorem}

\begin{fact}
  If $d = p_1^{m_1}p_2^{m_2}\dots p_s^{m_s}$, then
  $\quot{\Zb}{d\Zb} \cong \quot{\Zb}{p_1^{m_1}\Zb} \oplus
  \quot{\Zb}{p_2^{m_2}\Zb} \oplus \dots \oplus \quot{\Zb}{p_s^{m_s}\Zb}$.
\end{fact}

\begin{theorem}[Chinese Remainder theorem]
  Let $R$ be a commutative ring with $1$ and $I_1, \dots, I_n$ be ideals of $R$.
  Then
  \[
    \arraycolsep=3pt
    \begin{array}{rccl}
      \varphi: & R & \to & \quot{R}{I_1} \times \dots \times \quot{R}{I_n} \\
               & r & \mapsto & (\ob{r}, \dots, \ob{r})
    \end{array}
    \text{~is a ring homo.}
  \]
  and
  \begin{enumerate}[(1)]
    \item if $I_i, I_j$ are coprime $\forall i \ne j$, then
      $I_1I_2\dots I_n = I_1 \cap I_2 \cap \dots \cap I_n$.
    \item $\varphi$ is surjective $\iff$ $I_i, I_j$ are coprime
      $\forall i \ne j$.
    \item $\varphi$ is injective $\iff$ $I_1 \cap I_2 \cap \dots \cap I_n
      = \{ 0 \}$.
  \end{enumerate}
  So if $I_i, I_j$ are coprime $\forall i \ne j$, then
  \[
    \quot{R}{I_1I_2\dots I_n} \cong
    \quot{R}{I_1} \times \dots \times \quot{R}{I_n}.
  \]

  $I_i, I_j$ are coprime $\iff$ $I_i + I_j = R$.

  \begin{proof}
    we only need to prove (1), (2).

    \begin{enumerate}[(1)]
      \item By induction on $n$. $n = 2$, need $I_1\cap I_2 \subseteq I_1I_2$.
        Indeed, $I_1\cap I_2 = (I_1\cap I_2)R = (I_1\cap I_2)(I_1+I_2)
        \subseteq I_1I_2$.

        For $n > 2$, since $I_i + I_n = R \quad \forall i = 1,\dots, n-1$,
        $\exists x_i \in I_i, y_i \in I_n$ s.t. $x_i + y_i = 1 \quad
        \forall i = 1,\dots, n-1$.

        So $x_1x_2\dots x_{n-1} = (1-y_1)(1-y_2)\dots(1-y_{n-1}) = 1 - y,
        y \in I_n
        \implies I_1I_2\dots I_{n-1} + I_n = R$.

        Now, $I_1I_2\dots I_n = (I_1\dots I_{n-1})I_n =
        (I_1\dots I_{n-1})\cap I_n = I_1\cap \dots \cap I_n$.
      \item ``$\Rightarrow$'': WLOG, we may let $I_i = I_1, I_j = I_2$.
        We have $x \in R$ s.t.
        \[
          \varphi(x) = (\ob{1}, \ob{0}, \dots, \ob{0})
          \quad \text{i.e.~}
          \ob{x} = \ob{1} \text{~in~} \quot{R}{I_1}
        \]
        Write $x \equiv 1 \pmod {I_1}$.
        Since $1 - x \in I_1, x \in I_2$ and $(1 - x) + x = 1$, $I_1 + I_2 = R$.

        ``$\Leftarrow$'': $\forall y \in \text{RHS}$,
        $y = (\ob{r_1}, \dots, \ob{r_n})$.
        If we may find that $x_i \in R$ s.t.
        $\varphi(x_i) = (\ob{0}, \dots, \ob{1}, \ob{0}, \dots, \ob{0})$,
        then
        \[
          \varphi\left(\sum_{i=1}^n r_ix_i \right) = y
        \]

        It is enough to show, for example, $\exists x \in R$ s.t.
        $\varphi(x) = (\ob{1}, \ob{0}, \dots, \ob{0})$.

        Since $I_1 + I_i = R \quad \forall i = 2, \dots, n$,
        $\exists x_i \in I_1, y_i \in I_i$ s.t. $x_i + y_i = 1
        \forall i = 2, \dots, n$.

        So let $x = y_2\dots y_n = (1-x_2)\dots(1-x_n)$.
        We have $x \in I_2, \dots, I_n$ and $x \equiv 1 \pmod {I_1}$.
    \end{enumerate}
  \end{proof}
\end{theorem}

\begin{example}
  $\abs{G} = 72$ and $G$ is abelian:
  \[
    72 = 2 \times 36 = 3 \times 24 = 2 \times 2 \times 18
    = 6 \times 12 = 2 \times 6 \times 6
  \]
  Invariant factors

  Elementary divisors
\end{example}

\begin{definition}
  The exponent of $G$ with $\abs{G} < \infty$ is
  \[
    \Exp(G) \defeq \min \left\{
      m \in \Nb \middle| g^m = 1 \quad \forall g \in G
    \right\}
  \]
\end{definition}

\begin{exercise} \mbox{}
  \begin{enumerate}
    \item Let $G$ be abelian with $\abs{G} = n$. Show that if $d \Div n$, then
      $\exists H \le G$ s.t. $\abs{H} = d$.
    \item If $n=540, d=90$, then construct all possible $G$ and
      corresponding $H$.
  \end{enumerate}
\end{exercise}

\begin{exercise}
  Let $G$ be abelian with $\abs{G} < \infty$. Show that $G$ is cyclic
  $\iff \Exp(G) = \abs{G}$.
\end{exercise}

\begin{exercise}
  Let $f_i(x) \in \Zb[x], i = 1, \dots, k$ with $\deg f_i = d$ and
  $p_1, \dots, p_k$ be distinct primes.
  Show that $\exists f(x) \in \Zb[x]$ with $\deg f = d$ s.t.
  $\ob{f}(x) = \ob{f_i}(x)$ in 
  $\quot{\Zb}{p_i\Zb}[x] \quad \forall i = 1, \dots k$.

  $f(x) = a_d x^d + \dots + a_0,
  \ob{f}(x) = \ob{a_d} x^d + \dots + \ob{a_0}$
\end{exercise}

\subsubsection{Sylow theorems}

\begin{definition}
  Let $\abs{G} = p^\alpha r$ with $p \nmid r$.
  \begin{enumerate}
    \item If $H \le G$ with $\abs{H} = p^\alpha$, then we call $H$ a Sylow
      $p$-subgroup of $G$.
    \item $\text{Syl}_p(G) =$ the set of all Sylow $p$-subgroups of $G$.
    \item $n_p = \abs{\text{Syl}_p(G)}$.
  \end{enumerate}
\end{definition}

\begin{lemma}[Key lemma]
  Let $P \in \text{Syl}_p(G)$ and $Q$ be a $p$-subgroup of $G$. Then
  $Q \cap N_G(P) = Q \cap P$.

  \begin{proof}
    By Lagrange theorem, $H = Q \cap N_G(P)$ is also a $p$-subgroup of
    $N_G(P)$ since $\abs{H} \Div \abs{Q}$.

    Since $\begin{cases}
      P \lhd N_G(P) \\
      H \le N_G(P)
    \end{cases} \implies HP \le N_G(P)$, we have
    \[
      \abs{HP} = \frac{\abs{H}\abs{P}}{\abs{H\cap P}} = p^{\alpha+k-s}
    \]
    where $\abs{H\cap P} = p^s, s \le k$. Then
    $p^{\alpha+k-s} \Div \abs*{N_G(P)} \Div \abs{G} = p^\alpha r$.

    So $k = s \implies H = H \cap P \implies H \le P \cap Q$.
  \end{proof}
\end{lemma}

\begin{theorem}[Sylow \RNum{1}]
  $\forall 0 \le k \le \alpha$, $\exists H \le G$ s.t. $\abs{H} = p^k$.
  In particular, $\text{Syl}_p(G) \ne \phi$.

  \begin{proof}
    By induction on $\abs{G}$. If $\abs{G} = 1$, then $k = 0$, $H = \{1\}$.

    Assume $\abs{G} > 1, k \ge 1, \alpha \ge 1$.

    \begin{description}
      \item[case 1:] $p \Div \abs{Z_G}$. By Cauchy theorem,
        $\exists a \in Z_G$ with $\ord(a) = p$.
        Then $\gen{a} \lhd G$ and $\abs*{\quot{G}{\gen{a}}} = p^{\alpha-1} r
        \le \abs{G}$.
        If $k=1$, then $H = \gen{a}$.
        Otherwise, we may assume that $1\le k-1\le \alpha-1$. By induction
        hypothesis, $\exists H' = \quot{G}{\gen{a}}$ s.t. $\abs{H'} = p^{k-1}$.
        By 3rd isom. thm., we can write $H' = \quot{H}{\gen{a}}$ and thus
        $\abs{H} = p^k$.
      \item[case 2:] $p \nDiv \abs{Z_G}$. By the class equation,
        $\abs{G} = \abs{Z_G} + \sum_{i=1}^m \frac{\abs{G}}{\abs{Z_G(a_i)}},
        a_i \in Z_G$.

        In this cases, $\exists a_j$ s.t.
        $p \nDiv \frac{\abs{G}}{\abs{Z_G(a_j)}} \implies
        p^\alpha \Div \abs{Z_G(a_j)}$. And $Z_G(a_j) \lneq G$ since
        $a_j \not\in Z_G$.
        By induction hypothesis, $\exists H \le Z_G(a_j) \le G$ s.t.
        $\abs{H} = p^k$. \qedhere
    \end{description}
  \end{proof}
\end{theorem}

\begin{theorem}[Sylow \RNum{2}]
  Let $P \in \text{Syl}_p(G)$ and $Q$ be a $p$-subgroup of $G$. Then
  $\exists a \in G$ s.t. $Q \le aPa^{-1}$.
  In particular, $\forall P_1, P_2 \in \text{Syl}_p(G), \exists a \in G$
  s.t. $P_2 = aP_1a^{-1}$.
  \begin{proof}
    Let $X = \{\, \text{left cosets of $P$} \,\}$ and consider
    $\arraycolsep=1pt \begin{array}{rcl}
      Q \times X & \to & X \\
      (a, xP) & \mapsto & axP
    \end{array}$.

    Observe that $xP \in \Fix Q \iff axP = xP \quad \forall a \in Q \iff
    x^{-1}axP = P \quad \forall a \in Q \iff
    x^{-1}ax \in P \quad \forall a \in Q \iff
    a \in xPx^{-1} \quad \forall a \in Q$.

    We know $\abs{\Fix Q} \equiv \abs{X} \pmod p$ and $p \nmid r \implies$
    $\abs{\Fix Q} \ne 0 \iff \exists a \in G, Q \le aPa^{-1}$.

    In particular, $\begin{cases}
      P_2 \le aP_1a^{-1} \\
      \abs{P_2} = \abs{aP_1a^{-1}}
    \end{cases} \implies P_2 = aP_1a^{-1}$.
  \end{proof}
\end{theorem}

\begin{theorem}[Sylow \RNum{3}]
  $n_p \equiv 1 \pmod p$ and $n_p \mid r$.
  \begin{proof}
    \begin{itemize}
      \item Consider $\arraycolsep=1pt \begin{array}{ccrcl}
          &P \times &\text{Syl}_p(G) & \to & \text{Syl}_p(G) \\
          (&a, &Q) & \mapsto & aQa^{-1}
        \end{array}$ where $P \in \text{Syl}_p(G)$.

        $P' \in \Fix P \iff aP'a^{-1} = P' \quad \forall a \in P
        \iff P \le N_G(P') \cap P = P' \cap P \iff P' = P$.

        So $\Fix P = \{ P \} \implies n_p \equiv \abs{\Fix P} = 1 \pmod p$.

      \item Consider $\arraycolsep=1pt \begin{array}{ccrcl}
          &G \times &\text{Syl}_p(G) & \to & \text{Syl}_p(G) \\
          (&a, &Q) & \mapsto & aQa^{-1}
        \end{array} \implies$ There is only one orbit $\text{Syl}_p(G)$.
        
        We know $\abs{\text{Syl}_p(G)} = \frac{\abs{G}}{\abs{G_Q}}$
        and $G_Q = N_G(Q)$. Then $n_p = \frac{\abs{G}}{\abs{G_Q}} \Div \abs{G}$.
        So $n_p \Div p^\alpha r \implies n_p \Div r$.
    \end{itemize}
  \end{proof}
\end{theorem}

\begin{prop}
  Let $\abs{G} = pq$ where $p, q$ are primes with $\begin{cases}
    p < q \\
    q \not\equiv 1 \pmod p.
  \end{cases}$
  Then $G \cong C_{pq}$.
  \begin{proof}
    $n_p = 1+kp \mid q \implies n_p = 1$ i.e.
    $H \in \text{Syl}_p(G) \implies H \lhd G$.

    $n_q = 1+kq \mid p \implies n_q = 1$ i.e.
    $K \in \text{Syl}_q(G) \implies K \lhd G$.

    Since $\gcd(p, q) = 1$, $H \cap K = 1$.
    Hence $G = H \times K \cong C_p \times C_q \cong C_{pq}$.
  \end{proof}
\end{prop}

\begin{example}
  Consider $\abs{G} = 255 = 3 \times 5 \times 17$.
  \begin{enumerate}
    \item 找兩個 normal subgroup (17, 5 or 3)
    \item quot 掉後發現剩下的是 abelian $\leadsto$ $[G, G]$ 在裡面
    \item $[G, G] = 1$
    \item 唱 f.g. xxx thm. 得到 $G \cong \Zb_3 \times \Zb_5 \times \Zb_{17}$.
    \item 中國剩飯定理 $G \cong C_{255}$.
  \end{enumerate}
\end{example}

\begin{exercise}
  If $\abs{G} = 7 \times 11 \times 19$, then $G$ is abelian.
\end{exercise}

\begin{example}
  No group $G$ of order $48 = 2^4 \times 3$ is simple.
  \begin{enumerate}
    \item $n_2 = 1 + 2k \Div 3 \leadsto n_2 = 1 \text{~or~} 3$.
    \item $n_2 = 1$ then OK.
    \item Assume $n_2 = 3$. Let $P \in \text{Syl}_2(G),
      X = \{\, \text{left cosets of $P$} \,\}$ ($\abs{X} = 3$).
    \item Consider $\arraycolsep=1pt \begin{array}{ccrcl}
        &G \times &X & \to & X \\
        (&a, &xP) & \mapsto & axP
      \end{array} \leadsto \varphi: G \to S_3$.
    \item 考慮 $\ker\varphi$.
  \end{enumerate}
\end{example}

\begin{exercise}
  No group $G$ of order $36$ is simple.
\end{exercise}

\begin{exercise}
  No group $G$ of order $30$ is simple.
\end{exercise}

\begin{exercise}
  Let $\abs{G} = 385$. Show that $\exists P \in \text{Syl}_7(G)$ s.t.
  $P \le Z_G$.
\end{exercise}
