%! TEX root=../main.tex
\subsection{Week 3}
\subsubsection{Coset and Quotient Group}
Let $f: G_1 \to G_2$ be a group homo. Define $\Image f \defeq f(G_1)$.

Notice that $\Image f \le G_2$.
\begin{proof}
  Let $z_1 = f(a_1), z_2 = f(a_2)$, then
  $z_1z_2^{-1} = f(a_1)f(a_2)^{-1} = f(a_1)f(a_2^{-1}) = f(a_1a_2^{-1}) \in
  \Image f$.
\end{proof}

\begin{definition}
  $\ker f \defeq \{\, x \in G_1 \mid f(x) = 1 \,\} \le G_1$.
\end{definition}

\begin{fact} \mbox{}
  \begin{enumerate}
    \item $x \in (\ker f) a \iff f(x) = f(a)$.
    \item $\ker f = \{ 1 \} \iff f$ is 1-1.
  \end{enumerate}
\end{fact}

\begin{definition}
  Let $H \le G$, $\forall a \in G, Ha$ is called a {\bf right coset} of $H$
  in $G$.
\end{definition}

\begin{fact} \mbox{}
  \begin{enumerate}
    \item For 2 right cosets $Ha, Hb$, either $Ha = Hb$ or $Ha \cap Hb = \varnothing$
      must hold.
    \item $\Set{ Ha : a\in G }$ forms a partition of $G$.
  \end{enumerate}
\end{fact}

\begin{theorem}[Lagrange]
  Let $\abs{G} < \infty$ and $H \le G$, $\abs{H} \mid \abs{G}$.
  \begin{proof}
    Let $r = \abs{\Set{ Ha : a \in G }}$. Since either $Ha = Hb$ or $Ha \cap Hb = \varnothing$
    for any $a, b\in G$, we only need to show that $\abs{Ha} = \abs{Hb}$ for any
    $a, b \in G$.
    Now define $f: Ha \to Hb$ as $f: ha \mapsto hb$ for all $h \in H$.
    $f$ is clearly onto, and $f$ is 1-1 since $h_1b = h_2b \implies h_1 = h_2
    \implies h_1a = h_2a$.
    So $f$ is bijective $\implies \abs{Ha} = \abs{Hb}$.
    We can conclude that $r \cdot \abs{H} = \abs{G}$.
  \end{proof}
\end{theorem}

\begin{remark}
  $r$ is called the {\bf index} of $H$ in $G$, denoted by $[G:H]$.
  (The concept of index can be extended to infinite $G, H$.)
\end{remark}

\begin{exercise}
  no subgroup of $A_4$ has order $6$.
  (converse of Lagrange thm. is false.)
\end{exercise}

\begin{coro}
  If $\abs{G} = p$ is a prime in $\Zb$, then $G$ is cyclic.
  \begin{proof}
    Since $\abs{G} = p > 1$, we can pick $a \in G$ s.t. $a \ne 1$.
    Consider $H = \gen{a} \le G$, then $\abs{H} \mid \abs{G}$ by Lagrange's Theorem.
    Since $\abs{G} = p$ is a prime, we have $\abs{H} = 1$ or $p$.
    But $a \ne 1$ and $1, a \in H$ by the definition, we have $\abs{H} \ge 2$.
    So $\abs{H} = p \implies H = G \implies G = \gen{a}$.
  \end{proof}
\end{coro}

\begin{coro}
  If $\abs{G} < \infty, a \in G$, then $a^{\abs{G}} = 1$.
  \begin{proof}
    Let $H = \gen{a}$, we have $\abs{H} \mid \abs{G}$. Write $\abs{G} = r \abs{H}$.
    Now $a^{\abs{H}} = 1$ by the definition of $H$. So
    $a^{\abs{G}} = (a^{\abs{H}})^r = 1$.
  \end{proof}
\end{coro}

\begin{remark} \mbox{}
  \begin{enumerate}
    \item Let $H \le G, a \in G$, $aH$ is called a {\bf left coset}.
    \item $\{ \text{right cosets of $H$} \} \leftrightarrow
      \{ \text{right cosets of $H$} \}$ by $Ha \mapsto a^{-1}H$.
  \end{enumerate}
\end{remark}

\underline{Ques}: How to make $\{\, aH : a \in G \,\}$ to be a group?
For $aH, bH$, we must have $(aH)(bH) = abH$.

In general, $(aH)(bH) = abH$ is not well-defined.

\begin{example}
  Let $H = \gen{\cycle{1,2}} \le S_3$. $a_1 = \cycle{1,3}, a_2 = \cycle{1,2,3},
b_1 = \cycle{1,3,2}, b_2 = \cycle{2,3}$. 出慘點
\end{example}

If we hope $a_1b_1H = a_2b_2H$, then we need $(a_1b_1)^{-1}a_2b_2 \in H$.
\[
  b_1^{-1}a_1^{-1}a_2b_2 = b_1^{-1}b_2b_2^{-1}a_1^{-1}a_2b_2
\]
Notice that $b_1^{-1}b_2, a_1^{-1}a_2 \in H$, so we need
$b_2^{-1}a_1^{-1}a_2b_2 \in H$.

\begin{definition}
  Let $H \le G$. $H$ is said to be {\bf normal subgroup} of $G$ if
  $\forall g \in G, h \in H, g^{-1}hg \in H \quad
  (\text{or~} g^{-1}Hg \subseteq H)$, denoted by $H \lhd G$.
\end{definition}

\begin{definition}
  Let $H \lhd G$. The set $\{\, aH \mid a \in G \,\}$ forms a group under
  $(aH)(bH) = abH, a,b \in G$. We call it the {\bf quotient group}
  of $G$ by $H$, denoted by $\quot{G}{H}$.

  (Note: The indentity is $H = hH$ and $(aH)^{-1} = a^{-1}H$.)
\end{definition}

\begin{remark}
  Define $q: G \to \quot{G}{H}, a \mapsto aH$, called the quotient homomorphism.
\end{remark}

\begin{exercise}
  Let $H \le G$. Then TFAE
  \begin{enumerate}[(a)]
    \item $H \lhd G$.
    \item $\forall x \in G, xHx^{-1} = H$.
    \item $\forall x \in G, xH = Hx$. \label{eq:xh=hx}
    \item $\forall x, y \in G, (xH)(yH) = (xy)H$.
  \end{enumerate}
\end{exercise}

\underline{Ques}: How to find a normal subgroup of $G$?

\begin{prop} \mbox{}
  \begin{enumerate}
    \item If $G$ is abelian, then $\forall H \le G \leadsto H \lhd G$.
      (done by \ref{eq:xh=hx})
    \item If $H \le G$ with $[G:H] = 2$, then $H \lhd G$.
      \begin{example}
        $n \le 3, [S_n:A_n] = 2 \implies A_n \lhd S_n$.
      \end{example}
      \begin{proof}
        We can write $G = H \cup Ha = H \cup aH \implies aH = Ha,
        \forall a \notin H$.
      \end{proof}
  \end{enumerate}
\end{prop}

\begin{definition}
  Define the center of $G$ to be $Z_G = \{\, a \in G
  \mid ax = xa, \forall x \in G \,\} \le G$.
\end{definition}

\begin{prop} \mbox{}
  \begin{enumerate}
    \item $Z_G \lhd G$. (by \ref{eq:xh=hx} and def.)
    \item If $\quot{G}{Z_G}$ is cyclic, then $G$ is abelian.
      \begin{proof}
        Let $\quot{G}{Z_G} = \gen{aZ_G}$, (let $\ob{a} \defeq aZ_G$) for some
        $a \in G$.
        For $x_1, x_2 \in G$, let $x_1 = a^{k_1}z_1, x_2 = a^{k_2}z_2$, then
        $x_1x_2 = a^{k_1+k_2}z_1z_2 = x_2x_1$. ($z_i$ 可以各種交換)
      \end{proof}
  \end{enumerate}
\end{prop}

\begin{definition}
  The commutator of $G$ is define to be $[G,G] = \gen{xyx^{-1}y^{-1} \mid
  x,y \in G}$.
\end{definition}

\begin{prop}
  $[G,G] \lhd G$ ; $[G,G] = 1 \iff G$ is abelian.
  \begin{proof}
    $\forall x \in G, a \in [G,G], xax^{-1} = xax^{-1}a^{-1}a$ and
    $xax^{-1}a^{-1}, a \in [G,G]$.
  \end{proof}
\end{prop}

\begin{exercise} \mbox{}
  \begin{enumerate}
    \item If $H \le S_n$ and $\exists \sigma \in H$ is odd, then $[H:H\cap A_n] = 2$.
    \item For $n \ge 3$, $[S_n, S_n] = A_n$.
  \end{enumerate}
\end{exercise}

\begin{exercise}
  Let $H \le G$. Then  $H \lhd G$ and $\quot{G}{H}$ is abelian $\iff$
  $[G,G] \le H$.
  (hint: $\quot{G}{[G,G]}$ is "max" among all abelian quotient groups)
\end{exercise}


\subsubsection{Isomorphism theorems \& Factor theorem}
\begin{theorem}[1st isomorphism theorem]
  Let $f: G_1 \to G_2$ be a group homo. Then $\quot{G_1}{\ker f} \cong \Image f$.
  \begin{proof}
    Define $\varphi: a \ker f \mapsto f(a)$.
    \begin{itemize}
      \item well-defined: $a \ker f = b \ker f \implies a^{-1}b \in \ker f
        \implies f(a^{-1}b) = 1 \implies f(a)^{-1}f(b) = 1 \implies f(a)=f(b)$.
      \item group homo: $\varphi\left((a \ker f)(b\ker f)\right) = 
        \varphi(ab \ker f) = f(ab) = f(a)f(b) =
        \varphi(a\ker f)\varphi(b\ker f)$.
      \item onto: by def. of $\Image f$.
      \item 1-1: $f(a) = f(b) \implies a \ker f = b \ker f$ (easy).
      \end{itemize}
  \end{proof}
\end{theorem}

\begin{theorem}[Factor theorem]
  Let $f: G_1 \to G_2$ be a group homo. and $H \lhd G_1, H \le \ker f$. Then
  $\exists$ a group homo. $\varphi: \quot{G}{H} \to G_2$ s.t.
  \[
    \begin{tikzcd}
      G_1 \arrow{r}{q} \arrow[swap]{dr}{f} & \quot{G}{H} \arrow{d}{\varphi} \\
      & G_2
    \end{tikzcd}
  \]
\end{theorem}

\begin{example}
  Let $G = \gen{a}$ with $\ord(a) = n$. Then $G \cong \quot{\Zb}{n\Zb}$.
  (1st isom. thm.)
\end{example}

\begin{example}
  $\varphi: \Zb \to \quot{\Zb}{2\Zb}, 4\Zb \le 2\Zb$, so by factor thm.,
  $\quot{\Zb}{4\Zb} \to \quot{\Zb}{2\Zb}$.
\end{example}

\begin{example}
  $\det: \text{GL}(n, \Fb) \to \Fbx \implies
  \quot{\text{GL}(n, \Fb)}{\text{SL}(n, \Fb)} \cong \Fbx$
\end{example}

\begin{example}
  $\sgn: S_n \to \{ \pm 1 \} \implies \quot{S_n}{A_n} \cong \{ \pm 1 \}$
\end{example}

\begin{theorem}[2nd isomorphism theorem]
  Let $H \le G, K \lhd G$. Then $\quot{HK}{K} \cong \quot{H}{H\cap K}$.
  \begin{proof}
    First, $\begin{cases}H\le G \\ K \lhd G\end{cases} \implies HK = KH
      \implies HK \le G$ ; $K \lhd G \implies K \lhd HK$.

    Define $\varphi: H \to \quot{HK}{K}, h \mapsto hK$. which is a group homo.
    \begin{itemize}
      \item onto: $\forall (hk) K, hkK = hK$, so $\varphi(h) = hK = hkK$.
      \item Find $\ker \varphi$: $a \in \ker \varphi \iff \begin{cases}
          a \in H \\
          aK = K
        \end{cases} \iff a \in H \cap K$, so $\ker \varphi = H\cap K$.
    \end{itemize}
    Then by 1st isom. thm.
  \end{proof}
\end{theorem}

\begin{example}
  $G = \text{GL}(2, \Cb), H = \text{SL}(2, \Cb), K = \Cbx I_2 = Z_G \lhd G$.

  By 2nd isom. thm., $\quot{G}{K} \cong \quot{H}{\{\pm I_2\}}$.
  ($G = HK, \{\pm I_2 \} = H \cap K$)

  projective linear group: $\text{PGL}(2, \Cb) = \quot{G}{K}$.

  projective special linear group: $\text{PSL}(2, \Cb) = \quot{H}{H\cap K}$.
\end{example}

齊次座標...OTL

\begin{exercise} \mbox{}
  \begin{enumerate}
    \item Let $H_1 \lhd G_1, H_2 \lhd G_2$. Then $(H_1 \times H_2) \lhd
      (G_1 \times G_2)$ and $\quot{G_1\times G_2}{H_1\times H_2} \cong
      \quot{G_1}{H_1} \times \quot{G_2}{H_2}$.
    \item Let $H \lhd G, K \lhd G$ s.t. $G = HK$. Then
      $\quot{G}{H\cap K} \cong \quot{G}{H} \times \quot{G}{K}$.
  \end{enumerate}
\end{exercise}

\begin{exercise}
  Let $H \lhd G$ with $[G:H] = p$, which is a prime in $\Zb$. Then
  $\forall K \le G$, either \begin{enumerate*}[(1)]
    \item $K \le H$ or
    \item $G = HK$ and $[K:K\cap H] = p$.
  \end{enumerate*}
\end{exercise}

\begin{theorem}[3rd isomorphism theorem]
  Let $K \lhd G$.
  \begin{enumerate}
    \item There is a 1-1 correspondence between $\{\, H \le G \mid K \le H \,\}$
      and $\{ \text{subgroups of $\quot{G}{K}$} \}$. ($H \lhd G$ ... normal)
      \begin{proof}
        Define $\varphi: H \mapsto \quot{H}{K}$. ($\quot{H}{K} \le \quot{G}{K}$)
        \begin{itemize}
          \item 1-1: Assume $\quot{H_1}{K} = \quot{H_2}{K}$.
            For $a \in H_1$, $aK \in \quot{H_1}{K} = \quot{H_2}{K}$.
            so $\exists b \in H_2$ s.t. $aK = bK \implies b^{-1}a \in K \le H_2
            \implies a \in b H_2 = H_2$. So $H_1 \le H_2$. By symmetry,
            $H_2 \le H_1$, and thus $H_1 = H_2$.
          \item onto: Given a subgroup $Q$ of $\quot{G}{K}$, consider
            $H = q^{-1}(Q)$ where $q: G\to \quot{G}{K}$.
            \begin{itemize}
              \item $H \le G$: $\forall a, b \in H, q(a), q(b) \in Q \implies
                q(a)q(b)^{-1} \in Q \implies q(ab^{-1}) \in Q \implies
                ab^{-1} \in H \implies H \le G$.
              \item $K \le H$: $\forall a \in K, q(a) = aK = K \in Q \implies
                a \in H \implies K \le H$.
              \item $Q = \quot{H}{K}$: $\forall aK \in Q, aK = q(a) \implies
                a \in H \implies aK \in \quot{H}{K} \implies
                Q \subseteq \quot{H}{K}$.
                And $\forall aK \in \quot{H}{K} (a \in H), q(a) \in Q \implies
                  \quot{H}{K} \subseteq Q$. So $Q = \quot{H}{K}$.
            \end{itemize}
          \item $H \lhd G, K \le H \iff \forall g\in G, gHg^{-1} = H, K \le H
            \iff \forall \ob{g} \in \quot{G}{K}, \ob{g}(\quot{H}{K})\ob{g}^{-1}
            = \quot{H}{K} \iff \quot{H}{K} \lhd \quot{G}{K}$. \qedhere
        \end{itemize}
      \end{proof}
    \item If $H \lhd G$ with $K \le H$, then $\quot{(\quot{G}{K})}{(\quot{H}{K})}
      \cong \quot{G}{H}$.
      \begin{proof}
        Define $\varphi: G \to\quot{(\quot{G}{K})}{(\quot{H}{K})}$ with
        $\varphi: a \mapsto aK(\quot{H}{K})$.
        \begin{itemize}
          \item onto: ... easy.
          \item Find $\ker \varphi$: $a \in \ker \varphi \iff aK(\quot{H}{K})
            = \quot{H}{K} \iff aK \in \quot{H}{K} \iff a \in H$.
        \end{itemize}
        By 1st isom. thm., $\quot{(\quot{G}{K})}{(\quot{H}{K})} \cong
        \quot{G}{H}$.
      \end{proof}
  \end{enumerate}
\end{theorem}

\begin{example}
  $\quot{m\Zb + n\Zb}{m\Zb} \cong \quot{n\Zb}{m\Zb \cap n\Zb}$.
  ($m\Zb + n\Zb = \gcd(m,n)\Zb, m\Zb \cap n\Zb = \lcm(m,n)\Zb$)
\end{example}

\underline{Ques}: $\quot{G}{K} \cong \quot{G'}{K'}$ and $K \cong K'
\centernot\implies G \cong G'$.

\begin{example}
  $Q_8$ and $D_4$
  交給陳力
\end{example}

Extension problem: given two groups $A, B$, how to find $G$ and $K \lhd G$,
s.t. $K \cong A, \quot{G}{K} \cong B$?
($1 \to H \to G \to \quot{G}{H} \to 1$, short exact sequence)

 (e.g. $G = A \times B, K = A \times \{1\}$)
