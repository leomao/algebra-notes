%! TEX root=../main.tex
\subsection{Week 9}
\subsubsection{Classification}

To classify groups of small orders: 
\begin{itemize}
  \item $\abs{G} = 1$: $G = \{1\}$
  \item $\abs{G} = 2$: $G \cong C_2$
  \item $\abs{G} = 3$: $G \cong C_3$
  \item $\abs{G} = 4$: $G \cong \Zb_4 \text{~or~} \Zb_2 \times \Zb_2$
  \item $\abs{G} = 5$: $G \cong C_5$
  \item $\abs{G} = 6$: $n_3 = 1, n_2 = 1 \text{~or~} 3$. Let $H \in \Syl_3(G)$ and $H \lhd G$. Let $K \in \Syl_2(G)$. Also $H \cap K = \{1\}$ and $HK = G$ then $G \cong K \times_{\tau} H$
    \begin{itemize}
      \item If $\tau$ is trivial: $G \cong K \times H \cong C_2 \times C_3 \cong C_6$
      \item $\tau: b \mapsto \phi_2: \gen{a} \to \gen{a}$: $G \cong K \times_{\tau} H \cong \gen{a,b \mid a^3 = 1, b^2 = 1, bab^{-1} = a^2 = a^{-1}} \cong D_3$
    \end{itemize}
  \item $\abs{G} = 7$: $G \cong C_7$
  \item $\abs{G} = 8$:
    \begin{itemize}
      \item If abelian: $\Zb_8$ or $\Zb_4 \times \Zb_2$ or $\Zb_2 \times \Zb_2 \times \Zb_2$
      \item If non-abelian:
        \begin{itemize}
          \item $\not\exists a \in G$ with $\ord(a) = 8$
          \item Not each $a \in G$ with $a^2 = 1$, otherwise $G$ is abelian.
          \item $\exists a \in G$ with $\ord(a) = 4$: Let $H = \gen{a}$ and $H \lhd G$ since $[G:H] = 2$. Pick $b \in G \setminus H$ and $K = \gen{b}$
            \begin{itemize}
              \item $\ord(b) = 2$: $H \cap K = \{1\}$ and $HK = G$ then $G \cong K \times_{\tau} H$, $\tau: b \mapsto \phi: a \mapsto a^3$: $G \cong K \times_{\tau} H \cong \gen{a,b \mid a^4 = 1, b^2 = 1, bab^{-1} = a^3 = a^{-1}} \cong D_4$
              \item $\ord(b) = 4$: $H \cap K = \gen{a^2 = b^2}$. Then consider $bab^{-1} \in H \implies bab^{-1} = 1,a,a^2,a^3$
                \begin{enumerate}
                  \item $1,a$ obviously wrong.
                  \item $bab^{-1} = a^2$: $a = a^2aa^{-2} = b^2ab^{-2} = a^4 \implies a^3 = 1$ 矛盾
                  \item So $bab^{-1} = a^3 = a^{-1}$.
                \end{enumerate}
                $G \cong \gen{a,b \mid a^4 = 1, b^4 = 1, a^2 = b^2, bab^{-1} = a^3 = a^{-1}} \cong Q_8$
            \end{itemize}
        \end{itemize}
    \end{itemize}
  \item $\abs{G} = 9$: $G \cong \Zb_9 \text{~or~} \Zb_3 \times \Zb_3$
  \item $\abs{G} = 10$: $G \cong K \times H \cong C_2 \times C_5 \cong C_{10}$ or $G \cong D_5$
  \item $\abs{G} = 11$: $G \cong C_{11}$
  \item $\abs{G} = 12$: Claim: If $\abs{G} = 12$, then either $G$ has a normal Sylow $3$-subgroup or $G \cong A_4$.
    \begin{proof}
      By Sylow 3, $n_3 = 1+3k \mid 4 \implies n_3 = 1 \text{~or~} 4$.
      \begin{itemize}
        \item If $n_3 = 1$, then $G$ has a normal Sylow $3$-subgroup.
        \item Otherwise, let $P \in \Syl_3(G)$ and $X = \left\{ \text{left cosets of }P \right\}, \abs{X} = 4$.
          Consider $G \times X \to X$ defined by $(a,xP) \mapsto axP$ with $\phi: G \to S_4$.
          And $\ker \phi \le P$, $\abs{P} = 3$ and $P \centernot{\lhd} G$ (since $n_3=4$), so $\ker \phi = \{1\}$.

          And since $n_3 = 4$, there are $8$ elements of order $3$ which corresponds 
          to $8$ $3$-sycles in $A_4$, thus $\abs{\Image \phi \cap A_4} \ge 8$.
          But $\abs{\Image \phi \cap A_4} \Div \abs{A_4} = 12 \implies \Image \phi = A_4$
      \end{itemize}
    \end{proof}
    Now, for the case where $\exists H \in \Syl_3(G)$ and $H \lhd G$.
    Let $K \in \Syl_2(G)$, then $K \cap H = \{1\}$ and $KH = G \implies G \cong K \times_\tau H$
    for some $\tau: K \to \Aut(H) = \{\text{id}, \phi_2\}$
    \begin{itemize}
      \item $\tau$ is trivial: $\Zb_{12}$ or $\Zb_2 \times \Zb_6$.
      \item $\gen{b} = K \cong \Zb_4$: $\tau(b) = \phi_2 \implies G = \gen{a,b \mid a^3=1, b^4=1, bab^{-1} = a^{-1}} \not\cong D_6,A_4$
      \item $\gen{b} = K \cong \Zb_2 \times \Zb_2$: Let $K = \gen{b,c\mid b^2=1, c^2=1, bc=cb}$,
        then $\tau: b \mapsto \phi_2$ and $c \mapsto \text{id}$ (the other cases are equivalent to this one),
        $G = \gen{a,b,c \mid a^3 = 1, b^2 = 1, c^2 = 1, bc=cb,bab^{-1} = a^{-1}, cac^{-1} = a} \cong \gen{a,b\mid a^3=1,b^2=1, bab^{-1} = a^{-1}} \times \gen{c} \cong D_3 \times C_2 \cong D_6$
        \begin{fact}
          For odd $n$, $D_{2n} \cong D_n \times \quot{\Zb}{2\Zb}$.
          \begin{proof}
            \[D_{2n} = \gen{a,b\mid a^{2n}=1,b^2=1, bab^{-1} = a^{-1}}\]
            \[H = \gen{a^2,b\mid (a^2)^{n}=1,b^2=1, b(a^2)b^{-1} = a^{-2}} \cong D_n\]
            \[K = \gen{a^n} \cong C_2\]
            And $n$ is odd, so $H \cap K = \{1\}$ and $D_{2n} \cong D_n \times C_2$
          \end{proof}
        \end{fact}
    \end{itemize}
  \item $\abs{G} = 13$: $G \cong C_{13}$
  \item $\abs{G} = 14$: $G \cong C_{14} \text{~or~} D_7$
  \item $\abs{G} = 15$: $G \cong C_{15}$
\end{itemize}

\begin{exercise} \mbox{}
  Assume that $K$ is cyclic and $H$ is an arbitrary group.
  Let $\tau_1: K \to \Aut(H)$, $\tau_2: K \to \Aut(H)$ with $\tau_1(K) \sim \tau_2(K)$ (conjugate).
  If $\abs{K} = \infty$, then assume that $\tau_1$ and $\tau_2$ are injective. Show that $K \times_{\tau_1} H \cong K \times_{\tau_2} H$.
\end{exercise}

\begin{exercise}
  Classify $G$ if $\abs{G} = p^3$ with $p$ an odd prime and each nontrivial element of $G$ has order $p$.
\end{exercise}

\begin{exercise}
  Classify groups of order $30$.
\end{exercise}

\subsubsection{Free groups}
A free group generate by a non-empty set $X$ is that
there are no relations satisfied by any of elements in $X$.

\begin{definition}
  A free group on $X$ is a group $F$ with an inclusion map $i: X \to F$ satisfying the
  following universal property: For any group $G$ and any map $f: X \to G$,
  exists a unique group homo $\varphi : F \to G$ that the following diagram commutes.
  \[
    \begin{tikzcd}
    X \arrow[r]{i} \arrow[swap, dr]{f} & F \arrow{d}{\varphi} \\
    & G
    \end{tikzcd}
  \]
\end{definition}

\begin{theorem}
  $F$ exists and is unique up to isomorphism. (Denote it as $F(X) = F$).
\end{theorem}

\begin{proof}
  For $X$, we create a new disjoint set $X^{-1} = \{ x^{-1} : x \in X\}$
  and an element $1 \notin X \cup X^{-1}$.

  Define $F(X) = \{ 1 \} \cup \left\{ x_1^{\delta_1} x_2^{\delta_2} \cdots x_m^{\delta_m} :
   m \in \mathbb{N} , x_i \in X, \delta_i = \pm 1, x_{i+1}^{\delta_{i+1}} \neq
   \left( x_i^{\delta_i} \right)^{-1}\right\}$, and
  \[ x_1^{\delta_1} x_2^{\delta_2} \cdots x_m^{\delta_m} =
    y_1^{\epsilon_1} y_2^{\epsilon_2} \cdots y_m^{\epsilon_m}  \iff n = m \text{ and }
    \delta_i = \epsilon_i \text{ and } x_i = y_i , \forall i \]

  For each $y \in X \cup X^{-1}$, we define $\sigma_y : F(X) \to F(X)$ by
  \[
    \sigma_y (x_1^{\delta_1} x_2^{\delta_2} \cdots x_m^{\delta_m})
    =
    \begin{cases}
      y x_1^{\delta_1} x_2^{\delta_2} \cdots x_m^{\delta_m} & \text{if } x_1^{\delta_1} \neq y^{-1} \\
      \begin{cases}
        x_1^{\delta_1} x_2^{\delta_2} \cdots x_m^{\delta_m} & (m \geq 2) \\
        1 & (m = 1)
      \end{cases} & \text{if } x_1^{\delta_1} = y^{-1}
    \end{cases}
  \]

  Then $\sigma_y$ is a permutation of $F(X)$, since if
  $ \sigma_y(x_1^{\delta_1} x_2^{\delta_2} \cdots x_m^{\delta_m}) =
  \sigma_y(y_1^{\epsilon_1} y_2^{\epsilon_2} \cdots y_m^{\epsilon_m}) $.
  \begin{itemize}
    \item[m = n:] either $x_1^{\delta_1} = y_1^{\epsilon_1} = y^{-1}$ or not,
      then either $x_2^{\delta_1} x_3^{\delta_2} \cdots x_m^{\delta_m} =
  y_2^{\epsilon_1} y_3^{\epsilon_2} \cdots y_m^{\epsilon_m}$ or
  $y x_1^{\delta_1} x_2^{\delta_2} \cdots x_m^{\delta_m} =
  y y_1^{\epsilon_1} y_2^{\epsilon_2} \cdots y_m^{\epsilon_m}$. Both of them leads to
  $x_1^{\delta_1} x_2^{\delta_2} \cdots x_m^{\delta_m} =
  y_1^{\epsilon_1} y_2^{\epsilon_2} \cdots y_m^{\epsilon_m}$.
  \item[m = n+2:] Omimi
  \end{itemize}
  Also $\sigma_y$ is onto since omimi. And notice that $\sigma_{y^{-1}} \circ \sigma_y = id_{F(X)}$

  Define $A = \Gen{ \sigma_x : x \in X } \leq S_{F(X)}$. and define $\phi: F(X) \to A$ by
  $\phi(1) = id_{F(X)}$ and \\ $x_1^{\delta_1} \cdots x_m^{\delta_m} \mapsto \sigma_{x_1}^{\delta_1}
  \cdots \sigma_{x_m}^{\delta_m}$. The it is omimi that $\phi$ is a bijection. So we define
  $x::X \cdot y::X = \phi^{-1}( \phi(x) \circ \phi(y) )$.

  The $\phi$  in the universal property could be defined as
  $\phi(x_1^{\delta_1} x_2^{\delta_2} \cdots x_m^{\delta_m}) = f(x_1)^{\delta_1} \cdots f(x_m)^{\delta_m}$.
\end{proof}

\begin{prop}
  Let $G = \gen{ a_1, \dots, a_n }$ and $X = \{ x_1, \dots, x_m \}$. Then
  $G \cong \quot{F(X)}{K}$ for some normal subgroup $K$. $K$ is called the subgroup of relations
  connecting the generators.

  Define $f = x_i :: X_i \to a_i :: G$. By universal property, $\exists \phi
  = x_i :: F(X) \mapsto a_i :: G$. Then
  $\quot{F(x)}{\ker \phi} \cong G$.
\end{prop}

\begin{definition}
  Let $X = \{x_1, x_2, \cdots, x_n\}$ and $R \subset F(X)$.
  Let $N(R)$ be the smallest normal subgroup of $F(X)$ containing $R$,
  Then $G = \quot{F(X)}{N(R)}$ is written as $\Gen{ x_1, \dots, x_n \given \text{elements of } R }$,
  which is called a presentation of $G$. If $\abs{R} < \infty$, then $G$ is said to be finitely
  presented.
\end{definition}

\begin{example}
  \[ D_n = \gen*{
    \begin{bmatrix}
      \cos \frac{2\pi}{n} & -\sin \frac{2\pi}{n} \\
      \sin \frac{2\pi}{n} &  \cos \frac{2\pi}{n}
    \end{bmatrix},
    \begin{bmatrix}
      1 & 0 \\
      0 & -1
    \end{bmatrix}
    }
  \]

  We find that $x^n , y^2 , xyxy \in \ker \phi$. Then $R = \{ x^n , y^2 , xyxy \} \subseteq \ker \phi
  \implies N(R) \leq \ker \phi$.  By factor theorem, $\exists \bar{\phi} :: \quot{F(X)}{N(R)} \to D_n$.
  But notice that
  \[ \abs*{\quot{F(x)}{N(R)}}  \leq 2n \]
  since $xyxy = 1 \implies xy = yx^{-1}$, so every element could be turn into
  $x^i y^j$. Hence $\bar{\phi}$ is an isomorphism.
\end{example}

\begin{prop}
  Let $X = \{x_1, x_2, \cdots, x_n\}$. Then $\quot{F(X)}{[F(X), F(X)]} \cong \mathbb{Z}^n$.
\end{prop}

\begin{proof}
  Define $f = x_i :: X \mapsto e_i :: \mathbb{Z}^n$. Then $\phi = x_i :: F(X) \mapsto e_i :: \mathbb{Z}^n$.
  By 1st isomorphism theorem $\quot{F(X)}{\ker \phi} \cong \mathbb{Z}^n$ which is abelian,
  so $[F(X), F(X)] \leq \ker \phi$.
  By factor theorem, 一個ㄛ圖.

  Claim that $\bar{\phi}$ is 1-1.
  \begin{proof}
    Since $\quot{F(X)}{[F(X), F(X)]}$ is abelian, $\forall a \in \quot{F(X)}{[F(X), F(X)]}$, we can write
    $a = \bar{x}_1^{n_1} \bar{x}_2^{n_2} \cdots \bar{x}_m^{n_m}$.
    If $\bar{\phi}(\bar{a}) = (m_1, \cdots, m_n) = 0$ in $\mathbb{Z}^n$, then $m_i = 0,\, \forall i
    \implies a = 1$
  \end{proof}
\end{proof}

