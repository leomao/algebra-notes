%1 TEX root=../main.tex
\subsection{Week 6}
\subsubsection{Group actions \RNum{2}}

\begin{definition}
  Let $G \acts X$ and $\abs{X} < \infty$.
  Write $\Fix G \defeq \{\, x \in X \mid gx = x \quad \forall g \in G \,\}$.
\end{definition}

\begin{itemize}
  \item $x \in \Fix G$, $Gx = \{ x \}$.
  \item $x \not\in \Fix G$, $\abs{Gx} = [G : G_x]$.
\end{itemize}

Let $\{ G_{x_1}, \dots, G_{x_n} \}$ be the set of distinct orbits.
After rearrangement, assume $x_1, \dots, x_r \in \Fix G,
x_{r+1}, \dots, x_n \not\in \Fix G$. Then
\[
  \abs{X} = \abs{\Fix G} + \sum_{i=r+1}^{n} [G : G_{x_i}]
\]

\begin{theorem}[class equation]
  Let $\abs{G} < \infty$. Then either $G = Z_G$ or
  $\exists a_1, \dots, a_m \in G \setminus Z_G$ s.t.
  \[
    \abs{G} = \abs{Z_G} + \sum_{i=1}^{n} [G : G_{a_i}]
  \]
  \begin{proof}
    Consider the action $(g, x) \mapsto gxg^{-1}$, then
    \[
      \Fix G = \{\, x \in G \mid gxg^{-1} = x \quad \forall g \in G \,\}
      = Z_G
    \]
    It follows from the above argument.
  \end{proof}
\end{theorem}

\begin{definition}
  $G$ is called a $p$-group if $\abs{G} = p^n$, where $p$ is a prime,
  $n \in \Nb$.
\end{definition}

\begin{prop}
  If $G$ is a $p$-group, then $Z_G \ne \{ 1 \}$.
  \begin{proof}
    Let $\abs{G} = p^n$. If $G = Z_G$, then done.
    Otherwise, by the class equation (use action by conjugation),
    $\abs{G} = \abs{Z_G} + \sum_{i=1}^{n} [G : G_{a_i}], \quad a_i \not\in Z_G$.

    $G_{a_i} = Z_G(a_i)$, so $a_i \not\in Z_G \implies Z_G(a_i) \lneq G
    \implies p \mid [G:Z_G(a_i)] = \frac{\abs{G}}{\abs{Z_G(a_i)}}$.

    So $\abs{Z_G} = \abs{G} - \sum_{i=1}^{n} [G : Z_G(a_i)]
    \implies p \mid \abs{Z_G} \implies Z_G \ne \{ 1 \}$.
  \end{proof}
  \label{prop:pgroup}
\end{prop}

\begin{prop}
  If $\abs{G} = p^2$, then $G$ is abelian.
  ($\quot{\Zb}{p\Zb} \times \quot{\Zb}{p\Zb}$ and $\quot{\Zb}{p^2\Zb}$)
  \begin{proof}
    Assume that $G$ is not abelian.
    By prop \ref{prop:pgroup}, $\abs{Z_G} = p \implies \abs{\quot{G}{Z_G}} = p
    \implies \quot{G}{Z_G}$ is cyclic $\implies G$ is abelian. (contradiction)
  \end{proof}
\end{prop}

\begin{prop}
  If $\abs{G} = p^3$ and $G$ is not abelian, then $\abs{Z_G} = p$.

  (Abelian: $\quot{\Zb}{p\Zb} \times \quot{\Zb}{p\Zb} \times \quot{\Zb}{p\Zb},
  \quot{\Zb}{p^2\Zb} \times \quot{\Zb}{p\Zb}, \quot{\Zb}{p^3\Zb}$)

  \label{prop:w6p3}
\end{prop}

\begin{prop}
  Let $\abs{G} = p^n$. Then $\forall 0 \le k \le n, \exists G_k \lhd G$ s.t.
  $\abs{G_k} = p^k$ and $G_i \lneq G_{i+1}$.

  In general, for a finite group $G$, $\exists {\{1\}} =
  G_r \lhd G_{r-1} \lhd \dots \lhd G_1 \lhd G_0 = G$ s.t. $\quot{G_i}{G_{i+1}}$
  is cyclic.

  we call $G$ a solvable group.

  \begin{proof}
    By induction on $n$, $n = 1$ is trivial.
    For $n > 1$, assume that the statement a holds for $n-1$.
    By prop \ref{prop:pgroup}, $Z_G \ne \{1\}$. $\exists a \in Z_G, a \ne 1$.
    Let $\ord(a) = p^l$, then $\ord(a^{p^{l-1}}) = p$.
    $\implies$ in any case, $\exists a \in Z_G$ with $\ord(a) = p$.

    Now $\abs*{\quot{G}{\gen{a}}} = p^{n-1}$, so by induction hypothesis,
    $\forall 0 \le k \le n - 1, \exists \ob{G_k} \lhd \quot{G}{\gen{a}}$ s.t.
    $\abs*{\ob{G_k}} = p^k, \ob{G_i} \lneq \ob{G_{i+1}}$.

    By 3rd isom. thm., $\exists G_{k+1} \lhd G$ s.t. $\ob{G_k} =
    \quot{G_{k+1}}{\gen{a}}, G_j \lneq G_{j+1}$ and $\abs{G_{k+1}} = p^{k+1}$.

  \end{proof}
\end{prop}

\begin{prop}
  Let a $p$-group $G \acts X$ with $\abs{X} < \infty$.
  Then $\abs{X} \equiv \abs{\Fix G} \pmod p$.
  \label{prop:useful}
\end{prop}

\begin{theorem}[Cauchy theorem]
  Let $p \Div \abs{G}$. Then $\exists a \in G$ s.t. $\ord(a) = p$. Consider
  \[ X = \{\, (a_1, \dots, a_p) \mid a_i \in G, a_1a_2\dots a_p = 1\,\} \]
  and the action $\quot{\Zb}{p\Zb} \times X \to X$:
  \[
    (\ob{k}, (a_1, \dots, a_p)) \mapsto (a_{k+1}, \dots, a_p, a_1, \dots, a_k)
  \]
  (This is well-defined since $ab = 1 \implies ba = 1$ in a group.)
  We find that $(a_1, \dots, a_p) \in \Fix \quot{\Zb}{p\Zb} \iff a_1 = a_2
  \dots a_p$.
  By prop \ref{prop:useful}, $\abs*{\Fix \quot{\Zb}{p\Zb}} \equiv \abs{X}
  \pmod p$. And $\abs{X} = \abs{G}^{p-1} \equiv 0 \pmod p$.
  Since $(1, \dots, 1) \in \Fix \quot{\Zb}{p\Zb}, \abs*{\quot{\Zb}{p\Zb}} \ne 0
  \implies \abs*{\quot{\Zb}{p\Zb}} \ge p$.

  So $\exists (a, \dots, a) \in \Fix \quot{\Zb}{p\Zb} \implies a^p = 1$w
\end{theorem}

\underline{Application}: Let $\abs{G} = p^3$ and $G$ be non-abelian
($p$ is odd).
By prop \ref{prop:w6p3}, $\abs*{\quot{G}{Z_G}} = p^2$. Since $G$ is non-abelian,
we have $\quot{G}{Z_G} \cong \quot{\Zb}{p\Zb} \times \quot{\Zb}{p\Zb}$.
That is, $\forall a \in G, a^p \in Z_G$.

So,
\[
  \exists \varphi: G \to Z_G \cong C_p \text{~with~}
  \varphi: a \mapsto a^p
\]

Since $\quot{G}{Z_G}$ is abelian, $[G,G] \le Z_G$. And
\[
  \begin{cases}
    \abs{[G,G]} \Div \abs{Z_G} = p \\
    G \text{~is non-abelian}
  \end{cases}
  \implies [G,G] = Z_G
\]

\begin{definition}
  $[x, y] = x^{-1}y^{-1}xy \in [G,G], [x,y]^p = 1$.
\end{definition}

So $a^p b^p = a^p b^p [b, a]^p$ ... 換換換 總共需要 $p(p-1)/2$
\[ a^p b^p = (ab)^p [b,a]^{\frac{p(p-1)}{2}} = (ab)^p \]

So $\varphi$ is a group homo.

Now if $\Ker \varphi = G \quad (\forall a \in G, a^p = 1)$,
i.e. $\varphi$ is trivial, then $\varphi$ is useless.
Else, $\exists a \in G$ s.t. $\ord(a) = p^2$, then
$H = \gen{a} \lhd G$. ($[G:H] = p$ is the smallest prime dividing $\abs{G}$)

Also, in this case, $\varphi: G \onto Z_G \implies
\quot{G}{\Ker \varphi} \cong Z_G$. Let $E = \Ker \varphi$, $\abs{E} = p^2$.
By the def. of $\Ker \varphi$, $E \cong \quot{\Zb}{p\Zb} \times
\quot{\Zb}{p\Zb}$.

We find that $H \cap E = \gen{a^p}$. Pick $b \in E \setminus H$ and let
$K = \gen{b} \implies \abs{K} = p, H \cap K = \{ 1 \}, HK = G$.

\subsubsection{Semidirect product}

\begin{fact}
$K \lhd G, H \lhd G, K \cap H = \{1\} \implies KH = K \times H$ \\
($\forall k\in K, h \in H, khk^{-1} h^{-1} \in H \cap K = \{1\}, \implies kh=hk$)
\end{fact}

\begin{fact}
Let $K, H$ be two groups, and $G=K \times H \implies K \times \{1\} \lhd K \times H, \{1\} \times H \lhd K \times H$
\end{fact}

\begin{observation}
$K \leq G, H \lhd G, K \cap H = \{1\}$ (K 慘 H 好,簡稱慘好集) \\
$\implies$ elements in $KH$ has unique representation ? 好事喔\\
$KH \iff K \times H$ 1-1 corresp, $(kh) \iff (k, h)$
\end{observation}

Group operation :
$\forall k_1, k_2 \in K, h_1, h_2 \in H, (k_1 h_1) (k_2 h_2) = k_1 k_2 (k_2^{-1} h_1 k_2) h_2$ \\
Let $\tau : K \implies \Aut(H), k \implies \tau(k) : h\implies khk^{-1}$ (類似 $\in \Inn(H)$ )

\begin{definition}[Semi-Direct Product (慘好積)]
  $K \times_{\tau} H = \{(k, h) | k\in K, h \in H\}$ with group operation :
$(k_1, h_1)(k_2, h_2) = (k_1 k_2, \tau(k_2^{-1})(h_1)(h_2))$
where $\tau : K \implies \Aut(H)$  (need not to be inner homomorphism)
\end{definition}

Properties:
\begin{itemize}
  \item Associativity: Good, ex
  \item The identity = $(1, 1)$
  \item Inverse : $(k, h)^{-1} = (k^{-1}, \tau(k)(h^{-1}))$
  \item $K \cong K \times \{1\} \leq K \times_{\tau} H$ :
    $(k_1, 1)(k_2, 1) = (k_1 k_2, \tau(k_2^{-1})(1)1) = (k_1 k_2, 1) \in K \times \{1\}$
    $H \cong \{1\} \times H \leq K \times{\tau} H : (1, h+1), (1, h_2) = (1, \tau(1^{-1})(h_1)h_2) = (1, h_1 h_2) \in \{1\} \times K $
  \item $H \lhd K \times_t H : (k, h) (1, h')(k, h)^{-1} = (k, hh')(k^{-1}, \tau(k)(h^{-1}))
    = (1, \tau(k)(hh')\tau(k)(h^{-1})) \in H$

  \item $\tau(k)(h) = khk^{-1}$ :
    $(k, 1)(1, h)(k^{-1}, 1) = (k, h)(k^{-1}, 1) = (1, \tau(k)(h))$
  \item If $\tau$ is trivial $\implies K \times_t H \cong K \times H$
\end{itemize}

\begin{remark}
Some definition swaps the order of $H$ and $K$, i.e. $(h_1, k_1) (h_2, k_2) = (h_1 \phi(k_1)(h_2), k_1 k_2)$
\end{remark}

\begin{exercise}
Show that $H \rtimes_\phi K$ is a group and satisfies the above properties.
\end{exercise}

\begin{example}
Construct a non-abelian group of order 21.
\end{example}

\begin{fact}
$\Aut(\Zb/p\Zb) \cong (\Zb/p\Zb)^\times \cong C_{p-1}$
\end{fact}
Sol : $\phi_k: \Zb/p\Zb \implies \Zb/p\Zb, \bar{1} \implies \bar{k}$

$\phi_{k_2} o \phi_{k_1} (T) = \phi_{k_2}(\bar(k_1)) = \phi_{k_2}(T+\cdots+T)
= \bar(k_2) + \cdots \bar{k_2} = \bar{k_1 k_2}$

Let $K = C_3, H = C_7$, define $\tau : C_3 \implies \Aut(C_7) \cong C_6, a \implies \phi_2$
$phi_k : b \implies b^k$

$G = \gen{a, b | a^3=1, b^7=1, aba^{-1}=b^2}$

\begin{example}
p : odd, $|G| = p^3$, $G$ is non-abelian.
\end{example}
(sol)
$\phi: G \implies Z(G), a \implies a^p$ non trivial
case $\exists a \in G $ with $\ord(a) = p^2$.
Let $H = <a>$ here $\phi$ is onto and $E = \ker \phi \cong \Zb/p\Zb \times \Zb/p\Zb$
And $|H \cap E| = p$
$H \lhd G$ because $[G: H]=p$
Pick $b \in E \ H$ and let $K = <b> \implies |K| = p, K \cap H = \{1\}$
so $|G| = |KH| = p^3$

\begin{fact}
$\Aut(Z/p^2 Z) \cong (Z/p^2 Z)^\times$
\end{fact}
Sol : $\phi_k: \Zb/p^2\Zb \implies \Zb/p^2\Zb, \bar{1} \implies \bar{k}, \gcd(k, p) = 1$

Find a group homo $\tau : K \implies \Aut(H)$
because $(1+p)^p \equiv 1 \mod p^2$, $\ord(\bar{1+p}) \in \Zb/p^2\Zb$
Let $P = \gen{\bar{1+p}}$ is the only subgroup of order $p$.
(if $\exists |Q| = p$, $P \neq Q$ then $P \cap Q = 1$, $|PQ| = p^2$, miserable.)
So let $\tau : b \implies \phi_{1+p} : a \implies a^{1+p}$
so $G = \gen{a, b | a^{p^2}=1, b^p = 1, bab^{-1} = a^{1+p}}$ is a non-abelian group of order $p^3$.

\begin{example}
Isometry of $R^n$
\end{example}

\begin{definition}[Isometry]
An isometry of $R^n$ is a function $h: R^n \implies R^n$ that preserves the distance between vectors.
\end{definition}
$h = t \circ k$ where $t$ is translation, $k$ is an isometry fixing the origin, i.e. $k \in O(n)$.
Let $T$ be the group of translations on $R^n$, $T \cong (R^n, +, 0), t \implies t(0)$.

Let $\tau : O(n) \implies \Aut(T), A \implies L_A : R^n \implies R^n, v \implies Av$

$\implies \Isom(R^n) = O(n) \times_\tau R^n$

\begin{example}
Quaternium $Q_8 = \{\pm 1, \pm i, \pm j, \pm k\}$ is not a semi-deriect product of any two proper subgroups.
\end{example}
pf: since $\{\pm 1\}$ is contained in any non-trivial subgroups, can't find $H \cap K = \{1\}$.

\begin{example}
$A_4$, $V_4 = \{1, (12)(34), (14)(23), (13)(24)\} \lhd A_4, V_4 \cong \Zb/2\Zb \times \Zb/2\Zb$
\end{example}
Let $H_{ao} = \gen{(123)} \cong C_3$, define $\tau : H_{ao} \implies \Aut(V_4) \cong GL_2(\Zb/2\Zb)$
$(123) \implies (\bar{0} \bar{1}; \bar{1} \bar{1})$
so $A_4 \cong C_3 \times_\tau V_4$.

\begin{exercise}
Construct $D_n$ as a semi-direct product of $\Zb/n\Zb$ and $\Zb/2\Zb$.
\end{exercise}

\begin{exercise}
  \begin{enumerate}
    \item Show that $S_4$ is a semi-direct product of $V_4$ and $H = \{\sigma \in S_4 | \sigma(4) = 4\} \sim S_3$.
    \item Show that $S_n$ is a semi-direct product of $A_n$ and $H = \gen{(12)}$.
  \end{enumerate}
\end{exercise}

\begin{remark}
  \begin{itemize}
    \item $\Aut(\Zb/p\Zb \times \Zb/p\Zb) \cong GL_2(\Zb/p\Zb)$ (regarded as a vector space over $\Zb/p\Zb$)
    \item $\Aut(\Zb/p\Zb \times \Zb/q\Zb) \cong \Aut(\Zb/p\Zb) \times \Aut(\Zb/q\Zb) \cong C_{p-1} \times C_{q-1}$
  \end{itemize}
\end{remark}
