%! TEX root=../main.tex
\subsection{Koszul complex (week 16)}

\begin{remark}
  In this section, we assume that $R$ is commutative with $1$.
\end{remark}

\begin{definition}
  Let $L \in \Mod_R$, with $f: L \to R$ an $R$-linear map, define
  $$\deffunc{\dd_f}{\Lambda^n L}{\Lambda^{n-1} L}{x_1 \we \cdots \we x_n}
  {\sum\limits_{i=1}^n (-1)^{i+1} f(x_i) x_1 \we \cdots \we \hat{x}_i \we
  \cdots \we x_n}$$
  where $\Lambda^n L$ is the $n$-th exterior power of $L$, and $\hat{x}_i$ means
  omitting $x_i$.

  Then we can define a chain complex called {\bf Koszul complex}:
  $$K_\bullet(f) : \cdots \to \Lambda^n L \xto{\dd_f} \Lambda^{n-1} L \to \cdots \to
  \Lambda^2 L \xto{\dd_f} L \xto{f} R$$

  Also, $\dd_f$ can be considered as a graded $R$-homomorphism of degree $-1$:
  $$\deffunc{\dd_f}{\Lambda L}{\Lambda L}{x \we y}
  {\dd_f(x) \we y + (-1)^{\deg x}\cdot x \we \dd_f(y)}$$
  where $\Lambda L$ is the exterior algebra of $L$, and $x, y$ are any homogeneous 
  elements of $\Lambda L$.
\end{definition}

\begin{definition}
  Let $(C_\bullet, d), (C_\bullet', d')$ be chain complexes of $R$-modules, 
  define their {\it tensor product} to be a chain complex $C_\bullet \Ot C_\bullet'$ with
  $$\left( C_\bullet \Ot C_\bullet' \right)_n = 
  \bigoplus\limits_{i=0}^n \left( C_i \Ot_R C_{n-i}' \right)$$
  with the boundary maps being
  $$\deffunc{d \Ot d'}{(C_\bullet \Ot C_\bullet')_n}{(C_\bullet \Ot C_\bullet')_{n-1}}
  {\sum\limits_{i=0}^n x_i \Ot y_{n-i}}{\sum\limits_{i=0}^n 
  \left( d(x_i) \Ot y_{n-i} + (-1)^i \cdot x_i \Ot d'(y_{n-i}) \right)}$$

  One can verify that 
  \begin{align*}
    (d \Ot d') \circ (d \Ot d') (x \Ot y) 
    &= (d \Ot d') (d(x) \Ot y + (-1)^{\deg x} \cdot x \Ot d'(y)) \\
    &= d \circ d (x) \Ot y + (-1)^{\deg x - 1} \cdot d(x) \Ot d'(y) \\
    &\qquad + (-1)^{\deg x} \cdot d(x) \Ot d'(y) + x \Ot d' \circ d'(y) \\
    &= 0
  \end{align*}
\end{definition}

\begin{prop}
  Let $L_1, L_2 \in \Mod_R, f_1 \in \Hom_R(L_1, R), f_2 \in \Hom_R(L_2, R)$.
  Define
  $$\deffunc{f = f_1 + f_2}{L_1 \Op L_2}{R}{(x, y)}{f_1(x) + f_2(y)}, $$
  then 
  \begin{eqnarray*}
    K_\bullet(f_1) \Ot K_\bullet(f_2) &\cong& K_\bullet(f) \\
    \bigoplus\limits_{i=0}^n\left(\Lambda^i L_1 \Ot_R \Lambda^{n-i} L_2\right)
    &\cong& \Lambda^n(L_1 \Op L_2)
  \end{eqnarray*}
  with
  $\dd_{f_1} \Ot \dd_{f_2} = \dd_f$.

  \begin{proof}
    Exercise 16-1(2).
  \end{proof}
\end{prop}

\begin{definition}
  Let $L = \bigoplus\limits_{i=1}^n R e_i$ be a free $R$-module, and 
  $\mathbf{x} =(x_1, \cdots, x_n)$ with $x_i \in R$, define
  $$
  K_\bullet(\mathbf{x}) \defeq K_\bullet(f), \textrm{with }
  \deffunc{f}{L}{R}{e_i}{x_i}.
  $$
\end{definition}

\begin{coro}
  $K_\bullet(\mathbf{x}) \cong K_\bullet(x_1) \Ot \cdots \Ot K_\bullet(x_n)$
  with 
  $K_\bullet(x_i) : 0 \to R \xto{x_i} R$.
\end{coro}

\begin{prop}
  Let $x \in R$ and $(C_\bullet, \partial)$ be a chain complex of $R$-modules, then exists
  $$
  0 \to C_\bullet \xto{\rho} C_\bullet \Ot K_\bullet(x) \xto{\pi} C_\bullet(-1) \to 0
  $$
  exact, where $\left( C_\bullet(-1) \right)_n = C_{n-1}$.

  \begin{proof}
    Since $K_\bullet(x) : 0 \to R \xto{x} R$, so
    $$
    (C_\bullet \Ot K_\bullet(x))_n = (C_i \Ot_R R) \Op (C_{i-1} \Ot_R R),
    $$
    and the boundary map is
    $$
    \deffunc{\dd}{(C_i \Ot_R R) \Op (C_{i-1} \Ot_R R)}{(C_{i-1} \Ot_R R) \Op (C_{i-2} \Ot_R R)}
    {(z_1 \Ot r_1, z_2 \Ot r_2)}
    {\left( \partial z_1 \Ot r_1 + (-1)^{i-1} z_2 \Ot x r_2,
    \partial z_2 \Ot r_2 \right)}.
    $$
    Under the isomorphism $C_i \Ot_r R \cong C_i$, the boundary map become
    $$
    \deffunc{\dd}{C_i \Op C_{i-1}}{C_{i-1} \Op C_{i-2}}
    {
    \begin{pmatrix}
      r_1 z_1 \\ r_2 z_2
    \end{pmatrix}
    }
    {
    \begingroup
    \setlength\arraycolsep{4pt}
    \begin{pmatrix}
      \partial & (-1)^{i-1} x \\ 0 & \partial
    \end{pmatrix}
    \begin{pmatrix}
      r_1 z_1 \\ r_2 z_2
    \end{pmatrix}
    \endgroup
    }
    $$

    Let 
    $$
    \begin{align*}
      \deffunc{\rho_i}{C_i}{C_i \Op C_{i-1}}{z_1}{(z_1, 0)} \\
      \deffunc{\pi_i}{C_i \Op C_{i-1}}{C_{i-1}}{(z_1, z_2)}{z_2} \\
    \end{align*}
    $$
    then
    $$
    \begin{tikzcd}[cramped]
      0 \ar[r] & C_i \ar[r, "\rho_i"] \ar[d, "\partial"] & 
      C_i \Op C_{i-1} \ar[r, "\pi_i"] \ar[d, "\dd"] 
      & C_{i-1} \ar[r] \ar[d, "\partial"] & 0 \\
      0 \ar[r] & C_{i-1} \ar[r, "\rho_{i-1}"] & 
      C_{i-1} \Op C_{i-2} \ar[r, "\pi_{i-1}"] & C_{i-2} \ar[r] & 0
    \end{tikzcd}
    $$
    commutes and exact:
    \begin{itemize}
      \item $\dd \circ \rho (z_1) = \dd (z_1, 0) = (\partial z_1, 0)$
      \item $\rho \circ \partial (z_1) = \rho(\partial z_1) = (\partial z_1, 0)$
      \item $\partial \circ \pi (z_1, z_2) = \partial(z_2) = \partial z_2$
      \item $\pi \circ \dd (z_1, z_2) = \pi(\partial z_1 + (-1)^{i-1} x z_2, \partial z_2)
        = \partial z_2$
    \end{itemize}
  \end{proof}
\end{prop}

\begin{definition}
  We call $x$ to be $C_\bullet$-regular, if $x$ is not a zero divisor of $C_i$
  and $C_i / x C_i \neq 0$, for all $i \ge 0$.
\end{definition}

\begin{prop}
  If $x$ is $C_\bullet$-regular, then $H_i(C_\bullet \Ot K_\bullet(x)) \cong H_i(C_\bullet / x C_\bullet)$ 
  for all $i \ge 0$.

  \begin{proof}
    Let 
    $$
    \deffunc{\phi_i}{C_i \Op C_{i-1}}{C_i / x C_i}{(z_1, z_2)}{\ob{z_1}},
    $$
    then
    $$
    \begin{tikzcd}[cramped]
      C_i \Op C_{i-1} \ar[r, "\phi_i"] \ar[d, "\dd_i"] & C_i/x C_i \ar[d, "\ob{\partial}_i"] \\
      C_{i-1} \Op C_{i-2} \ar[r, "\phi_{i-1}"] & C_{i-1}/x C_{i-1} 
    \end{tikzcd}
    $$
    commutes.
    \begin{itemize}
      \item $\ob{\partial} \circ \phi_i(z_1, z_2) = \ob{\partial}(z_1) = \ob{\partial z_1}$.
      \item $\phi_{i-1} \circ \dd (z_1, z_2) = \phi_{i-1}(\partial z_1 + (-1)^{i-1} xz_2, \partial z_2)
        = \ob{\partial z_1}$, since $xz_2 \in xC_{i-1}$.
    \end{itemize}

    Now we need to show the induced maps
    $$
    \deffunc{\phi_{*i}}{\ker \dd_i / \im \dd_{i+1}}{\ker \ob{\partial}_i / \im \ob{\partial}_{i+1}}
    {\ob{(z_1, z_2)}}{\ob{\ob{z_1}} = \ob{z_1} + \im \ob{\partial}_{i+1}}
    $$
    are isomorphisms.

    \begin{itemize}
      \item Onto: \\


    \end{itemize}
  \end{proof}
\end{prop}
