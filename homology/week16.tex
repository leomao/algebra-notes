%! TEX root=../main.tex
\subsection{Koszul complex (week 16)}
In this section, we assume that $R$ is commutative with $1$.
\begin{definition}
  Let $L \in \Mod_R$, with $f: L \to R$ an $R$-linear map, define
  $$\deffunc{\dd_f}{\Lambda^n L}{\Lambda^{n-1} L}{x_1 \we \cdots \we x_n}
  {\sum\limits_{i=1}^n (-1)^{i+1} f(x_i) x_1 \we \cdots \we \hat{x}_i \we
  \cdots \we x_n}$$
  where $\Lambda^n L$ is the $n$-th exterior power of $L$, and $\hat{x}_i$ means
  omitting $x_i$.

  Then we can define a chain complex called {\bf Koszul complex}:
  $$K_\bullet(f) : \cdots \to \Lambda^n L \xto{\dd_f} \Lambda^{n-1} L \to \cdots \to
  \Lambda^2 L \xto{\dd_f} L \xto{f} R$$

  Also, $\dd_f$ can be considered as a graded $R$-homomorphism of degree $-1$:
  $$\deffunc{\dd_f}{\Lambda L}{\Lambda L}{x \we y}
  {\dd_f(x) \we y + (-1)^{\deg x}\cdot x \we \dd_f(y)}$$
  where $\Lambda L$ is the exterior algebra of $L$, and $x, y$ are any homogeneous 
  elements of $\Lambda L$.
\end{definition}

\begin{definition}
  Let $(C_\bullet, d), (C_\bullet', d')$ be chain complexes of $R$-modules, 
  define their {\it tensor product} to be a chain complex $C_\bullet \Ot C_\bullet'$ with
  $$\left( C_\bullet \Ot C_\bullet' \right)_n = 
  \bigoplus\limits_{i=0}^n \left( C_i \Ot_R C_{n-i}' \right)$$
  with the boundary maps being
  $$\deffunc{d \Ot d'}{(C_\bullet \Ot C_\bullet')_n}{(C_\bullet \Ot C_\bullet')_{n-1}}
  {\sum\limits_{i=0}^n x_i \Ot y_{n-i}}{\sum\limits_{i=0}^n 
  \left( d(x_i) \Ot y_{n-i} + (-1)^i \cdot x_i \Ot d'(y_{n-i}) \right)}$$

  One can verify that 
  \begin{align*}
    (d \Ot d') \circ (d \Ot d') (x \Ot y) 
    &= (d \Ot d') (d(x) \Ot y + (-1)^{\deg x} \cdot x \Ot d'(y)) \\
    &= d \circ d (x) \Ot y + (-1)^{\deg x - 1} \cdot d(x) \Ot d'(y) \\
    &\qquad + (-1)^{\deg x} \cdot d(x) \Ot d'(y) + x \Ot d' \circ d'(y) \\
    &= 0
  \end{align*}
\end{definition}

\begin{prop}
  Let $L_1, L_2 \in \Mod_R, f_1 \in \Hom_R(L_1, R), f_2 \in \Hom_R(L_2, R)$.
  Define
  $$\deffunc{f = f_1 + f_2}{L_1 \Op L_2}{R}{(x, y)}{f_1(x) + f_2(y)}, $$
  then 
  \begin{eqnarray*}
    K_\bullet(f_1) \Ot K_\bullet(f_2) &\cong& K_\bullet(f) \\
    \bigoplus\limits_{i=0}^n\left(\Lambda^i L_1 \Ot_R \Lambda^{n-i} L_2\right)
    &\cong& \Lambda^n(L_1 \Op L_2)
  \end{eqnarray*}
  with
  $\dd_{f_1} \Ot \dd_{f_2} = \dd_f$.

  \begin{proof}
    Exercise 16-1(2).
  \end{proof}
\end{prop}

\begin{definition}
  Let $L = \bigoplus\limits_{i=1}^n R e_i$ be a free $R$-module, and 
  $\mathbf{x} =(x_1, \cdots, x_n)$ with $x_i \in R$, define
  $$
  K_\bullet(\mathbf{x}) \defeq K_\bullet(f), \textrm{with }
  \deffunc{f}{L}{R}{e_i}{x_i}.
  $$
\end{definition}

\begin{coro}
  $K_\bullet(\mathbf{x}) \cong K_\bullet(x_1) \Ot \cdots \Ot K_\bullet(x_n)$
  with 
  $K_\bullet(x_i) : 0 \to R \xto{x_i} R$.
\end{coro}

\begin{prop}
  Let $x \in R$ and $(C_\bullet, \partial)$ be a chain complex of $R$-modules,
  then there exist $\rho, \pi$ s.t.
  $$
  0 \to C_\bullet \xto{\rho} C_\bullet \Ot K_\bullet(x) \xto{\pi} C_\bullet(-1) \to 0
  $$
  is exact, where $\left( C_\bullet(-1) \right)_n = C_{n-1}$.

  \begin{proof}
    Since $K_\bullet(x) : 0 \to R \xto{x} R$, so
    $$
    (C_\bullet \Ot K_\bullet(x))_n = (C_i \Ot_R R) \Op (C_{i-1} \Ot_R R),
    $$
    and the boundary map is
    $$
    \deffunc{\dd}{(C_i \Ot_R R) \Op (C_{i-1} \Ot_R R)}{(C_{i-1} \Ot_R R) \Op (C_{i-2} \Ot_R R)}
    {(z_1 \Ot r_1, z_2 \Ot r_2)}
    {\left( \partial z_1 \Ot r_1 + (-1)^{i-1} z_2 \Ot x r_2,
    \partial z_2 \Ot r_2 \right)}.
    $$
    Under the isomorphism $C_i \Ot_r R \cong C_i$, the boundary map become
    $$
    \deffunc{\dd}{C_i \Op C_{i-1}}{C_{i-1} \Op C_{i-2}}
    {
    \begin{pmatrix}
      r_1 z_1 \\ r_2 z_2
    \end{pmatrix}
    }
    {
    \begingroup
    \setlength\arraycolsep{4pt}
    \begin{pmatrix}
      \partial & (-1)^{i-1} x \\ 0 & \partial
    \end{pmatrix}
    \begin{pmatrix}
      r_1 z_1 \\ r_2 z_2
    \end{pmatrix}
    \endgroup
    }
    $$
    Let 
    $$
      \deffunc{\rho_i}{C_i}{C_i \Op C_{i-1}}{z_1}{(z_1, 0)}
      \quad \text{and} \quad
      \deffunc{\pi_i}{C_i \Op C_{i-1}}{C_{i-1}}{(z_1, z_2)}{z_2}
    $$
    then
    $$
    \begin{tikzcd}[cramped]
      0 \ar[r] & C_i \ar[r, "\rho_i"] \ar[d, "\partial"] & 
      C_i \Op C_{i-1} \ar[r, "\pi_i"] \ar[d, "\dd"] 
      & C_{i-1} \ar[r] \ar[d, "\partial"] & 0 \\
      0 \ar[r] & C_{i-1} \ar[r, "\rho_{i-1}"] & 
      C_{i-1} \Op C_{i-2} \ar[r, "\pi_{i-1}"] & C_{i-2} \ar[r] & 0
    \end{tikzcd}
    $$
    commutes and exact:
    \begin{itemize}
      \item $\dd \circ \rho (z_1) = \dd (z_1, 0) = (\partial z_1, 0)$
      \item $\rho \circ \partial (z_1) = \rho(\partial z_1) = (\partial z_1, 0)$
      \item $\partial \circ \pi (z_1, z_2) = \partial(z_2) = \partial z_2$
      \item $\pi \circ \dd (z_1, z_2) = \pi(\partial z_1 + (-1)^{i-1} x z_2, \partial z_2)
        = \partial z_2$
    \end{itemize}
  \end{proof}
\end{prop}

\begin{coro}
  This induces a long exact sequence
  $$
  \cdots \to H_i(C_\bullet) \xto{\rho_*} H_i(C_\bullet \Ot K_\bullet(x)) \xto{\pi_*} 
  H_i(C_\bullet(-1)) \xto{\pm x} H_{i-1}(C_\bullet) \to \cdots.
  $$
  \begin{proof}
    We only need to show the connection homomorphism is indeed $\pm x$.

    Given $z \in C_{i-1}$ with $\partial z = 0$, 
    $$z \xmapsto{\pi^{-1}} (0, z) \xmapsto{d} \left((-1)^{i-1}xz, 0\right) \xmapsto{\rho^{-1}}
    (-1)^{i-1} xz.$$
  \end{proof}
\end{coro}

\begin{definition}
  We call $x$ to be $C_\bullet$-regular, if $x$ is not a zero divisor of $C_i$
  and $C_i / x C_i \neq 0$, for all $i \ge 0$.
\end{definition}

\begin{prop}
  If $x$ is $C_\bullet$-regular, then $H_i(C_\bullet \Ot K_\bullet(x)) \cong H_i(C_\bullet / x C_\bullet)$ 
  for all $i \ge 0$.

  \begin{proof}
    Let 
    $$
    \deffunc{\phi_i}{C_i \Op C_{i-1}}{C_i / x C_i}{(z_1, z_2)}{\ob{z_1}},
    $$
    then
    $$
    \begin{tikzcd}[cramped]
      C_i \Op C_{i-1} \ar[r, "\phi_i"] \ar[d, "\dd_i"] & C_i/x C_i \ar[d, "\ob{\partial}_i"] \\
      C_{i-1} \Op C_{i-2} \ar[r, "\phi_{i-1}"] & C_{i-1}/x C_{i-1} 
    \end{tikzcd}
    $$
    commutes.
    \begin{itemize}
      \item $\ob{\partial} \circ \phi_i(z_1, z_2) = \ob{\partial}(z_1) = \ob{\partial z_1}$.
      \item $\phi_{i-1} \circ \dd (z_1, z_2) = \phi_{i-1}(\partial z_1 + (-1)^{i-1} xz_2, \partial z_2)
        = \ob{\partial z_1}$, since $xz_2 \in xC_{i-1}$.
    \end{itemize}

    Now we need to show the induced maps
    $$
    \deffunc{\phi_{*i}}{\ker \dd_i / \im \dd_{i+1}}{\ker \ob{\partial}_i / \im \ob{\partial}_{i+1}}
    {\ob{(z_1, z_2)}}{\ob{\ob{z_1}} = \ob{z_1} + \im \ob{\partial}_{i+1}}
    $$
    are isomorphisms.

    \begin{itemize}
      \item Onto: \\
        For $\ob{z} \in \ker \ob{\partial}_i$ with $\partial z = xz' \in xC_{i-1}$, 
        $z' \in C_{i-1}$.
        Then $\phi_i(z, (-1)^i z') = \ob{z}$, and 
        $\dd(z, (-1)^i z') = (\partial z - xz', (-1)^i \partial z') = (0, 0)$, so
        $(z, (-1)^i z') \in \ker \dd_i$.
        (Since $x \partial z' = \partial(xz') = \partial^2 z = 0$, and $x$ is not a
        zero divisor of $C_i$, so $\partial z' = 0$.)

        Now, $\phi_{*i} \left( \ob{(z, (-1)^i z')} \right) = \ob{\ob{z}}$, so $\phi_{*i}$ is onto.

      \item 1-1: \\
        Let $(z, z') \in \ker \dd_i$ with $\phi_i(z, z') = \ob{z} \in \im \ob{\partial}_{i+1}$, i.e.
        $\ob{z} = \partial \ob{z''}$ with $z'' \in C_{i+1}$.
        This means $z - \partial z'' = x z'''$ with $z''' \in C_i$, so
        $\partial (z - \partial z'') = \partial z = x \partial z'''$.

        On the other hand, $\dd(z, z') = (\partial z + (-1)^{i-1} xz', \partial z') = (0, 0)$, so
        $\partial z = (-1)^i xz', \partial z' = 0$.

        So $\dd(z'', (-1)^i z''') = (\partial z'' + (-1)^{2i} x z'''), (-1)^i \partial z''')
        = (z, z')$, i.e. $(z, z') \in \im d_{i+1}$. 
        ($\partial z = x \partial z''' = (-1)^i xz'$, since $x$ is not a zero divisor, 
        so $\partial z''' = (-1)^i z'$.)

        Hence, $\phi_{*i}\left(\ob{(z_1, z_2)}\right) = \ob{0}$ implies 
        $\ob{(z_1, z_2)} = \ob{0}$, so $\phi_{*i}$ is 1-1.
    \end{itemize}
  \end{proof}
\end{prop}

\begin{definition}
  Let $M \in \Mod_R$. A sequence $\set{a_1, \cdots, a_m}, m \ge 0$ is said to be
  $M$-regular if
  \begin{itemize}
    \item $M/\gen{a_1, \cdots, a_m}M \neq 0$.
    \item $a_{i+1}$ is not a zero divisor of $M/\gen{a_1, \cdots, a_i}M$ for 
      $0 \le i \le m-1$.
  \end{itemize}
\end{definition}

\begin{theorem}
  If $\mathbf{x} = (x_1, \cdots, x_n)$ is an $R$-regular sequence, then
  $K_\bullet(\mathbf{x}) \to R/\gen{x_1, \cdots, x_n} \to 0$ is a free resolution of 
  $R / \gen{x_1, \cdots, x_n}$.

  \begin{proof}
    Since its modules are
    $K_i(\mathbf{x}) = \Lambda^i R^n \cong R^{n \choose i}$
    , i.e. free $R$-modules, so we only need to show the exactness.

    By induction on $n$, 
    \begin{itemize}
      \item $n = 1$:  
        $K_\bullet(x_1) : 0 \to R \xto{x_1} R \to R/\gen{x_1} \to 0$ exact.

      \item $n > 1$:  
        Assume that $\mathbf{x}' = (x_1, \cdots, x_{n-1})$ and 
        $K_\bullet(\mathbf{x}') \to R/\gen{x_1, \cdots, x_{n-1}} \to 0$ exact, 
        i.e. $H_i(K_\bullet(\mathbf{x}')) = 0$ for $i > 0$.

        Since we have $K_\bullet(\mathbf{x}) \cong K_\bullet(\mathbf{x}') \Ot 
        K_\bullet(x_n)$ and a long exact sequence
        $$
        \cdots \to H_i(K_\bullet(\mathbf{x}')) \to H_i(K_\bullet(\mathbf{x})) \to 
        H_i(K_\bullet(\mathbf{x}')(-1)) \xto{\pm x_n} H_{i-1}(K_\bullet(\mathbf{x})) \to \cdots
        $$
        where $H_i(K_\bullet(\mathbf{x}')(-1)) = H_{i-1}(K_\bullet(\mathbf{x}'))$.

        For $i > 1$, the sequence becomes
        $$
        \cdots \to 0 \to H_i(K_\bullet(\mathbf{x})) \to 0 \xto{\pm x_n} \cdots,
        $$
        so $H_i(K_\bullet(\mathbf{x})) = 0$. 

        For $i = 1$, we have $H_0(K_\bullet(\mathbf{x})) \cong R/\gen{x_1, \cdots, x_{n-1}}$, so
        $$
        0 \to H_1(K_\bullet(\mathbf{x})) \to R/\gen{x_1, \cdots, x_{n-1}} \xto{\pm x_n} 
        R/\gen{x_1, \cdots, x_{n-1}}
        $$

        But $x_n$ is not a zero divisor of $R/\gen{x_1, \cdots, x_{n-1}}$, so the map
        $\pm x_n$ is 1-1, then $H_1(K_\bullet(\mathbf{x})) \cong \ker (\pm x_n) = 0$. 
    \end{itemize}
  \end{proof}
\end{theorem}

\begin{example}
  Let $\mathbf{x} = (x_1, x_2)$, then
  $$
  K_\bullet(\mathbf{x}) : 0 \to R \xto{\alpha} R^2 \xto{\beta} R \xto{q} R/\gen{x_1, x_2} \to 0
  $$
  with $\alpha : r \mapsto (-x_2r, x_1r)$ and $\beta : (r_1, r_2) \mapsto x_1r_1 + x_2r_2$.
\end{example}

\begin{coro}
  Let $I = \gen{x_1, \cdots, x_n} \subset R$ be an ideal with $\set{x_1, \cdots, x_n}$ be $R$-regular, 
  then $R/I$ has {\it projective dimension} $\mathrm{pd}(R/I) = n$, 
  i.e. the shortest projective resolution of $R/I$ has length $n$.

  \begin{proof}
    $K_\bullet(\mathbf{x})$ is already a projective resolution of length $N$, so 
    we only need to show that there's no shorter ones.

    The left side of $K_\bullet(\mathbf{x})$ reads
    $$
    0 \to \Lambda^n R^n \xto{d_n} \Lambda^{n-1} R^n \to \cdots
    $$

    But 
    $$
    \Lambda^n R^n = R(e_1 \we \cdots \we e_n) \cong R, \quad
    \Lambda^{n-1} R^n = \bigoplus\limits_{i=1}^n  R (e_1 \we \cdots \we \hat{e_i} \we \cdots \we e^n)
    \cong R^n
    $$
    so
    $$
    \deffunc{d_n}{R}{R^n}{r}{(x_1 r, -x_2 r, \cdots, (-1)^{n-1} x_n r)}
    $$

    Taking tensor with $R/I$, we get
    $$
    0 \to R \Ot_R R/I \xto{d_n \Ot \id} R^n \Ot_R R/I \to \cdots
    $$
    but $R \Ot_R R/I \cong R/I, R^n \Ot_R R/I \cong (R/I)^n$, so
    $$
    \deffunc{d_n \Ot \id}{R/I}{(R/I)^n}{\ob{r}}{(\ob{x_1r}, \ob{-x_2r}, \cdots, \ob{(-1)^{n-1}x_n r})}
    $$

    Now, 
    $$
    \Tor_n(R/I, R/I) = H_n(K_\bullet(\mathbf{x}) \Ot R/I) = \ker (d_n \Ot \id)
    = \Ann_{R/I} I = \set{\ob{r} \in R/I \mid rI = I} = R/I \neq 0.
    $$
    ($R/I \neq 0$ is because $\set{x_1, \cdots, x_n}$ is $R$-regular.)
    Thus, any projective resolution can't have length
    shorter than $n$ since that will imply $\Tor_n(R/I, R/I) = 0$.
  \end{proof}
\end{coro}

\begin{remark}
  Let $I = \gen{x_1, \cdots, x_n}$ generated by $R$-regular sequence 
  $\set{x_1, \cdots, x_n}$, then
  \begin{itemize}
    \item $\Tor_n(R/I, M) \cong \Ann_M I$.
    \item $\Ext^n(R/I, M) \cong M/IM$.
  \end{itemize}
\end{remark}

\subsection{Derived category}
\begin{definition} \mbox{}
  \begin{itemize}
    \item $\Cc$ is a pre-additive category if $\Hom_\Cc(X, Y)$ is an abelian group
      $\forall X, Y \in \Cc$ s.t.
      \[
        \begin{tikzcd}
          X \ar[r, "u"] & Y \ar[r, "f", shift left] \ar[r, "g"', shift right]
                        & Z \ar[r, "v"]
                        & T
        \end{tikzcd}
      \]
      with $(f+g) u = fu + gu$ and $v(f+g) = vf + vg$.
    \item addivitve category: a pre-additive category $\Cc$ s.t.
      \begin{itemize}
        \item There exists a zero object $0$ s.t. $\forall X,\, \Hom_\Cc(0, X)
          = \set{0} = \Hom_\Cc(X, 0)$.
        \item Finite sum and finite products exist.
          % TODO graph
      \end{itemize}
  \end{itemize}
\end{definition}

\begin{definition} \hfill
  \begin{itemize}
    \item $f \in \Hom(B, C)$ is called a monomorphism if
      $\forall X \xto{g} B \xto{f} C$ with $f \circ g = 0 \implies g = 0$.
    \item $f \in \Hom(B, C)$ is called a epimorphism if
      $\forall B \xto{f} C \xto{h} D$ with $h \circ f = 0 \implies h = 0$.
    \item a kernel of $f \in \Hom(B, C)$ is a morphism $i: A \to B$ s.t.
      $f \circ i = 0$ and $\forall g: X \to B$ with $f\circ g = 0$, we have
      \[
        \begin{tikzcd}[cramped]
          A \ar[r, "i"] & B \ar[r, "f"] & C \\
                        & X \ar[lu, dashed, "\existsuniq"] \ar[u, "g"] \ar[ru, "0"]
        \end{tikzcd}
      \]
    \item a cokernel of $f \in \Hom(B, C)$ is a morphism $p: C \to D$ s.t.
      $p \circ f = 0$ and $\forall h: C \to Y$ with $h\circ f = 0$, we have
      \[
        \begin{tikzcd}[cramped]
          B \ar[r, "f"] \ar[rd, "0"] & C \ar[r, "p"] \ar[d, "h"]
                                     & D \ar[ld, dashed, "\existsuniq"] \\
                                     & Y
        \end{tikzcd}
      \]
  \end{itemize}
\end{definition}

\begin{remark} \mbox{}
  \begin{itemize}
    \item If $i$ is a kernel of $f$, then $i$ is a monomorphism.
    \item If $p$ is a kernel of $f$, then $p$ is a epimorphism.
  \end{itemize}
\end{remark}

\begin{remark}
  An epimonrphism may not be a cokernel. Consider $\Zb \xto{\times 3} \Zb$
  which is an epimorphism in the category of f.g. free $\Zb$-modules.
  If $\Zb \xto{\times 3} \Zb$ is the cokernel of $G \xto{f} \Zb$, then
  \[
    \begin{tikzcd}[cramped]
      G \ar[r, "f"] \ar[rd, "0"] & \Zb \ar[r, "\times 3"] \ar[d, "\times 2"]
                                 & \Zb \ar[ld, dashed, "\existsuniq \tilde f"] \\
                                 & \Zb
    \end{tikzcd}
  \]
  This implies $\tilde f: 1 \mapsto \frac{2}{3}$, which is impossible.
\end{remark}

\begin{definition}
  $\Ac$ is an {\bf abelian category} if it is an additive category s.t.
  \begin{itemize}
    \item kernels and cokernels always exist in $\Ac$.
    \item every monomorphism is a kernel and every epimorphism is a cokernel.
  \end{itemize}
\end{definition}

\begin{fact} If $\Ac$ is an abelian category, then:
  \begin{itemize}
    \item every morphism is expressible as the composite of an epimorphism
      and a monomorphism. Given $f: B \to C$, we have
      \[
        \begin{tikzcd}[column sep=tiny]
          B \ar[rr, "f"] \ar[rd, twoheadrightarrow] & & C \\
               & \Image f \ar[ru, hook] &
        \end{tikzcd}
      \]
      where $\Image f$ is unique up to isomorphism.

      \begin{proof}
        Consider the following diagram:
        \[
          \begin{tikzcd}[row sep=25pt]
            \ker f \ar[r, "i", hookrightarrow] &
            B \ar[r, "f"] \ar[d, "p'", twoheadrightarrow]
            \ar[rd, "\exists ! \mu", dashed, very near start, sloped] &
            C \ar[r, "p", twoheadrightarrow] \ar[d, "i'", hookleftarrow]
            \ar[ld, "\exists ! \gamma", dashed, leftarrow, pos=0.4, sloped] &
            \coker{f} \\
            & \coker{i} \ar[r, "\exists ! \sigma"', dashed, thick, pos=0.4] &
            \ker{p} &
          \end{tikzcd}
        \]
        Where $\mu, \gamma$ exists because $i, p$ are kernel and cokernel.
        Now, $i' \mu i = f i = 0$, and since $i'$ is a monomorphism,
        $\mu i = 0$. Moreover, since $p$ is the cokernel of $i$,
        there exists a unique $\sigma$ letting the diagram commute.

        By exercise, $\sigma$ is both a monomorphism and epimorphism.
        In an abelian catagory, this implies that $\sigma$ is
        actually an isomorphism (i.e., $\sigma^{-1}$ exists).
      \end{proof}
    \item $0 \to A \xto{f} B \xto{g} C \to 0$ is exact if
      $f$ is monomorphism, $g$ is epimorphism and $\Image f = \ker g$.
  \end{itemize}
\end{fact}

\begin{theorem}[Freyd-Mitchell theorem]
  A small abelian category is equivalent to a full subcategory of a category
  of $R$-modules.
\end{theorem}

\begin{definition} \hfill
  \begin{itemize}
    \item $I \in \Obj \Ac$ is injective if the functor $\Hom(-, I)$ is exact.
    \item An abelian category is said to be {\bf enough injectives} if
      for any $A \in \Obj \Ac$, there exists an injective object $I$ such that
      $A \hookrightarrow I$.
  \end{itemize}
\end{definition}

\begin{definition}
  Given a functor $F : \Ac \to \mathcal{B}$ satisfy:
  \begin{enumerate}
    \item $F$ is additive, which is to say $F$ is a group
      homomorphism $\Hom(A, A') \to \Hom(FA, FA')$.
    \item $F$ is left exact. If $0 \to A' \to A \to A'' \to 0$,
      then $0 \to FA' \to FA \to FA''$.
  \end{enumerate}
  Then the derived functor $R^i F : \Ac \to \mathcal{B}$ is defined as
  \[ R^i F(A) = \begin{cases}
      F(A), & \text{ if } i = 0 \\
      H^i(F(I^\bullet)), & \text{ else } \end{cases} \]
\end{definition}

Our goal is to construct the derived category $D^+(\Ac)$ and $D^+(\mathcal{B})$
letting $RF$ be a exact functor.

\begin{definition}
  Let $\Ac$ be an abelian category.
  \begin{itemize}
    \item $\Kom(\Ac)$ is the category of complexes over $\Ac$.
    \item $K(\Ac)$ is the homotopy category of $\Ac$, defined by
      $\Obj(K(\Ac)) = \Obj(\Kom(\Ac))$ and
      \[ \Hom_{K(\Ac)}(A^\bullet, B^\bullet) = \left. \Hom_{\Kom(\Ac)}(A^\bullet, B^\bullet) \right/ \sim, \]
      where $\sim$ indicates homotopy equivalences.
  \end{itemize}
\end{definition}

\begin{remark} \hfill
  \begin{itemize}
    \item $\Hom_{K(\Ac)}(I^\bullet_A, I^\bullet_B) \cong \Hom_\Ac(A, B)$ by comparison theorem
      (\ref{thm:comparison-theorem}).
    \item It could be shown that $K(\Ac)$ is additive but may not be abelian.
  \end{itemize}
\end{remark}

\begin{definition}
  $f \in \Hom_{K(\Ac)}(A^\bullet, B^\bullet)$ is called a quasi-isomorphism
  if $H^n(f)$ is an isomorphism between $H^n(A^\bullet)$ and $H^n(B^\bullet)$
  for each $n$.
\end{definition}

\begin{example}
  \begin{itemize}
    \item A quasi-isomorphism is often not invertible. For example:
    \[ \begin{tikzcd}[cramped]
        0 \ar[r] & \Zb \ar[r, "\times 2"] \ar[d] & \Zb \ar[r] \ar[d] & 0 \ar[r] \ar[d] & \cdots \\
        0 \ar[r] & 0 \ar[r] & \Zb / 2\Zb \ar[r] & 0 \ar[r] & \cdots \\
    \end{tikzcd} \]
    \item Given $0 \to A \to I^\bullet$,
      \[ \begin{tikzcd}[cramped]
          0 \ar[r] & A \ar[r] \ar[d] & 0 \ar[r] \ar[d] & 0 \ar[r] \ar[d] & \cdots \\
          0 \ar[r] & I^0 \ar[r] & I^1 \ar[r] & I^2 \ar[r] & \cdots \\
      \end{tikzcd} \]
    are two quasi-isomorphic complexes.
  \end{itemize}
\end{example}

\begin{definition}
  Let $\Bc$ be a category. A class of morphism $S \subset \Mor(\Bc)$
  is said to be {\bf localizing} if
  \begin{enumerate}
    \item $S$ is closed under composition with $\Id_X \in S$ for each object $X$
      in $\Bc$.
    \item Extension condition holds: For each $f \in \Mor \Bc, \, s \in S$,
      exists $g \in \Mor \Bc, \, t \in S$ such that $ft = sg$.
      The dual version should hold as well.
    \item For any $f, g \in \Hom(X, Y)$,
      \[ \exists s \in S \text{ s.t. } sf = sg \iff \exists t \in S \text{ s.t. } ft = gt. \]
  \end{enumerate}
\end{definition}

\begin{theorem}
  If $S$ is localizing, then exists a category $\Bc[S^{-1}]$ with
  a functor $Q : \Bc \to \Bc[S^{-1}]$ such that
  \begin{enumerate}
    \item $Q(s)$ is an isomorphism for each $s \in S$.
    \item Given another functor $F : B \to B'$ satisfy condition 1, there exists
      a unique functor $G : \Bc[S^{-1}] \to B'$ such that $F = G \circ Q$.
  \end{enumerate}
  \begin{proof}
    Define a roof to be a pair $(s, t)$ with
    \[ \begin{tikzcd}[column sep=tiny, row sep=small]
        & X' \ar[ld, "S \ni s"'] \ar[rd, "t"] & \\
      X & & Y
    \end{tikzcd} \]
  Also, define $(s, f) \sim (t, g)$ if there exists $Z$ such that
  \[ \begin{tikzcd}[column sep=small]
        & X' \ar[ld, "s"'] \ar[rd, "f"] & \\
        X & Z \ar[u, "p"] \ar[d, "q"] & Y \\
        & X'' \ar[lu, "t"] \ar[ru, "g"']  &
    \end{tikzcd} \]
  with $sp = tq \in S$ and $fp = gq$.

  First we check that ``$\sim$'' is indeed an equivalence relation.
  $(s, f) \sim (s, f)$ and $(s, f) \sim (t, g) \implies (t, g) \sim (s, f)$
  are trivial. If $(s, f) \sim (t, g)$ and $(t, g) \sim (u, h)$, then
  we have the following diagram:
  \[
    \begin{tikzcd}
      & X' \ar[ldd, "s"', red] \ar[rdd, "f"] & & & \\
      & Z \ar[u, "p", red] \ar[d, "q"] & & & \\
      A & X'' \ar[l, "t", red] \ar[r, "g"'] & B &
      R \ar[llu, "\exists z"', thick] \ar[lld, "\exists a", thick] & T \ar[l, "\exists w", thick] \\
      & W \ar[u, "r", red] \ar[d, "v"] & & & \\
      & X''' \ar[luu, "u"] \ar[ruu, "h"'] & & & \\
    \end{tikzcd}
  \]
  Using definition 2. on $tr \in S$ and $sp$, there are morphism $z, a$
  with $z \in S$ and $spz = tra$. Moreover, $tqz = spz = tra$,
  if we let $b = qz, c = ra$, then by 3., morphism $w \in S$ exists
  with $bw = cw$. Define $x = pzw, y = vaw$,
  we have $sx = spzw = tqzw = tbw = tcw = traw = uvaw = uy$
  and $sx \in S$ since $sx = spzw$ and $sp, z, w$ are all in $S$.
  Similarly, $fx = hy$, thus $(s, f) \sim (u, h)$.
  Hence we've just proved that $\sim$ is an equivalence relation.

  Now we could construct the localized category as following:
  The objects are $\Obj(\Bc[S^{-1}]) = \Bc$ and
  $\Mor(\Bc[S^{-1}]) = \Set{ \text{ equivalence classes under} \sim \, }$.
  $[(t, g)] \circ [(s, f)] = [(su, gh)]$ could be defined as in the following
  diagram:
  \[ \begin{tikzcd}[column sep=tiny]
      & & {\color{red} \exists U}
      \ar[ld, "S \ni u"', red] \ar[rd, "h", red] & {\footnotesize \color{red} \text{by 2.}} & \\
      & X \ar[ld, "s"'] \ar[rd, "f"] & & Y \ar[ld, "t"'] \ar[rd, "g"] & \\
      A & & B & & C \\
    \end{tikzcd} \]
  \end{proof}

  Finally, define functor $Q$ by $Q(X) = X, \, \forall X \in \Obj(\Bc)$
  and $Q(f) = [(\Id_X, f)]$. For the universal property, if $F$
  is another functor making every morphism in $S$ be invertible,
  then the functor $G$ exists uniquely by $G([(s, f)]) = F(f) F(s)^{-1}$.
\end{theorem}
