%! TEX root=../main.tex
\subsection{Ext and Tor (week 15)}

Given $M, N \in \Mod_R$, there are two ways to define $\Ext^n(M, N)$:

\begin{definition}[Ext functor] \mbox{}
  \begin{itemize}
    \item Find any projective resolution $P_\bullet \xto{\alpha} M \to 0$,
      and let $P_M : P_\bullet \to 0$ (called a {\it deleted resolution}).
      We can define $\Ext_{\proj}^{n}(M, N) = H^n(\Hom(P_M, N))$.
    \item Find any injective resolution $0 \xto{\alpha} N \to E^\bullet$,
      and let $E_N : 0 \to E^\bullet$.
      We can define $\Ext_{\inj}^{n}(M, N) = H^n(\Hom(M, E_N))$.
  \end{itemize}
\end{definition}

\begin{prop}
  $\Ext_\proj^0(M, N) \cong \Ext_\inj^0(M, N) \cong \Hom(M, N)$.
  \begin{proof}
    $$\Hom(P_M, N) : 0 \xto{\ob{d_0}} \Hom(P_0, N) \xto{\ob{d_1}} \Hom(P_1, N) \to \cdots$$
    so $\Ext_\proj^0(M, N) = \ker \ob{d_1} / \im \ob{d_0} = \ker \ob{d_1} = \im \alpha = \Hom(M, N)$.
  \end{proof}
  Similarly, $\Ext_\inj^0(M, N) = \Hom(M, N)$.
\end{prop}

\begin{lemma} \mbox{}
  \begin{itemize}
    \item If $M$ is projective, then $\Ext_\proj^n(M, N) = 0$ for all $n > 0, N \in \Mod_R$.
    \item If $N$ is injective, then $\Ext_\inj^n(M, N) = 0$ for all $n > 0, M \in \Mod_R$.
  \end{itemize}
  
  \begin{proof}
    If $M$ is projective, then $0 \to M \xto{\id_M} M \to 0$ is a projective resolution
    of $M$.
    Its deleted resolution is then $P_M : 0 \to M \to 0$.
    Hence for $n > 0$, $\Ext_\proj^n(M, N) = H^n(\Hom(P_M, N)) = 0$.

    The argument applies similarly to injective case.
  \end{proof}
\end{lemma}

\begin{theorem}[Equivalence of $\Ext_\proj$ and $\Ext_\inj$]
  $$\Ext_\proj^n(M, N) \cong \Ext_\inj^n(M, N).$$

  \begin{proof}
    Let $P_\bullet \to M \to 0$ and $0 \to N \to E^\bullet$ be projective and 
    injective resolutions, then we have 
    $0 \to K_0 \to P_0 \to M \to 0$ and $0 \to N \to E^0 \to L^1 \to 0$ exact.

    \begin{minipage}{0.5\textwidth}
    $$
      \begin{tikzcd}[cramped, sep=10pt]
        \cdots \ar[r] & P_2 \ar[rr] \ar[rd] && P_1 \ar[rr] \ar[rd] && P_0 \ar[r] & M \ar[r] & 0 \\
                      & & K_1 \ar[ru] \ar[rd] && K_0 \ar[ru] \ar[rd] \\
                      & 0 \ar[ru] && 0 \ar[ru] && 0
      \end{tikzcd}
    $$
    \end{minipage}
    \begin{minipage}{0.5\textwidth}
    $$
      \begin{tikzcd}[cramped, sep=10pt]
        0 \ar[r] & N \ar[r] & E^0 \ar[rr] \ar[rd] && E^1 \ar[rr] \ar[rd] && E^2 \ar[r] & \cdots \\
                 &&& L^1 \ar[ru] \ar[rd] && L^2 \ar[ru] \ar[rd] \\
                 && 0 \ar[ru] && 0 \ar[ru] && 0
      \end{tikzcd}
    $$
    \end{minipage}

    \vspace{10pt}

    We can construct long exact sequences of homology of $\Hom(\cdot, E_N)$:
    $$0 \to \Hom(M, N) \to \Hom(M, E^0) \to \Hom(M, L^1) \to \Ext_\inj^1(M, N) \to \Ext_\inj^1(M, E^0) = 0$$
    $$0 \to \Hom(P_0, N) \to \Hom(P_0, E^0) \to \Hom(P_0, L^1) \to 0$$
    $$0 \to \Hom(K_0, N) \to \Hom(K_0, E^0) \to \Hom(K_0, L^1) \to \Ext_\inj^1(K_0, N) \to \Ext_\inj^1(K_0, E^0) = 0$$

    The second sequence is short because $P_0$ is projective (so $\Hom(P_0, \cdot)$ preserves exactness). 

    Similarly, for $\Hom(P_M, \cdot)$ we have:
    $$0 \to \Hom(M, N) \to \Hom(P_0, N) \to \Hom(K_0, N) \to \Ext_\proj^1(M, N) \to \Ext_\proj^1(P_0, N) = 0$$
    $$0 \to \Hom(M, E^0) \to \Hom(P_0, E^0) \to \Hom(K_0, E^0) \to 0$$
    $$0 \to \Hom(M, L^1) \to \Hom(P_0, L^1) \to \Hom(K_0, L^1) \to \Ext_\proj^1(M, L^1) \to \Ext_\proj^1(P_0, L^1) = 0$$

    Combining these sequences together, we got the following 2D diagram:

    $$
      \begin{tikzcd}[cramped, sep=15pt]
        & 0 \ar[d] & 0 \ar[d] & 0 \ar[d] \\
        0 \ar[r] & \Hom(M, N) \ar[r] \ar[d] & \Hom(M, E^0) \ar[r, "\phi"] \ar[d] &
        \Hom(M, L^1) \ar[r] \ar[d] & \Ext_\inj^1(M, N) \ar[r] & 0 \\
        0 \ar[r] & \Hom(P_0, N) \ar[r] \ar[d, "\alpha"] & \Hom(P_0, E^0) \ar[r, "\sigma"] \ar[d, "\beta"] &
        \Hom(P_0, L^1) \ar[r] \ar[d, "\gamma"] & 0 \\
        0 \ar[r] & \Hom(K_0, N) \ar[r] \ar[d] & \Hom(K_0, E^0) \ar[r, "\tau"] \ar[d] &
        \Hom(K_0, L^1) \ar[r] \ar[d] & \Ext_\inj^1(K_0, N) \ar[r] & 0 \\
        & \Ext_\proj^1(M, N) \ar[d] & 0 & \Ext_\proj^1(M, L^1) \ar[d] \\
        & 0 & & 0
      \end{tikzcd}
    $$

    By Snake lemma, there is an exact sequence
    $$(\ker \alpha \to )\ker \beta \to \ker \gamma \to \coker \alpha \to \coker \beta( \to \coker \gamma)$$
    and this reads
    $$\Hom (M, E^0) \xto{\phi} \Hom(M, L^1) \to \Ext_\proj^1(M, N) \to 0$$

    Thus $\Ext_\proj^1(M, N) \cong \coker \phi \cong \Ext_\inj^1(M, N)$.\\
    (From now on, we don't need to distinguish $\proj / \inj$ for $\Ext^1$ !)

    \vspace{10pt}

    Since $\sigma$ is onto, $\im \gamma = \im (\gamma \circ \sigma)$.
    Similarly, $\im \tau = \im (\tau \circ \beta)$.

    By the commutativity of the diagram, $\im \gamma = \im \tau$, so
    $$\Ext^1(K_0, N) \cong \coker \gamma = \Hom(K_0, L^1)/\im \gamma
    \cong \coker \tau \cong \Ext^1(M, L^1).$$

    Write $K_{-1} \defeq M, L^0 \defeq N$, then $\Ext^1(K_0, L^0) = \Ext^1(K_{-1}, L^1)$ ($\star$).

    Similarly, from the exact sequences
    $$ 0 \to K_j \to P_j \to K_{j-1} \to 0 $$
    $$ 0 \to L^i \to E^i \to L^{i+1} \to 0 $$,
    we can obtain $\Ext^1(K_j, L^i) \cong \Ext^1(K_{j-1}, L^{i+1})$ for $i, j \geq 0$.

    Now, observe that
    $$0 \to L^{n-1} \to E^{n-1} \xto{d_{n-1}} E^n \xto{d_n} \cdots$$
    is an injective resolution of $L^{n-1}$, and
    $\Ext^1(M, L^{n-1}) \cong \ker \ob{d_n} / \im \ob{d_{n-1}} \cong \Ext_\inj^n(M, N)$.

    Similarly, for projective resolution we have $\Ext^1(K_{n-2}, N) \cong \Ext_\proj^n(M, N)$.

    Finally, by ($\star$),  
    $$\Ext_\inj^n(M, N) \cong \Ext^1(K_{-1}, L^{n-1}) \cong \Ext^1(K_0, L^{n-2}) \cong \cdots
    \cong \Ext^1(K_{n-2}, L^0) \cong \Ext_\proj^n(M, N).$$

  \end{proof}
\end{theorem}

\begin{definition}[Tor functor]
  Let $M, N \in \Mod_R$, and $P_\bullet \to M \to 0$ be a projective resolution of $M$, 
  similar to the Ext case, for $n \ge 0$ we can define
  $$\Tor_n(M, N) = H_n(P_M \Ot N).$$
\end{definition}

\begin{fact}
  By Horseshoe lemma, short exact sequence $0 \to M_1 \to M_2 \to M_3 \to 0$
  induces a long exact sequence:
  $$\cdots \to \Tor_1(M_1, N) \to \Tor_1(M_2, N) \to \Tor_1(M_3, N) \to 
  M_1 \Ot N \to M_2 \Ot N \to M_3 \Ot N \to 0$$
\end{fact}

\begin{prop}
  If $M$ is flat, then $\Tor_n(M, N) = 0$ for $n > 0, N \in \Mod_R$.
  \begin{proof}
    Since $0 \to M \xto{\id_M} M \to 0$ is a flat resolution of $M$. 
  \end{proof}
\end{prop}

\begin{theorem}[Tor for flat resolutions]
  Let $U_\bullet \to M \to 0$ be a flat resolution of $M$, then for $n \ge 0$,
  $$\Tor_n(M, N) \cong H_n(U_M \Ot N).$$
  \begin{proof}
    $$
      \begin{tikzcd}[cramped, sep=10pt]
        \cdots \ar[r] & U_2 \ar[rr] \ar[rd] && U_1 \ar[rr] \ar[rd] && U_0 \ar[r] & M \ar[r] & 0 \\
                      && W_1 \ar[ru] \ar[rd] && W_0 \ar[ru] \ar[rd] \\
                      & 0 \ar[ru] && 0 \ar[ru] && 0
      \end{tikzcd}
    $$

    $$H_\bullet(U_M \Ot N) : \cdots \to U_2 \Ot N \xto{d_2 \Ot \id} 
    U_1 \Ot N \xto{d_1 \Ot \id} U_0 \Ot N \to 0$$

    \begin{itemize}
      \item $n=0$: \\ 
        Since tensor is right exact, 
        $U_1 \Ot N \xto{d_1 \Ot \id} U_0 \Ot N \xto{\alpha \Ot \id} M \Ot N \to 0$
        is exact.
        Hence
        $$H_0(U_M \Ot N) = (U_0 \Ot N) / \im(d_1 \Ot \id) = (U_0 \Ot N) / \ker(\alpha \Ot \id)
        \cong M \Ot N$$

        Any projective resolution also has this property, so $\Tor_0(M, N) = H_0(U_M \Ot N)$.

      \item $n=1$: \\
        $0 \to W_0 \to U_0 \to M \to 0$ induces a long exact sequence:
        $$\cdots \to \Tor_1(W_0, N) \to 0 \to \Tor_1(M, N) 
        \to W_0 \Ot N \xto{i \Ot 1} U_0 \Ot N \to M \Ot N \to 0$$

        where $\Tor_1(U_0, N) = 0$ because $U_0$ is flat.
        We can see that $\Tor_1(M, N) \cong \ker(i \Ot 1)$.

    $$
      \begin{tikzcd}[cramped, sep=15pt]
        U_2 \Ot N \ar[rr, "d_2 \Ot 1"] \ar[rd, "\beta' \Ot 1"] && 
        U_1 \Ot N \ar[rr, "d_1 \Ot 1"] \ar[rd, "\alpha' \Ot 1"] 
        && U_0 \Ot N\\
        & W_1 \Ot N\ar[ru, "j \Ot 1"] \ar[rd] && 
        W_0 \Ot N \ar[ru, "i \Ot 1"] \ar[rd] \\
        && 0 && 0
      \end{tikzcd}
    $$

    Since $\alpha' \Ot 1$ is onto, $W_0 \Ot N \cong (U_1 \Ot N) / \ker (\alpha' \Ot 1)$.
    Also, $x \in \ker(d_1 \Ot 1) \iff (d_1 \Ot 1)(x) = 0 \iff (\alpha' \Ot 1)(x) \in \ker (i \Ot 1)$,
    so $\ker (i \Ot 1) = (\alpha' \Ot 1)( \ker (d_1 \Ot 1)) \cong \ker (d_1 \Ot 1) / \ker(\alpha' \Ot 1)$. \\
    ($\alpha' \Ot 1$ can be considered a quotient map, then 
    $\ker (d_1 \Ot 1)$ descends to $\ker (d_1 \Ot 1) / \ker (\alpha' \Ot 1)$.)

    Now, in the diagram $W_1 \Ot N \to U_1 \Ot N \to W_0 \Ot N \to 0$ exact, so 
    $\ker (\alpha' \Ot 1) = \im (j \Ot 1)$. But $\beta' \Ot 1$ is onto, thus 
    $\im (j \Ot 1) = \im (d_2 \Ot 1)$. 

    Finally, 
    $$\Tor_1(M, N) \cong \ker(i \Ot 1) \cong \ker (d_1 \Ot 1) / \ker(\alpha' \Ot 1)
    = \ker (d_1 \Ot 1) / \im(d_2 \Ot 1) = H_1(U_M \Ot N).$$

    \item $n \ge 2$: \\
      Let's see further in the previous long exact sequences:
      $$
        \cdots \to \Tor_2(W_0, N) \to 0 \to \Tor_2(M, N)
        \xto{\sim} \Tor_1(W_0, N) \to 0 \to \Tor_1(M, N) \to \cdots
      $$
      we can see that $\Tor_n(M, N) \cong \Tor_{n-1}(W_0, N)$ for $n \ge 2$.

      Now, 
      $$\cdots U_3 \xto{d_3} U_2 \xto{d_2} U_1 \to W_0 \to 0$$
      is a flat resolution of $W_0$, and its homology is 
      $H_{n-1}(U_{W_0} \Ot N) = \ker(d_n \Ot 1) / \im(d_{n-1} \Ot 1) = H_n(U_M \Ot N)$.

      By induction, assume it's true for $n-1$, then
      $$
        H_n(U_M \Ot N) = H_{n-1}(U_{W_0} \Ot N) \cong \Tor_{n-1}(W_0, N) \cong \Tor_n(M, N).
      $$
      \qedhere
    \end{itemize}
  \end{proof}
\end{theorem}

\begin{example}
  $R = \Zb, M = \Zb/m\Zb$ with $m \ge 2$. Then
  $$P : 0 \to \Zb \xto{m} \Zb \xto{/m\Zb} \Zb/m\Zb \to 0$$
  is a projective resolution of $\Zb/m\Zb$. So for any $N \in \Mod_\Zb$,
  $$H^n(\Hom_\Zb(P_{\Zb/m\Zb}, N)) : 0 \to \Hom_\Zb(\Zb, N) \xto{\ob{m}} \Hom_\Zb(\Zb, N) \to 0,$$
  \begin{eqnarray*}
    \Ext_\Zb^0(\Zb/m\Zb, N) &=& \Hom_\Zb(\Zb/m\Zb, N) \cong {}_m N \defeq \{a \in N \mid ma = 0\}\\
    \Ext_\Zb^1(\Zb/m\Zb, N) &\cong& N / mN \\
    \Ext_\Zb^n(\Zb/m\Zb, N) &=& 0\ \ (\textrm{for } n \ge 2)
  \end{eqnarray*}
\end{example}

\begin{example}
  $\Qb = \Zb_{\langle 0 \rangle}$ is a localization, thus a flat $\Zb$ module.
  Then
  $$U : 0 \to \Zb \xto{i} \Qb \to \Qb/\Zb \to 0$$
  is a flat resolution of $\Qb/\Zb$. For $G \in \Mod_\Zb$ (i.e. an abelian group),
  $$H_n(G \Ot U_{\Qb/\Zb}) : 0 \to G \Ot \Zb \xto{\id \Ot i} G \Ot \Qb \to 0$$
  \begin{eqnarray*}
    \Tor_0(G, \Qb/\Zb) &\cong& G \Ot \Qb/\Zb \\
    \Tor_1(G, \Qb/\Zb) &=& \ker(\id \Ot i) \cong t(G) \defeq
    \set{a \in G \mid ma = 0 \textrm{ for some } m \in \Nb} \\
    \Tor_n(G, \Qb/\Zb) &=& 0\ \ (\textrm{for } n \ge 2)
  \end{eqnarray*}
\end{example}

\begin{definition}
  Let $M$ be a left $R$-module, then define
  $M^* \defeq \Hom_\Zb(M, \Qb/\Zb)$
  as a right $R$-module by 
  $$\deffunc{fr}{M}{\Qb/\Zb}{x}{f(rx)}$$
\end{definition}

\begin{fact} \mbox{}
  \begin{enumerate}
    \item $\Qb/\Zb$ is injective.
    \item $A = 0 \iff A^* = 0$.
    \item $B \toone C \iff C^* \onto B^*$.
  \end{enumerate}

  \begin{proof} \mbox{}
    \begin{enumerate}
      \item 
        For $m \in \Zb \setminus \set{0}$, $m(\Qb/\Zb) = \Qb/\Zb$ by 
        $m(\frac{a}{mb} + \Zb) \mapsfrom \frac{a}{b} + \Zb$, so $\Qb/\Zb$ is divisible.
        But $\Zb$ is a PID, so $\Qb/\Zb$ is injective.

      \item ($\Rightarrow$)
        $A^* = \Hom_\Zb(0, \Qb/\Zb) = 0$.

        ($\Leftarrow$)
        If $A \neq 0$, then $\exists a \in A, a \neq 0$, so $0 \to \Zb a \xto{i} A$ is an inclusion.

        Since $\Zb a$ is a cyclic abelian group, there is a nonzero $g : \Zb a \to \Qb/\Zb$.
        (If $\Zb a \cong \Zb/m\Zb$, let $g : a \mapsto \frac{1}{m}$; 
        if $\Zb a \cong \Zb$, let $g : a \mapsto \frac{1}{2}$.)

        But $\Qb/\Zb$ is injective, so $\exists f : A \to \Qb/\Zb$ (i.e.
        $f \in A^*$), and $f \circ i = g \neq 0$ so $f \neq 0$, thus $A^* \neq 0$.

        $$
        \begin{tikzcd}
          0 \ar[r] & \Zb a \ar[r, "i"] \ar[d, "g"] & A \ar[ld, "\exists f"] \\
                   & \Qb/\Zb 
        \end{tikzcd}
        $$

      \item 
        Since $\Qb/\Zb$ is injective, $\Hom(\cdot, \Qb/\Zb)$ is exact.
        Let $0 \to \ker f \to B \xto{f} C$ exact, applying $\Hom(\cdot, \Qb/\Zb)$
        results in
        $C^* \xto{f^*} B^* \to (\ker f)^* \to 0$ exact.
        Thus $\coker f^* = (\ker f)^*$.

        By 2., $B \toone C \iff \ker f = 0 \iff (\ker f)^* = 0 \iff \coker f^* = 0 
        \iff C^* \onto B^*$.
    \end{enumerate}
  \end{proof}
\end{fact}

\begin{prop}
  Let $M$ be an $R$-module, then TFAE
  \begin{enumerate}
    \item $M$ is flat.
    \item $M^*$ is injective (as a $R$-module).
    \item $\Tor_1(R/I, M) = 0$ for all ideal $I \subseteq R$.
    \item $I \Ot_R M \cong IM$ for all ideal $I \subseteq R$.
  \end{enumerate}

  \begin{proof} \mbox{}
    \begin{itemize}
      \item $3. \iff 4.$ \\
        For any ideal $I \subseteq R$,
        $0 \to I \xto{i} R \xto{q} R/I \to 0$
        is exact.
        This induces a long exact sequence:
        $$\Tor_1(R, M) \to \Tor_1(R/I, M) \to I \Ot_R M \xto{i \Ot \id} R \Ot_R M \xto{q \Ot \id} R/I \Ot_R M \to 0$$
        \begin{itemize}
          \item $\Tor_1(R, M) = 0$ since $R$ is a flat $R$-module.
          \item $R \Ot_R M \cong M$.
          \item $R/I \Ot_R M \cong M/IM$ by $(r + I) \Ot a \mapsto (ra + IM)$.
        \end{itemize}
        So we have
        $$0 \to \Tor_1(R/I, M) \to I \Ot_R M \xto{i'} M \xto{q'} M/IM \to 0$$
        exact, with $q' : M \to M/IM$ being exactly the quotient map (one can check
        that $q \Ot \id \cong q'$).

        Now it's clear that $\Tor_1(R/I, M) = 0 \iff I \Ot_R M \cong \ker(q') \cong IM$.

        (The reverse direction requires $I \Ot_R M \cong IM$ being the natural isomorphism
        $r \Ot b \mapsto rb$, so $i' : IM \to M$ can then be the natural inclusion.)

      \item $1. \iff 2.$ \\
        Let $0 \to N' \xto{f} N$, then $\Hom_R(N, M^*) \xto{\ob{f}} \Hom_R(N', M^*)$.

        By the adjoint relation, 
        $$\Hom_R(N, M^*) = \Hom_R(N, \Hom_\Zb(M, \Qb/\Zb))
        \cong \Hom_\Zb(N \Ot_R M, \Qb/\Zb) = (N \Ot_R M)^*,$$
        we have another map $(N \Ot_R M)^* \xto{(f \Ot \id)^*} (N' \Ot_R M)^*$
        isomorphic to the previous one, with its unstarred map
        $N' \Ot_R M \xto{f \Ot \id} N \Ot_R M$.

        Now, $M^*$ is injective $\iff \ob{f}$ is surjective $\forall N, N' \iff (f \Ot \id)^*$ 
        is surjective $\forall N, N' \iff f \Ot \id$ is injective $\forall N, N' \iff M$ is flat.

      \item $2. \iff 4.$ \\
        Similar to the previous section, by Baer's criterion, \\ 
        \begin{eqnarray*}
          M^*\textrm{ is injective} &\iff& 
          \Hom_R(R, M^*) \onto \Hom_R(I, M^*), \forall I \subseteq R \\
          &\iff& (R \Ot_R M)^* \onto (I \Ot_R M)^*, \forall I \subseteq R \\
          &\iff& I \Ot_R M \toone R \Ot_R M \cong M, \forall I \subseteq R \\
          &\iff& I \Ot_R M \cong IM, \forall I \subseteq R.
        \end{eqnarray*}

        Similarly, this requires the isomorphism of $I \Ot_R M \cong IM$ be 
        natural (the following $f$).

        The map $\deffunc{f}{I \Ot_R M}{IM}{r \Ot a}{ra}$ is always onto, but may not
        be 1-1. If it is, $I \Ot_R M \cong IM$.
    \end{itemize}
  \end{proof}
\end{prop}

\begin{prop}
  For $I, J \subseteq R$ being ideals, then $\Tor_1(R/I, R/J) \cong (I \cap J) / IJ$.
  \begin{proof}
    $0 \to I \xto{i} R \to R/I \to 0$ induces a long exact sequence
    $$0 \to \Tor_1(R/I, R/J) \to I \Ot_R R/J \xto{i \Ot 1}
    R \Ot_R R/J \to R/I \Ot_R R/J \to 0,$$
    where $\Tor_1(R, R/J) = 0$ since $R$ is flat.

    Also $I \Ot_R R/J \cong I/IJ, R \Ot_R R/J \cong R/J$, so we have
    $I/IJ \xto{i'} R/J$ with $\Tor_1(R/I, R/J) \cong \ker(i \Ot 1) \cong \ker i'$.

    But $\deffunc{i'}{I/IJ}{R/J}{x+IJ}{x+J}$, so $\ob{x} \in \ker i' \iff
    x \in I \textrm{ and } x \in J \iff x \in I \cap J$, hence
    $\ker i' \cong (I \cap J) / IJ$.
  \end{proof}
\end{prop}
