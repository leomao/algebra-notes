%! TEX root=../main.tex
\begin{definition}
  The {mapping cone} of a chain map $f$ between two chain $X^\bullet \xto{f} Y^\bullet$
  is defined as a chain with $\cone(f)^n = X^{n+1} \oplus Y^n$,
  and the chain map is defined as
  \[ \begin{tikzcd}[row sep=0, ampersand replacement=\&]
      d_{\cone(f)} : \&[-3em] \cone(f)^n = X^{n+1} \oplus Y^n \ar[r] \& \cone(f)^{n+1} = X^{n+2} \oplus Y^{n+1} \\
      \&[-3em] (x_{n+1}, y_n) \ar[r, mapsto, "{\begin{pmatrix} -d_X & 0 \\ f & d_Y \end{pmatrix}}"']
      \& \big(-d_X(x^{n+1}), f(x^{n+1}) + d_Y(y_n)\big) \\
    \end{tikzcd} \]
\end{definition}

It is easy to see that $d_{\cone(f)}^2 = 0$.

\begin{prop}
  Suppose that $f : X^\bullet \to Y^\bullet$ is a chain map,
  then there is a short exact sequence
  \[ \begin{tikzcd}[row sep=0pt]
      0 \ar[r] & Y \ar[r] & \cone(f) \ar[r] & X[+1] \ar[r] & 0 \\
      & y \ar[mapsto, r] & (0, y) & & \\
      & & (x, y) \ar[mapsto, r] & x & \\
    \end{tikzcd} \]

  \begin{proof}
    It is easy to see that the rows are exact.
    Tracing the diagram shows that the diagram commutes.
  \end{proof}
\end{prop}

\begin{coro}
  There exists a long exact sequence of homology:
  \[ \cdots \to H^{m}(Y^\bullet) \to H^{m}(\cone(f)) \to H^{m+1}(X^\bullet)
    \xto{\delta} H^{m+1}(Y^\bullet) \to H^{m+1}(\cone(f)) \to \cdots \]
  Where the connecting homomorphism $\delta = f^*$.

  \begin{proof}
    Tracing the diagram below as in the snake lemma,
    \[ \begin{tikzcd}[cramped]
        & X^{m} \oplus Y^{m-1} \ar[r] \ar[d] & X^{m} \ar[d] \\
        Y^{m} \ar[r] & X^{m+1} \oplus Y^{m} \ar[r]  & X^{m+1} \\
    \end{tikzcd} \]
  Suppose $\bar{x} \in H^m(X^\bullet)$, then $d_X(x) = 0$,
  so $d (x, 0) = (-d_X(x), f(x)) = (0, f(x))$, which implies
  $f(x) :: Y^m \mapsto d (x, 0) :: X^{m+1} \oplus Y^m$, then
  $\delta(\bar{x}) = \ob{f(x)}$, so $\delta = f^*$.
  \end{proof}
\end{coro}

\begin{coro} \label{coro:cone-exact-iff-quasi-isomorphic}
  $\cone(f)$ acyclic (exact) $\iff$ $f$ quasi-isomorphic.

  \begin{proof}
    Directly by the exact sequence
    \[ H^{m-1}(\cone(f)) \to H^m(X^\bullet) \to H^m(Y^\bullet) \to H^{m}(\cone(f)) \]
  \end{proof}
\end{coro}

Notice that $X[-k]$ is defined as $X[-k]^n = X^{n-k}$ with $d_{X[-k]} = (-1)^k d_X$ below.

\begin{theorem}
  Let $\Ac$ be an abelian category and $K(\Ac)$ be the homotopy category.
  Then the class of quasi-isomorphisms are localizing.

  \begin{proof}
    We check that:
    \begin{enumerate}
      \item It is closed under composition:
        If $f, g$ are quasi-isomorphic, then $(fg)^* = f^* g^*$ is
        a isomorphism since both $f^*, g^*$ are, thus $fg$ is quasi-isomorphic.
      \item The diagram could be completed:
        \[ \begin{tikzcd}
            \exists W^\bullet \ar[r, dashed] \ar[d, dashed] &
            Z^\bullet \ar[d, "g:\ \text{q-iso}"] \\
            X^\bullet \ar[r, "f"] & Y^\bullet
          \end{tikzcd} \]

        Consider the following diagram:
        \[ \begin{tikzcd}[column sep=2.5cm, row sep=1.5cm]
            \cone(\pi f)[-1] \ar[r, "k", "{(x_n, z_n, y_{n-1}) \mapsto x_n}"']
            \ar[d, "{h[-1]}", "{(x_n, z_n, y_{n-1}) \mapsto z_n}"']
            &
            X^\bullet \ar[r, "\pi f"] \ar[d, "f"] &
            \cone(g) \ar[d, equal]  \\
            Z^\bullet \ar[r, "k", "{z_n \mapsto g(z_n)}"'] &
            Y^\bullet \ar[r, "\pi", "{y_n \mapsto (0, y_n)}"'] &
            \cone(g)
          \end{tikzcd} \]
        Where $\cone(\pi f)^n \cong X^{n+1} \oplus \cone(g)^{n} \cong X^{n+1} Z^{n+1} Y^{n}$.

        We claim that $fk \simeq g h[-1]$. Since $(fk - gh[-1])(x_n, z_n, y_{n-1})
        = f(x_n) + g(z_n)$. Define
        \[ \begin{tikzcd}[row sep=0]
            \varphi: &[-3em] \cone(\pi f)[-1]^n = \cone(\pi f)^{n-1} \ar[r] & Y^{n-1} \\
                     &[-3em] (x_n, z_n, y_{n-1}) \ar[r, mapsto] & -y_{n-1}
          \end{tikzcd} \]
        Then
        \begin{align*}
          \varphi d_{C(\pi f)[-1]}(x_n, (z_n, y_{n-1}))
          &= \varphi (d(x_n), - \pi f(x_n) - d(z_n, y_{n-1})) \\
          &= \varphi (d(x_n), - (0, f(x_n)) - (d(z_n), g(z_n) + d(y_{n-1}))) \\
          &= \varphi (d(x_n), - d(z_n),  - f(x_n) - g(z_n) - d(y_{n-1})) \\
          &= f(x_n) + g(z_n) + d(y_{n-1})
        \end{align*}
        and $d_Y \varphi(x_n, z_n, y_{n-1}) = -d(y_{n-1})$,
        so $\varphi d_{C(\pi f)[-1]} + d_Y \varphi = fk - gh[-1]$, thus
        $fk \simeq gh[-1]$.
      \item Let $f : X^\bullet \to Y^\bullet$ in $K(\Ac)$. We shall prove that
        \[ \exists s: Y^\bullet \to Z^\bullet \text{ s.t. } sf = 0
          \iff \exists t: W^\bullet \to X^\bullet \text{ s.t. } ft = 0 \]

        Let $h^i : X^i \to Z^{i-1}$ be a homotopy bewteen $sf$ and $0$.
        Consider the diagram:
        \[ \begin{tikzcd}[column sep=1.5cm, row sep=1.5cm]
            \cone(s)[-1] \ar[rr, leftarrow, "g", "{(f(x_n), -h(x_n)) \mapsfrom x_n}"']
            \ar[d, equal] & &
            X^\bullet \ar[r, leftarrow, "t"] \ar[d, "f"] &
            \cone(g)[-1] = W^\bullet & \\
            \cone(s)[-1] \ar[rr, "{p[-1]}"] & &
            Y^\bullet \ar[r, "s"] &
            Z^\bullet \ar[r, "\pi"] &
            \cone(s)
          \end{tikzcd} \]
        One can easily check that $g$ is a chain map, which congruent
        with the boundary map (because of $h^i$). Now, we have $ft = p[-1]gt$,
        but $gt \simeq 0$ by
        \[ \begin{tikzcd}[row sep=0]
            k_n: & W^n = X^n \oplus Y^{n-1} \oplus Z^{n-2} \ar[r]
            & C(s)[-1]^{n-1} = Y^{n-1} \oplus Z^{n-2} \\
            & (x_n, y_{n-1}, z_{n-2}) \ar[r, mapsto] & (y_{n-1}, z_{n-2})
          \end{tikzcd} \]
        since
        \begin{align*}
          k d (x_n, y_{n-1}, z_{n-2})
          &= k ( -(dx_n, g(x_n) + d(y_{n-1}, z_{n-2}))) \\
          &= k (- dx_n, - (f(x_n), -h(x_n)) + (-d y_{n-1}, g(y_{n-1}) + d z_{n-2})) \\
          &= (-f(x_n) - dy_{n-1}, h(x_n) + g(y_{n-1}) + d z_{n-2}) \\
        \end{align*}
        and $dk(x_n, y_{n-1}, z_{n-2}) = d(y_{n-1}, z_{n-2}) = (d y_{n-1}, -g(y_{n-1}) - d z_{n-2})$.
        Thus $dk + kd = -gt \implies gt \simeq 0$.

        Now, since $s$ is quasi-isomorphic, by corollary~\ref{coro:cone-exact-iff-quasi-isomorphic},
        $\cone(s)$ is acyclic, and thus $t$ is quasi-isomorphic.
        Hence we've find $t$ so that $ft \simeq 0$.
    \end{enumerate}
    We could then define the derived category as $D(\Ac) = K(\Ac)[S^{-1}]$ now.
  \end{proof}
\end{theorem}

\begin{prop}
  The derived category is additive.
  \begin{proof}
    Let $\varphi, \varphi' : X \to Y$ in $D(\Ac)$ with
    $\varphi = [(s, f)], \varphi' = [(s', f')]$, that is, we have the following two diagram
    \[ \begin{tikzcd}[cramped]
        & Z \ar[ld, "s"'] \ar[rd, "f"] & & & & Z' \ar[ld, "s'"'] \ar[rd, "f'"] & \\
        X &   & Y & & X & & Y 
      \end{tikzcd} \]
    using 2. in the definition of localizing,
    exists $U$ so that
    \[ \begin{tikzcd}[cramped]
        \exists U \ar[r, "r'"] \ar[d, "r"] & Z' \ar[d, "s'"] \\
        Z \ar[r, "s"] & X
      \end{tikzcd} \]
    with one of $r, r'$ is guaranteed to be quasi-isomorphic, say $r$.
    But then $H^n(U) \cong H^n(Z) \cong H^n(X) \cong H^n(Z')$
    since $r, s, s'$ are all quasi-isomorphic. This implies $r'$
    is also quasi-isomorphic, so we'll have the new roof for $\varphi$
    \[ \begin{tikzcd}[cramped, column sep=small]
        & & U \ar[ld, "r"'] \ar[lldd, bend right] \ar[dd, "g"] \\
        & Z \ar[ld, "s"'] \ar[rd, "f"] & \\
        X & & Y
      \end{tikzcd} \]
    Similarly, this applies to $\varphi'$. Since $rs = r's'$, we could
    define $\varphi + \varphi' = [(rs, g+g')]$.
  \end{proof}
\end{prop}

\begin{definition}
  Let $\Ac, \Bc$ be abelian categories, $F: \Ac \to \Bc$ be an additive functor.
  \begin{itemize}
    \item Define $D^+(\Ac)$ as a subcategory of $D(\Ac)$
      consist of all the objects (chains) $X^\bullet$ in $D(\Ac)$
      such that $X^i = 0$ for all $i \leq i_0(X^\bullet)$.
      $K^+(\Ac)$ is defined similarly.
    \item Assume that $F$ act on complexes component wise.
      $K^+(F) : K^+(\Ac) \to K^+(\Bc)$.
    \item A triangle in $K^+(\Ac)$ is a diagram of the form
      $\triangle: X^\bullet \to Y^\bullet \to Z^\bullet \to X^\bullet[1]$
    \item $\triangle$ is said to be distinguished if
      \[ \begin{tikzcd}
          X^\bullet \ar[r, "f"] \ar[d, "\rotatebox{90}{\(\sim\)}"] &
          Y^\bullet \ar[r] \ar[d, "\rotatebox{90}{\(\sim\)}"] &
          Z^\bullet \ar[r] \ar[d, "\rotatebox{90}{\(\sim\)}"] &
          X^\bullet[1] \ar[d, "\rotatebox{90}{\(\sim\)}"] \\
          \bar{X}^\bullet \ar[r, "\bar{f}"] &
          \bar{Y}^\bullet \ar[r] &
          \cone(\bar{f}) \ar[r] &
          \bar{X}^\bullet[1]
      \end{tikzcd} \]
    In this case, we denote it as $\tridot$.
  \end{itemize}
\end{definition}

Recall that $\bar{Y}^\bullet \to \cone(\bar{f}) \to \bar{X}^\bullet$
induces a long exact sequence
\[ \cdots \to H^i(\bar{Y}) \to H^i(\cone(\bar{f}))
  \to H^i(\bar{X}[1]) \to H^{i+1}(\bar{Y}) \to \cdots \]

\begin{prop}
  Let $F: \Ac \to \Bc$ be an exact functor, then
  \begin{enumerate}
    \item The exact functor $D^+(F): D^+(\Ac) \to D^+(\Bc)$ exists.
    \item $D^+(F)$ preserves distinguished triangle, (i.e., $\tridot \mapsto \tridot$.)
  \end{enumerate}
  \begin{proof}
    First, we have the following observation:
    \begin{itemize}
      \item $F$ sends acyclic chain to acyclic chain:
        If $X^\bullet$ acyclic, then $X^\bullet$ could be decomposed to
        many short exact sequence:
        \[ 0 \to \ker d_X^i \to X^i \to \ker d_X^{i+1} \to 0 \]
        Apply $F$ we would then get
        \[ 0 \to F(\ker d_X^i) \to F(X^i) \to \ker d_X^{i+1} \to 0 \]
        which we could connect them and get the desired exact sequence
        \[ \cdots \to F(X^{i-1}) \to F(X^i) \to F(X^{i+1}) \to \cdots \]
      \item If $f : X^\bullet \to Y^\bullet$, then $F(f): F(X)^\bullet \to F(Y)^\bullet$,
        and we have $F(\cone(f)) \cong \cone(F(f))$,
        since $F(\cone(f))^n = F(X^{n+1} \oplus Y^n) \cong F(X^{n+1}) \oplus F(Y^n)
        = \cone(F(f))^n$ because $F$ is additive.
        Moreover, the boundary map $d_{\cone(F(f))}$ is
        \[ \begin{pmatrix}
            -F(d_X) & 0 \\ F(f) & F(d_Y)
        \end{pmatrix} = F \begin{pmatrix}
            d_X & 0 \\ f & d_Y
          \end{pmatrix} = d_{F(\cone(f))}
        \]
        Since $F$ transform the object and morphisms consistently, thus
        $F(\cone(f)) \cong \cone(F(f))$. Similarly we have
        $F(\cyl(f)) \cong \cyl(F(f))$.
    \end{itemize}
    Now, return to our proof:
    \begin{enumerate}
      \item If $f$ quasi-isomorphic, then $\cone(f)$ acyclic by
        corollary~\ref{coro:cone-exact-iff-quasi-isomorphic},
        and $F(\cone(f)) \cong \cone(F(f))$ acyclic by the discussion
        above, and finally $F(f)$ acyclic by the same corollary.
        Thus $F$ preserves quasi-isomorphisms.

        Moreover, we could complete the following diagram
        \[ \begin{tikzcd}[column sep=large]
            K^+(\Ac) \ar[r, "K^+(F)"] \ar[d, "Q_A"] &
            K^+(\Bc) \ar[d, "Q_B"] \\
            K^+(\Ac)[S_A^{-1}] \ar[r, "\exists ! D^+(F)"] & K^+(\Bc)[S_B^{-1}]
          \end{tikzcd} \]
        since $F$ send quasi-isomorphisms to quasi-isomorphism
        and by the universal property of the localized category.
        Thus $D^+(f)$ exists.
      \item Apply $D^+(F)$ to the diagram
        \[ \begin{tikzcd}
            X^\bullet \ar[r, "f"] \ar[d, "\rotatebox{90}{\(\sim\)}"] &
            Y^\bullet \ar[r] \ar[d, "\rotatebox{90}{\(\sim\)}"] &
            Z^\bullet \ar[r] \ar[d, "\rotatebox{90}{\(\sim\)}"] &
            X^\bullet[1] \ar[d, "\rotatebox{90}{\(\sim\)}"] \\
            \bar{X}^\bullet \ar[r, "\bar{f}"] &
            \bar{Y}^\bullet \ar[r] &
            \cone(\bar{f}) \ar[r] &
            \bar{X}^\bullet[1]
        \end{tikzcd} \]
        We get
        \[ \begin{tikzcd}
            F X^\bullet \ar[r, "F f"] \ar[d, "\rotatebox{90}{\(\sim\)}"] &
            F Y^\bullet \ar[r] \ar[d, "\rotatebox{90}{\(\sim\)}"] &
            F Z^\bullet \ar[r] \ar[d, "\rotatebox{90}{\(\sim\)}"] &
            F X^\bullet[1] \ar[d, "\rotatebox{90}{\(\sim\)}"] \\
            F \bar{X}^\bullet \ar[r, "F \bar{f}"] &
            F \bar{Y}^\bullet \ar[r] &
            F \cone(\bar{f}) \ar[r] &
            F \bar{X}^\bullet[1]
        \end{tikzcd} \]
        Where the quasi-isomorphisms are preserved by the discussion above.
    \end{enumerate}
  \end{proof}
\end{prop}

\begin{definition}
  A class $R$ of objects in $\Obj \Ac$ is said to
  be adapted to a left exact functor $F$ if
  \begin{enumerate}
    \item It is stable under finite direct sums
    \item $F$ sends acyclic chain in $\Kom^+(R)$ to acyclic chain (in $\Kom^+(\Bc)$).
    \item For each $X \in \Obj \Ac$, exists $I \in R$ such that $0 \to X \to I$.
  \end{enumerate}
\end{definition}

\begin{theorem}
  Let $F$ be a left exact functor, $R$ be a class of objects adpated to $F$.
  Define $S_R$ to be the class of quasi-isomorphisms on $K^+(R)$
  which is localizing since it is stable with the construction
  of mapping cones. Then $D^+(\Ac) \cong K^+(R)[S_R^{-1}]$.

  \begin{proof}
    First we claim that for all $C^\bullet \in D^+(\Ac)$ (which
    we assume $C^i = 0, \, \forall i < 0$),
    There exists $I^\bullet \in K^+(R)$ such that $C^\bullet \cong I^\bullet$.

    We shall construct quasi-isomorphism $t^n : C^n \to I^n$.
    Using induction on $n$:
    \begin{enumerate}
      \item[$n = 0$:] By the definition of adapting class we have
        $0 \to C^0 \xto{t^0} I^0$ for some $I^0$.
        Consider the following diagram:
        \[ \begin{tikzcd}
            & 0 \ar[d] & & \\
            0 \ar[r] & C^0 \ar[r, "d_C"] \ar[d, "t^0"] &
            C^1 \ar[r, "t^1 = c a"] \ar[d, "a", near start] & I^1 \\
            & I^0 \ar[r, "b"'] \ar[rru, "d_I", near start] &
            I^0 \amalg_{C^0} C^1 \ar[ru, "c"'] & \\
            & 0 \ar[ru] & &
        \end{tikzcd} \]
      Where $I^0 \amalg_{C^0} C^1 \triangleq (I^0 \oplus C^1)
      \big/ \Set{(t^0(x), -d_C(x)) \mid x \in C^0}$.

      We shall prove that $t^0$ is an isomorphism
      between $H^0(C^\bullet) = \ker d^1_C$ and
      $H^0(I^\bullet) = \ker d^1_I$. It is
      obviously 1-1 since $0 \to C^0 \xto{t^0} I^0$,
      so we need to check it is onto. For any $y \in \ker d^1_I
      = \ker b$ since $c$ is monomorphism.
      Then $b(y) = 0 \implies (y, 0) = (t^0(x), -d^1_C(x))$
      for some $x \in C^0$. So $y = t^0(x)$
      with $d^1_C(x) = 0 \implies x \in \ker d^1_C$.

    \item [$n = 1$:] Consider the diagram now:
        \[ \begin{tikzcd}
            & & C^1 \ar[r, "d^2_C"] \ar[d] \ar[ld, "t^1"'] &
            C^2 \ar[r, "t^2"] \ar[d, "a'"] & I^2 \\
            I^0 \ar[r, "d_I^1"] &
            I^1 \ar[r, "f"'] \ar[rrru, "d_I^2"] &
            \coker d^1_I \ar[r, "b'"] &
            \coker d^1_I \amalg_{C^1} C^2 \ar[ru, "c'"] \\
            & & 0 \ar[ru] & &
        \end{tikzcd} \]
      Similarly, we shall prove that
      \[ H^1(t) : \ \frac{\ker d^2_C}{\Image d^1_C}
        \xrightarrow{\quad\sim\quad} \frac{\ker d^2_I}{\Image d^1_I} \]
      is an isomorphism.

      \begin{itemize}
        \item 1-1: Let $t^1(x) \in \Image d^1_I$. Since $t^1 = ca$
          and $d^1_I = cb$, there is $y$ such that
          $ca(x) = cb(y)$. Since $c$ 1-1, $a(x) = b(y) \implies (0, x) = (y, 0)$.
          in the pushout, so $(y, -x) = (t^0(z), -d^1_C(z))$ for some $z \in C^0$.
          Thus $x = d^1_c(z) \in \Image d^1_C$.
        \item onto: For each $y \in \ker d_I^2 = \ker b'p$ since $c'$ 1-1.
          Then
          \[ b'p(y) = 0 \implies (y + \Image d^1_I, 0) = (t'(x) + \Image d^1_I, -d^2_C(x))
            \text{ for some } x \in C^1 \]
          in the pushout, so we have $y - t'(x) \in \Image d^1_I$ and
          $ x \in \ker d^2_C$ and thus $H^1(t)(\bar{x}) = \bar{y}$.
      \end{itemize}
    \item[$n > 1$:] Similar as $n = 1$.
    \end{enumerate}
    \medskip

    After proving this claim, we shall show that
    $\Hom_{K^+(R)[S_R^{-1}]}(X^\bullet, Y^\bullet)
    \cong \Hom_{K^+(A)[S^{-1}_A]}(X^\bullet, Y^\bullet)$. \\
    {\bf We will use left roofs instead of right roofs defined before here.}
    \begin{itemize}
      \item 1-1: If $(f, s) \cong (g, t)$ in $K^+(\Ac)[S^{-1}_\Ac]$, then
        \[ \begin{tikzcd}
            & P \ar[d] & & \\
          X \ar[ru] \ar[rd] & T \in \Ac \ar[rr, bend left, "\sim", "\exists u"', sloped, red] &
          Y \ar[lu, "\sim"', sloped, near end] \ar[ld, "\sim", sloped] & {\color{red} Z \in R} \\
          & Q \ar[u] & &
        \end{tikzcd} \]
      where $u$ exists by the previous claim.
      \item onto: Given a roof in $\Ac$
        \[ \begin{tikzcd}[column sep=small]
            & {\color{red} Z \in R} & \\
            & X' \in \Ac \ar[u, "\rotatebox{90}{\(\sim\)}", red] & \\
            X \ar[ru, "f"] & & Y \ar[lu, "s"']
        \end{tikzcd} \]
      We could find a roof in $R$ which is equivalent to it again
      by the previous claim.
    \end{itemize}
  \end{proof}
\end{theorem}

Finally, if $F: \Ac \to \Bc$ is an additive left exact functor, then
we will have $K^+(F): K^+(\Ac) \to K^+(\Bc)$ which
sends acyclic chain in $K^+(R)$ to acyclic chain in $K^+(\Bc)$.
This implies that $K^+(F)$ sends quasi-isomorphism in
$K^+(R)$ to quasi-isomorphism in $K^+(\Bc)$. So
we have the following diagram:
\[ \begin{tikzcd}
  K^+(R) \ar[r, "K^+(F)"] \ar[d, "Q_R"] & K^+(\Bc) \ar[d, "Q_\Bc"] \\
  I^\bullet \in K^+(R)[S_R^{-1}] \ar[r, "{\color{red} \exists ! \bar{F}}"] & D^+(\Bc) \\
    D^+(\Ac) \ar[u, "\rotatebox{90}{\(\sim\)}"] \ar[ru, "RF"']
\end{tikzcd} \]
Where $\bar{F}$ exists by the universal property of localization.
Then the derived functor $RF$ could be defined with $R^i F(C^\bullet) = H^i(RF(C^\bullet))$.

The universal property of $RF$ is as following:
$RF : D^+(\Ac) \to D^+(\Bc)$ is exact and the diagram commutes:
\[ \begin{tikzcd}[column sep=small]
  & D^+(\Ac) \ar[rd, "RF"] & \\
  K^+(\Ac) \ar[ru, "Q_\Ac"] \ar[rd, "K^+(F)"] & & D^+(\Bc) \\
  & K^+(\Bc) \ar[ru, "Q_\Bc"] &
\end{tikzcd} \]
with $\epsilon_F: Q_\Bc \circ K^+(F) \to RF \circ Q_\Ac$ being
a morphism of functors (???). \\
Moreover, if $G : D^+(\Ac) \to D^+(\Bc)$ is another exact functor
with $\epsilon_G : Q_\Bc \circ K^+(F) \to G \circ Q_\Ac$,
then there is an unique $y: RF \to G$ such that
\[ \begin{tikzcd}[column sep=small]
  & Q_\Bc \circ K^+(F) \ar[ld, "\epsilon_F"] \ar[rd, "\epsilon_G"] & \\
  RF \circ Q_\Ac \ar[rr, "y \circ Q_\Ac"] & & G \circ Q_\Ac
\end{tikzcd} \]

Now, one may ask that whether $RG \circ RF \cong R(G \circ F)$,
the answer is no in generally, but there are spectral sequences so that ...
Which is another story then...
