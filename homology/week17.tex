%! TEX root=../main.tex
\begin{definition}
  The {mapping cone} of a chain map $f$ between two chain $X^\bullet \xto{f} Y^\bullet$
  is defined as a chain with $\cone(f)^n = X^{n+1} \oplus Y^n$,
  and the chain map is defined as
  \[ \begin{tikzcd}[row sep=0, ampersand replacement=\&]
      d_{\cone(f)} : \& \cone(f)^n = X^{n+1} \oplus Y^n \ar[r] \& \cone(f)^{n+1} X^{n+2} \oplus Y^{n+1} \\
      \& (x_{n+1}, y_n) \ar[r, mapsto, "{\begin{pmatrix} -d_X & 0 \\ f & d_Y \end{pmatrix}}"']
      \& \big(-d_X(x^{n+1}), f(x^{n+1}) + d_Y(y_n)\big) \\
    \end{tikzcd} \]
\end{definition}

It is easy to see that $d_{\cone(f)}^2 = 0$.

\begin{prop}
  Suppose that $f : X^\bullet \to Y^\bullet$ is a chain map,
  then there is a short exact sequence
  \[ \begin{tikzcd}[row sep=0pt]
      0 \ar[r] & Y \ar[r] & \cone(f) \ar[r] & X[+1] \ar[r] & 0 \\
      & d \ar[mapsto, r] & (0, d) & & \\
      & & (c, d) \ar[mapsto, r] & -c & \\
    \end{tikzcd} \]

  \begin{proof}
    It is easy to see that the rows are exact.
    Tracing the diagram shows that the diagram commutes.
  \end{proof}
\end{prop}

\begin{coro}
  There exists a long exact sequence of homology:
  \[ \cdots \to H^{m}(Y^\bullet) \to H^{m}(\cone(f)) \to H^{m+1}(X^\bullet)
    \xto{\delta} H^{m+1}(Y^\bullet) \to H^{m+1}(\cone(f)) \to \cdots \]
  Where the connecting homomorphism $\delta = f^*$.

  \begin{proof}
    Tracing the diagram below as in the snake lemma,
    \[ \begin{tikzcd}[cramped]
        & X^{m} \oplus Y^{m-1} \ar[r] \ar[d] & X^{m} \ar[d] \\
        Y^{m} \ar[r] & X^{m+1} \oplus Y^{m} \ar[r]  & X^{m+1} \\
    \end{tikzcd} \]
  Suppose $\bar{x} \in H^m(X^\bullet)$, then $d_X(x) = 0$,
  so $d (-x, 0) = (dx, -f(x))$ with $dx = 0$, which implies
  $-f(x) :: Y^m \mapsto d (-x, 0) :: X^{m+1} \oplus Y^m$,
  so $\delta = -f^*$ (Chu Wen Ti)...
  \end{proof}
\end{coro}

\begin{coro} \label{coro:cone-exact-iff-quasi-isomorphic}
  $\cone(f)$ exact $\iff$ $f$ quasi-isomorphic.

  \begin{proof}
    Directly by the exact sequence
    \[ H^{m-1}(\cone(f)) \to H^m(X^\bullet) \to H^m(Y^\bullet) \to H^{m}(\cone(f)) \]
  \end{proof}
\end{coro}

Notice that $X[-k]$ is defined as $X[-k]^n = Z^{n-k}$ with $d_{X[-k]} = (-1)^k d_X$ below.

\begin{theorem}
  Let $\Ac$ be an abelian category and $K(\Ac)$ be the homotopy category.
  Then the class of quasi-isomorphisms are localizing.

  \begin{proof}
    We check that:
    \begin{enumerate}
      \item It is closed under composition:
        If $f, g$ are quasi-isomorphic, then $(fg)^* = f^* g^*$ is
        a isomorphism since both $f^*, g^*$ are, thus $fg$ is quasi-isomorphic.
      \item The diagram could be completed:
        \[ \begin{tikzcd}
            \exists W^\bullet \ar[r, dashed] \ar[d, dashed] &
            Z^\bullet \ar[d, "g:\ \text{q-iso}"] \\
            X^\bullet \ar[r, "f"] & Y^\bullet
          \end{tikzcd} \]

        Consider the following diagram:
        \[ \begin{tikzcd}[column sep=2.5cm, row sep=1.5cm]
            \cone(\pi f)[-1] \ar[r, "k", "{(x_n, z_n, y_{n-1}) \mapsto x_n}"']
            \ar[d, "{h[-1]}", "{(x_n, z_n, y_{n-1}) \mapsto z_n}"']
            &
            X^\bullet \ar[r, "\pi f"] \ar[d, "f"] &
            \cone(g) \ar[d, equal]  \\
            Z^\bullet \ar[r, "k", "{z_n \mapsto g(z_n)}"'] &
            Y^\bullet \ar[r, "\pi", "{y_n \mapsto (0, y_n)}"'] &
            \cone(g)
          \end{tikzcd} \]
        Where $\cone(\pi f)^n \cong X^{n+1} \oplus \cone(g)^{n} \cong X^{n+1} Z^{n+1} Y^{n}$.

        We claim that $fk \simeq g h[-1]$. Since $(fk - gh[-1])(x_n, z_n, y_{n-1})
        = f(x_n) + g(z_n)$. Define
        \[ \begin{tikzcd}[row sep=0]
            \varphi: & \cone(\pi f)[-1]^n = \cone(\pi f)^{n-1} \ar[r] & Y^{n-1} \\
            & (x_n, z_n, y_{n-1}) \ar[r, mapsto] & -y_{n-1}
          \end{tikzcd} \]
        Then
        \begin{align*}
          \varphi d_{C(\pi f)[-1]}(x_n, (z_n, y_{n-1}))
          &= \varphi (d(x_n), - \pi f(x_n) - d(z_n, y_{n-1})) \\
          &= \varphi (d(x_n), - (0, f(x_n)) - (d(z_n), g(z_n) + d(y_{n-1}))) \\
          &= \varphi (d(x_n), - d(z_n),  - f(x_n) - g(z_n) - d(y_{n-1})) \\
          &= f(x_n) + g(z_n) + d(y_{n-1})
        \end{align*}
        and $d_Y \varphi(x_n, z_n, y_{n-1}) = -d(y_{n-1})$,
        so $\varphi d_{C(\pi f)[-1]} + d_Y \varphi = fk - gh[-1]$, thus
        $fk \simeq gh[-1]$.
      \item Let $f : X^\bullet \to Y^\bullet$ in $K(\Ac)$. We shall prove that
        \[ \exists s: Y^\bullet \to Z^\bullet \text{ s.t. } sf = 0
          \iff \exists t: Y^\bullet \to Z^\bullet \text{ s.t. } ft = 0 \]

        Let $h^i : X^i \to Z^{i-1}$ be a homotopy bewteen $sf$ and $0$.
        Consider the diagram:
        \[ \begin{tikzcd}[column sep=1.5cm, row sep=1.5cm]
            \cone(s)[-1] \ar[rr, leftarrow, "g", "{(f(x_n), -h(x_n)) \mapsfrom x_n}"']
            \ar[d, equal] & &
            X^\bullet \ar[r, leftarrow, "t"] \ar[d, "f"] &
            \cone(g)[-1] = W & \\
            \cone(s)[-1] \ar[rr, "{p[-1]}"] & &
            Y^\bullet \ar[r, "s"] &
            Z^\bullet \ar[r, "\pi"] &
            \cone(s)
          \end{tikzcd} \]
        We have $ft = p[-1]gt$, but $gt \simeq 0$ by
        \[ \begin{tikzcd}[row sep=0]
            k_n: & W^n = X^n \oplus Y^{n-1} \oplus Z^{n-2} \ar[r]
            & C(s)[-1]^{n-1} = Y^{n-1} \oplus Z^{n-2} \\
            & (x_n, y_{n-1}, z_{n-2}) \ar[r, mapsto] & (y_{n-1}, z_{n-2})
          \end{tikzcd} \]
        since
        \begin{align*}
          k d (x_n, y_{n-1}, z_{n-2})
          &= k ( -(dx_n, g(x_n) + d(y_{n-1}, z_{n-2}))) \\
          &= k (- dx_n, - (f(x_n), -h(x_n)) + (-d y_{n-1}, g(y_{n-1}) + d z_{n-2})) \\
          &= (-f(x_n) - dy_{n-1}, h(x_n) + g(y_{n-1}) + d z_{n-2}) \\
        \end{align*}
        and $dk(x_n, y_{n-1}, z_{n-2}) = d(y_{n-1}, z_{n-2}) = (d y_{n-1}, -g(y_{n-1}) - d z_{n-2})$.
        Thus $dk + kd = -gt \implies gt \simeq 0$.

        Now, since $s$ is quasi-isomorphic, by corollary~\ref{coro:cone-exact-iff-quasi-isomorphic},
        $\cone(s)$ is acyclic, and thus $t$ is quasi-isomorphic (???????, 山山門口是頁).
        Hence we've find $t$ so that $ft \simeq 0$. (????? $h$ 在哪裡用??)
    \end{enumerate}
    We could then define the derived category as $D(\Ac) = K(\Ac)[S^{-1}]$ now.
  \end{proof}
\end{theorem}

\begin{prop}
  The derived category is additive.

  \begin{proof}
    Let $\varphi, \varphi' : X \to Y$ in $D(\Ac)$ with
    $\varphi = [(s, f)], \varphi' = [(s', f')]$, that is, we have the following two diagram
    \[ \begin{tikzcd}[cramped]
        & Z \ar[ld, "s"'] \ar[rd, "f"] & & & & Z' \ar[ld, "s'"'] \ar[rd, "f'"] & \\
        X &   & Y & & X & & Y 
      \end{tikzcd} \]
    using 2. in the definition of localizing,
    exists $U$ so that
    \[ \begin{tikzcd}[cramped]
        \exists U \ar[r, "r'"] \ar[d, "r"] & Z' \ar[d, "s'"] \\
        Z \ar[r, "s"] & X
      \end{tikzcd} \]
    with one of $r, r'$ is guaranteed to be quasi-isomorphic, say $r$.
    But then $H^n(U) \cong H^n(Z) \cong H^n(X) \cong H^n(Z')$
    since $r, s, s'$ are all quasi-isomorphic. This implies $r'$
    is also quasi-isomorphic, so we'll have the new roof for $\varphi$
    \[ \begin{tikzcd}[cramped, column sep=small]
        & & U \ar[ld, "r"'] \ar[lldd, bend right] \ar[dd, "g"] \\
        & Z \ar[ld, "s"'] \ar[rd, "f"] & \\
        X & & Y
      \end{tikzcd} \]
    Similarly this applies to $\varphi'$. Since $rs = r's'$, we could
    define $\varphi + \varphi' = [(rs, g+g')]$.
  \end{proof}
\end{prop}

\begin{definition}
  Let $\Ac, \Bc$ be abelian categories, $F: A \to B$ be an additive functor.
  \begin{itemize}
    \item Define $D^+(\Ac)$ as a subcategory of $D(\Ac)$
      consist of all the objects (chains) $X^\bullet$ in $D(\Ac)$
      such that $X^i = 0$ for all $i \leq i_0(X^\bullet)$.
      $K^+(\Ac)$ is defined similarly.
    \item Assume that $F$ act on complexes component wise.
      $K^+(F) : K^+(\Ac) \to K^+(\Bc)$.
    \item A triangle in $K^+(\Ac)$ is a diagram of the form
      $\triangle: X^\bullet \to Y^\bullet \to Z^\bullet \to X^\bullet[1]$
    \item $\triangle$ is said to be distinguished if
      \[ \begin{tikzcd}
          X^\bullet \ar[r, "f"] \ar[d, "\rotatebox{90}{\(\sim\)}"] &
          Y^\bullet \ar[r] \ar[d, "\rotatebox{90}{\(\sim\)}"] &
          Z^\bullet \ar[r] \ar[d, "\rotatebox{90}{\(\sim\)}"] &
          X^\bullet[1] \ar[d, "\rotatebox{90}{\(\sim\)}"] \\
          \bar{X}^\bullet \ar[r, "\bar{f}"] &
          \bar{Y}^\bullet \ar[r] &
          \cone(\bar{f}) \ar[r] &
          \bar{X}^\bullet[1]
      \end{tikzcd} \]
    In this case, we denote it as $\tridot$.
  \end{itemize}
\end{definition}
