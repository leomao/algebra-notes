%! TEX root=../main.tex
\section{Introduction to Homological Algebra}

\subsection{Projective, Injective and Flat modules (week 14)}

\begin{definition} \mbox{}
  \begin{itemize}
    \item $M \in \Mod_R$ is {\bf projective} if $\Hom(M, \cdot)$ preserves the
      {\it right} exactness.
    \item $N \in \Mod_R$ is {\bf injective} if $\Hom(\cdot, N)$ preserves the
      {\it right} exactness.
    \item $M \in \Mod_R$ is {\bf flat} if $M\otimes \cdot$ preserves the
      {\it left} exactness.
  \end{itemize}
\end{definition}

\begin{fact} \mbox{}
  \begin{itemize}
    \item $M$ is projective $\iff$
      $\begin{tikzcd}[cramped, sep=15pt]
         & M \ar[dl, "\exists \tilde f"'] \ar[d, "f"] \\
        M_2 \ar[r] & M_3 \ar[r] & 0
      \end{tikzcd}$
    \item $N$ is injective $\iff$
      $\begin{tikzcd}[cramped, sep=15pt]
        0 \ar[r] & M_1 \ar[d, "g"'] \ar[r] & M_2 \ar[dl, "\exists \tilde g"] \\
         & N
      \end{tikzcd}$
    \item free $\implies$ projective:
      If $X = \Set{ x_i \given i \in \Lambda}$ and $f: x_i \mapsto a_i$.
      Since $\beta$ onto, exists $b_i$ so that $\beta(b_i) = a_i$.
      we can then set $\tilde f: x_i \mapsto b_i$ by the universal
      property of free module.
      \[\begin{tikzcd}[cramped, sep=15pt]
        & F(X) \ar[dl, "\exists \tilde f"'] \ar[d, "f"] \\
        M_2 \ar[r, "\beta"] & M_3 \ar[r] & 0
      \end{tikzcd}\]
    \item free $\implies$ flat:
      Let $F \cong R^{\oplus \Lambda}$ be a free module,
      and $M_1, M_2$ be two modules such that $0 \to M_1 \to M_2$.
      Since $R \otimes_R M \cong M$, we have
      \begin{alignat*}{3}
        & 0 \to M_1 \to M_2 & \text{ exact } \\
        \implies & 0 \to R \otimes M_1 \to R \otimes M_2 & \text{ exact } \\
        \implies & 0 \to \bigoplus_{i \in \Lambda} R \otimes M_1 \to
        \bigoplus_{i \in \Lambda} R \otimes M_2 \ & \text{ exact } \\
        \stackrel{\text{(a)}}{\implies} & 0 \to R^{\oplus \Lambda} \otimes M_1 \to
        R^{\oplus \Lambda} \otimes M_2 & \text{ exact } \\
        \implies & 0 \to F \otimes M_1 \to F \otimes M_2 & \text{ exact }
      \end{alignat*}
      Where (a) is by the fact that $(A \oplus B) \otimes C \cong
      (A\otimes C) \oplus (B \otimes C)$. Thus $F$ flat.

    \item If $S$ is a multiplication closed set in $R$ with $1 \in S$, then
      \[ 0 \to M \to N \to L \to 0 \implies 0 \to M_S \to N_S \to L_S \to 0. \]
      We know that $M_S \cong R_S \otimes_R M$. So $R_S$ is a flat $R$-module.
      e.g. $\Qb$ is a flat $\Zb$-module.
  \end{itemize}
\end{fact}

For any $M \in \Mod_R$, a projective module $N$ such that $N \to M \to 0$ could
be easily found: Simply let $N = F$, a free module on the generating set of $M$.

Now we shall ask for any module $M$, does there exist $N \in \Mod_R$ such that
$N$ is injective and $0 \to M \to N$?

\begin{theorem}[Baer's criterion] \label{thm:boers-criterion}
  $N$ is injective $\iff \forall I \subset R$, and a homomorphism
  $f$, there exists a homomorphism $h$
  such that the following diagram commutes:

  \[ \begin{tikzcd}[cramped, sep=15pt]
    0 \ar[r] & I \ar[d, "f"'] \ar[r] & R \ar[dashed, dl, "\exists h"] \\
     & N
    \end{tikzcd}
  \]
  \begin{proof}
    ``$\Rightarrow$'': See $I$ as an $R$ module, then it is obivous by the
    definition of injective module.

    ``$\Leftarrow$: Consider the following diagram:
    \[
      \begin{tikzcd}[cramped, sep=15pt]
      0 \ar[r] & M_1 \ar[d, "g"] \ar[r] & M_2  \\
        & N
      \end{tikzcd}
    \]
    Let $S \triangleq \Set{ (M, \rho) \mid M_1 \subseteq M \subseteq M_2
      \text{ and } \rho : M \to N \text{ extends } g} \neq \varnothing$
    since $(M_1, g) \in S$.

    By the routinely proof using Zorn's lemma, exists a maximal element $(M^*, \mu) \in S$.

    We claim that $M^* = M_2$.
    If not, pick $a \in M_2 \setminus M^*$
    and let $M' \triangleq M^* + Ra \supsetneq M^*$,
    $I \triangleq \Set{ r \in R \mid ra \in M^* }$.
    Define $f : I \to N$ with $r \mapsto \mu(ra)$.
    Then we have an extension $h : R \to N$ of $f$.

    Now, let $\mu' :  M' \to N = x + ra \mapsto \mu(x) + h(r)$.
    We shall prove that this map is well-defined:
    If $x_1 + r_1a = x_2 + r_2a$, then $(r_1 - r_2) a = x_2 - x_1 \in M \implies
    r_1 - r_2 \in I$.
    So $h(r_1) - h(r_2) = f(r_1 - r_2) = \mu((r_1 - r_2)a) = \mu(x_2) - \mu(x_1)$,
    which prove $\mu'$ is well defined, and the existence of $\mu'$
    contradicts the fact that $(M^*, \mu)$ is maximal.
  \end{proof}
\end{theorem}

\begin{definition}
  $M$ is {\bf divisible} if $\forall x\in M, r \in R \setminus \set{0}$, there exists
  $y \in M$ such that $x = ry$, i.e. $rM = M \quad \forall r \in R \setminus \set{0}$.
\end{definition}

\begin{prop} \mbox{} \label{prop:injective-and-divisible}
  \begin{enumerate}
    \item Every injective module $N$ over an integral domain is divisible.
      \begin{proof}
        For any $x_0 \in N$ and $r_0 \in R \setminus \{0\}$. Let
        $I = \gen{r_0} \subset R$. As long as $R$ is an integral domain,
        $I \cong R$ as an $R$-module, so the $R$-module homomorphism
        $f : I \to N = r r_0 \mapsto r x_0$ is well-defined.
        Since $N$ is injective, this map extends to $h : R \to N$.
        Let $y_0 \triangleq h(1)$, then $r_0 y_0 = r_0 h(1) = h(r_0) = x_0$.
        Thus $N$ is divisible.
      \end{proof}
    \item Every divisible module $N$ over an PID is injective.
      \begin{proof}
        For any $I \subseteq R$ and a homomorphism $f : I \to N$, if $I = 0$ then
        $h = x \mapsto 0$ is always an extension of $f$.
        So assume $I \neq 0$. Since $R$ is a PID,
        $I = \gen{r_0}$ for some $r_0 \neq 0 \in R$.
        By the fact that $N$ divisible, exists $y_0 \in N$
        such that $r_0 y_0 = x_0 \triangleq f(r_0)$.

        Now we could define $h : R \to N$ by $1 \mapsto y_0$.
        Then $h(r_0) = r_0 h(1) = r_0 y_0 = x_0$, thus
        $h$ is an extension of $f$ and $N$ is injective.
     \end{proof}
     \item If $R$ is a PID, then any quotient $N$ of an injective $R$-module $M$
       is injective.
       \begin{proof}
        By 2., $rM = M$ for any $r \neq 0$, thus $rN = N$ for any $r \neq 0$,
        and hence $N$ is injective.
       \end{proof}
  \end{enumerate}
\end{prop}

\begin{theorem} \label{thm:every-module-is-in-injective-module}
 For any $M \in \Mod_R$, there exists an injective module $N$ containing $M$.
 \begin{proof} $ $
   \begin{enumerate}[label={\bf Case \arabic*:}]
      \item $R = \Zb$. \\
        Let $X = \Set{x_i}_{i \in \Lambda}$ be a generating set for $M$
        and $F$ is free on $X$. Let $f$ be the natural map from $F$
        to $M$. then $M \cong \quot{F}{\ker f}$.

        Define $F' = \bigoplus_{i \in \Lambda} \Qb e_i \supset F$,
        which is obviously a divisible $\Zb$-module.
        Then $M \subseteq F' / \ker f \triangleq M'$,
        where $M'$ is injective by proposition~\ref{prop:injective-and-divisible}.

      \item $R$ arbitrary. \\
        We can regard any $M$ as a $\Zb$-module, then there exists an injective
        module $N_0 \supset M$.
        Now, we have an $R$-module $N \triangleq \Hom_\Zb(R, N_0)$
        with multiplication $rf \triangleq x \mapsto f(x r)$.

        We claim that $N$ is injective. For any $f : M_1 \to N$,
        and a homomorphism $\alpha : M_1 \to M_2$,
        first we can regard $\alpha$ as a $\Zb$-module homomorphism, then
        we define $f' : M_1 \to N_0$ as $x \mapsto f(x)(1)$. Since
        $N_0$ injective (in $\Mod_\Zb$), there exists a $\Zb$-module
        homomorphism $h'$ from $M_2$ to $N_0$.
        \[
          \begin{tikzcd}[contains/.style = {draw=none, 
            "\in" description ,sloped}]
                        & x \ar[d, contains] \ar[ddd, mapsto, bend right] \\
              0 \ar[r]  & M_1 \ar[r,"\alpha"] \ar[d, "f"] & M_2 \ar[ld, 
              "\exists h (?)",dashed] \\
                        & N \ar[r, draw=none, "\cong" description]&  
                        \Hom_\Zb(R, N_0) \\
                        & f(x) \ar[u, contains] \\
            \end{tikzcd}
            \begin{tikzcd}[contains/.style = {draw=none, 
            "\in" description ,sloped}]
                        & x \ar[d, contains] \ar[ddd, mapsto, bend right] \\
              0 \ar[r]  & M_1 \ar[r,"\alpha"] \ar[d, "f'"] & M_2 \ar[ld, 
              "\exists h'"] \\
                        & N_0 \\
                        & f(x)(1) \ar[u, contains] \\
          \end{tikzcd}
        \]
        Now, define
        \[
        \begin{tikzcd}[row sep=0.1]
          h: &[-3em] M_2 \ar[r] & N \\
             &[-3em] y \ar[r, mapsto] & h(y) : &[-3em] R \ar[r] & N_0 \\
             &[-3em] & &[-3em] 1 \ar[r, mapsto] & h'(y) \\
             &[-3em] & &[-3em] r \ar[r, mapsto] & h'(ry) \\
        \end{tikzcd}
        \]
        We check that $h$ is well-defined.
        \begin{itemize}
          \item $h(y) \in \Hom_\Zb(R, N_0)$
            \[
              h(y)(r_1 + r_2) = h'((r_1 + r_2)y) = h'(r_1 y + r_2 y)
              = h'(r_1y) + h'(r_2y) = h(y)(r_1) + h(y)(r_1)
            \]
          \item $h \in \Hom_R(M_2, N)$
            \[
              \begin{aligned}
                h(r_1y_1 + y_2)(r) &= h'(r(r_1y_1 + y_2))
                = h'(rr_1y_1 + ry_2) \\
                &= h'(rr_1y_1) + h'(ry_2) \\
                &= h(y)(rr_1) + h(y_2)(r) \\
                &= (r_1h(y))(r) + h(y_2)(r)
              \end{aligned}
            \]
          \item Show diagram commute $f = h \circ \alpha$.
            Fix $y \in M_1$, then $\forall r \in R$:
            \[
              \begin{aligned}
                (h \circ \alpha)(y)(r) &= h(\alpha(y))(r)  = h'(r\alpha(y))\\
                                      &= h'(\alpha(ry)) = f'(ry) \\
                                      &= f(ry)(1) = rf(y)(1) \\
                                      &= f(y)(r)
              \end{aligned}
            \]
        \end{itemize}
        Thus $N$ injective.

        Now, notice that $\Hom_\Zb(R, \cdot)$ is a left exact functor,
        so $M \toone N_0$ implies $\Hom_\Zb(R, M) \toone \Hom_\Zb(R, N_0)$,
        thus $M \cong \Hom_R(R, M) \toone \Hom_\Zb(R, M) \toone 
        \Hom_\Zb(R,N_0) = N$.
        \qedhere
    \end{enumerate}
 \end{proof}
\end{theorem}

\begin{prop} \label{prop:tfae-of-projective-module}
  TFAE
  \begin{enumerate}
    \item $M$ is projective.
    \item Every exact sequence $0 \to M_1 \to M_2 \to M \to 0$ split.
    \item $\exists M'$ s.t. $M \oplus M' \cong F$: free.
  \end{enumerate}
  \begin{proof} $ $\\
    (1) $\Rightarrow$ (2) : Since $M$ projective, the map $\lambda$ with
    $\beta \circ \lambda = \Id$ exists in the following diagram:\\
    \[
      \begin{tikzcd}
                            & M \ar[d, "\Id"] \ar[dl,"\exists \lambda"swap] \\
        M_2 \ar[r, "\beta"] & M \ar[r]        & 0
      \end{tikzcd}
    \]
    Then $\lambda$ is a lifting, so $M_2 \cong M_1 \oplus M$ and $0 \to M_1 \to M_2 \to M \to 0$ split. \\

    (2) $\Rightarrow$ (3): Let $F$ be a free module on a generating set of $M$,
    and $\beta :: F \to M$ be the natural map,
    then $0 \to \ker \beta \to F \to M \to 0$ split,
    so $F \cong \ker \beta \oplus M$.

    (3) $\Rightarrow$ (1):
    For any $M_2 \to M_3 \to 0$, since $M' \oplus M$ free and thus projective,
    $\lambda'$ exists in the following diagram:
    \[
      \begin{tikzcd}
        0 \ar[r] & M' \ar[r] & M' \oplus M \ar[r, "\pi" swap, 
        yshift=-0.7ex] \ar[d, "\exists \lambda'"]
        & M \ar[r] \ar[d, "f"] \ar[l, "\mu" swap, yshift=0.7ex] & 0 \\
         & & M_2 \ar[r, "\beta"] & M_3 \ar[r] & 0
      \end{tikzcd}
    \]
    Define $\lambda = \lambda' \circ \mu$. Then $\beta \circ \lambda= 
    \beta \circ \lambda' \circ \mu = f \circ \pi \circ \mu = f$.
  \end{proof}
\end{prop}

\begin{prop}
  TFAE
  \begin{enumerate}
    \item $M$ is injective.
    \item Each exact sequence $0 \to M \to M_2 \to M_3 \to 0$ split.
  \end{enumerate}
  \begin{proof}
    (1) $\Rightarrow$ (2): Similar to the projective case, $\mu$ exists
    in the following diagram:
    \[
    \begin{tikzcd}
      0 \ar[r] & M \ar[r, "\alpha"] \ar[d, "\Id"] & M_2 \ar[dl, "\exists
      \mu", dashed] \\
               & M
    \end{tikzcd}
    \]
    So $M_2 = M \oplus M_3$.

    (2) $\Rightarrow$ (1):
    By theorem~\ref{thm:every-module-is-in-injective-module},
    there is a module $N \subset M$ so that $N$ is injective.

    Consider
    $
    \begin{tikzcd}
    0 \ar[r] & M \ar[r, "i", yshift=0.7ex] &  N \ar[r] \ar[l, "\exists \mu", 
      yshift=-0.7ex] & \coker i \ar[r]& 0
    \end{tikzcd}
    $
    split exact and $\mu \circ i = \Id_M$. Since $N$ injective,
    $h'$ exists in the following diagram:
    \[
    \begin{tikzcd}
      0 \ar[r] & M_1 \ar[r, "\alpha"] \ar[d , "f"] \ar[dd, "i 
      \circ f" swap, bend right = 60] & M_2 \ar[ddl, "\exists h'"]\\
               & M \ar[d, xshift=-0.7ex, "i" swap , hook] \\
               & N \ar[u, xshift=0.7ex, "\mu" swap]
    \end{tikzcd}
    \]
    Let $h = \mu \circ h'$, then $h \circ \alpha = \mu \circ h' 
    \circ \alpha = \mu \circ i \circ f = f$.
  \end{proof}
\end{prop}

\begin{prop}
  projective $\implies$ flat.
  \begin{proof}
    Observe that $\bigoplus\limits_{i \in \Lambda} M_i$ is flat if and only if $M_i$ is
    flat for each $i$, since if $0 \to N_1 \xrightarrow{\alpha} N_2$ exact, then
    \[
      \begin{tikzcd} [cong/.style = {draw=none, "\cong" description, sloped}]
        0 \ar[r] & \left( \bigoplus M_i \right) \otimes N_1
        \ar[r, "1 \otimes \alpha"] \ar[d,equal] &
        \left(\bigoplus M_i \right) \otimes N_2 \ar[d, equal] \\
        0 \ar[r] & \bigoplus (M_i \otimes N_1)
        \ar[r, "\oplus (1 \otimes \alpha)"] \ar[d, Leftrightarrow]& 
        \bigoplus (M_i \otimes N_2)\\
        0 \ar[r] & M_i \otimes N_1 \ar[r, "1 \otimes \alpha"] 
        & M_i \otimes N_2 & \forall i \in \Lambda 
      \end{tikzcd}
    \]

    If $M$ is projective, then by proposition~\ref{prop:tfae-of-projective-module}
    $\exists M'$ such that $M \oplus M' \cong F$ is free.
    Since free implies flat, by above, $M$ is flat.
  \end{proof}
\end{prop}

\begin{definition} \mbox{}
  \begin{itemize}
    \item A chain complex $C_\bullet$ of $R$-modules is a sequence and maps:
      \[
        C_\bullet : \dotsm \to C_{n+1} \xto{d_{n+1}} C_n \xto{d_n} C_{n-1}
        \to \dotsm \to C_1 \xto{d_1} C_0 \to 0
      \]
      with $d_n \circ d_{n+1} = 0, \quad \forall n$.
      (i.e. $\Image d_{n+1} \subseteq \ker d_n$)

      Then define
      \begin{itemize}
        \item $Z_n(C_\bullet) \triangleq \ker d_n$ is the $n$-cycle.
        \item $B_n(C_\bullet) \triangleq \Image d_{n+1}$ is the $n$-boundary.
        \item $H_n(C_\bullet) \triangleq \quot{Z_n(C_\bullet)}{B_n(C_\bullet)}$
          is called the $n$-th homology.
      \end{itemize}
    \item A cochain complex $C^\bullet$ of $R$-modules is a sequence and maps:
      \[
        C^\bullet : 0 \to C^0 \xto{d^1} C^1 \to \dotsm \to
        C^{n-1} \xto{d^n} C^n \xto{d^{n+1}} C^{n+1} \to \dotsm
      \]
      with $d^{n+1} \circ d^n = 0, \quad \forall n$.
      (i.e. $\Image d^n \subseteq \ker d^{n+1}$)

      Then define
      \begin{itemize}
        \item $Z^n(C^\bullet) \triangleq \ker d^{n+1}$ is the $n$-cocycle.
        \item $B^n(C^\bullet) \triangleq \Image d^n$ is the $n$-coboundary.
        \item $H^n(C^\bullet) \triangleq \quot{Z^n(C^\bullet)}{B^n(C^\bullet)}$
          is called the $n$-th cohomology.
      \end{itemize}
    \item $\varphi: C_\bullet \to \tilde C_\bullet$ is a chain map if the
      following diagram commutes:
      \[
      \begin{tikzcd}
        \dotsm \ar[r] & C_{n+1} \ar[r, "d_{n+1}"] \ar[d, "\varphi_{n+1}"]
                      & C_n \ar[r, "d_n"] \ar[d, "\varphi_n"]
                      & C_{n-1} \ar[r] \ar[d, "\varphi_{n-1}"]
                      & \dotsm \\
        \dotsm \ar[r] & \tilde C_{n+1} \ar[r, "\tilde{d}_{n+1}"] & \tilde C_n \ar[r, "\tilde{d}_n"]
                      & \tilde C_{n-1} \ar[r] & \dotsm
      \end{tikzcd}
      \]
      Observe that $\varphi_n(\ker d_n) \subseteq \ker \tilde d_n$ and
      $\varphi_n(\Image d_{n+1}) \subseteq \Image \tilde d_{n+1}$.
      This will induce the following maps:
      \[
        \deffunc{\varphi_*}{H_n(C_\bullet)}{H_n(\tilde C_\bullet)}
        {x + B_n(C_\bullet)}{\varphi_n(x) + B_n(\tilde C_\bullet)}
      \]
    \item $f: C_\bullet \to \tilde C_\bullet$ is null homotopic if
      $\exists s_n: C_n \to \tilde C_{n+1}$ s.t.
      $f_n = \tilde d_{n+1} \circ s_n + s_{n-1} \circ d_n, \quad \forall n$.
      \[
      \begin{tikzcd}
        \dotsm \ar[r] & C_{n+1} \ar[r, "d_{n+1}"] \ar[d, "f_{n+1}"]
                      & C_n \ar[r, "d_n"] \ar[d, "f_n"] \ar[ld, red, "s_n" red]
                      & C_{n-1} \ar[r] \ar[d, "f_{n-1}"]  \ar[ld, red, "s_{n-1}" red]
                      & \dotsm \\
        \dotsm \ar[r] & \tilde C_{n+1} \ar[r, "d_{n+1}"] & \tilde C_n \ar[r, "d_n"]
                      & \tilde C_{n-1} \ar[r] & \dotsm
      \end{tikzcd}
      \]

      \begin{prop} \label{prop:null-homotopic-then-fstar-is-zero}
      If $f$ is null homotopic, then $f_* = 0$.
      \begin{proof}
        $f_*(x) = \tilde{d}_{n+1} s_{n}(x) + s_{n-1} d_{n}(x) =
        \tilde{d}_{n+1} s_n (x) \in B_n(\tilde{C}_\bullet) \implies f_*(\bar{x}) = 0$.
      \end{proof}
      \end{prop}
    \item Two chain map $f, g: C_\bullet \to \tilde C_\bullet$ are homotopic
      if $f - g$ is null homotopic. ($f_* = g_*$)
    \item Let $M \in \Mod_R$. A projection resolution of $M$ is an exact sequence:
      \[
        \dotsm \to P_2 \xto{d_2} P_1 \xto{d_1} P_0 \xto{\alpha} M \to 0
      \]
      where $P_i$ is projective for all $i$.

      For any $M$, projection resolution always exists.
      Let $P_0$ be a free module on the generators of $M$.
      We get $P_0 \xto{\alpha} M \to 0$. Similarly, let $P_1$ be
      free on $\ker \alpha$, then we could extend the map to
      $P_1 \to P_0 \to M \to 0$. Continue the process we would
      get a diagram as below, where $K_i$ are the kernels:
      \[
        \begin{tikzcd}[sep=10pt]
          \cdots \ar[r] & P_2 \ar[rr] \ar[rd] && P_1 \ar[rr] \ar[rd] && P_0 \ar[r] & M \ar[r] & 0 \\
                        & & K_1 \ar[ru] \ar[rd] && K_0 \ar[ru] \ar[rd] \\
                        & 0 \ar[ru] && 0 \ar[ru] && 0
        \end{tikzcd}
      \]
  \end{itemize}
\end{definition}

\begin{theorem}[Comparison theorem] \label{thm:comparison-theorem}
  Given two chain as following:
  \[
    \begin{tikzcd}
      \dotsm \ar[r] & P_2 \ar[r, "d_2"] \ar[d, dashed, "\exists f_2"]
                    & P_1 \ar[r, "d_1"] \ar[d, dashed, "\exists f_1"]
                    & P_0 \ar[r, "\alpha"] \ar[d, dashed, "\exists f_0"]
                    & M \ar[r] \ar[d, "f"] & 0
                    & \text{(projective resolution)} \\
      \dotsm \ar[r] & \tilde C_2 \ar[r, "d_2'"]
                    & \tilde C_1 \ar[r, "d_1'"]
                    & \tilde C_0 \ar[r, "\alpha'"]
                    & N \ar[r] & 0
                    & \text{(exact sequence)}
    \end{tikzcd}
  \]
  Then $\exists f_i : P_i \to C_i$ s.t. $\Set{f_i}$ forms a chain map making the
  completed diagram commutes. And any two such chain maps are homotopic.
  \begin{proof}
    Using induction on $n$.

    For $n = 0$, the existence of $f_0$ is guaranteed by the definition of projective module.
    \[ \begin{tikzcd}
        & P_0 \ar[d, "f \circ \alpha"] \ar[ld, dashed, "\exists f_0"'] & \\
        C_0 \ar[r] & N \ar[r] & 0
    \end{tikzcd} \]

    For $n > 0$, we claim that $f_{n-1} d_n (P_n) \subseteq \Image d'_{n}$,
    since $d'_{n-1} f_{n-1} d_n (x) = f_{n-2} d_{n-1} d_n (x) = 0$ and
    by the fact that $C$ is exact, $f_{n-1} d_n (x) \in \ker d'_{n-1} = \Image d'_n$.
    So using the diagram and again by the definition of projective module,
    $f_n$ exists.

    \[ \begin{tikzcd}
        & P_n \ar[d, "f_{n-1} \circ d_n"] \ar[ld, dashed, "\exists f_n"'] & \\
        C_n \ar[r] & \Image d'_n \ar[r] & 0
    \end{tikzcd} \]

    Now, for another chain map $\Set{g_i: P_i \to C_i}$, we shall construct suitable $\Set{s_n}$
    to prove they are homotopic. For $s_{-1} : M \to C_0$ we could simply pick the
    zero map. Again, if we could prove that
    $\Image (g_n - f_n - s_{n-1} d_n) \subset \ker d'_n$, then by
    the definition of projective module, we would obtain $s_n$ with
    \[ d_{n+1} s_n = g_n - f_n - s_{n-1} d_n \implies g_n - f_n = d_{n+1} s_n + s_{n-1} d_n \]
    immediately. For this we calculate $d'_n (g_n - f_n - s_{n-1} d_n)
    = g_{n-1} d_n - f_{n-1} d_n - d'_n s_{n-1} d_n$. Notice that
    $d'_n s_{n-1} = g_{n-1} - f_{n-1} - s_{n-2} d_{n-1}$, and
    with $d_{n-1} d_n = 0$, we get $d'_n (g_n - f_n - s_{n-1} d_n) = 0$.
  \end{proof}
\end{theorem}

\begin{definition}
  Let $M \in \Mod_R$ and $\dotsm \to P_2 \xto{d_2} P_1 \xto{d_1} P_0 \xto{\alpha} M \to 0$
  be a projective resolution of $M$. Fix $N \in \Mod_R$. Applying $\Hom_R(\cdot, N)$ will
  get a complex:
  \[
    0 \to \Hom_R(M, N) \xto{\bar \alpha} \Hom_R(P_0, N) \xto{\bar d_1}
    \Hom_R(P_1, N) \to \dotsm
  \]
  Define
  \begin{itemize}
    \item $\Ext^0_R(M, N) = \ker \bar d_1 = \Image \bar \alpha \cong \Hom_R(M, N)$.
    \item $\Ext^n_R(M, N) = H^n(\Hom(P_\bullet, N)), \quad \forall n \ge 1$.
  \end{itemize}
\end{definition}

\begin{theorem}[Indenpedency of the choice of projective resolutions]
  $\Ext^n(M, N)$ is independent of the choice of the projective resolution used.

  \begin{proof}
    First, consider two projective resolutions of $M, \tilde{M}$, and map $f: M \to \tilde{M}$,
    and two liftings $\Set{f_i}, \Set{g_i}$. Use $\bar{\cdot}$ to
    denote the natural transformation from $X \to Y$ to $\Hom(Y, N) \to \Hom(X, N)$
    by $\bar{f} \triangleq g \mapsto g \circ f$. Then we shall prove that
    $\bar{f_\bullet}^* = \bar{g_\bullet}^*$, which is to say
    $\bar{f_\bullet}^*$ is independent of the lifting used.

    By comparison theorem (\ref{thm:comparison-theorem}), $\Set{f_i}, \Set{g_i}$ are homotopic,
    and we could write down the diagram below:
    \[
      \begin{tikzcd}
        \dotsm \ar[r] & P_2 \ar[r, "d_2"] \ar[d, "g_2", "f_2"']
                      & P_1 \ar[r, "d_1"] \ar[d, "g_1", "f_1"'] \ar[ld, "s_1"' near start]
                      & P_0 \ar[r, "\alpha"] \ar[d, "f_0"', "g_0"] \ar[ld, "s_0"' near start]
                      & M \ar[r] \ar[d, "f"] & 0 \\
        \dotsm \ar[r] & \tilde P_2 \ar[r, "\tilde d_2"]
                      & \tilde P_1 \ar[r, "\tilde d_1"]
                      & \tilde P_0 \ar[r, "\tilde\alpha"]
                      & \tilde M \ar[r] & 0
      \end{tikzcd}
    \]
    Notice that $\bar{\cdot}$ act linearly, that is, $\ob{f + g} = \bar{f} + \bar{g}$,
    and $\ob{fg} = \bar{g} \bar{f}$.
    So we have
    \[ g_n - f_n = s_{n-1} d_n + \tilde d_{n+1} s_n \implies
      \bar g_n - \bar f_n = \bar d_n \bar s_{n-1} + \bar s_n \bar d_{n+1} \]
    and $\bar f_n, \bar g_n$ are homotopic.
    Thus by proposition~\ref{prop:null-homotopic-then-fstar-is-zero},
    $\bar f_\bullet^* = \bar g_\bullet^*$.

    Now, let $P_\bullet, {P'}_\bullet$ be two projective resolutions.
    Consider the diagram:
    \[
      \begin{tikzcd}[cramped]
        \dotsm \ar[r] & P_1 \ar[r] \ar[d, "f_1"] \ar[dd, bend right, "\Id"' near start]
                      & P_0 \ar[r] \ar[d, "f_0"] \ar[dd, bend right, "\Id"' near start]
                      & M \ar[r] \ar[d, "\Id"] & 0 \\
        \dotsm \ar[r] & P'_1 \ar[r] \ar[d, "g_1"]
                      & P'_0 \ar[r] \ar[d, "g_0"]
                      & M \ar[r] \ar[d, "\Id"] & 0 \\
        \dotsm \ar[r] & P_1 \ar[r]
                      & P_0 \ar[r]
                      & M \ar[r] & 0 \\
      \end{tikzcd}
    \]
    Then $g_i \circ f_i$ and $\Id$ are two liftings, and thus by previous
    we have $\bar g_i^* \circ \bar f_i^* = \Id^*$.
    By symmetry, $\bar f_i^* \circ \bar g_i^* = \Id^*$, which means that
    the homology calculated using different resolution are isomorphic.

  \end{proof}
\end{theorem}

\begin{theorem}[Horseshoe Lemma] \label{thm:horseshoe-lemma}
  Given $0 \to L \to M \to N \to 0$ and projective resolutions
  $P_\bullet \to L \to 0, \, \tilde P_\bullet \to N \to 0$.
  Then there is a projective resolution for $M$ such that
  the following diagram commutes:
  \[ \begin{tikzcd}[cramped]
      & \vdots \ar[d] & \vdots \ar[d] & \vdots \ar[d] & \\
      0 \ar[r] & P_1 \ar[r] \ar[d] & \bar P_1 \ar[r] \ar[d] & \tilde P_1 \ar[r] \ar[d] & 0 \\
      0 \ar[r] & P_0 \ar[r] \ar[d] & \bar P_0 \ar[r] \ar[d] & \tilde P_0 \ar[r] \ar[d] & 0 \\
      0 \ar[r] & L   \ar[r] \ar[d] & M        \ar[r] \ar[d] & N          \ar[r] \ar[d] & 0 \\
               & 0 & 0 & 0 & \\
  \end{tikzcd} \]

  \begin{proof}
    Let $\bar P_n \triangleq P_n \oplus \tilde P_n$. $\bar P_n$ is projective
    by the fact that direct sum of projective modules are projective. Also
    $0 \to P_n \to P_n \oplus \tilde P_n \to \tilde P_n \to 0$ by
    injection and projection. It remains to show that the
    maps in the middle column exists.

    Consider the following diagram:
    \[ \begin{tikzcd}[cramped]
        & 0 \ar[d] & 0 \ar[d] & 0 \ar[d] & \\
        {\color{blue} 0} \ar[r, blue] &
        \ker \alpha \ar[r] \ar[d] &
        \ker \bar \alpha \ar[r] \ar[d] &
        \ker \tilde \alpha \ar[r, blue] \ar[d] \ar[dddll, dashed, orange, controls={+(3,-0.8) and +(-3,0.8)}]
        & {\color{blue} 0} \\
        0 \ar[r] &
        P_0 \ar[r] \ar[d, "\alpha"] &
        P_0 \oplus \tilde P_0 \ar[r] \ar[d, red, "\bar \alpha" red] &
        \tilde P_0 \ar[r] \ar[d, "\tilde \alpha"] \ar[ld, "\exists \sigma"] & 0 \\
        0 \ar[r] & L \ar[r, "f"'] \ar[d] &
        M \ar[r, "g"'] \ar[d] &
        N \ar[r] \ar[d] & 0 \\
        & 0 & 0 & 0 & \\
    \end{tikzcd} \]
    $\sigma$ exists because $\tilde P_0$ is projective.
    Define
    \[ \begin{tikzcd}[row sep=0pt]
        \bar\alpha : &[-3em] P_0 \oplus \tilde P_0 \ar[r] & M \\
                     &[-3em] (z, y) \ar[r, mapsto] & f \circ \alpha(z) + \sigma(y)
    \end{tikzcd} \]
    It easy to see that $\bar \alpha$ let the diagram commutes.
    So we show that $P_0 \oplus \tilde P_0 \xto{\bar \alpha} M \to 0$:

    For any $x \in M$, consider $g(x) \in N$. Since
    $\tilde P_0 \xto{\tilde \alpha} N \to 0$, there exists $y \in \tilde P_0$
    such that $\tilde\alpha(y) = g(x) \implies g\circ\sigma(y) = g(x)$.
    Then $x - \sigma(y) = \ker g = \Image f$, so there exists $w \in L$
    such that $f(w) + \sigma(y) = x$. Now, since $P_0 \xto{\alpha} L \to 0$,
    there exists $z \in P_0$ such that $\alpha(z) = w$. Then we have
    $\bar\alpha(z, y) = x$. So $P_0 \oplus \tilde P_0 \xto{\bar \alpha} M \to 0$.
    
    Now apply the snake lemma, we can obtain
    $
      0 \to \ker \alpha \to \ker \bar \alpha \to \ker \tilde \alpha \to 0.
    $

    Use $d_{-1} \triangleq \alpha$ and so on, we can do induction on $n$ by
    using $\ker d_{n-1}, \ker \bar d_{n-1}, \ker \tilde d_{n-1}$ to replace
    $L, M, N$. Then we are done.
  \end{proof}
\end{theorem}

\begin{theorem}[Long exact sequence for $\Ext$]
  If $0 \to L \to M \to N \to 0$ exact, then there is a long exact sequence:
  \begin{multline*} 0 \to \Hom(N, K) \to \Hom(M, K) \to \Hom(L, K) \\
    \to \Ext^1(N, K) \to \Ext^1(M, K) \to \Ext^1(L, K) \to \Ext^2(N, K) \to \dots
  \end{multline*}

  \begin{proof}
    Taking $\Hom({-}, K)$ in the diagram of Horseshoe lemma
    (\ref{thm:horseshoe-lemma}) and delete the first row,
    we get
  \[ \begin{tikzcd}[cramped]
      & \vdots \ar[leftarrow, d] & \vdots \ar[leftarrow, d] & \vdots \ar[leftarrow, d] & \\
      0 \ar[leftarrow, r] & \Hom(P_1, K) \ar[leftarrow, r] \ar[leftarrow, d] & \Hom(\bar P_1, K) \ar[leftarrow, r] \ar[leftarrow, d] & \Hom(\tilde P_1, K) \ar[leftarrow, r] \ar[leftarrow, d] & 0 \\
      0 \ar[leftarrow, r] & \Hom(P_0, K) \ar[leftarrow, r] \ar[leftarrow, d] & \Hom(\bar P_0, K) \ar[leftarrow, r] \ar[leftarrow, d] & \Hom(\tilde P_0, K) \ar[leftarrow, r] \ar[leftarrow, d] & 0 \\
      & 0 & 0 & 0 & \\
     \end{tikzcd} \]
   Notice that $\Hom(M \oplus N, K) \cong \Hom(M, K) \oplus \Hom(N, K)$, so
   each row is indeed exact.

   By exercise 14.7, the long exact sequence in the statement exists.
   (one can check the kernels of the first row are indeed
   $\Hom(N, K), \Hom(M, K), \Hom(L, K)$.)
  \end{proof}
\end{theorem}

