%! TEX root=../main.tex
\section{Introduction to Homological Algebra}

\subsection{Projective, Injective and Flat modules (week 14)}

\begin{definition} \mbox{}
  \begin{itemize}
    \item $M \in \Mod_R$ is {\bf projective} if $\Hom(M, \cdot)$ preserves the
      {\it right} exactness.
    \item $N \in \Mod_R$ is {\bf injective} if $\Hom(\cdot, N)$ preserves the
      {\it right} exactness.
    \item $M \in \Mod_R$ is {\bf flat} if $M\otimes \cdot$ preserves the
      {\it left} exactness.
  \end{itemize}
\end{definition}

\begin{fact} \mbox{}
  \begin{itemize}
    \item $M$ is projective $\iff$
      $\begin{tikzcd}[cramped, sep=15pt]
         & M \ar[dl, "\exists \tilde f"'] \ar[d, "f"] \\
        M_2 \ar[r] & M_3 \ar[r] & 0
      \end{tikzcd}$
    \item $N$ is injective $\iff$
      $\begin{tikzcd}[cramped, sep=15pt]
        0 \ar[r] & M_1 \ar[d, "g"'] \ar[r] & M_2 \ar[dl, "\exists \tilde g"] \\
         & N
      \end{tikzcd}$
    \item free $\implies$ projective:
      If $X = \Set{ x_i \given i \in \Lambda}$ and $f: x_i \mapsto a_i$.
      Since $\beta$ onto, exists $b_i$ so that $\beta(b_i) = a_i$.
      we can then set $\tilde f: x_i \mapsto b_i$ by the universal
      property of free module.
      \[\begin{tikzcd}[cramped, sep=15pt]
        & F(X) \ar[dl, "\exists \tilde f"'] \ar[d, "f"] \\
        M_2 \ar[r, "\beta"] & M_3 \ar[r] & 0
      \end{tikzcd}\]
    \item free $\implies$ flat:
      Let $F \cong R^{\otimes \Lambda}$ be a free module,
      and $M_1, M_2$ be two modules such that $0 \to M_1 \to M_2$.
      Since $R \otimes_R M \cong M$, we have
      \begin{alignat*}{3}
        & 0 \to M_1 \to M_2 & \text{ exact } \\
        \implies & 0 \to R \otimes M_1 \to R \otimes M_2 & \text{ exact } \\
        \implies & 0 \to \bigoplus_{i \in \Lambda} R \otimes M_1 \to
        \bigoplus_{i \in \Lambda} R \otimes M_2 \ & \text{ exact } \\
        \stackrel{(a)}{\implies} & 0 \to R^{\otimes \Lambda} \otimes M_1 \to
        R^{\otimes \Lambda} \otimes M_2 & \text{ exact } \\
        \implies & 0 \to F \otimes M_1 \to F \otimes M_2 & \text{ exact }
      \end{alignat*}
      Where (a) is by the fact that $(A \oplus B) \otimes C \cong
      (A\otimes C) \oplus (B \otimes C)$. Thus $F$ flat.

    \item If $S$ is a multiplication closed set in $R$ with $1 \in S$, then
      \[ 0 \to M \to N \to L \to 0 \implies 0 \to M_S \to N_S \to L_S \to 0. \]
      We know that $M_S \cong R_S \otimes_R M$. So $R_S$ is a flat $R$-module.
      e.g. $\Qb$ is a flat $\Zb$-module.
  \end{itemize}
\end{fact}

For any $M \in \Mod_R$, a projective module $N$ such that $N \to M \to 0$
could be easily find: Simply let $N = F$, a free module on the set $M$.

Now we shall ask for any module $M$, does there exist $N \in \Mod_R$ such that
N is injective and $0 \to M \to N$?

\begin{theorem}[Baer's criterion] \label{thm:boers-criterion}
  $N$ is injective $\iff \forall I \subset R$, and a homomorphism
  $f$, there exists a homomorphism $h$
  such that the following diagram commutes:

  \[ \begin{tikzcd}[cramped, sep=15pt]
    0 \ar[r] & I \ar[d, "f"'] \ar[r] & R \ar[dashed, dl, "\exists h"] \\
     & N
    \end{tikzcd}
  \]
  \begin{proof}
    ``$\Rightarrow$'': See $I$ as an $R$ module, then it is immediate by the
    definition of injective module.

    ``$\Leftarrow$: Consider the following diagram:
    \[
      \begin{tikzcd}[cramped, sep=15pt]
      0 \ar[r] & M_1 \ar[d, "g"] \ar[r] & M_2  \\
        & N
      \end{tikzcd}
    \]
    Let $S \triangleq \Set{ (M, \rho) \mid M_1 \subseteq M \subseteq M_2
      \text{ and } \rho : M \to N \text{ extends } g} \neq \varnothing$
    since $(M_1, g) \in S$.

    By the routinely proof using Zorn's lemma, exists a maximal element $(M^*, \mu) \in S$.

    We claim that $M^* = M_2$.
    If not, pick $a \in M_2 \setminus M^*$
    and let $M' \triangleq M^* + Ra \supsetneq M^*$,
    $I \triangleq \Set{ r \in R \mid ra \in M^* }$.
    Define $f : I \to N$ with $r \mapsto \mu(ra)$.
    Then we have an extension $h :: R \to N$ of $f$.

    Now, let $\mu' ::  M' \to N = x + ra \mapsto \mu(x) +
    h(r)$.
    We shall prove that this map is well-defined:
    If $x_1 + r_1a = x_2 + r_2a$, then $(r_1 - r_2) a = x_2 - x_1 \in M \implies
    r_1 - r_2 \in I$.
    So $h(r_1) - h(r_2) = f(r_1 - r_2) = \mu((r_1 - r_2)a) = \mu(x_2) - \mu(x_1)$,
    which prove $\mu'$ is well defined, and the existence of $\mu'$
    contradict the fact that $(M^*, \mu)$ is maximal.
  \end{proof}
\end{theorem}

\begin{definition}
  $M$ is {\bf divisible} if $\forall x\in M, r \in R \setminus \set{0}$, there exists
  $y \in M$ such that $x = ry$, i.e. $rM = M \quad \forall r \in R \setminus \set{0}$.
\end{definition}

\begin{prop} \mbox{} \label{prop:injective-and-divisible}
  \begin{enumerate}
    \item Every injective module $N$ over an integral domain is divisible.
      \begin{proof}
        For any $x_0$ and $r_0 \in R \setminus \{0\}$. Let
        $I = \gen{r_0} \subset R$. As long as $R$ is an integral domain,
        $I \cong R$ as an $R$-module, so the $R$-module homomorphism
        $f :: I \to N = r x_0 \mapsto r r_0$ is well-defined.
        Since $N$ injective, this map extends to $h :: R \to N$.
        Let $y_0 \triangleq h(1)$, then $r_0 y_0 = r_0 h(1) = h(r_0) = x_0$.
        Thus $N$ injective.
      \end{proof}
    \item Every divisible module $N$ over an PID is injective.
      \begin{proof}
        For any $I \subseteq R$ and a homomorphism $f :: I \to N$, if $I = 0$ then
        $h = x \mapsto 0$ is always an extension of $f$.
        So assume $\forall I \neq 0$. Since $R$ is a PID,
        $I = \gen{r_0}$ for some $r_0 \neq 0 \in R$.'
        By the fact that $N$ divisible, exists $y_0 \in N$
        such that $r_0 y_0 = x_0 \triangleq f(r_0)$.

        Now we could define
        $h :: R \to N$ by $1 \mapsto y_0$.
        Then $h(r_0) = r_0 h(1) = r_0 y_0 = x_0$, thus
        $h$ is an extension of $f$ and $N$ injective.
     \end{proof}
     \item If $R$ is a PID, then any quotient $N$ of a injective $R$-module $M$
       is injective.
       \begin{proof}
        By 2., $rM = M$ for any $r \neq 0$, thus $rN = N$ for any $r \neq 0$,
        and hence $N$ injective.
       \end{proof}
  \end{enumerate}
\end{prop}

\begin{theorem} \label{thm:every-module-is-in-injective-module}
 For any $M \in \Mod_R$, exists $N$ injective and contains $M$.
 \begin{proof} $ $
   \begin{enumerate}[label={\bf Case \arabic*:}]
      \item $R = \Zb$. \\
        Let $X = \Set{x_i}_{i \in \Lambda}$ be a generating set for $M$
        and $F$ is free on $X$. Let $f$ be the natural map from $f$
        to $M$. then $M \cong \quot{F}{\ker f}$.

        Define $F' = \bigoplus_{i \in \lambda} \Qb e_i \subset F$,
        which is obviously a divisible $\Zb$-module.
        Then $M \subseteq F' / \ker f \triangleq M'$,
        where $M'$ is injective by proposition~\ref{prop:injective-and-divisible}.

      \item $R$ arbitrary. \\
        We can regard any $M$ as a $\Zb$-module, then there exists an injective
        module $N_0 \supset M$.
        Now, we have an $R$-module $N \triangleq \Hom_\Zb(R, N_0)$
        with multiplication $r f(x) \triangleq x \mapsto f(x r)$.

        We claim that $N$ is injective. For any $f :: M_1 \to N$,
        and a homomorphism $\alpha :: M_1 \to M_2$,
        then $\alpha$ could be take as a $\Zb$-module homomorphism.
        Define $f' :: M_1 \to N_0$ by $x \mapsto f(x)(1)$. Since
        $N_0$ injective, exists $h'$, a $\Zb$ module
        homomorphism from $M_2$ to $N_0$.
        \[
          \begin{tikzcd}[contains/.style = {draw=none, 
            "\in" description ,sloped}]
                        & x \ar[d, contains] \ar[ddd, mapsto, bend right] \\
              0 \ar[r]  & M_1 \ar[r,"\alpha"] \ar[d, "f"] & M_2 \ar[ld, 
              "\exists h (?)",dashed] \\
                        & N \ar[r, draw=none, "\cong" description]&  
                        \Hom_\Zb(R, N_0) \\
                        & f(x) \ar[u, contains] \\
            \end{tikzcd}
            \begin{tikzcd}[contains/.style = {draw=none, 
            "\in" description ,sloped}]
                        & x \ar[d, contains] \ar[ddd, mapsto, bend right] \\
              0 \ar[r]  & M_1 \ar[r,"\alpha"] \ar[d, "f'"] & M_2 \ar[ld, 
              "\exists h'"] \\
                        & N_0 \\
                        & f(x)(1) \ar[u, contains] \\
          \end{tikzcd}
        \]
        Now, define
        \[
        \begin{tikzcd}[row sep=0.1]
          h: &[-3em] M_2 \ar[r] & N \\
             &[-3em] y \ar[r, mapsto] & h(y) : &[-3em] R \ar[r] & N_0 \\
             &[-3em] & &[-3em] 1 \ar[r, mapsto] & h'(y) \\
             &[-3em] & &[-3em] r \ar[r, mapsto] & h'(ry) \\
        \end{tikzcd}
        \]
        We check that $h$ is well-defined.
        \begin{itemize}
          \item $h(y) \in \Hom_\Zb(R, N_0)$
            \[
              h(y)(r_1 + r_2) = h'((r_1 + r_2)y) = h'(r_1 y + r_2 y)
              = h'(r_1y) + h'(r_2y) = h(y)(r_1) + h(y)(r_1)
            \]
          \item $h \in \Hom_R(M_2, N)$
            \[
              \begin{aligned}
                h(r_1y_1 + y_2)(r) &= h'(r(r_1y_1 + y_2))
                = h'(rr_1y_1 + ry_2) \\
                &= h'(rr_1y_1) + h'(ry_2) \\
                &= h(y)(rr_1) + h(y_2)(r) \\
                &= (r_1h(y))(r) + h(y_2)(r)
              \end{aligned}
            \]
          \item Show diagram commute $f = h \circ \alpha$
            Fix $y \in M_1$, then $\forall r \in R$:
            \[
              \begin{aligned}
                (h \circ \alpha)(y)(r) &= h(\alpha(y))(r)  = h'(r\alpha(y))\\
                                      &= h'(\alpha(ry)) = f'(ry) \\
                                      &= f(ry)(1) = rf(y)(1) \\
                                      &= f(y)(r)
              \end{aligned}
            \]
        \end{itemize}
        Thus $N_0$ injective.

        Now notice that, $\Hom_\Zb(R, \cdot)$ is a left exact functor,
        so $M \toone N_0$ implies $\Hom_\Zb(R, M) \toone \Hom_\Zb(R, N_0)$,
        thus $M \cong \Hom_R(R, M) \toone \Hom_\Zb(R, M) \toone 
        \Hom_\Zb(R,N_0) = N$.
        \qedhere
    \end{enumerate}
 \end{proof}
\end{theorem}

\begin{prop} \label{prop:tfae-of-projective-module}
  TFAE
  \begin{enumerate}
    \item $M$ is projective.
    \item Every exact sequence $0 \to M_1 \to M_2 \to M \to 0$ split.
    \item $\exists M'$ s.t. $M \oplus M' \cong F$: free.
  \end{enumerate}
  \begin{proof} $ $\\
    (1) $\Rightarrow$ (2) : Since $M$ projective, the map $\lambda$ with
    $\beta \circ \lambda = \Id$ exists in the following diagram:\\
    \[
      \begin{tikzcd}
                            & M \ar[d, "\Id"] \ar[dl,"\exists \lambda"swap] \\
        M_2 \ar[r, "\beta"] & M \ar[r]        & 0
      \end{tikzcd}
    \]
    Then $\lambda$ is a lifting, so $M_2 \cong M_1 \oplus M$ and $0 \to M_1 \to M_2 \to M \to 0$ split. \\

    (2) $\Rightarrow$ (3): Let $F$ be a free module on a generating set of $M$,
    and $\beta :: F \to M$ be the natural map,
    then $0 \to \ker \beta \to F \to M \to 0$ split,
    so $F \cong \ker \beta \oplus M$.

    (3) $\Rightarrow$ (1):
    For any $M_2 \to M_3 \to 0$, since $M' \oplus M$ free and thus projective,
    $\lambda'$ exists in the following diagram:
    \[
      \begin{tikzcd}
        0 \ar[r] & M' \ar[r] & M' \oplus M \ar[r, "\pi" swap, 
        yshift=-0.7ex] \ar[d, "\exists \lambda'"]
        & M \ar[r] \ar[d, "f"] \ar[l, "\mu" swap, yshift=0.7ex] & 0 \\
         & & M_2 \ar[r, "\beta"] & M_3 \ar[r] & 0
      \end{tikzcd}
    \]
    Define $\lambda = \lambda' \circ \mu$. Then $\beta \circ \lambda= 
    \beta \circ \lambda' \circ \mu = f \circ \pi \circ \mu = f$.
  \end{proof}
\end{prop}

\begin{prop}
  TFAE
  \begin{enumerate}
    \item $M$ is injective.
    \item Each exact sequence $0 \to M \to M_2 \to M_3 \to 0$ split.
  \end{enumerate}
  \begin{proof}
    (1) $\Rightarrow$ (2): Similar to the projective case, $\mu$ exists
    in the following diagram:
    \[
    \begin{tikzcd}
      0 \ar[r] & M \ar[r, "\alpha"] \ar[d, "\Id"] & M_2 \ar[dl, "\exists
      \mu", dashed] \\
               & M
    \end{tikzcd}
    \]
    So $M_2 = M \oplus M_3$.

    (2) $\Rightarrow$ (1):
    By theorem~\ref{thm:every-module-is-in-injective-module},
    there is a module $N \subset M$ so that $N$ is injective.

    Consider
    $
    \begin{tikzcd}
    0 \ar[r] & M \ar[r, "i", yshift=0.7ex] &  N \ar[r] \ar[l, "\exists \mu", 
      yshift=-0.7ex] & \coker i \ar[r]& 0
    \end{tikzcd}
    $
    split exact and $\mu \circ i = \Id_M$. Since $N$ injective,
    $h'$ exists in the following diagram:
    $$
    \begin{tikzcd}
      0 \ar[r] & M_1 \ar[r, "\alpha"] \ar[d , "f"] \ar[dd, "i 
      \circ f" swap, bend right = 60] & M_2 \ar[ddl, "\exists h'"]\\
               & M \ar[d, xshift=-0.7ex, "i" swap , hook] \\
               & N \ar[u, xshift=0.7ex, "\mu" swap]
    \end{tikzcd}
    $$
    Let $h = \mu \circ h'$, then $h \circ \alpha = \mu \circ h' 
    \circ \alpha = \mu \circ i \circ f = f$
  \end{proof}
\end{prop}

\begin{prop}
  projective $\implies$ flat.
  \begin{proof}
    Observe that $\bigoplus\limits_{i \in \Lambda} M_i$ is flat if and only if $M_i$ is
    flat for each $i$, since if $0 \to N_1 \xrightarrow{\alpha} N_2$ exact, then
    \[
      \begin{tikzcd} [cong/.style = {draw=none, "\cong" description, sloped}]
        0 \ar[r] & \left( \bigoplus M_i \right) \otimes N_1
        \ar[r, "1 \otimes \alpha"] \ar[d,equal] &
        \left(\bigoplus M_i \right) \otimes N_2 \ar[d, equal] \\
        0 \ar[r] & \bigoplus (M_i \otimes N_1)
        \ar[r, "\oplus (1 \otimes \alpha)"] \ar[d, Leftrightarrow]& 
        \bigoplus (M_i \otimes N_2)\\
        0 \ar[r] & M_i \otimes N_1 \ar[r, "1 \otimes \alpha"] 
        & M_i \otimes N_2 & \forall i \in \Lambda 
      \end{tikzcd}
    \]

    If $M$ is projective, then by proposition~\ref{prop:tfae-of-projective-module}
    $\exists M'$ such that $M \oplus M' \cong F$ is free.
    Since free implies flat, by above, $M$ is flat.
  \end{proof}
\end{prop}


