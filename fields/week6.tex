%! TEX root=../main.tex
\subsection{Solution by radicals}

\begin{definition} \mbox{}
  \begin{enumerate}
    \item Given $L/K$ and $\alpha \in L$, $\alpha$ is called a radical over $K$
      if $\alpha^n \in K$ for some $n\in \Nb$.
    \item $L/K$ is called an extension by radicals if there exist
      $L = L_n \supset L_{n-1} \supset \dots \supset L_1 \supset L_0 = K$
      s.t. $\forall i = 1,\dots, n, \quad L_i = L_{i-1}(\alpha_i)$ with
      $\alpha_i$ a radical over $L_{i-1}$.
    \item $f(x) \in K[x]$ is solvable by radicals if there exists $L/K$,
      an extension by radicals, s.t. $f$ splits over $L$.
  \end{enumerate}
\end{definition}

\begin{definition}
  (Recall) Let $G$ be a finite group. $G$ is solvable if
  $\exists\: \{1\} = G_m \lhd G_{m-1} \lhd \dots \lhd G_0 = G$ s.t.
  $\quot{G_{i-1}}{G_i}$ is cyclic $\forall i$.
\end{definition}

\begin{theorem}[Main Theorem]
  Under some proper assumption on $\Char K$, a separable poly. $f(x) \in K[x]$
  is solvable by radicals $\iff$ the Galois group of $f$ is solvable.
\end{theorem}

\begin{lemma} \label{lemma:galois-radical-ext}
  Given a Galois extension $L/K$ and $M = L(\alpha)$ is an extension by
  a radical, where $\alpha^n = a \in L$. Assume that $\Char K \nmid n$. Then
  $\exists\: N$ s.t. $N/M$ is an extension by radicals and $N/K$ is Galois
  and $N$ contains $\zeta_n$.
  \begin{proof}
    We know that $M(\zeta_n) = L(\zeta_n, \alpha)$ is a splitting field of
    $x^n - a$ over $L$. If we set
    \[ f(x) = \prod_{\sigma \in \Gal(L/K)} (x^n - \sigma(a)), \]
    then the coefficients of $f(x)$ are elementary symmetric poly. in
    $\Set{\sigma(a) \given \sigma \in \Gal(L/K) }$, which are fixed by
    $\Gal(L/K)$, so $f(x) \in K[x]$.

    Let $L$ be a splitting field of $g(x)$ over $K$. (since $L/K$ is Galois)
    Choose $N$ as a splitting field of $f(x)g(x)$ over $K$.
    By def., $N/K$ is Galois. Let $L = K(\beta_1,\dots,\beta_s)$ where
    $\beta_1, \dots, \beta_s$ are the roots of $g(x)$, then
    \[
      N = K\big(\beta_1,\dots,\beta_s, \zeta_n,
        \alpha_\sigma : \sigma \in \Gal(L/K)
      \big),
      \qquad \alpha_\sigma^n = \sigma(a) \in L
    \]
    $M = L(\alpha)$,
    so $N = M(\alpha_\sigma : \sigma \in \Gal(L/K) \setminus \{ \Id \})$
    $\implies$ $N/M$ is an extension by radicals.
  \end{proof}
\end{lemma}

\begin{lemma} \label{lemma:radical-ext-chain-galois}
  Let $L = L_m \supset L_{m-1} \supset \dots \supset L_0 = K$ s.t.
  $L_i = L_{i-1}(\alpha_i)$ with $\alpha^{n_i} = a_i \in L_{i-1}$.
  If $\Char K \nmid n_1n_2\cdots n_m$, then there exists $N/L$ s.t.
  $N/K$ is a Galois extension by radicals and $\zeta_{n_i} \in N \quad
  \forall i = 1, \dots, m$.

  \begin{proof}
    By induction on $m$. For $m = 1$, $L_1 \supset L_0 = K$ and
    $L_1 = L_0(\alpha_1) = K(\alpha_1)$ where $\alpha_1^{n_1} \in K$ for some
    $n_1\in \Nb$. Set $N = L(\zeta_n) = K(\zeta_n, \alpha_1)$, done.

    For $m > 1$, by induction hypothesis, $\exists\: N'/L_{m-1}$ s.t.
    $N'/K$ is Galois extension by radicals and $N'$ contains
    $\zeta_{n_i} \quad \forall i = 1, \dots, m-1$.
    By lemma \ref{lemma:galois-radical-ext}, $\exists\: N/N(\alpha_m)$ is
    an extension by radicals s.t. $N/K$ is Galois and $N$ contains $\zeta_{n_m}$.
  \end{proof}
\end{lemma}

\begin{theorem}[Part A]
  Let $L = L_m \supset \dots \supset L_0 = K$ s.t. $L_i = L_{i-1}(\alpha_i)$
  with $\alpha^{n_i} = a_i \in L_{i-1}$ and $\Char K \nmid n_1\cdots n_m$.
  If a separable poly. $f(x) \in K[x]$ splits over $L$, then the Galois group
  of $f$ over $K$ is solvable.
  
  \begin{proof}
    By lemma \ref{lemma:radical-ext-chain-galois}, we assume that $L/K$ is
    Galois and so is $L/L_i, \quad i=1,\dots,m$.
    If we set $n = \lcm(n_1, \dots, n_m)$, then, by lemma \ref{lemma:radical-ext-chain-galois},
    $L$ also contains $\zeta = \zeta_n = \zeta_{n_1}^{r_1} \cdots \zeta_{n_m}^{r_m}$.

    Consider $L = L(\zeta) \supset L_{m-1}(\zeta) \supset \dots \supset
    L_0(\zeta) = K(\zeta)$ (Note that $K(\zeta) \supset K$ and $L/K$ is Galois)
    and let $G_i = \Gal(L/L_i(\zeta)) \quad i = 0, \dots, m$.
  \end{proof}
\end{theorem}

