%! TEX root=../main.tex
\subsection{Solution by radicals}

\begin{definition} \mbox{}
  \begin{enumerate}
    \item Given $L/K$ and $\alpha \in L$, $\alpha$ is called a radical over $K$
      if $\alpha^n \in K$ for some $n\in \Nb$.
    \item $L/K$ is called an extension by radicals if there exist
      $L = L_n \supset L_{n-1} \supset \dots \supset L_1 \supset L_0 = K$
      s.t. $\forall i = 1,\dots, n, \quad L_i = L_{i-1}(\alpha_i)$ with
      $\alpha_i$ a radical over $L_{i-1}$.
    \item $f(x) \in K[x]$ is solvable by radicals if there exists $L/K$,
      an extension by radicals, s.t. $f$ splits over $L$.
  \end{enumerate}
\end{definition}

\begin{definition}
  (Recall) Let $G$ be a finite group. $G$ is solvable if
  $\exists\: \{1\} = G_m \lhd G_{m-1} \lhd \dots \lhd G_0 = G$ s.t.
  $\quot{G_{i-1}}{G_i}$ is cyclic $\forall i$.
\end{definition}


\begin{lemma} \label{lemma:galois-radical-ext}
  Given a Galois extension $L/K$ and $M = L(\alpha)$ is an extension by
  a radical, where $\alpha^n = a \in L$. Assume that $\Char K \nmid n$. Then
  $\exists\: N$ s.t. $N/M$ is an extension by radicals and $N/K$ is Galois
  and $N$ contains $\zeta_n$.
  \begin{proof}
    We know that $M(\zeta_n) = L(\zeta_n, \alpha)$ is a splitting field of
    $x^n - a$ over $L$. If we set
    \[ f(x) = \prod_{\sigma \in \Gal(L/K)} (x^n - \sigma(a)), \]
    then the coefficients of $f(x)$ are elementary symmetric polynomials in
    $\Set{\sigma(a) \given \sigma \in \Gal(L/K) }$, which are fixed by
    $\Gal(L/K)$, so $f(x) \in K[x]$.

    Let $L$ be a splitting field of $g(x)$ over $K$. (since $L/K$ is Galois)
    Choose $N$ as a splitting field of $f(x)g(x)$ over $K$.
    By def., $N/K$ is Galois. Let $L = K(\beta_1,\dots,\beta_s)$ where
    $\beta_1, \dots, \beta_s$ are the roots of $g(x)$, then
    \[
      N = K\big(\beta_1,\dots,\beta_s, \zeta_n,
        \alpha_\sigma : \sigma \in \Gal(L/K)
      \big),
      \qquad \alpha_\sigma^n = \sigma(a) \in L
    \]
    So $N = M(\zeta_n, \alpha_\sigma : \sigma \in \Gal(L/K) \setminus \{ \Id \})$
    $\implies$ $N/M$ is an extension by radicals.
  \end{proof}
\end{lemma}

\begin{lemma} \label{lemma:radical-ext-chain-galois}
  Let $L = L_m \supset L_{m-1} \supset \dots \supset L_0 = K$ s.t.
  $L_i = L_{i-1}(\alpha_i)$ with $\alpha^{n_i} = a_i \in L_{i-1}$.
  If $\Char K \nmid n_1n_2\cdots n_m$, then there exists $N/L$ s.t.
  $N/K$ is a Galois extension by radicals and $\zeta_{n_i} \in N, \,
  \forall i = 1, \dots, m$.

  \begin{proof}
    By induction on $m$. For $m = 1$, $L_1 \supset L_0 = K$ and
    $L_1 = L_0(\alpha_1) = K(\alpha_1)$ where $\alpha_1^{n_1} \in K$ for some
    $n_1\in \Nb$. Set $N = L(\zeta_{n_1}) = K(\zeta_{n_1}, \alpha_1)$, done.

    For $m > 1$, by induction hypothesis, $\exists\: N'/L_{m-1}$ s.t.
    $N'/K$ is Galois extension by radicals and $N'$ contains
    $\zeta_{n_i},\, \forall i = 1, \dots, m-1$.
    By lemma \ref{lemma:galois-radical-ext}, $\exists\: N/N'(\alpha_m)$ is
    an extension by radicals s.t. $N/K$ is Galois and $N$ contains $\zeta_{n_m}$.
  \end{proof}
\end{lemma}

\begin{prop} \label{prop:quot-solvable}
  Let $H \lhd G$. Then $G$ is solvable $\iff$ $H, \quot{G}{H}$ are solvable.

  \begin{proof}
    ``$\Leftarrow$'': Let $q: G \to \quot{G}{H}$ be the quotient map,
    $Q = \quot{G}{H}$. The solvable series is given by
    \[
      G = q^{-1}(Q) = q^{-1}(Q_0) \rhd q^{-1}(Q_1) \rhd \dots \rhd q^{-1}(Q_n)
      = H = H_0 \rhd H_1 \rhd \dots \rhd H_m = \{ 1 \}
    \]
    ``$\Rightarrow$'': \\
    \underline{Claim:} Define $G^{(i)} = [G^{(i-1)}, G^{(i-1)}], \quad
    i \in \Nb; G^{(0)} = G$.
    Then $G$ is solvable $\iff$ $G^{(n)} = \{1 \}$ for some $n$.
    \begin{proof}
      ``$\Leftarrow$'': O.K.

      ``$\Rightarrow$'': Given $G = G_0 \rhd G_1 \rhd \dots \rhd G_m = \{1\}$
      with $\quot{G_{i-1}}{G_i}$ abelian.
      We have $G^{(1)} \le G_1 \leadsto G^{(2)} \le [G_1, G_1] \le G_2 \leadsto
      \dots \leadsto G^{(n)} \le G_n = \{ 1 \} \leadsto G^{(n)} = \{1\}$.
    \end{proof}
    By the claim above:
    \begin{itemize}
      \item $H^{(n)} \le G^{(n)} = \{1\} \leadsto H^{(n)} = \{1\} \implies H$ 
        is solvable.
      \item $q\big([G, G]\big) = [q(G), q(G)] = [\quot{G}{H}, \quot{G}{H}] =
        \left(\quot{G}{H}\right)^{(1)} \leadsto \dots \leadsto
        q(G^{(n)}) = \left(\quot{G}{H}\right)^{(n)} \implies \quot{G}{H}$
        is solvable.
    \end{itemize}
  \end{proof}
\end{prop}

\begin{theorem}[Main Theorem] \label{thm:solvable-iff-solvable}
  Under some proper assumption on $\Char K$, a separable polynomial $f(x) \in K[x]$
  is solvable by radicals $\iff$ the Galois group of $f$ is solvable.

  \begin{enumerate}[label={\bf Part \Alph*:}]
    \item Let $L = L_m \supset \dots \supset L_0 = K$ s.t. $L_i = L_{i-1}(\alpha_i)$
      with $\alpha^{n_i} = a_i \in L_{i-1}$ and $\Char K \nmid n_1\cdots n_m$.
      If a separable poly. $f(x) \in K[x]$ splits over $L$, then the Galois group
      of $f$ over $K$ is solvable.

  \begin{proof}
    By lemma \ref{lemma:radical-ext-chain-galois}, we can first extend
    the extension tower and thus assume that $L/K$ is
    Galois with each $\zeta_{n_i}$ in $L$. Then each $L/L_i$ is Galois.
    If we set $n = \lcm(n_1, \dots, n_m)$,
    $L$ also contains $\zeta = \zeta_n = \zeta_{n_1}^{r_1} \cdots \zeta_{n_m}^{r_m}$.

    Consider $L = L(\zeta) \supset L_{m-1}(\zeta) \supset \dots \supset
    L_0(\zeta) = K(\zeta)$ (Note that $K(\zeta) \supset K$ and $L/K$ is Galois)
    and let $G_i = \Gal(L/L_i(\zeta))$ for each $i = 0, \dots, m$.

    Define $L'_i \triangleq L_i(\zeta)$ for all $i$. We can find that
    \begin{itemize}
      \item $G_m = \{1\}, G_0 = \Gal(L/K(\zeta))$.
      \item Since $\zeta_n \in L_{i-1}$, $L_i / L_{i-1}$ is normal,
        so
        \[ \quot{G_{i-1}}{G_i} = \quot{\Gal(L/L'_{i-1})}{\Gal(L/L'_i)}
        \cong \Gal(L'_{i-1} / L'_i) =  \Gal(L'_i(\alpha_i) / L'_i) \] is cyclic.
    \end{itemize}
    So $G_0$ is solvable.
    Moreoevr, $K(\zeta)$ is a splitting field of $x^n - 1$ over $K$ and
    $\Gal(K(\zeta)/K) \le \left(\quot{\Zb}{n\Zb}\right)^\times$, which is
    abelian, so it is solvable. Also, $\Gal(K(\zeta)/K) \cong \quot{\Gal(L/K)}{G_0}$
    is solvable. $\implies \Gal(L/K)$ is solvable.
    Let $N$ be a splitting field of $f$ over $K$ $\leadsto L \supset N \leadsto
    \Gal(N/K) \cong \quot{\Gal(L/K)}{\Gal(L/N)}$.

    By prop \ref{prop:quot-solvable}, $\Gal(N/K)$ is solvable.
  \end{proof}

  \item Let $f \in K[x]$ be separable and $L$ be a splitting field of $f$ over $K$.
  Assume $\Char K \nmid \abs{\Gal(L/K)}$. If $\Gal(L/K)$ is solvable, then
  $f$ is solvable by radicals.

  \begin{proof}
    Let $n = \abs{\Gal(L/K)}$ and $\zeta = \zeta_n$.
    Let $N$ be a splitting field of $f$ over $K(\zeta)$, i.e. $N = LK(\zeta)$.
    $\implies \Gal(N/K(\zeta)) \cong \Gal(L/L\cap K(\zeta)) \le \Gal(L/K)$.

    So $\Gal(N/K(\zeta))$ is solvable, say $\Gal(N/K(\zeta)) = G_0 \rhd G_1
    \rhd \dots \rhd G_m = 1$, $\quot{G_{i-1}}{G_i}$ is cyclic.
    
    If we set $N_j = N^{G_j}$, then
    $N = N_m \supset N_{m-1} \supset \dots \supset N_0 = K(\zeta)$ and
    $G_j = \Gal(N/N_j)$, $\quot{G_{i-1}}{G_i} \cong \Gal(N_i/N_{i-1})$ is
    cyclic $\implies N_i = N_{i-1}(\alpha_i), \alpha_i^{n_i} \in N_{i-1}$.
    (kummer extension)
    \begin{mdframed}
      Note that $n_i = [L_i : L_{i-1}] = \abs{G_{i-1}} / \abs{G_i}$ dividing
      $\abs{G_0}$ and $\abs{G_0} \mid n$, so $\zeta_n$ generates $\zeta_{n_i}$
      and $\Char K \nmid n_i$.
    \end{mdframed}
    $\implies N/K(\zeta)$ is an extension by radicals $\leadsto$
    $N/K$ is an extension by radicals.
  \end{proof}
\end{enumerate}
\end{theorem}

\begin{remark}
  In Part A of theorem \ref{thm:solvable-iff-solvable},
  $\Gal(K(\zeta)/K) \le \left(\quot{\Zb}{n\Zb}\right)^\times$ may be proper
  subgroup. We can check the if $[K(\zeta) : K] \overset{?}{=} 4 = \varphi(5)$.
\end{remark}

\subsection{Ruffini-Abel theorem}

\begin{theorem}[Main theorem]
  Assume $\Char F = 0$.
  The general equation of the $n$-th degree is not solvable by radicals if
  $n \ge 5$. In fact, let $f(x) = x^n - t_1x^{n-1} + t_2 x^{n-2} - \dots +
  (-1)^nt_n \in \underbrace{F(t_1, \dots, t_n)}_{=K}[x]$ with
  $t_1, \dots, t_n$ variables and $L$ be a splitting field of $f$ over $K$.
  Then $\Gal(L/K) \cong S_n$. $S_n$ is not solvable for $n \ge 5$.
\end{theorem}

\begin{lemma} \label{lemma:symm-poly-symm-group}
  Let $L = F(x_1, \dots, x_n)$ and $s_1, \dots, s_n$ be the elementary
  symmetric polynomials in $x_1, \dots, x_n$.
  \[
    s_k = \sum_{1 \le j_1 < \dots < j_k \le n} \prod_{i=1}^k x_{j_i}
  \]
  If $K = F(s_1, \dots, s_n) \subset L$, then $L/K$ is Galois and
  $\Gal(L/K) \cong S_n$.

  write $f(x) = (x - x_1) \cdots (x - x_n) = x^n - s_1x^{n-1} + s_2x^{n-2} -
  \dots + (-1)^ns_n \in K[x]$.
  Clearly, $L$ is a splitting field of $f$ over $K \leadsto L/K$ is Galois
  and $\Gal(L/K) \toone S_n$.

  Now, for $\sigma \in S_n$, $\sigma$ can be regarded as an element in
  $\Gal(L/K)$:
  \[
    \arraycolsep=1pt
    \begin{array}{rcl}
      \sigma: & L & \to L \\
              & x_i & \mapsto x_{\sigma(i)}
    \end{array}
  \]
  Since $\Set{ \sigma(x_1), \dots, \sigma(x_n) } = \Set{ x_1, \dots, x_n }
  \leadsto \sigma(s_i) = s_i \quad \forall i \leadsto \sigma\big|_K = \Id_K
  \leadsto \sigma \in \Gal(L/K)$.
\end{lemma}

\begin{coro}
  $L^{S_n} = K = F(s_1, \dots, s_n)$. \\
  $L^{S_n} = \Set{ f(x_1, \dots, x_n) \in L \given f(x_{\sigma(1)}, \dots,
  x_{\sigma(n)}) = f(x_1, \dots, x_n) \quad \forall \sigma \in S_n }$ is
  all symmetric poly.
\end{coro}

\begin{coro}
  For any finite group $G$, by Cayley thm, $G \toone S_n$ for some $n$.
  so $\Gal(L/L^G) \cong G$.
\end{coro}

Now we prove the Main theorem:
\begin{proof}
  Let $L = K(z_1, \dots, z_n)$. Since $t_1, \dots, t_n$ are the symmetric
  poly. w.r.t. $z_1, \dots, z_n$, $L = F(z_1, \dots, z_n)$.

  Let $F(s_1, \dots, s_n)$ and $F(x_1, \dots, x_n)$ be given as in lemma
  \ref{lemma:symm-poly-symm-group}.

  since $t_1, \dots, t_n$ are variables,
  $\exists\: \tau: F[t_1, \dots, t_n] \onto F[s_1, \dots, s_n]$ with $\tau:
  t_i \mapsto s_i$.
  Also, Since $x_1, \dots, x_n$ are variables, $\exists\: \sigma:
  F[x_1, \dots, x_n] \onto F[z_1, \dots, z_n]$ with $\sigma: x_i \mapsto z_i$.

  now, $\sigma \circ \tau(t_i) = \sigma(s_i)
  = \sigma\left(\sum x_{j_1}\cdots x_{j_i}\right)
  = \left(\sum z_{j_1}\cdots z_{j_i}\right) = t_i \implies \sigma\circ\tau =
  \Id \implies \tau$ is 1-1 and thus an isom.
  So there exists an extension $\tau': F(t_1, \dots, t_n) \isoto F(s_1, \dots, s_n)$.
  Note $\bar{\tau}': f(x) \mapsto g(x) = x^n - s_1x^{n-1} + \cdots + (-1)^ns_n$.

  Let $F(z_1, \dots, z_n)$ be a splitting field of $f$ over $F(t_1, \dots, t_n)$
  and $F(x_1, \dots, x_n)$ be a splitting field of $g$ over $F(s_1, \dots, s_n)$
  where $g = \bar{\tau}'(f)$.
  There exists $\sigma': F(z_1, \dots, z_n) \isoto F(x_1, \dots, x_n)$ with
  $\sigma'\big|_{F(t_1, \dots, t_n)} = \tau'$.
  So $\Gal(L/K) \cong S_n$ by lemma \ref{lemma:symm-poly-symm-group}.
\end{proof}

\begin{remark} \mbox{}
  \begin{itemize}
    \item Since $S_n$ is transitive, $f$ is irr.
    \item $\Char F = 0 \leadsto f$ is separable.
  \end{itemize}
\end{remark}
