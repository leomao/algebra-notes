%! TEX root=../main.tex
\subsection{Algebra closure (week 3)}

\begin{definition} \hfill
  \begin{itemize}
    \item $L$ is called an {\bf algebraic closure}\index{algebraic closure} of $K$ if $L/K$ is algebraic and
      each polynomial $f(x) \in K[x]$ splits over $L$.
    \item $L$ is said to be {\bf algebraically closed}\index{algebraically closed} if for each $f(x) \in L[x]$,
      $f(x)$ has a root in $L$.
  \end{itemize}
\end{definition}

\begin{prop}
  Given $L/K$, if $L$ is algebraically closed, then $L^\text{alg} \triangleq
  \Set{\alpha \in L \given \alpha \text{ is algebraic over } K}$ is an
  algebraic closure of $K$.

  \begin{proof}
    By prop~\ref{prop:alg-elements-form-a-field}, $L^\text{alg}$ is a field, and
    by definition, $L^\text{alg}/K$ is algebraic.

    Now we show that for any $f(x) \in K[x]$, $f(x)$ splits over $L$.
    Using induction, $\deg f = 1$ is trivial. If $\deg f > 1$, then since
    $f(x) \in K[x] \subseteq L[x]$, $f$ has a root in $L$, say $\alpha$. so we could
    write $f(x) = (x - \alpha) g(x)$. Then $g(x) \in K(\alpha)[x] \subseteq L[x]$.
    By induction, $g(x)$ splits and hence $f(x)$ splits.

    So for any $f(x) \in K[x]$, $f$ splits over $L$. Write
    $f(x) = (x - \alpha_1) \dotsm (x - \alpha_n)$, then each $\alpha_i$ is
    algebraic over $K \implies \alpha_i \in L^\text{alg}$ and hence
    $f(x)$ splits over $L^\text{alg}[x]$.
  \end{proof}
\end{prop}

\begin{coro}
  If $K$ is algebraically closed, then $K$ is an algebraic closure of $K$ itself.
\end{coro}

\begin{prop}
  If $L$ is an algebraic closure of $K$, then $L$ is algebraically closed.

  \begin{proof}
    For $f(x) \in L[x]$, let $\alpha$ be a root of $f(x)$. Since $L(\alpha)/L$ and
    $L/K$ is algebraic, by prop~\ref{prop:alg-tower-implies-alg}, $L(\alpha)/K$ is algebraic.
    So $\alpha$ must be in $L$, hence $f(x)$ has a root in $L$.
  \end{proof}
\end{prop}

\begin{prop}
  The following are equivalent.
  \begin{enumerate}
    \item $K$ has no nontrivial algebraic extension.
    \item For all irreducible polynomial in $K[x]$ has degree $1$.
    \item Every polynomial of  positive degree in $K[x]$ has at least one root in $K$.
    \item Every polynomial of  positive degree in $K[x]$ splits over $K$.
  \end{enumerate}
\end{prop}

In below we would use the Zorn's lemma heavily.
\begin{lemma}[Zorn's lemma]
  Suppose a partially order set $P$ has the property that every chain (i.e., a total order subset)
  has an upper bound in $P$, then the set $P$ contains at least one maximal element.
\end{lemma}

\begin{lemma} \label{lemma:max-ideal-exists}
  In a commutative ring $R$ with $1$, any proper ideal $I \subsetneq R$ is contained in a maximal ideal.

  \begin{proof}
    Consider $S = \Set{J \subsetneq R \given I \subseteq J} \neq \varnothing$ since $I \in S$.
    Define a partial order on $S$ by $J_1 \preceq J_2 \iff J_1 \subseteq J_2$.

    Given a chain $\Set{J_i \given i \in \Lambda}$, let $J = \bigcup_{i \in \Lambda} J_i$. $J$ is an
    ideal, since if $x, y \in J$, then $x \in J_1, y \in J_2$.
    Let $\tilde{J} = \max(J_1, J_2)$, then $x, y \in \tilde{J}$
    which implies $x + y \in \tilde{J}$, and it is easy to check that for any $x \in R, y \in J$, $xy \in J$.

    Also, $J$ is proper since $1 \notin J$, or else $1 \in J_i$ and thus $J_i = R$
    which leads to a contradiction.

    By Zorn's lemma, there exists a maximal element in $S$, and thus it is a
    maximal ideal which contains $I$.
  \end{proof}
\end{lemma}

\begin{theorem}
  If $K$ is a field, then there exists an algebraic closure $L$ of $K$.

  \begin{proof}
    Let $S = \Set{x_f \given f(x) \in K[x] \text{ with } \deg f \geq 1}$ be the
    set of variables indexed by non-constant polynomial in $K[x]$. Consider
    the polynomial ring $K[S]$ and $I = \gen{ f(x_f) : f \in K[x] \text{ with } \deg f \geq 1 }$,
    which is an ideal in $K[S]$.

    We claim that $I \neq K[S]$. If not, then $1 \in I \implies 1 = \sum_{i = 1}^n g_i f_i(x_{f_i})$.
    Write $x_i \triangleq x_{f_i}$ for $i = 1, 2, \cdots, n$. Also, by definition
    $g_i$ only involves a finite number of variable in $S$, so we could set
    $g_i \in K[x_1, x_2, \dots, x_m]$ with $m \geq n$. That is,
    $1 = \sum_{i = 1}^n g_i(x_1, x_2, \dots, x_m) f_i(x_i)$. Let $\Sigma$ be a
    splitting field for $f(x) = f_1(x) f_2(x) \dotsm f_n(x)$ and define $\alpha_i \in \Sigma$
    which satisfies $f_i(\alpha_i) = 0$ and $a_i = 0$ for $n+1 \leq i \leq m$. Then
    $1 = \sum_{i = 1}^n g(\alpha_1, \alpha_2, \dots, \alpha_m) f_i(\alpha_i) = 0$
    , which leads to a contradiction.

    By lemma~\ref{lemma:max-ideal-exists}, there exists a maximal ideal $M$ s.t. $I \subseteq M$.

    Consider $K \toone F_1 \triangleq K[S] / M$, and then for all $f \in K[x]$,
    $f(\bar{x}_f) = \bar{0}$ in $F_1$.  By induction, $\exists F_1 \subseteq F_2 \subseteq F_3 \subseteq \cdots$
    which satisfies $f(x) \in F_n[x]$ has a root in $F_{n+1}$
    Let $F = \bigcup_{i = 1}^\infty F_i$ which is algebraically closed since if
    $f(x) \in F[x]$ then $f(x) \in F_m[x]$ for some $m$ and thus $f(x)$ has a root in $F_{m+1} \subseteq F$.

    Finally $L \triangleq \Set{ \alpha \in F \given \alpha \text{ is algebraic over } K }$
    is an algebraic closure of $K$.
  \end{proof}

  \begin{lemma} \label{lemma:homo-extend-to-alg-closed-extension}
    If $L_1 / K$ is algebraic and $\tau: K \to L_2$ is a non-zero homomorphism
    with $L_2$ being algebraically closed, then $\tau$ could be extend to $\sigma: L_1 \to L_2$.

    \begin{proof}
      Consider $S = \Set*{ (M, \theta) \given K \subset M \subset L_1,\ \theta: M \to L_2
      \text{ with } \theta\big|_K = \tau }$, which is not an empty set since $(K, \tau) \in S$.

      Define a partial order on $S$ by $(M_1, \theta_1) \preceq (M_2, \theta_2)
      \iff M_1 \subseteq M_2 \,\land\, \theta_2 \big|_{M_1} = \theta_1$.
      Given any chain $\Set{(M_i, \theta_i) \colon i \in \Lambda }$, let
      $N = \bigcup_{i = 1}^\infty M_i$ and $\theta = \alpha :: N \mapsto \theta_i(\alpha)$
      if $\alpha \in M_i$. It could be check easily that this map is well defined,
      and $(N, \theta)$ is a least upper bound in $S$ for this chain.
      By Zorn's lemma, there exists a max element $(M, \sigma)$ in $S$.

      Now, if $M \neq L_1$, then pick $\alpha \in L_1 \setminus M$. Since $L_1/K$ is algebraic,
      the minimal polynomial $m_{\alpha, K}$ exists. Since $L_2$ algebraically closed, $\bar\sigma(m_{\alpha, K})$
      has a root in $L_2$, and thus by lemma~\ref{lemma:extension-exists-condition},
      $\sigma$ could be extend to $\sigma': M(\alpha) \to L_2$ which contradicts
      the maximality of $(M, \sigma)$. Thus $M = L_1$.
    \end{proof}
  \end{lemma}

  \begin{theorem}
    Any two algebraic closures $L_1, L_2$ of $K$ are isomorphic.
    \begin{proof}
      Consider the inclusion map $\Id_K :: K \toone L_1$.
      By Lemma~\ref{lemma:homo-extend-to-alg-closed-extension},
      $\Id_K$ could be extend to $\sigma :: L_2 \to L_1$ such that $\sigma\big|_K = \Id_K$.
      Since $\sigma \neq 0$, $\sigma(L_2) \cong L_2$.
      Also, $L_2$ is algebraically closed implies $\sigma(L_2)$ is algebraically closed.
      So for any $\alpha \in L_1$, $\alpha$ is algebraic over $K$ and thus over $\sigma(L_2)$,
      which implies $\alpha \in \sigma(L_2)$, so $\sigma$ is onto, hence $\sigma$ is an
      isomorphism between $L_1$ and $L_2$.
    \end{proof}
  \end{theorem}

  \begin{example}
    Let $p$ be a prime.
    \begin{itemize}
      \item Any finite field $L$ with $\Char L = p$, $L \cong \Fb_{p^n}$ for
        some $n \in \Nb$.
      \item $\Gal\left(\Fb_{p^n} / \Fb_p\right) = \gen{\sigma_p}$ with
        $p = \alpha :: \Fb_{p^n} \mapsto \alpha^p :: \Fb_{p^n}$.
      \item A subfield $L$ of $\Fb_{p^n}$ is isomorphic to $\Fb_{p^m}$ with
        $m \mid n$ since $[\Fb_{p^n} : \Fb_{p^m}] = d \leadsto p^{md} = p^n$.
      \item $\bigcup_{n=1}^\infty \Fb_{p^n}$ is a field, and it is the
        algebraic closure of $\Fb_p$.
    \end{itemize}
  \end{example}
\end{theorem}

\subsection{Separable extension}

\begin{definition} \hfill
  \begin{itemize}
    \item $\alpha$ is separable over $K$ if $m_{\alpha, K}$ is separable over $K$.
    \item $L/K$ is called a {\bf separable extension}\index{Field extension!separable extension}
      if $\forall \alpha \in L$, $\alpha$ is separable over $K$.
  \end{itemize}
\end{definition}

\begin{example}
  Let $\Char K = p$ and $K^p \subsetneq K$. Pick $b \in K \setminus K^p$ and consider $L$ to be
  the splitting field of $x^p - b$ over $K$, say $\alpha \in L$ with $\alpha^p = b$.
  Notice that $x^p - b = x^p - a^p = (x - a)^p$, and $x^p - b$ is irreducible in $K$, or else
  if $x^p - b = g(x) h(x)$ in $K[x]$, then write $g(x) = (x - \alpha)^k, h(x) = (x - \alpha)^{n-k}$,
  but then expand $g(x)$ and we would get $\alpha^k \in K$, since $\alpha^p \in K$ and
  $\gcd(k, p) = 1$ implies $\alpha \in K$ which leads to a contradiction.

  By above we know that $x^p - b$ is inseparable.
\end{example}

\begin{definition}
  $K$ is said to be \emph{\Index{perfect}} if either $\Char K = 0$ or ``$\Char K = p$ and $K = K^p$''.
\end{definition}

\begin{example}
  If $\Char K = p$ and $\quot{K}{\Fb_p}$ is algebraic, then $K$ is perfect.

  \begin{proof}
    Consider \deffunc{\sigma_p}{K}{K}{\alpha}{\alpha^p}, which is a monomorphism which fixes $\Fb_p$.
    Since $\quot{K}{\Fb_p}$ is algebraic, by the exercise problem, $\sigma_p$ is an automorphism, so $K = K^p$.
  \end{proof}
\end{example}

\begin{fact}
  $K$ is perfect if and only if for any irreducible polynomial $f(x) \in K[x]$, $f$ is separable.

  Also, we can find that an irreducible polynomial $f(x) \in K[x]$ is not separable over $K$
  if and only if $\Char K = p > 0$ and $f(x) = g(x^p)$ for some $g(x) \in K[x]$, where
  $g(x)$ is irreducible and not all coefficients of $g$ is in $K^p$.

  Finally, if $\Char K = 0$, then $K$ is separable.
\end{fact}

\begin{prop} \label{prop:separable-and-split-have-most-embeddings}
  Give $\quot{K(\alpha)}{K}$ with degree $m_{\alpha, K} = d$ and $\tau :: K \to L \neq 0$.
  If $\alpha$ is separable over $K$ and $\bar\tau(m_{\alpha, K})$ splits over $L$, then
  there are exactly $d$ monomorphisms $\sigma :: K(\alpha) \to L$ with $\sigma\big|_K = \tau$.

  Otherwise, if $\alpha$ is not separable or $\bar\tau(m_{\alpha, K})$ doesn't split over $L$,
  then there are $r < d$ such monomorphisms.

  \begin{proof}
    Observe that $m_{\alpha, K}$ is separable over $K$ if and only if $\bar\tau(m_{\alpha, K})$
    is separable over $\tau(K)$.
    Extend $K$ to $\Sigma$, $\tau(K)$ to $\Sigma'$, where $\Sigma, \Sigma'$ are the splitting
    field of $m_{\alpha, K}$ and $\bar\tau(m_{\alpha, K})$, respectively.
    Since $K \cong \tau(K)$, by theorem~\ref{thm:two-splitting-field-are-isom}, $\Sigma \cong \Sigma'$.
    Let $\tau'$ be the isomorphism which is an extension of $\tau$.

    If $m_{\alpha, K} = (x - \alpha_1) (x - \alpha_2) \cdots (x - \alpha_d)$, then
    $\bar\tau(m_{\alpha, K}) = (x - \tau'(\alpha_1)) (x - \tau'(\alpha_2)) \cdots (x - \tau'(\alpha_n))$.
    where $\tau' :: \Sigma \isoto \Sigma'$ and $\alpha_i \neq \alpha_j \iff \tau'(\alpha_i) \neq \tau'(\alpha_j)$.
    Thus if $\alpha$ is separable, $\bar\tau(m_{\alpha, K})$ has $d$ distinct roots in $L$.
    By corollary~\ref{coro:num-of-extensions}, there are exactly $d$ monomorphisms
    $\sigma$ with $\sigma\big|_K = \tau$.

    Otherwise, there are $r$ roots in $L$ where $r < d$, and thus there are $r < d$ such monomorphisms.
  \end{proof}
\end{prop}

\begin{prop}\label{prop:separable-field-split-have-most-embeddings}
  Let $[K': K] = d$ and $\tau:: K \to L \neq 0$. Then $\quot{K'}{K}$ is separable and $\forall \alpha \in K'$,
  $\bar\tau(m_{\alpha, K})$ splits over $L$, if and only if there are exactly
  $d$ monomorphisms $\sigma::K' \to L$ with $\sigma\big|_k = \tau$.
  Otherwise $\exists r < d$ of such monomorphisms.

  \begin{proof}
    By induction on $d$, if $d = 1$ we could simply let $\sigma = \tau$.

    For $d > 1$, consider $\alpha \in K' \setminus K$.
    By prop \ref{prop:separable-and-split-have-most-embeddings}, there exists
    exactly $[K(\alpha): K]$ monomorphisms $\tau_1: K(\alpha) \to L$.

    Now, for any $\beta \in \quot{K'}{K(\alpha)}$,
    $m_{\beta, K(\alpha)} \mid m_{\beta, K}$ and thus $m_{\beta, K(\alpha)}$ is separable
    and $\bar\tau_1(m_{\beta, K(\alpha)})$ splits over $L$ since $\bar\tau(m_{\beta, K})$ splits.
    These imply that $\quot{K'}{K(\alpha)}$ is separable and $\forall \beta \in K'$,
    $m_{\beta, K(\alpha)}$ splits over $L$. Thus, $K(\alpha)$ satisfies the hypothesis,
    and by induction, there are exactly $[K': K(\alpha)]$ monomorphisms $\sigma :: K' \to L$
    such that $\sigma\big|_{K(\alpha)} = \tau_1$, thus there are $[K': K(\alpha)][K(\alpha): k]
    = [K': K]$ such monomorphisms.

    Otherwise, we could choose $\alpha \in K'$ such that $\bar\tau(m_{\alpha, K})$ has fewer
    then $[K(\alpha): K]$ roots in L, then there are $r' < [K(\alpha): K]$ monomorphism $\tau_1 :: K(\alpha)
    \to L$. By induction, each $\tau_1$ has $r''$ extensions $\sigma :: K' \to L$ and $r'' \leq [K': K(\alpha)]$
    Hence the number of monomorphism equals $r' r'' < [K': K]$.
  \end{proof}
\end{prop}

\begin{lemma} \label{lemma:element-in-alg-ext-splits}
  If $\quot{K(\alpha_1, \alpha_2, \dots, \alpha_n)}{K}$ is algebraic and $L$ is a splitting field of
  $f(x) = \prod_{i=1}^n m_{\alpha_i, K}$
  over $K$, then for all $\beta \in K(\alpha_1, \alpha_2, \dots, \alpha_n)$,
  $m_{\beta, K}$ also splits over $L$.

  \begin{proof}
    Let $L = K(R)$ with $R$ being the set of all roots of $f(x)$. Pick any root $\gamma$ of $m_{\beta, K}$.
    Observe the following diagram:
  \[
    \begin{tikzcd}
      K(R) \arrow[rr, "\sim", "\text{(2) } \sigma"'] & & K(R, \gamma) \\
      K(\beta) \arrow[u, hookrightarrow] \arrow[rr, "\sim", "\text{(1) } \tau"'] & & K(\gamma) \arrow[u, hookrightarrow] \\
      & K \arrow[ur, hookrightarrow] \arrow[ul, hookrightarrow] &
    \end{tikzcd}
  \]
  Where (1) holds because these fields are both isomorphic to $\quot{K[x]}{\gen{ m_{\beta, K} }}$. \\
  (2) holds because $\tau$ obviously fixes $K$, and hence $K(R)$ is a splitting field of $f$
  and $K(R, \gamma)$ is a splitting field of $\bar\tau(f)$. By theorem~\ref{thm:two-splitting-field-are-isom},
  $K(R)$ and $K(R, \gamma)$ are isomorphic.

  Thus we have $[K(R): K] = [K(R, \gamma): K]$ along with
  $[K(R, \gamma): K] = [K(\gamma, R): K(R)][K(R): K]$.
  This implies $[K(\gamma, R): K(R)] = 1$, hence $\gamma \in R$.
  \end{proof}
\end{lemma}

\begin{theorem} \label{thm:separable-elements-form-separable-field}
  Given $\quot{K(\alpha_1, \alpha_2, \dots, \alpha_n)}{K}$, if $\alpha_i$ is
  separable over $K_{i-1} \triangleq K(\alpha_1, \dots, \alpha_{i-1})$, then
  $\quot{K(\alpha_1, \alpha_2, \dots, \alpha_n)}{K}$ is separable.

  \begin{proof}
    Let $L$ be a splitting field of $f(x) = \prod m_{\alpha_i, K}$.

    We claim that there are $[K_j: K]$ monomorphisms $\tau_j:: K_j \to L$ with $\tau_j \big|_K = \Id_K$.
    Use induction on $j$, if $j = 0$, then there are only $1$ such monomorphism, namely itself $\Id_K$.

    For $j > 0$, observe that $m_{\alpha_j, K_{i-1}} \mid m_{\alpha_j, K}$, and since $\bar\tau_{j-1}
    (m_{\alpha_j, K}) = m_{\alpha_j, K}$ splits over $L$, $m_{\alpha_j, K_{i-1}}$ also splits over $L$.
    By hypothesis, $\alpha_j$ is separable over $K_{j-1}$, so by
    prop~\ref{prop:separable-and-split-have-most-embeddings},
    there are $[K_j: K_{j-1}]$ such monomorphisms $\tau_j:: K_j \to L$ with $\tau_j \big|_{K-1} = \tau_{j-1}$.
    By induction, there are $[K_{j-1}: K]$ monomorphisms $\tau_{j-1}:: K_{j-1} \to L$
    with $\tau_j \big|_K = \Id_K$. Compose these monomorphisms, we know that there
    exist exactly $[K_j: K_{j-1}][K_{j-1}: K] = [K_j: K]$ monomorphisms $\tau_j:: K_j \to L$
    such that $\tau_j \big|_K = \Id_K$.

    So there are exactly $[K_n: K]$ monomorphisms $\tau :: K(\alpha_1, \dots, \alpha_n) \to L$
    with $\tau\big|_K = \Id_K$.
    By prop~\ref{prop:separable-field-split-have-most-embeddings}, $K(\alpha_1, \dots, \alpha_n)$ is separable.
  \end{proof}
\end{theorem}

\begin{theorem}
  $\quot{L}{K}$ is separable if and only if $\quot{L}{M},\, \quot{M}{K}$ are separable.

  \begin{proof}
    ``$\Rightarrow$'': If $L/K$ is separable, then $M/K$ is obviously separable. For any $\beta \in L$,
    $m_{\beta, M} \mid m_{\beta, K}$ so $m_{\beta, M}$ is separable which implies $L/M$ is separable.

    ``$\Leftarrow$'': For any $\alpha \in L$, write $m_{\alpha, M} = x^n + a_{n-1}x^{n-1} + \dots + a_1 x + a_0$.
    Then $m_{\alpha, M}$ is separable implies $\alpha$ is separable over $K(a_0, \dots, a_{n-1})$.
    Note that $a_0, \dots, a_{n-1} \in M$ are separable over $K$.
    By theorem~\ref{thm:separable-elements-form-separable-field},
    $K(a_0, a_1, \dots, a_{n-1}, \alpha) / K$ is separable, hence each $\alpha$ is separable over $K$,
    thus $L/K$ is separable.
  \end{proof}
\end{theorem}

\begin{theorem}[Primitive element theorem] \hfill
  \begin{itemize}
    \item A finite extension is simple if and only if there are only finitely many intermediate fields.
    \item If $L/K$ is finite and separable, then $L/K$ is simple.
  \end{itemize}
\end{theorem}
