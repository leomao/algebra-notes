\subsection{Week 3}

\begin{prop}
  The following are equivalent.
  \begin{enumerate}
    \item $K$ has no nontrivial algebraic extension.
    \item For all irreducible polynomial in $K[x]$ has degree $1$.
    \item Every polynomial of  positive degree in $K[x]$ has at least one root in $K$.
    \item Every polynomial of  positive degree in $K[x]$ splits over $K$.
  \end{enumerate}
\end{prop}

\begin{lemma} \label{lemma:max-ideal-exists}
  In a commutative ring $R$ with $1$, any proper ideal $I \subsetneq R$ is contained in a maximal ideal.

  \begin{proof}
    Consider $S = \{ J \subsetneq R \mid I \subseteq J \} \neq \varnothing$ since $I \in S$.
    Define a partial order on $S$ by $J_1 \prec J_2 \iff J_1 \subseteq J_2$.

    Given a chain $\{ J_i \mid i \in \Lambda \}$, let $J = \bigcup_{i \in \Lambda} J_i$ which is a proper
    ideal, since if $x, y \in J$, then $x \in J_1, y \in J_2$. Let $\tilde{J} = \max(J_1, J_2)$, then $x, y \in \tilde{J}$
    which implies $x + y \in \tilde{J}$. So $J$ is a least upper bound for this chain.

    By Zorn's leemma, exists a max element in $S$ which is a max ideal in $R$ containing $I$.
  \end{proof}
\end{lemma}

\begin{theorem}
  If $K$ is a field, then exists an algebraic closure $L$ of $K$.

  \begin{proof}
    Let $S = \{ x_f \mid f(x) \in K[x] \text{ with } \deg f \geq 1 \}$ be the set of variables indexed by non-constant
    polynomial in $K[x]$. Consider the polynomial ring $K[S]$ and $I = \langle f(x_f) : f \in K[x] \text{ with } \deg f \geq 1 \rangle$,
    which is an ideal in $K[S]$. We claim that $I \neq K[S]$. If not then $1 \in I \implies 1 = \sum_{i = 1}^n g_i f_i(x_{f_i})$.
    Write $x_i \triangleq x_{f_i}$ for $i = 1, 2, \cdots, n$. Also, by definition $g_i$ only involves a finite number of
    variable in $S$, so we could set $g_i \in K[x_1, x_2, \cdots, x_m]$ with $m \geq n$. That is, $1 = \sum_{i = 1}^n g_i(x_1, x_2,
    \cdots, x_m) f_i(x_i)$. Let $\Sigma$ be a splitting field for $f(x) = f_1(x) f_2(x) \cdots f_n(x)$ and define $\alpha_i \in \Sigma$
    which satisfied $f_i(\alpha_i) = 0$ and $a_i = 0$ for $n+1 \leq i \leq m$. Then
    $1 = \sum_{i = 1}^n g(\alpha_1, \alpha_2, \cdots, \alpha_m) f_i(\alpha_i) = 0$ which leads to an contradiction.

    By lemma~\ref{lemma:max-ideal-exists}, exists a maximal ideal $M$ satisfied $I \subseteq M$.

    Consider $K \hookrightarrow F_1 \triangleq K[S] / M$, and then for all $f \in K[x]$, $f(\bar{x}_f) = \bar{0}$ in $F_1$.
    By induction, exists $F_1 \subseteq F_2 \subseteq F_3 \subseteq \cdots$ which satisfied $f(x) \in F_n[x]$ has a root in $F_{n+1}$
    Let $F = \bigcup_{i = 1}^\infty F_i$ which is algebraically closed since if $f(x) \in F[x]$ then $f(x) \in F_m[x]$
    for some $m$ and thus $f(x)$ has a root in $F_{m+1} \subseteq F$.

    Finally $L \triangleq \{ \alpha \in F \mid \alpha \text{ is algebraic over } K \}$ is an algebraic closure of $K$.
  \end{proof}

  \begin{lemma} \label{lemma:homo-extend-to-alg-closed-extension}
    If $L_1 / K$ is algebraic and $\tau: K \to L_2$ is a non-zero homomorphism with $L_2$ being algebraically closed,
    then $\tau$ extends to $\sigma: L_1 \to L_2$.

    \begin{proof}
      Consider $S = \{ (M, \theta) \mid K \subset M \subset L_1, \theta: M \to L_2 \text{ with } \theta\big|_K = \tau\}$
      which is not an empty set since $(K, \tau) \in S$.

      Define a partial order on $S$, $(M_1, \theta_1) \prec (M_2, \theta_2) \iff M_1 \subseteq M_2, \theta_2 \big|_{M_1} = \theta_1$.
      Given any chain $\{(M_i, \theta_i) : i \in \Lambda \}$, let $N = \bigcup_{i = 1}^\infty M_i$ and
      $\theta = \alpha :: N \mapsto \theta_i(\alpha)$ if $\alpha \in M_i$. This map is well defined, and is
      a  least upper bound for this chain. By Zorn's lemma, exists a max element $(M, \sigma)$ in $S$.
    \end{proof}
  \end{lemma}

  \begin{theorem}
    Any two algebraic closures $L_1, L_2$ of $K$ are isomorphic.
    \begin{proof}
      Consider the inclusion map $\text{id}_K :: K \hookrightarrow L_1$.
      By Lemma~\ref{lemma:homo-extend-to-alg-closed-extension},
      $\tau$ could be extend to $\sigma :: L_2 \to L_1$ such that $\sigma\big|_K = \text{id}_K$.
      Also, $L_2$ is algebraically closed which implies $\sigma(L_2)$ is algebraically closed.
      So for any $\alpha \in L_1$, $\alpha$ is algebraic over $K$ and thus over $\sigma(L_2)$,
      which implies $\alpha \in \sigma(L_2)$, so $\sigma$ is onto, hence $\sigma$ is an
      isomorphism between $L_1$ and $L_2$.
    \end{proof}
  \end{theorem}

  \begin{example}
    Let $p$ be a prime.
    \begin{itemize}
      \item Any finite field $L$ with $\Char L = p$
    \end{itemize}
  \end{example}
\end{theorem}

