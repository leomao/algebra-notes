%! TEX root=../main.tex
\subsection{Abelian extension (week 5)}

\begin{definition}
  $L/K$ is called an abelian extension if $L/K$ is Galois and $\Gal(L/K)$ is abelian.
\end{definition}

\begin{example}
  For an extension $\Fb_{q^n} / \Fb_q$ of a finite field, $\Fb_{q^n}$ is a splitting field of $x^{q^n}-x$
  over $\Fb_p$, so $\Fb_{q^n} / \Fb_q$ is Galois by theorem~\ref{thm:splitting-field-iff-finite-normal}.
  By theorem~\ref{thm:aut-of-finite-field}, we know that $\Gal(F_{q^n} / F_q) = \gen{\sigma_q}$
  is a cyclic group.
\end{example}

\begin{definition} \hfill
  \begin{itemize}
    \item The cyclotomic field $\Qb(\zeta_n)$ is the splitting field of $x^n - 1$ over $\Qb$.
    \item $\zeta$ is called an $n$th root of unity if $\zeta^n = 1$. $\mathcal{U} = \gen{\zeta}$
      is the multiplicative group of $n$th roots of unity.
    \item $\zeta_n$ is called a primitive $n$th root of unity if $\zeta^n = 1$ but
      $\zeta^m \ne 1, \, \forall 0 < m < n$.
    \item The $n$th cyclotomic polynomial is defined as
      \[ \Phi_n \triangleq \prod_{\gcd(k, n) = 1} (x - \zeta_n^k) \implies \deg \Phi_n = \varphi(n) \]
  \end{itemize}
\end{definition}

\begin{prop} \hfill
  \begin{itemize}
    \item $x^n - 1 = \prod_{d \mid n} \Phi_d$.
      \begin{proof}
        First, Both sides have no multiple root. Then since $\alpha^n = 1 \iff \ord_{\times}(\alpha) \mid n$,
        we know that two sides has equal roots.
      \end{proof}
    \item $\Phi_n \in \Zb[x]$.
      \begin{proof}
        By induction on $n$. $n = 1$ is trivial.
        Assume that the statement is true for all $k < n$, then since
        \[ x^n - 1 = \Phi_n \prod_{\substack{d \mid n, d < n}} \Phi_d \triangleq \Phi_n \Phi_{< n} \]
        But notice that $\Phi_{<n}$ is monic, so by the long division algorithm, it is easy to
        see that $\Phi_n = (x^n - 1) / \Phi_{<n}$ has all coefficients in $\Zb$.
      \end{proof}
    \item $\Phi_n$ is irreducible.
      \begin{proof}
        Suppose $\Phi_n = f(x) g(x)$ with $f$ irreducible, and both $f, g$ are monic.
        By Gauss's lemma, we could assume $f(x), g(x) \in \Zb[x]$.
        Let $\zeta_n$ be a primitive $n$th root of unity which satisfied $f(\zeta_n) = 0$
        and $p$ be a prime with $p \nmid n$.

        Assume that $g(\zeta_n^p) = 0$, $m_{\zeta_n, \Qb} = f \implies f \mid g(x^p)$,
        say $g(x^p) = f(x) h(x)$.
        By the long division algorithm, we know that $h(x) \in \Zb[x]$, since $f(x) \in \Zb[x]$
        and monic.

        In $\Zb / p\Zb[x]$, we have $\bar{g}(x)^p = \bar{g}(x^p) = \bar{f}(x) \bar{h}(x)$,
        which implies $\bar{g}, \bar{f}$ has common root, thus $\bar\Phi_n = \bar{f}\bar{g}$ and
        hence $x^n - \bar{1}$ has a multiple root.
        But $(x^n - \bar{1})' = nx^{n-1} \neq 0$, and $0$ is not a root of $x^n - \bar{1}$,
        which leads to a contradiction.

        So we conclude that $f(\zeta_n^p) = 0$ for any $p \nmid n$, which could be extended
        and show that $f(\zeta_n^k) = 0$ for any $\gcd(k, n) = 1$, thus $f = \Phi_n$.
      \end{proof}
    \item $\Qb(\zeta_n) / \Qb$ is Galois with $[\Qb(\zeta_n): \Qb] = \deg \Phi_n = \varphi(n)$.
    \item $\Gal(\Qb(\zeta_n) / \Qb) \cong (\quot{\Zb}{n\Zb})^\times$.

      \begin{proof}
        Let $\sigma_k = (\zeta_n \mapsto \zeta_n^k) \in \Gal(\Qb(\zeta_n) / \Qb)$.
        The isomorphism is given by $\sigma_k \mapsto \bar{k}$.
        Clearly, it is a homomorphism since $\sigma_k \sigma_h = (\zeta_n \mapsto \zeta_n^{kh}) = \sigma_{kh}$.
        Also $\sigma_k = 1 \iff \bar{k} = 1$. Finally, $\abs{\Gal(\Qb(\zeta_n) / \Qb)} = \abs{\Fb_n^\times} = \varphi(n)$,
        so the map is onto.
      \end{proof}
    \item Suppose $n = p_1^{n_1} p_2^{n_2} \cdots p_k^{n_k}$ with $p_1, \dots, p_k$ are distinct primes.
      Define $L_i \triangleq \Qb\big(\zeta_{p_i^{n_i}}\big)$. Obviously,
      $L_i \subseteq \Qb(\zeta_n)$ hence $L_1 L_2 \dotsm L_k \subseteq \Qb(\zeta_n)$,
      but $\zeta_n = \zeta_{p_1^{n_1}} \zeta_{p_2^{n_2}} \dotsm \zeta_{p_k^{n_k}}$,
      so $L_1 L_2 \dotsm L_k \supseteq \Qb(\zeta_n)$. Thus we have
      $L_1 L_2 \dotsm L_k = \Qb(\zeta_n)$.
  \end{itemize}
\end{prop}

\begin{example}
  Let $n = p$ be a prime.
  \begin{itemize}
    \item $\Gal(\Qb(\zeta_p) / \Qb) = \Fb_p^\times = \Zb / (p-1)\Zb$.
    \item For $H \leq \Gal(\Qb(\zeta_p) / \Qb)$, we shall find $\Qb(\zeta_p)^H$.
      Let $\alpha = \sum_{\tau \in H} \tau(\zeta_p)$, then it is easy to
      see that $\alpha \in \Qb(\zeta_p)^H$. Also, since $[\Qb(\zeta_p): \Qb] = p-1$,
      $\zeta_p, \zeta_p^2, \dots, \zeta_p^{p-1}$ is linearly independent,
      so if some $\sigma \in G$ satisfy $\sigma(\alpha) = \alpha$, then since
      both $\sigma(\alpha), \alpha$ are a sum of linearly independent elements,
      $\sigma$ must send $\zeta_p$ to an element $\tau(\zeta_p)$ for some $\tau \in H$,
      then $\sigma = \tau \implies \sigma \in H$. Thus $\Qb(\zeta_p)^H = \Qb(\alpha)$.
  \end{itemize}
\end{example}

\begin{lemma}
  If $L_1 / K$, $L_2 / K$ are Galois, then $L_1 \cap L_2 / K$, $L_1 L_2 / K$ are Galois and
  \[ \Gal( L_1 L_2 / K) \cong \big\{ (\sigma, \tau) \bigm| \sigma\big|_{L_1 \cap L_2} = \tau\big|_{L_1 \cap L_2} \big\}
    \leq \Gal(L_1 / K) \times \Gal(L_2 / K) \]
  In particular, if $L_1 \cap L_2 = K$, then $\Gal(L_1 L_2 / K) \cong \Gal(L_1 / K) \times \Gal(L_2 / K)$.

  \begin{proof}
    We know that $L_1 \cap L_2 / K$ is finite and separable. Also, for each $\alpha \in L_1 \cap L_2$,
    $m_{\alpha, K}$ splits in both $L_1, L_2$ since they are normal, thus $m_{\alpha, K}$ splits
    in $L_1 \cap L_2$, hence $L_1 \cap L_2 / K$ is galois.

    Similary, $L_1 L_2$ is finite and separable. Let $L_1$ be the splitting field of $f_1$,
    and $L_2$ be the splitting field of $f_2$, then $L_1 L_2$ is the splitting field
    of the square-free part of $f_1 f_2$, hence $L_1 L_2 / K$ normal.

    Define $\varphi = \sigma :: \Gal(L_1 L_2 / K) \mapsto \big( \sigma \big|_{L_1}, \sigma \big|_{L_2} \big)
    :: \Gal(L_1 / K) \times \Gal(L_2 / K)$. Since $L_1, L_2$ are normal,
    by proposition~\ref{prop:TFAE-of-normal-extension}, $\sigma \big|_{L_i}(L_i) = L_i$ so they are well-defined.
    Also, it is clear that the map is 1-1.

    Now we count the number of the pair $(\sigma \big|_{L_1}, \sigma \big|_{L_2})$,
    There are $[L_1: K]$ of $\tau = \sigma \big|_{L_1}$, and fixing one, each $\sigma \big|_{L_2}$
    is an extension of $\tau \big|_{L_1 \cap L_2}$, so there are $[L_2 : L_1 \cap L_2]$ of such.
    On the other hand, we have $\abs{\Gal(L_1 L_2 / K)} = [L_1 L_2 : K] = [L_1 L_2 : L_1]
    [L_1 : K] = [L_2 : L_1 \cap L_2] [L_1 : K]$, thus $\Gal(L_1 L_2 / K)$ and
    $\Set{(\sigma \big|_{L_1}, \sigma \big|_{L_2})}$ has the same size, and hence the
    map is onto.
  \end{proof}
\end{lemma}

Back to our problem, $[L_1 L_2 \cdots L_k: \Qb] = [\Qb(\zeta_n): \Qb] = \varphi(n) = \varphi(p_1^{n_1})
\cdots \varphi(p_k^{n_k}) = [L_1 : \Qb] [L_2 : \Qb] \cdots [L_k : \Qb]$, thus
\[ \Gal\big( \Qb(\zeta_n) / \Qb \big) \cong
  \Gal\big( \Qb(\zeta_{p_1^{n_1}}) / \Qb \big) \times
  \Gal\big( \Qb(\zeta_{p_1^{n_2}}) / \Qb \big) \times \cdots \times
  \Gal\big( \Qb(\zeta_{p_1^{n_k}}) / \Qb \big) \]

\begin{theorem}
  Let $G$ be a finite abelian group. Then there exists a subfield $L$ of
  a cyclotomic field satisfying $\Gal(L / \Qb) \cong G$.

  \begin{proof}
    By the FTFGAG,
    \[ G \cong \Zb / n_1 \Zb \times \Zb / n_2 \Zb \times \dots \times \Zb / n_k \Zb \]
    By dirichlet theorem, there are infinitely many prime $p$ such that $n \mid p - 1$.
    Let $p_i$ be a prime such that $n_i \mid p_i - 1$ and all $p_i$ are distinct.
    Then $G$ is a subgroup of $\Zb / (p_1 - 1) \Zb \times \dots \times \Zb / (p_k - 1) \Zb \cong \Gal(\Qb(\zeta_n) / \Qb)$
    where $n = p_1 p_2 \dotsm p_k$.
  \end{proof}
\end{theorem}

\subsubsection{Kummer extension}
In this section, we assume that $\Char K \nmid n$ and $\zeta$ is a primitive $n$th root of unity.

\begin{definition} \hfill
  \begin{itemize}
    \item $L/K$ is called a kummer extension of exponent $n$ if $\zeta \in K$ and $L$ is a splitting field
      of $(x^n - a_1) (x^n - a_2) \dotsm (x^n - a_k)$ over $K$.
    \item Let $\abs{G} < \infty$, the exponent $e(G)$ of $G$ is the least positive integer $m$
      satisfying $g^m = 1$ for any $g \in G$.
  \end{itemize}
\end{definition}

\begin{theorem}
  Let $L$ be a splitting field of $x^n - a$ over $K$, then $\Gal(L / K(\zeta))$ is cyclic of
  degree dividing $n$. More over $x^n - a$ is irreducible over $K(\zeta)$ $\iff$ $[L: K(\zeta)] = n$.

  \begin{proof}
    If $\alpha$ is a root of $x^n - a$, then $\alpha, \alpha \zeta, \dots, \alpha \zeta^{n-1}$
    are roots of $x^n - a$, so $L = K(\alpha, \zeta) = K(\zeta)(\alpha)$.

    Consider $\deffunc{\varphi}{\Gal(L / K(\zeta))}{\Zb / n\Zb}{(\alpha \mapsto \alpha \zeta^k)}{\bar{k}}$.
    It is easy to see that $\varphi$ is a homomorphism. Also, if $\varphi(\sigma) = 0$,
    $\sigma = (\alpha \mapsto \alpha) = \Id$. Thus $\varphi$ is 1-1 and
    $\Gal(L / K(\zeta)) \toone \quot{\Zb}{n\Zb}$.
  \end{proof}
\end{theorem}

\begin{definition}
  $L/K$ is called a cyclic extension if $L/K$ is Galois and $\Gal(L/K)$ is cyclic.
\end{definition}

\begin{theorem} \label{thm:kummer-base-theorem}
  If $L/K$ is a cyclic extension of degree $n$ and $\zeta \in K$, then $L$ is a splitting field of
  some irreducible polynomial $x^n - a$ over $K$.

  \begin{proof}
    Recall a result proved in HW problem: Distinct automorphisms of $L$ are linearly independent over $L$.
    
    Let $\Gal(L/K) = \gen{\sigma}$ with $\ord(\sigma) = n$. Then
    $\Id_L + \zeta \sigma + \zeta^2 \sigma^2 + \dots + \zeta^{n-1} \sigma^{n-1} \neq 0$
    \[ \implies \exists c \in L, \text{ s.t. } \alpha = c + \zeta \sigma(c) + \zeta^2 \sigma^2(c)
    + \dots + \zeta^{n-1} \sigma^{n-1}(c) \neq 0 \]

    Observe that $\sigma(\alpha) = \zeta^{-1} \alpha$, so $\alpha \notin K$. Also $\sigma(\alpha^n)
    = \sigma(\alpha)^n = \zeta^{-n}\alpha^n = \alpha^n$, so $\alpha^n$ is fixed by
    $\Gal(L/K)$, thus $a \triangleq \alpha^n \in K$, and hence $K(\alpha)$
    is a splitting field of $x^n - a$ over $K$.

    Now $\sigma(\alpha) = \zeta^{-1}\alpha \in K(\alpha)$, so $\sigma\big|_{K(\alpha)} \in \Gal(K(\alpha) / K)$.
    Also $\sigma^k(\alpha) = \zeta^{-k} \alpha \implies \ord(\sigma) = n$.
    Thus
    \[
      n = [L: K] \geq [K(\alpha): K] = \Gal(K(\alpha) / K) \geq n \implies L = K(\alpha)
      \qedhere
    \]
  \end{proof}
\end{theorem}

\begin{theorem}
  Let $L/K$ be a Galois extension such that $\Gal(L/K)$ is abelian of exponent $n$
  and $\zeta_n \in K$, then $L/K$ is a Kummer extension.

  \begin{proof}
    By induction on $[L: K]$. If $[L: K] = 1$ then $L = K$ and is trivial.

    Assume $[L: K] > 1$, then by FTFGAG, $G \triangleq \Gal(L/K) \cong \quot{\Zb}{d_1 \Zb}
    \times \quot{\Zb}{d_2 \Zb} \times \dots \times \quot{\Zb}{d_s \Zb}$ with $d_i \mid d_{i+1}$.
    If $s = 1$ then the theorem degenerates to theorem~\ref{thm:kummer-base-theorem}.

    So assume $s > 1$. Let $H = \Zb / d_1 \Zb \times \dots \times \Zb / d_{s-1} \Zb,
    N = \Zb / d_s \Zb$ be the corresponding subgroup in $\Gal(L/K)$.
    Set $M = L^N$, we have $[M: K] \lneq [L: K]$. Since any subgroup
    of abelian group is normal, we have $\Gal(M/K) \cong \Gal(L/K) / \Gal(L/M) = G / N = H$.

    Denote $m = d_{s-1}, n = d_{s}$, we have $m \mid n$.
    Then $\zeta_n \in K \implies \zeta_m = \zeta_n^{n/m} \in K$,
    thus we could pass down the induction, and assume $M$ is a kummer extension which is a splitting
    field of $g = (x^m - b_1) (x^m - b_2) \dotsm (x^m - b_{k-1})$ over $K$ with each $b_i \in K$.
    Let $\alpha_1, \dots, \alpha_{k-1}$ be all the roots of $g$, then $\alpha_i$
    is also a root of $(x^n - b_1^{n/m})$. Thus if we define $a_i \triangleq b_i^{n/m}$, then
    $M$ is also the splitting field of $(x^n - a_1) (x^n - a_2) \dotsm (x^n - a_{k-1})$ over $K$
    since $\zeta_n \in K$.

    Now, if $N = \gen{\sigma}$, then $G \cong H \times N = \{\sigma^k \tau : 0 \leq k < n, \tau \in H\}$.
    Since automorphisms are linearly independent, exists $c \in L$ satisfied
    \[ 0 \neq \alpha = \sum_{\tau \in H} \tau(c) + \zeta \sum_{\tau \in H} \sigma \tau(c)
    + \dots + \zeta^{n-1} \sum_{\tau \in H} \sigma^{n-1} \tau(c) \]
    Then $\sigma(\alpha) = \zeta^{-1} \alpha$, so $\alpha \notin M$. Also $\sigma(\alpha^n) = \alpha^n$
    and $\tau(\alpha^n) = \tau(\alpha)^n = \alpha^n$, so $a_k \triangleq \alpha^n \in K$.
    Thus $M(\alpha)$ is a splitting field of $(x^n - a_k)$ over $M$.

    Finally, $n = [L: M] \geq [M(\alpha): M] = \abs{\Gal(M(\alpha) / M)} \geq n$,
    thus $L = M(\alpha)$, and hence $L$ is a splitting field of
    $(x^n - a_1) (x^n - a_2) \dotsm (x^n - a_k)$.
  \end{proof}
\end{theorem}

\begin{theorem}
  If $L/K$ is a kummer extension of exponent $n$, then $\Gal(L/K)$ is abelian of exponent dividing $n$.

  \begin{proof}
    Let $L$ be the splitting field of $(x^n - a_1)(x^n - a_2) \dotsm (x^n - a_k)$ with $\alpha_i = \sqrt[n]{a_i}$.
    If $\sigma \in \Gal(L/K)$, then $\sigma$ sends each $\alpha_i$ to some $\zeta^{k_{\sigma, i}} \alpha_i$.
    So $\sigma^n = \alpha_i \mapsto \zeta^{k_{\sigma, i} n} \alpha_i = \alpha_i \mapsto \alpha_i = \Id$
    and $\sigma \circ \tau = \alpha_i \mapsto \zeta^{k_{\sigma, i} + k_{\tau, i}} \alpha_i = \tau \circ \sigma$.
    by the fact that  $\Set{\alpha_i}$ is the generating set of $L$. Hence $\Gal(L/K)$ is abelian and of
    exponent dividing $n$.
  \end{proof}
\end{theorem}

\subsubsection{Cubic equations}

\begin{lemma}
  Let $\Char K \neq 2$ and $f(x) \in K[x]$ with $\deg f = n$. Let
  $L = K(\alpha_1, \dots, \alpha_n)$ be a splitting field of $L$ over $K$.

  Define $\delta = \prod\limits_{1 \leq i < j \leq n} (\alpha_i - \alpha_j)$, then
  $L^{\Gal(L/K) \cap A_n} = K(\delta)$.  (Here $\Gal(L/K) \toone S_n$)

  \begin{proof}
    Notice that any transposition maps $\delta$ to $-\delta$, so
    $\forall \sigma \in \Gal(L/K) \cap A_n$, $\sigma(\delta) = \delta$, thus
    $K(\delta) \subseteq L^{\Gal(L/K) \cap A_n}$.

    Now, $\big| \quot{\Gal(L/K)}{\Gal(L/K) \cap A_n} \big|$ is either $1$ or $2$.
    If it is $1$, then $\Gal(L/K) \leq A_n$ ,thus $\delta \in K$ and is trivial.
    Assume it is $2$, then $\delta$ is not fixed by all permutation, thus
    $\delta \notin K$. But $D = \delta^2 \in K$ is the discriminant. So we have
    $2 = [K(\delta): K] \le [L^{\Gal(L/K) \cap A_n}: K] =
    \abs*{\Gal \big( L^{\Gal(L/K) \cap A_n}/K \big)} = 2$, thus
    $K(\delta) = L^{\Gal(L/K) \cap A_n}$.
  \end{proof}
\end{lemma}

\begin{prop}
  Let $f(x) = x^3 + px + q$ be irreducible in $K[x]$ and $L$ be a splitting field,
  \begin{itemize}
    \item If $\Gal(L/K) \cong S_3$ then $\sqrt{D} \notin K$.
    \item If $\Gal(L/K) \cong A_3$ then $\sqrt{D} \in K$.
  \end{itemize}
\end{prop}

\begin{definition}
  $H \leq S_n$ is said to be transitive if for any $i, j$, there exists
  $\sigma \in H$ such that $\sigma(i) = j$.
\end{definition}

\begin{fact}
  Let $f(x)$ be a separable polynomial with degree $n$, then
  \[ f(x) \text{ is irreducible} \iff
  \text{ The Galois group of } f \text{ is transitive in } S_n \]
\end{fact}
