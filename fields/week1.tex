%! TEX root=../main.tex
\section{Fields}

\subsection{Algebraic extensions}

\begin{definition} \hfill
  \begin{itemize}
    \item $L / K$ is called an {\bf field extension}\index{Field extension} if $L$ is a field and $K$ is a subfield of $L$.
    \item $\alpha \in L$ is {\bf algebraic}\index{algebraic element}
      over $K$ if exists $f(x) \in K[x]$ satisfied $f(\alpha) = 0$.
    \item $L / K$ is called an {\bf algebraic extension} \index{Field extension!algebraic extension}
      if $\forall \alpha \in L, \exists f(x) \in K[x]$
      such that $f(\alpha) = 0$.
    \item $K(\alpha_1, \alpha_2, \dots, \alpha_n) \triangleq \big\{ P(\alpha_1, \dots, \alpha_n)
      / Q(\alpha_1, \dots, \alpha_n) : P, Q \in K[x_1, x_2, \dots, x_n] \text{ and } Q \neq 0 \big\}$
  \end{itemize}
\end{definition}

\begin{theorem}[Eisenstein criterion] \mbox{} \\
  Let $f(x) = a_n x^n + \dots + a_1 x + a_0 \in \Zb[x]$ with $\gcd(a_0, a_1, \cdots, a_n) = 1$.
  Assume that there exists a prime $p$ s.t. $p \nmid a_n$ but $p \mid a_i$ for other $i \neq n$,
  and $p^2 \nmid a_0$, then $f$ is irreducible.

  \begin{proof}
    Since $f$ is primitive, by Gauss lemma, we only need to prove that it is irreducible in $\Qb[x]$.
    Consider $\bar{f}(x)$, by assumption, $\bar{f}(x) = \bar{a}_n x^n$. So if $f(x) = g(x) h(x)$
    with $\deg g, \deg h \geq 1$, let $g(x) = b_r x^r + \dots + b_0, h(x) = c_{n-r} x^{n-r} + \dots + c_0$,
    then $\bar{g}(x) = \bar{b}_r x^r, \bar{h}(x) = \bar{c}_{n-r} x^{n-r}$ for some
    $r$. But then we would find out that $\bar{b}_0 = \bar{c}_0 = 0$, and thus $p^2 \mid a_0$,
    which is a contradiction, hence $f$ is irreducible.
  \end{proof}
\end{theorem}

\begin{prop}
  Given $L/K$ and $\alpha \in L$, if $\alpha$ is algebraic over $K$, then
  there exists a unique monic irreducible polynomial $m_{\alpha, K}(x) \in K[x]$
  of minimal degree s.t. $m_{\alpha, K}(\alpha) = 0$ and for any other
  $f(x) \in K[x]$ with $f(\alpha) = 0$, we have $m_{\alpha, K} \mid f$.
  We call $m_{\alpha, K}$ the {\bf minimal polynomial} of $\alpha$ over $K$.

  \begin{proof}
    %Consider the evaluation map on $\alpha$:
    %\[ \deffunc{\text{ev}_\alpha}{K[x]}{K[\alpha]}{f(x)}{f(\alpha)} \]
    %The map is

    Let $I$ be the set of all polynomials such that $f(\alpha) = 0$, since $\alpha$ algebraic,
    $I \neq \varnothing$, so pick a monic polynomial $g(x)$ of minimal degree in $I$.
    For any other $f(x) \in I$, write $f(x) = g(x) q(x) + r(x)$ with $\deg r < \deg g$.
    If $r(x) \neq 0$, then $r(\alpha) = f(\alpha) - q(\alpha) g(\alpha)$.
    But then $r(\alpha) = f(\alpha) - q(\alpha) g(\alpha) = 0$ with $\deg r < \deg g$,
    which contradicts the minimality of $g$, thus $r = 0$, and hence $g \mid f$.

    Finally, if $g(x) = h_1(x) h_2(x)$ with $\deg h_1, \deg h_2 < \deg g$,
    then one of them, say $h_1(\alpha) = 0$ again contradicts the minimality of $g$,
    hence $g$ is irreducible.
  \end{proof}
\end{prop}

\begin{prop}
  Let $L/K$ be an extension and $\alpha \in L$, the following are equivalent:
  \begin{enumerate}[(\arabic*)]
    \item $\alpha$ is algebraic over $K$.
    \item $K[\alpha] = K(\alpha)$.
    \item $[K(\alpha): K] < \infty$.
  \end{enumerate}

  \begin{proof}
    (1) $\Rightarrow$ (2): ``$\subset$'' trivial. \\
    ``$\supset$'': For all $\beta \in K(\alpha), \beta = g(\alpha) / h(\alpha)$ with $h(\alpha) \neq 0$.
    So $m_{\alpha, K} \nmid h$. Since $m_{\alpha, K}$ is irreducible, $\gcd(m_{\alpha, K}, h) = 1$,
    hence there exists $a(x), b(x) \in K[x]$ such that $1 = a(x) h(x) + b(x) m_{\alpha, K}(x)$
    Subsitute $\alpha$ and we get $1/h(\alpha) = a(\alpha)$, hence $\beta = g(\alpha) a(\alpha) \in K[\alpha]$.

    (2) $\Rightarrow$ (1): Since $1 / \alpha \in K[\alpha]$, thus $1 / \alpha = f(\alpha)$ for some
    polynomial $f$, hence if we set $g(x) = xf(x) - 1$, $g(\alpha) = 0$ which implies $\alpha$ is algebraic.

    (1) $\Rightarrow$ (3): Assume that $\deg m_{\alpha, K} = n$, it is easy to see that
    $K[\alpha] = \gen{ 1, \alpha, \dots, \alpha^{n-1} }_K$. Since (1) $\implies$ (2),
    we have $[K(\alpha): K] = [K[\alpha], K] = n$.

    (3) $\Rightarrow$ (1): Since $[K(\alpha): K] = n$, consider $1, \alpha, \alpha^2, \dots, \alpha^n$.
    Some of these $n+1$ elements may be coincident, but nevertheless these elements are linearly dependent.
    Hence there exists $a_0, \dots, a_n$ not all zero in $K$ s.t.
    $a_0 + a_1 \alpha + \dots + a_n \alpha^n = 0 \implies \alpha$ is algebraic.
  \end{proof}
\end{prop}

\begin{prop}
  Given $M/L$ and $L/K$, $[M: K] = [M: L] [L: K]$.

  \begin{proof}
    If $[M:L] = m < \infty$ and $[L:K] = n < \infty$, then $L \cong K^{\oplus n}, M \cong L^{\oplus m}$.
    So $M \cong \left( K^{\oplus n} \right)^{\oplus m} \cong K^{\oplus mn}$, thus $[M: K] = mn$.

    Now if $[M: K] = l < \infty$, then there exists a basis $\{ z_1, z_2, \dots, z_l \}$
    which is a basis for $M$ over $K$. Then $M = K z_1 + \dots + K z_l \subset L z_1 +
    \dots + L z_l \subset M \implies M = L z_1 + \dots + L z_l$. Hence $[M: L] < \infty$.
    Also, since $L$ is a $K$-linear subspace of $M$, $[L: K] \leq l \implies [L: K] < \infty$.
    Thus if $[M: L] = \infty$ or $[L: K] = \infty$, then $[M: K] = \infty$.
  \end{proof}
\end{prop}

\begin{prop} \label{prop:alg-elements-form-a-field}
  Given $L/K$, define $L^{\text{alg}} \triangleq \{ \alpha \in L \mid \alpha \text{ is algebraic over } K \}$,
  then $L^{\text{alg}}$ is a subfield of $L$.

  \begin{proof}
    Notice that if $\alpha, \beta \in L^{\text{alg}}$, then $\beta$ is algebraic over $K$
    implies that $\beta$ is algebraic over $K(\alpha)$. Thus
    \[ [K(\alpha, \beta): K] = [K(\alpha)(\beta): K(\alpha)] [K(\alpha): K]
      < \infty \]

    Also, since $K(\alpha + \beta), K(\alpha - \beta), K(\alpha \beta), K(\alpha / \beta)$ are
    all contained in $K(\alpha, \beta)$, they are all algebraic over $K$, thus
    these elements are all algebraic, and hence $L^{\text{alg}}$ is a subfield.
  \end{proof}
\end{prop}

\begin{prop}
  $[L: K] < \infty$ if and only if $L = K(\alpha_1, \alpha_2, \dots, \alpha_n)$ with each $\alpha_i$
  algebraic over $K$. In this case, $L / K$ is algebraic.

  \begin{proof}
    ``$\Rightarrow$'': Let $[L: K] = n$, so there is a basis $\{ \alpha_1, \alpha_2, \dots, \alpha_n \}$
    for $L$ over $K$. It is easy to see that $L = K(\alpha_1, \dots, \alpha_n)$.
    Also $[K(\alpha_i): K] \leq [L: K] < \infty$, thus $\alpha_i$ is algebraic.

    ``$\Leftarrow$'': Since $\alpha_i$ is algebraic over $K$, $\alpha_i$ is algebraic over $K(\alpha_1, \dots, \alpha_{i-1})$.
    Thus
    \[ [L: K] = [K(\alpha_1, \dots, \alpha_n): K(\alpha_1, \dots, \alpha_{n-1})]
      [K(\alpha_1, \dots, \alpha_{n-1}): K(\alpha_1, \dots, \alpha_{n-2})] \dots [K(\alpha_1): K] < \infty \]
    Moreover, $\forall \alpha \in L, [K(\alpha): K] \leq [L: K] < \infty$, so $\alpha$ is algebraic over $K$.
  \end{proof}
\end{prop}

\begin{coro}
  Given $L/K$, and $S$ a subset of $L$, if $\forall \alpha \in S$, $\alpha$ is
  algebraic over $K$, then $K(S) / K$ is algebraic.

  \begin{proof}
    If $\beta \in K(S)$, by definition we know that there exists
    $\alpha_1, \dots, \alpha_n$ such that $\beta \in K(\alpha_1, \dots, \alpha_n)$.
    Thus $\beta$ is algebraic over $K$.
  \end{proof}
\end{coro}

\begin{prop} \label{prop:alg-tower-implies-alg}
  If $M/L$ and $L/K$ are algebraic, then $M/K$ is algebraic.

  \begin{proof}
    For all $\alpha \in M$, since $\alpha$ is algebraic over $L$,
    there exists $a_{n-1}, \dots, a_0$ so that $\alpha^n + a_{n-1} \alpha^{n-1} + \dots + a_0 = 0$,
    that is, $\alpha$ is algebraic over $K(a_0, \dots, a_{n-1})$.

    So $[K(a_0, \dots, a_{n-1}, \alpha): K] = [K(a_0, \dots, a_{n-1})(\alpha): K(a_0, \dots, a_{n-1})]
    [K(a_0, \dots, a_{n-1}): K] < \infty$, thus $\alpha$ is algebraic over $K$.
  \end{proof}
\end{prop}

\begin{definition}
  Given $L/L_1$ and $L/L_2$, $L_1 L_2$ is defined as the smallest subfield of $L$
  containing both $L_1$ and $L_2$.
\end{definition}

\begin{prop}
  Let $[L_1: K] = m$ and $[L_2: K] = n$.
  \begin{enumerate}[(\arabic*)]
    \item $[L_1 L_2: K] \leq mn$.
    \item If $\gcd(m, n) = 1$, then $[L_1 L_2: K] = mn$.
  \end{enumerate}

  \begin{proof}
    (1): Assume $L_1 = K(\alpha_1, \dots, \alpha_m), L_2 = K(\beta_1, \dots, \beta_n)$.
    We could find that $L_1 L_2 = K(\alpha_1, \dots, \alpha_m, \beta_1, \dots, \beta_n)$.
    Notice that $[K(\beta_1, \dots, \beta_m)(\alpha_i): K(\beta_1, \dots, \beta_m)] \leq [K(\alpha_i): K]$,
    and thus $[L_1 L_2: K] = [K(\alpha_1, \dots, \alpha_m, \beta_1, \dots, \beta_n)
    : K(\beta_1, \dots, \beta_n)] [K(\beta_1, \dots, \beta_m): K] \leq [K(\alpha_i, \dots, \alpha_n): K]
    [K(\beta_1, \dots, \beta_n): K] = [L_1: K][L_2: K]$.

    (2): Notice that $[L_i: K] \mid [L_1 L_2: K]$, so $mn \mid [L_1 L_2: K]$. By
    (1), $[L_1 L_2: K] \leq nm$, hence $[L_1 L_2: K] = nm$.
  \end{proof}
\end{prop}

\begin{definition}
  Let $R$ be a commutative ring with $1$, and $I$ be an ideal of $R$, then
  \begin{itemize}
    \item $I$ is called a {\bf maximal ideal}\index{Ideal!maximal ideal} if for any ideal $J$ satisfying
      $I \subseteq J$ we have $J = I \text{ or } J = R$.
    \item $I$ is called a {\bf prime ideal}\index{Ideal!prime ideal}
      if $I \neq R$ and $ab \in I \implies a \in I \text{ or } b \in I$.
  \end{itemize}
\end{definition}

\begin{prop} \label{prop:max-prime-to-field-int-domain}
  Suppose $R$ is a ring and $I \subsetneq R$ is an ideal, then
  \begin{enumerate}
    \item $I$ is maximal $\iff$ $R / I$ is a field.
    \item $I$ is a prime ideal $\iff$ $R / I$ is an integral domain.
  \end{enumerate}

  \begin{proof} \hfill
    \begin{enumerate}
      \item ``$\Rightarrow$'': For any $\bar{r} \in R/I$ with $\bar{r} \neq 0$, then $r \notin I$.
        Consider $\gen{ r } + I$ which contains $I$ and is not equal to $I$ because $r \notin I$.
        Since $I$ is maximal, $\gen{ r } + I = R$, and thus $\exists x \in R, y \in I$ such that
        $xr + y = 1$, so $\bar{x} \bar{r} = \bar{1}$. Hence every non-zero element has multiply inverse
        and $R / I$ is a field.

      ``$\Leftarrow$'': If $J$ is an ideal such that $I \subsetneq J$, pick $x \in J \setminus I$,
      then $\bar{x} \neq 0$, so $\exists r \in J$ such that $\bar{x} \bar{r} = 1$. Then
      $xr + I = 1 + I \implies \exists y \in I \text{ s.t. } xr + y = 1$. So $1 \in J$, and
      because $J$ is an ideal, $J = R$.

      \item By the fact that $(ab \in I \implies a \in I \text{ or } b \in I) \iff
        (\bar{a}\bar{b} = 0 \implies \bar{a} = 0 \text{ or } \bar{b} = 0)$ the proof is complete.
        \qedhere
    \end{enumerate}
  \end{proof}
\end{prop}

\begin{prop} \label{prop:irr-to-max-ideal}
  If $f(x) \in K[x]$ is irreducible, where $K$ is a field, then $\gen{ f(x) }$ is maximal ideal.
  \begin{proof}
    We know that $K[x]$ is a principle ideal domain, so if $\gen{ f(x) } \subseteq J$, then
    $J$ is generated by a element, say $g(x)$. Since $f(x) \in J$, we could write $f(x) = g(x) h(x)$.
    By the fact that $f(x)$ is irreducible, either $g(x)$ is an unit then $J = R$, or $h(x)$ is
    an unit then $J = \gen{ f(x) }$.
  \end{proof}
\end{prop}

\begin{example}
  $f(x) = x^2 + 1$ has roots $\alpha = \pm \sqrt{-1}$, so $\Rb(\sqrt{-1}) \cong \Rb[x] / \gen{ x^2 + 1 }$.
\end{example}

\begin{theorem} \label{thm:field-ext-1}
  Let $f(x) \in K[x]$ be monic, irreducible and of degree $n$. Then there exists
  $L / K$ and $\alpha \in L$ s.t. $f(\alpha) = 0, L = K(\alpha)$ and $[L: K] = n$.
  \begin{proof}
    Since $f(x)$ is irreducible, by prop. \ref{prop:irr-to-max-ideal}
    $\gen{ f(x) }$ is a maximal ideal. Then by prop.
    \ref{prop:max-prime-to-field-int-domain} $L = K[x] / \gen{ f(x) }$ is a field,
    and $K$ is a subfield of $L$ by the inclusion map $\alpha \mapsto \bar\alpha$.
    The map is 1-1 since $\bar{1} \neq 0$ and a field homomorphism is either a
    1-1 map or a zero(全洪)map.

    Notice that $L \cong K[\bar{x}]$, where $\bar{x}$ is the coset $x + \gen{ f(x) }$.
    Now let $\alpha = \bar{x}$, and it is easy to see that $f(\alpha) = f(x) + \gen{ f(x) } = 0$.
    Also $L \cong K[\bar{x}] \cong K(\alpha)$. Finally, $m_{\alpha, K} \mid f$ and by the fact that
    $f$ is monic and irreducible, $m_{\alpha, K} = f$ and thus $[L: K] = \deg m_{\alpha, K} = \deg f = n$.
  \end{proof}
\end{theorem}

\begin{theorem}
  Let $f(x) \in K[x]$ be of degree $n > 0$. Then there exists $L/K$ s.t. $f$
  splits over $L$, that is,
  \[ f(x) = \lambda (x - \alpha_1) (x - \alpha_2) \dotsm (x - \alpha_n) \text{ with }
    \alpha_1, \alpha_2, \dots, \alpha_n \in L,\, \lambda \in K \]
  In fact, $L$ can be chosen to be the smallest field over which $f$ splits and in this case $[L : K] \leq n!$.\\
  $L$ is called a \emph{splitting field} \index{splitting field} for $f$ over $K$.
\end{theorem}

\begin{proof}
  By induction on $n$, $n = 1$ is trivial, simply pick $L = K$.

  For $n > 1$, let $p(x)$ be an monic irreducible factor of $f(x)$.
  By theorem \ref{thm:field-ext-1}, there exists an extension $K(\alpha_1)$ s.t. $p(\alpha_1) = 0$.
  By division algorithm, $f(x) = (x - \alpha_1) f_1(x)$ where $f_1(x) \in K(\alpha_1)[x]$
  and $\deg f_1 = n - 1$. Using the induction hypothesis, we know that there exists $L$,
  which is an extension of $K(\alpha_1)$, s.t. $f_1$
  splits over $L$. Hence $\exists \alpha_2, \alpha_3, \dots, \alpha_n \in L$ s.t.
  $ f_1(x) = \lambda (x - \alpha_2) \dots (x - \alpha_n)$,
  thus $f(x) = \lambda (x - \alpha_1) (x - \alpha_2) \dots (x - \alpha_n)$. Compare the
  coefficient of $x^n$ we know that $\lambda \in K$.

  More over, observe that $K(\alpha_1, \dots, \alpha_n)$ is the smallest field containing $K$ and
  $\{ \alpha_1, \dots, \alpha_n\}$. So if we choose $L = K(\alpha_1, \alpha_2, \dots, \alpha_n)$,
  then
  \[ [L: K] = [K(\alpha_1, \alpha_2, \dots, \alpha_n): K(\alpha_1, \alpha_2, \dots, \alpha_{n-1})]
    \dotsm
    [K(\alpha_1): K] \leq n! \]
  Since $[K(\alpha_1, \alpha_2, \dots, \alpha_k): K(\alpha_1, \alpha_2, \dots, \alpha_{k-1})]
  = [K(\alpha_1, \alpha_2, \dots \alpha_{k-1})(\alpha_k): K(\alpha_1, \alpha_2, \dots, \alpha_{k-1})]$
  and $\alpha_k$ is a root of $p(x) \in K(\alpha_1, \alpha_2, \dots, \alpha_{k-1})[x]$
  where $f(x) = (x - \alpha_1)(x - \alpha_2) \dotsm (x - \alpha_{k-1}) p(x)$.
\end{proof}

\begin{example}
  Find a splitting field $L$ for $x^8 - 2$ over $\Qb$ and determine $[L : \Qb]$.

  The roots are $\alpha \zeta^k$ where $\alpha = \sqrt[8]{2}$ and $\zeta = e^{2 \pi \mathrm{i} / 8}$.
  But $\zeta = \sqrt{2}(1 + \mathrm{i}) / 2$ where $\sqrt{2} = \alpha^4$,
  so we know that $L = \Qb(\alpha, \zeta) = \Qb(\alpha, \mathrm{i})$.
  Thus $[L: \Qb] = [\Qb(\alpha, \mathrm{i}): \Qb(\alpha)][\Qb(\alpha): \Qb] = 2 \cdot 8 = 16$.
\end{example}

\begin{remark}
  $\quot{\Qb[x]}{\gen{x^8 - 2}} = \Qb(\bar{x}) \cong \Qb(\sqrt[8]{2})
  \cong \Qb(\sqrt[8]{2} \zeta)$
\end{remark}

\begin{prop} \label{prop:after-homo-still-irr}
  Let $K, L$ be two fields and $\tau: K \to L$ be a nontrivial homomorphism.
  We define $\bar\tau : K[x] \to \tau(K)[x] \subseteq L[x]$ by
  \[ a_n x^n + \dots + a_0 \mapsto \bar\tau(f) \triangleq \tau(a_n)x^n + \dots + \tau(a_0) \]
  which is an isomorphism. Also, $f$ is irreducible implies $\bar\tau(f)$ is irreducible in $\tau(K)[x]$.
\end{prop}

\begin{lemma} \label{lemma:extension-exists-condition}
  Let $K(\alpha) / K$ be algebraic and $\tau: K \to L$ be a nontrivial homo,
  then there exists an extension $\sigma$ of $\tau$ from $K(\alpha)$ to $L$ if
  and only if $\exists \beta \in L$ s.t. $\bar\tau(m_{\alpha, K})(\beta) = 0$.

  In this case $m_{\beta, \tau(K)} = \bar\tau(m_{\alpha, K})$.

\begin{proof}
  ``$\Rightarrow$'': Let $\beta = \sigma(\alpha)$ and $m_{\alpha, K} = x^n + a_{n-1} x^{n-1} + \dots + a_0$.
  Then $\bar\tau(m_{\alpha, K})(\beta) = \beta^n + \tau(a_{n-1})\beta^{n-1} + \dots + \tau(a_0)
  = \tau(\alpha^n + a_{n-1} \alpha^{n-1} + \dots + a_0) = 0$

  ``$\Leftarrow$'': Observe that $m_{\beta, \tau(K)} = \bar\tau(m_{\alpha, K})$ since
  $\bar\tau(m_{\alpha, K})(\beta) = 0$ and $\bar\tau(m_{\alpha, K})$ is monic and irreducible
  by prop \ref{prop:after-homo-still-irr}. $\sigma$ is then given by the following diagram.
  \[
    \begin{tikzcd}
      & K[x] \arrow[r, "\sim", "\bar\tau"'] \arrow[d, two heads]
      & \tau(K)[x] \arrow[d, two heads] \\
      K(\alpha) \arrow[r, Leftrightarrow, "\cong"]
      & \raisebox{.1em}{$K[x]$} \Big/ \raisebox{-.1em}{$\gen{ m_{\alpha, K} }$}
      \arrow[r, "\sim", "\sigma"']
      & \raisebox{.1em}{$\tau(K)[x]$} \Big/ \raisebox{-.1em}{$\gen{ m_{\beta, \tau(K)} }$}
      \arrow[r, Leftrightarrow, "\cong"]
      & \tau(K)(\beta) \subseteq L
    \end{tikzcd}
  \]
\end{proof}
\end{lemma}

\begin{coro} \label{coro:num-of-extensions}
  Let $K(\alpha)/K$ be an algebraic extension and $\tau: K \hookrightarrow L$.
  If $\bar\tau(m_{\alpha, K})$ has $r$ distinct roots in $L$, then there are exactly $r$ extensions of $\tau$.
\end{coro}

\begin{theorem} \label{thm:two-splitting-field-are-isom}
  Let $\tau: K \to K'$ be an isomorphism of fields.
  If $L$ is a splitting field for $f$ over $K$ and $L'$ is a splitting field for $\bar\tau(f)$
  over $K'$, then $L \cong L'$

\begin{proof} \label{coro:extension-exists-splitting-field}
  By induction on $n = \deg f$. When $n = 1$, $L = K, L' = K'$, so $L \cong L'$.

  Now if $n > 1$, assume $f(\alpha) = 0$ for $\alpha \in L$. Then
  $\bar\tau(m_{\alpha, K}) \mid \bar\tau(f)$ and by the fact that $L'$ is a
  splitting field for $\bar\tau(f)$, $\exists \beta \in L'$ s.t. $\bar\tau(m_{\alpha, K})(\beta) = 0$.
  By lemma \ref{lemma:extension-exists-condition}, $\exists \tau_{\circ}:
  K(\alpha) \xrightarrow\sim K'(\beta)$ with $\tau_\circ \big|_K = \tau$.

  Now, write $f = (x - \alpha) f_\circ$, then $\bar\tau(f) = \bar\tau_\circ(f) = (x - \tau_\circ(\alpha))
  \bar\tau_\circ(f_\circ) = (x - \beta) \bar\tau_\circ(f_\circ)$. Then $L$ and $L'$ is a splitting
  field for $f_{\circ}$ over $K(\alpha)$ and $\bar\tau_\circ(f_\circ)$ over $K(\beta)$ respectively.
  By induction hypothesis, $L \cong L'$.
\end{proof}
\end{theorem}

\begin{coro}
  Let $\tau: K \xrightarrow\sim K'$ be an isomorphism of fields, and
  $L$ is a splitting field of $f$ over $K$, $L'$ is a splitting field of $\bar\tau(f)$ over $K'$.
  Then $\tau$ could be extend to $\sigma: L \xrightarrow\sim L'$ such that $\sigma\big|_K = \tau$.
\end{coro}
