\section{Fields}

\subsection{Week 1}

\begin{definition}
  Let $R$ be a commutative ring with $1$, and $I$ be an ideal of $R$, then
  \begin{itemize}
    \item $I$ is called a maximal ideal if for any ideal $J$ satisfied
      $I \subseteq J$ then $J = I \text{ or } J = R$.
    \item $I$ is called a prime ideal if $I \neq R$ and $ab \in I \implies a \in I \text{ or } b \in I$.
  \end{itemize}
\end{definition}

\begin{prop} \label{prop:max-prime-to-field-int-domain}
  Suppose $R$ is a ring and $I \subsetneq R$ is an ideal, then
  \begin{enumerate}
    \item $I$ is maximal $\iff$ $R / I$ is a field.
    \item $I$ is a prime ideal $\iff$ $R / I$ is an integral domain.
  \end{enumerate}

  \begin{proof} \hfill \vspace*{-1em}
    \begin{enumerate}
      \item ``$\Rightarrow$'': For any $\bar{r} \in R/I$ with $\bar{r} \neq 0$, then $r \not\in I$.
        Consider $\langle r \rangle + I$ which contains $I$ and is not equal to $I$ because $r \not\in I$.
        Since $I$ is maximal, $\langle r \rangle + I = R$, and thus exists $x \in R, y \in I$ such that
        $xr + y = 1$, so $\bar{x} \bar{r} = \bar{1}$. Hence every non-zero element has multiply inverse
        and $R / I$ is a field.

      ``$\Leftarrow$'': If $J$ is an ideal such that $I \subsetneq J$, pick $x \in J \setminus I$,
      then $\bar{x} \neq 0$, so exists $r \in J$ such that $\bar{x} \bar{r} = 1$. Then
      $xr + I = 1 + I \implies \exists y \in I \text{ s.t. } xr + y = 1$. So $1 \in J$, and
      because $J$ is an ideal, $J = R$.

      \item By the fact that $(ab \in I \implies a \in I \text{ or } b \in I) \iff
        (\bar{a}\bar{b} = 0 \implies \bar{a} = 0 \text{ or } \bar{b} = 0)$ the proof is complete.
    \end{enumerate}
  \end{proof}
\end{prop}

\begin{prop} \label{prop:irr-to-max-ideal}
  If $f(x) \in K[x]$ is irreducible, where $K$ is a ring, then $\langle f(x) \rangle$ is maximal ideal.
  \begin{proof}
    We know that $K[x]$ is a principle ideal domain, so if $\langle f(x) \rangle \subseteq J$, then
    $J$ is generated by a element, say $g(x)$. Since $f(x) \in J$, we could write $f(x) = g(x) h(x)$.
    By the fact that $f(x)$ is irreducible, either $g(x)$ is an unit then $J = R$, or $h(x)$ is
    an unit then $J = \langle f(x) \rangle$.
  \end{proof}
\end{prop}

\begin{example}
  $f(x) = x^2 + 1$ has roots $\alpha = \pm \sqrt{-1}$, so $\Rb(\sqrt{-1}) \cong \Rb[x] / \langle x^2 + 1 \rangle$.
\end{example}

\begin{theorem} \label{thm:field-ext-1}
  Let $f(x) \in K[x]$ be monic, irreducible and of degree $n$. Then exists $L / K$ and $\alpha \in L$
  satisfied $f(\alpha) = 0, L = K(\alpha)$ and $[L: K] = n$.
\end{theorem}

\begin{proof}
  Since $f(x)$ is irreducible, by prop. \ref{prop:irr-to-max-ideal} $\langle f(x) \rangle$ is a maximal ideal.
  Then by prop. \ref{prop:max-prime-to-field-int-domain} $L = K[x] / \langle f(x) \rangle$ is a field, and $K$ is a subfield
  of $L$ by the inclusion map $\alpha \mapsto \bar\alpha$. The map is 1-1 since $\bar{1} \neq 0$ and
  a field homomorphism is either a 1-1 map or a zero(全洪)map.

  Notice that $L \cong K[\bar{x}]$, where $\bar{x}$ is the coset $x + \langle f(x) \rangle$.
  Now let $\alpha = \bar{x}$, and it is easy to see that $f(\alpha) = f(x) + \langle f(x) \rangle = 0$.
  Also $L \cong K[\bar{x}] \cong K(\alpha)$. Finally, $m_{\alpha, K} \mid f$ and by the fact that
  $f$ is monic and irreducible, $m_{\alpha, K} = f$ and thus $[L: K] = \deg m_{\alpha, K} = \deg f = n$.
\end{proof}

\begin{theorem}
  Let $f(x) \in K[x]$ be of degree $n > 0$. Then exists $L/K$ satisfied $f$ splits over $L$,
  that is,
  \[ f(x) = \lambda (x - \alpha_1) (x - \alpha_2) \cdots (x - \alpha_n) \text{ with }
    \alpha_1, \alpha_2, \cdots, \alpha_n \in L,\, \lambda \in K \]
  In fact, $L$ can be chosen to be the smallest field over which $f$ splits and in this case $[L : K] \leq n!$.\\
  $L$ is called a splitting field for $f$ over $K$.
\end{theorem}

\begin{proof}
  By induction on $n$, $n = 1$ is trivial.

  For $n > 1$, let $p(x)$ be an monic irreducible factor of $f(x)$.
  By theorem \ref{thm:field-ext-1}, exists $\alpha_1$ satisfied $p(\alpha_1) = 0$.
  By division algorithm, $f(x) = (x - \alpha_1) f_1(x)$ where $f_1(x) \in K(\alpha_1)[x]$
  and $\deg f_1 = n - 1$. By the induction hypothesis, exists $L$ satisfied $f_1$
  splits over $L$, say $\exists \alpha_2, \alpha_3, \cdots, \alpha_n \in L$ satisfied
  $ f_1(x) = \lambda (x - \alpha_2) \cdots (x - \alpha_n) \text{ for } \lambda  \in K(\alpha_1) $,
  thus $f(x) = \lambda (x - \alpha_1) (x - \alpha_2) \cdots (x - \alpha_n)$.

  Observe that $K(\alpha_1, \cdots, \alpha_n)$ is the smallest field containing $K$ and
  $\{ \alpha_1, \cdots, \alpha_n\}$.
\end{proof}

\begin{example}
  Find a splitting field $L$ for $x^8 - 2$ over $\Qb$ and determine $[L : \Qb]$.
\end{example}

\begin{remark}
  $\quot{\Qb[x]}{\gen{x^8 - 2}} = \Qb(\bar{x}) \cong \Qb(\sqrt[8]{2})
  \cong \Qb(\sqrt[8]{2} \zeta)$
\end{remark}

\begin{prop}
  Let $K, L$ be two fields and $\tau: K \to L$ be a nontrivial homomorphism.
  We define $\bar\tau : K[x] \to \tau(K)[x] \subseteq L[x]$ by
  \[ a_n x^n + \ldots + a_0 \mapsto \bar\tau(f) = \tau(a_n)x^n + \ldots + \tau(a_0) \]
  which is an isomorphism. Also, $f$ is irreducible implies $\bar\tau(f)$ is irreducible.
\end{prop}

\begin{lemma}
  Let $K(\alpha) / K$ be algebraic and $\tau: K \to L$ be a nontrivial homo,
  then exists an extension $\sigma$ of $\tau$ from $K(\alpha)$ to $L$ if and only if
  exists $\beta \in L$ satisfied $\bar\tau(m_{\alpha, K})(\beta) = 0$.

  In this case $m_{\beta, \tau(K)} = \bar\tau(m_{\alpha, K})$.
\end{lemma}

\begin{proof}
  ``$\Rightarrow$'': Let $\beta = \sigma(\alpha)$ and $m_{\alpha, K} = x^n + a_{n-1} x^{n-1} + \ldots + a_0$.
\end{proof}

\begin{theorem}
  Let $\tau: K \to K'$ be an isomorphism of fields.
  If L is a splitting field for $f$ over $K$ and $L'$ is a splitting field for $\bar\tau(f)$
  over $K'$, then $L \cong L'$
\end{theorem}

\begin{proof}
  By induction on $n = \deg f$. When $n = 1$, $L = K, L' = K'$, so $L \cong L'$.

  Now if $n > 1$, assume $f(\alpha) = 0$ for $\alpha \in L$, then $\bar\tau(m_{\alpha, K}) \mid \bar\tau(f)$.
\end{proof}

\begin{example}
  $L = \Qb(\sqrt{2}, \sqrt{3})$.
\end{example}
