%! TEX root=../main.tex
\subsection{Transcendental extensions}

\begin{definition}
  Let $L/K$ be an extension and $S \subset L$.
  \begin{itemize}
    \item $S$ is algebrarically dependent over $K$ if for some $n \in \mathbb{N}$,
      exists $f(x_1, \dots, x_n) \in K[x_1, \dots, x_n]$ satisfied
      $f(a_1, \dots, a_n) = 0$ for some distinct $a_1, \dots, a_n \in S$.
    \item $S$ is algebrarically independent over $K$ if $S$ is not algebrarically dependent.
    \item $S$ is called a transcendence base for $L/K$ if $S$ is algebrarically independent
      and $L/K(S)$ is algebraic.
  \end{itemize}
\end{definition}

\begin{theorem}
  Any two transcendence bases for $L/K$ have the same cardinality.

  \begin{proof}
    Pick any transcendence base $S = \{s_1, \dots, s_n\}$ for $L/K$.
    Let $T$ be another transcendence base for $L/K$.

    First we deal with the case which $S$ is finite.

    We claim that $\exists t_1 \in T$ such that
    $t_1$ is algebrarically independent over $K(s_2, \dots, s_n)$.
    \begin{proof}
      If not, then all elements of $T$ is algebraic dependent over $K(s_2, \dots, s_n)$.
      This implies $K(s_2, \dots, s_n)(T) / K(s_2, \dots, s_n)$ is algebraic.
      And $L/K(T)$ is algebraic implies $L/K(T)(s_2,\dots,s_n)$ is algebraic.
      Then $L/K(s_2, \dots, s_n)$ is algebraic, which is a contradiction ($s_1$ is not).
    \end{proof}
    By the claim, $\Set{t_1, s_2, \dots, s_n}$ is algebraic depedent.
    Also, there exists $f \ne 0$ in $K[x_1, \dots, x_{n+1}]$ s.t.
    $f(t_1, s_1, \dots, s_n) = 0$ since $t_1$ is algebraic over $K(s_1, \dots, s_n)$.
    Since $\Set{s_1, \dots, s_n}$ and $\Set{t_1, s_2, \dots, s_n}$ are both
    algebraic depedent, $t_1, s_1$ must occur in $f$ $\implies$
    $s_1$ is algebraic over $K(t, s_2, \dots, s_n)$.
    Then $K(t_1, s_1, \dots, s_n) / K(t_1, s_2, \dots, s_n)$ is algebraic.
    Since $L/K(t_1, s_1, \dots, s_n)$ is algebraic, $L/K(t_1, s_2, \dots, s_n)$
    is algebraic.

    Repeating this process, we get find $t_1, \dots, t_n \in T$ s.t.
    $L/K(t_1, \dots, t_n)$ is algebraic. But $T$ is a transcendence base,
    so we must have $T = \Set{t_1, \dots, t_n}$.

    Now assume $S$ is infinite.
    For another transcendence base $T$, we have $\abs{T} = \infty$.
    For $s \in S$, $s$ is algebraic over $K(T)$, and
    in fact is over $K(T_s)$ such that $T_s$ is finite,
    since algebraic relation involves.
    Let $m_{s, K(T)} \in K(T_s)[x]$ for some finite set $T_s \subset T$.
    We claim that $\bigcup\limits_{s\in S} T_s = T$.
    \begin{proof}
      Let $U = \bigcup\limits_{s\in S} T_s$.
      Clearly $U \subseteq T$. And by def, $K(U)(S)/K(U)$ is algebraic.
      Also, $L/K(U)(S)$ is algebraic. So $L/K(U)$ is algebraic
      $\implies T = U$ since $T$ is a transcendence base.
    \end{proof}
    By well ordering principle, we can well-order $S$ and denote its
    $1$st element by $s_1$.
    Let
    \[
      \begin{cases}
        T_{s_1}' = T_{s_1} \\
        T_{s}' = T_s \setminus \bigcup_{l < s} T_l
      \end{cases}
      \quad \leadsto \quad
      \Set{ T_s' }_{s\in S} \text{~are mutually disjoint}
    \]
    For all $T_s'$, choose a fixed ordering of the elements in $T_s'$, says
    $t_{s,1}, \dots, t_{s, {k_s}}$. Define an injection
    $\varphi: \bigcup\limits_{s\in S} T_s' \to S\times \Nb$ with
    $\varphi: t_{s,i} \mapsto (s, i)$.
    So $\abs{T} = \abs*{\bigcup\limits_{s\in S} T_s} \le \abs{S\times \Nb}
    = \abs{S}\abs{N} = \abs{S}$ since $\abs{S} = \infty$.
  \end{proof}
\end{theorem}

\begin{definition}
  Let $S$ be a transcendence base of $L/K$, then we use
  $\trdeg_K L$ to denote $\abs{S}$.
\end{definition}

\begin{remark}
  If $S_1, S_2$ are two transcendence base for $L/K$, then it is {\bf not
  necessarily true} that $K(S_1) = K(S_2)$.
\end{remark}

\begin{definition}
  $L/K$ is called purely transcendental if exists a transcendental base $S$
  such that $L = K(S)$.
\end{definition}

\begin{theorem}[L\"{u}roth's theorem]
  If $L$ is purely transcendental of degree $1$ over $K$, then any proper
  intermediate field $E$ is also purely transcendental of degree $1$.
\end{theorem}

\begin{lemma} \label{lemma:rational-trans-alg}
  Let $L = K(t)$ with $t$ being transcendental over $K$ and $u = f(t) / g(t) \in L \setminus K$
  with $\gcd(f(t), g(t)) = 1$.
  Assume $n = \max(\deg f, \deg g)$, then $L/K(u)$ is algebraic and $[L: K(u)] = n$.
  \begin{proof}
    Write
    \[ f(t) = a_n t^n + \dots + a_1 t + a_0, \quad g(t) = b_n t^n + \dots + b_1 t + b_0 \]
    (note that either $a_n \ne 0$ or $b_n \ne 0$)
    Let $F(x) = f(x) - ug(x) =-(a_n - ub_n) x^n + \dots + (a_1 - ub_1)x + (a_0 - ub_0)$.
    Since $a_n - ub_n \ne 0$, $F(x) \ne 0$ and $\deg F(x) > 0$.
    By def. of $u$, we have $F(t) = 0 \implies t$ is algebraic over $K(u)$ and
    $[K(t):K(u)] \le n$.
    Now we prove that $F(x)$ is irreducible over $K(u)$.
    By Gauss lemma, it suffices to show that $F(x)$ is irreducible in
    $K[u][x] = K[u, x]$.
    Assume that $F(x) = p(u, x) q(u, x)$ with $\deg_u p = 1$ and $q \in K[x]$.
    Since $F(x) = f(x) - ug(x)$, we have
    $q \mid f, q\mid g \implies q \mid \gcd(f, g) = 1 \implies q \in K$.
    So $[K(t):K(u)] = n$.
  \end{proof}
\end{lemma}

Now we prove the L\"{u}roth's theorem:
\begin{proof}
  For $v \in E \setminus K$, by lemma \ref{lemma:rational-trans-alg},
  $t$ is algebraic over $K(v) \leadsto t$ is algebraic over $E$.

  Let $m(x) = m_{t,E}$, then there exists $\beta(t) \in K(t)$ s.t.
  $\beta(t)m(x) = a_n(t)x^n + \dots + a_1(t)x + a_0(t)$ is primitive
  in $K[t][x] = K[t, x]$. Let $F(t, x) = \beta(t)m(x)$.

  Since $t$ is not algebraic over $K$, there exists some $u = \frac{a_i(t)}{a_n(t)} \not\in K$.
  Write $u = \frac{f(t)}{g(t)}$ with $\gcd(f, g) = 1$.
  (Note that $u \in E$)

  By lemma \ref{lemma:rational-trans-alg}, $[K(t): K(u)] = r \ge n$.
  Now we show that $r \le n$, then $r = n \implies E = K(u)$.

  Let $l = f(t)g(x) - g(t)f(x)$, which is skew-symmetric in $t$ and $x$.
  Notice that $g(t)^{-1} l \in E[x]$ and has $t$ as a zero.
  So $m(x) \mid g(t)^{-1} l$ in $E[x]$ $\implies \beta(t)m(x) \mid \beta(t)g(t)^{-1} l$.
  Since $\beta(t)g(t)^{-1} \in K[t]$, $F(t, x) \mid l$ in $K(t)[x]$.
  Since $F(t, x)$ is primitive in $K[t][x]$, $F(t, x) \mid l$ in $K[t][x]$.

  Say $l = Fq$ for some $q(t,x) \in K[t][x]$.
  Note that $\deg_t l \le r, \deg_t F \ge r \leadsto \deg_t l =\deg_t F = r,
  \deg_t q = 0$. So $q \in K[x] \leadsto q$ is primitive in $K[t][x]$.
  By Gauss lemma, $F, q$ are primitive, then $l$ is also primitive in $K[t][x]$.
  Since $l$ is skew-symmetric in $t$ and $x$, $l$ is also primitive in $K[x][t]$.
  But $q\in K[x]$ and $q \mid l$, we have $q \in K$.
  Hence $n = \deg_x F = \deg_x l = \deg_t l  = \deg_t F \ge r$.
\end{proof}

\subsection{Hilbert theorem 90 and Normal basis}

Let $L = K(\alpha)$ with $f = m_{\alpha, K} = x^n + a_{n-1} x^{n-1} + \dots + a_0$ being separable.
We have known that exists exactly $n$ monomorphisms $\sigma_i :: L \to \overline{K}$ fixing $K$,
and $\Set{\sigma_1(\alpha), \dots, \sigma_n(\alpha)}$ consists of all roots of $f$.
So
\begin{multline*} x^n + a_{n-1} x^{n-1} + \dots + \a_0 = (x - \sigma_1(\alpha)) \dotsm (x - \sigma_n(\alpha)) \\
  \implies -a_{n-1} = \sigma_1(\alpha) + \dots + \sigma_n(\alpha) \text{ and }
  (-1)^n a_0 = \sigma_1(\alpha) \dotsm \sigma_n(\alpha)
\end{multline*}

\begin{definition}
  Let $L/K$ be a Galois extension with $G = \Gal(L/K)$.
  for all $\alpha \in L$, define 
  \begin{align*}
    N_{L/K}(\alpha) &= \prod_{\sigma \in G} \sigma(\alpha) & N_{L/K} :: L^\times \to K^\times \text{multiplicative} \\
    \trace_{L/K}(\alpha) &= \sum{\sigma \in G} \sigma(\alpha) & \trace{L/K} :: L \to K \text{additive}
  \end{align*}
\end{definition}

\begin{theorem}
  Let $L/K$ is cyclic and $G = \langle \sigma \rangle$ with $\ord(\sigma) = n$, then
  \begin{enumerate}
    \item $\alpha \in L^\times$ and $N_{L/K}(\alpha) = 1$ $\iff$ $\exists \beta \in L^\times, \alpha = \beta / \sigma(\beta)$.
    \item $\alpha \in L$ and $\trace_{L/K}(\alpha) = 0$ $\iff$ $\exists \beta \in L, \alpha = \sigma(\beta) - \beta$.
  \end{enumerate}

  \begin{proof}
    \begin{enumerate}
      \item ``\Leftarrow'': 
      ``\Rightarrow'':  Since automorphisms are 
      \item ``\Leftarrow'': 
        ``\Rightarrow'':  Choose $c$ such that $\beta_1 = c + \sigma(c) + \dots + \sigma^{n-1}(c) \neq 0$, so $\sigma(\beta_1) = \beta_1$.
        Let $\beta_2$.
    \end{enumerate}
  \end{proof}
\end{theorem}

\begin{coro}
  Let $\Char K = p$ and $[L: K] = p$, then $L/K$ is cyclic $\iff$ $L = K(\alpha)$ where
  $\alpha$ is a root of $x^p - x - a$.

  \begin{proof}
    ``$\Rightarrow$'': Let $\Gal(L/K) = \langle \sigma \rangle$ with $\ord(\sigma) = p$. Then
    $\trace_{L/K}(1) = p = 0$. By Hilbert theorem 90, exists $\alpha$ satisfied $1 = \sigma(\alpha) - \alpha$.
    So $\alpha \not\in K$. Then we have $1 < [K(\alpha): K] $
  \end{proof}
\end{coro}

\begin{coro}
  If $x^2 + d y+2 = 1$ where $-d$ is not a square,
\end{coro}

\begin{definition}
  Let $L/K$ be Galois and $\Gal(L/K) = \Set{ \text{Id} = \sigma_1, \dots, \sigma_n }$.
  A basis for $L/K$ of the form $\Set{ \sigma_1(\alpha), \sigma_2(\alpha), \dots, \sigma_n(\alpha)}$
  with $\alpha \in L$ is called  a normal basis for $L/K$.
\end{definition}

\begin{lemma}
  $\alpha_1, \dots, \alpha_n \in L$ form a basis for $L/K$ if and only if
  \[
    \begin{vmatrix}
      \alpha_1 & \alpha_2 & \cdots & \alpha_n \\
      \sigma_1(\alpha_1) & \sigma_1(\alpha_2) & \cdots & \sigma_1(\alpha_n) \\
      \vdots & \vdots & \ddots & \vdots \\
      \sigma_n(\alpha_1) & \sigma_n(\alpha_2) & \cdots & \sigma_n(\alpha_n)
    \end{vmatrix} \neq 0
  \]
\end{lemma}

\begin{lemma}
  Let $\abs{K} = \infty$. Then $\sigma_1, \dots, \sigma_n$ are algebrarically independent over $L$.

  \begin{proof}
    Let $f(x_1, \dots, x_n) \in L[x_1, \dots, x_n]$ such that $f(\sigma_1, \dots, \sigma_n) = 0$.
    Let $\Set{\alpha_1, \dots, \alpha_n}$ be a basis for $L/K$. Then
    \[ 0 = f(\sigma_1, \dots, \sigma_n)\left( \sum_{i = 1}^n r_i \alpha_i \right)
      = f(\sigma_1, \dots, \sigma_n)\left( \sum_{i = 1}^n r_i \alpha_i \right)
      = \]
  \end{proof}
\end{lemma}

\begin{theorem}
  Any Galois extension $L/K$ has a normal basis.

  \begin{proof}
    Case 1: $L/K$ is cyclic (so all finite field is included). \\
    Let $\Gal(L/K) = \langle \sigma \rangle$ with $\ord(\sigma) = n$. $\sigma$ could be view as
    a linear transformation of $L$ over $K$. Thus $\sigma$ gives $L$ a
    $K[x]$-module structure by $(f(x), \alpha) \mapsto f(\sigma)(\alpha)$.

    Case 2: $\abs{K} = \infty$. Let $\Gal(L/K) = \Set{\sigma_1, \dots, \sigma_n}$.
  \end{proof}
\end{theorem}
