%! TEX root=../main.tex
\subsection{Normal extension}

\begin{definition}
  $\quot{L}{K}$ is called a {\bf normal extension}\index{Extension!normal extension}
  if $\forall \alpha \in L$, $m_{\alpha, K}$ splits over $L$.
\end{definition}

\begin{theorem} \label{thm:splitting-field-iff-finite-normal}
  $L$ is a splitting field of some polynomial $f(x)$ over $K$ if
  and only if $L/K$ is finite and normal.

  \begin{proof}
    ``$\Rightarrow$'': Let $\alpha_1, \alpha_2, \dots, \alpha_n$ be the roots of $f$,
    so $L = K(\alpha_1, \alpha_2, \dots, \alpha_n)$, and $L$ is also a splitting field
    of $\prod m_{\alpha_i, K}$ since $m_{\alpha_i, K} \mid f$. By lemma~\ref{lemma:element-in-alg-ext-splits},
    for any $\beta$ in $L$, $m_{\beta, K}$ splits, thus $L/K$ normal and obviously also finite.

    ``$\Leftarrow$'': Since $L/K$ is a finite extension, we could write
    $L = K(\alpha_1, \alpha_2, \dots, \alpha_n)$. Let $f = \prod m_{\alpha_i, K}$, then
    since $L/K$ normal, each $m_{\alpha_i, K}$ splits. It is also easy to see that $L$
    is the smallest field where $f$ splits, thus $L$ is a splitting field of $f$.
  \end{proof}
\end{theorem}

\begin{remark}
  If $L/K$ is normal, then for any $M$ with $K \subset M \subset L$, we have $L/M$ is normal,
  this is because $\forall \alpha, \, m_{\alpha, M} \mid m_{\alpha, K}$, and thus
  $m_{\alpha, M}$ splits since $m_{\alpha, K}$ splits.

  But $M/K$ need not to be normal. For example, Let $K = \Qb$, $L$ be the splitting field of $x^3 - 2$,
  by theorem~\ref{thm:splitting-field-iff-finite-normal} $L/K$ is normal.
  Then $L = \Qb(\sqrt[3]{2}, \omega)$ where $\omega \triangleq \mathrm{e}^{2 \pi \mathrm{i} / 3}$.
  Let $M = \Qb(\sqrt[3]{2})$ then $m_{\sqrt[3]{2}, K}$ doesn't split in $M$, so $M/K$ is not normal.
\end{remark}

\begin{prop}
  Let $\quot{L}{K}$ be a finite, normal extension and $L \supset M \supset K$, then the following
  are equivalent.

  \begin{enumerate}[(\alph*)]
    \item $\quot{M}{K}$ is normal.
    \item $\forall \sigma \in \Aut(\quot{L}{K})$, $\sigma(M) \subset M$.
    \item $\forall \sigma \in \Aut(\quot{L}{K})$, $\sigma(M) = M$.
  \end{enumerate}

  \begin{proof}
    (a) $\Rightarrow$ (b): $\forall \alpha \in M$, $m_{\alpha, K}(\sigma(\alpha)) = \sigma(m_{\alpha, K}(\alpha)) = 0$.
    So $\sigma(\alpha)$ is a root of $m_{\alpha, K}$. Since $\quot{M}{K}$ normal, $m_{\alpha, K}$ splits in $M$
    and thus every root of $m_{\alpha, K}$ is in $M$, hence $\forall m, \, \sigma(m) \in M \implies \sigma(M) \subset M$.

    (b) $\Rightarrow$ (c): Since $L/K$ algebraic and $\sigma$ 1-1, by a homework problem, $\sigma$ onto.

    (c) $\Rightarrow$ (a): For any $\alpha \in M$, let $\beta \in L$ be a root of $m_{\alpha, K}$.
    By theorem~\ref{thm:splitting-field-iff-finite-normal}, we could assume $L$ is a splitting field of $f$ over $K$.
    Consider the following diagram,
    \[
      \begin{tikzcd}
        L \arrow[rr, "\sim", "\sigma"'] & & L \\
        K(\alpha) \arrow[u, hookrightarrow] \arrow[rr, "\sim", "\tau"'] & & K(\beta) \arrow[u, hookrightarrow] \\
        & K \arrow[ur, hookrightarrow] \arrow[ul, hookrightarrow] &
      \end{tikzcd}
    \]
    Where isomorphism $\tau$ with $\tau(\alpha) = \beta$ exists since $\alpha, \beta$ share the same minimal polynomial,
    and $\sigma$ with $\sigma\big|_K = \tau$ exists by theorem~\ref{thm:two-splitting-field-are-isom}.
    Since $\sigma \in \Aut(\quot{L}{K})$, $\beta = \sigma(\alpha) \in M$, thus $M/K$ normal.
  \end{proof}
\end{prop}

\begin{definition}
  Let $L/K$ is called a \emph{Galois}\index{Extension!Galois extension} extension if $L/K$ is finite, normal and separable.
  That is, $L$ is a splitting field of some separable polynomial over $K$.
\end{definition}

\begin{theorem}
  If $L/K$ is Galois, then $\abs{\Aut(L/K)} = [L:K]$. Otherwise, $\abs{\Aut(L/K)} < [L:K]$.

  \begin{proof}
    Since $L/K$ is normal, for any $\alpha$, $m_{\alpha, K}$ splits over $L$.
    Since $L/K$ is separable, $m_{\alpha, K}$ has no multiple roots. So there are exactly $[L:K]$
    extensions $\sigma:: L \to L$ of $\text{id}_K$.
  \end{proof}
\end{theorem}

\begin{definition}
  Given a field $L$, define the {\bf fixed field}\index{fixed field} of $G$ by
  $L^G \triangleq \{ \alpha \in L \mid \sigma(\alpha) = \alpha, \, \forall \sigma \in G \}$.
\end{definition}

\begin{theorem}
  If $G$ is a subgroup of $\Aut(L)$ with $\abs{G} < \infty$, then $\abs{G} = [L: L^G]$, $G = \Aut(L / L^G)$
  and $L / L^G$ is Galois.

  \begin{proof}
    First we prove that $[L: L^G] \leq \abs{G}$. Assume not, then $\abs{G} < [L: L^G]$.
    Let $G = \{\sigma_1, \sigma_2, \dots, \sigma_n\}$ and $\alpha_1, \alpha_2, \dots, \alpha_{n+1} \in L$.
    Which $\alpha_i$ is linear independent.

    Consider the equations
    \[
      \left\{
        \begin{array}{ccc}
          \sigma_1(\alpha_1) x_1 + \dots + \sigma_1(\alpha_{n+1}) x_{n+1} &=& 0 \\
          \sigma_2(\alpha_1) x_1 + \dots + \sigma_2(\alpha_{n+1}) x_{n+1} &=& 0 \\
          \vdots &\vdots& \vdots \\
          \sigma_n(\alpha_1) x_1 + \dots + \sigma_n(\alpha_{n+1}) x_{n+1} &=& 0 \\
        \end{array}
      \right.
    \]
    Since the number of variables is more than the number of equations, there is a
    non-trivial solution. Choose one solution $(a_1, \dots, a_{n+1})$ having the
    least amount of nonzero element. By reordering, we could assume
    the solution is $(a_1, a_2, \dots, a_m, 0, 0, \dots, 0)$. If $m = 1$,
    then $1_G = \sigma_i$ for some $i$, and $\sigma_i(\alpha_1) a_1 = \alpha_1 a_1 = 0 \implies a_1 = 0$,
    which is a contradiction.
  \end{proof}
\end{theorem}

\begin{definition}
  Let $f(x) \in K[x]$ and $L$ be a splitting field of $f(x)$ over $K$. We call $\text{Gal}(L/K)$
  the {\bf Galois group}\index{Galois group} of $f(x)$.
\end{definition}

\begin{prop}
  Let $f(x) \in \Qb[x]$ be irreducible polynomial of degree $p$ where $p$ is a prime.
  If $f(x)$ has exactly $p-2$ roots and $2$ complex roots, then the Galois group of $f(x)$ is $S_p$.

  \begin{proof}
    Let $L$ be a splittinf field of $f$ over $\Qb$ and $R = \{ \alpha_1, \alpha_2, \dots, \alpha_p \}$ be
    the set of all roots of $f(x)$. Since $f(x)$ is irreducible, $f(x) / a_p = m_{\alpha_i, \Qb}$.
  \end{proof}
\end{prop}
