%! TEX root=../main.tex
\subsection{Normal extension}

\begin{definition}
  $\quot{L}{K}$ is called a {\bf normal extension}\index{Extension!normal extension}
  if $\forall \alpha \in L$, $m_{\alpha, K}$ splits over $L$.
\end{definition}

\begin{theorem} \label{thm:splitting-field-iff-finite-normal}
  $L$ is a splitting field of some polynomial $f(x)$ over $K$ if
  and only if $L/K$ is finite and normal.

  \begin{proof}
    ``$\Rightarrow$'': Let $\alpha_1, \alpha_2, \dots, \alpha_n$ be the roots of $f$,
    so $L = K(\alpha_1, \alpha_2, \dots, \alpha_n)$, and $L$ is also a splitting field
    of $\prod m_{\alpha_i, K}$ since $m_{\alpha_i, K} \mid f$. By lemma~\ref{lemma:element-in-alg-ext-splits},
    for any $\beta$ in $L$, $m_{\beta, K}$ splits, thus $L/K$ normal and obviously also finite.

    ``$\Leftarrow$'': Since $L/K$ is a finite extension, we could write
    $L = K(\alpha_1, \alpha_2, \dots, \alpha_n)$. Let $f = \prod m_{\alpha_i, K}$, then
    since $L/K$ normal, each $m_{\alpha_i, K}$ splits. It is also easy to see that $L$
    is the smallest field where $f$ splits, thus $L$ is a splitting field of $f$.
  \end{proof}
\end{theorem}

\begin{remark}
  If $L/K$ is normal, then for any $M$ with $K \subset M \subset L$, we have $L/M$ is normal,
  this is because $\forall \alpha, \, m_{\alpha, M} \mid m_{\alpha, K}$, and thus
  $m_{\alpha, M}$ splits since $m_{\alpha, K}$ splits.

  But $M/K$ need not to be normal. For example, Let $K = \Qb$, $L$ be the splitting field of $x^3 - 2$,
  by theorem~\ref{thm:splitting-field-iff-finite-normal} $L/K$ is normal.
  Then $L = \Qb(\sqrt[3]{2}, \omega)$ where $\omega \triangleq \mathrm{e}^{2 \pi \mathrm{i} / 3}$.
  Let $M = \Qb(\sqrt[3]{2})$ then $m_{\sqrt[3]{2}, K}$ doesn't split in $M$, so $M/K$ is not normal.
\end{remark}

\begin{prop} \label{prop:TFAE-of-normal-extension}
  Let $\quot{L}{K}$ be a finite, normal extension and $L \supset M \supset K$, then the following
  are equivalent.

  \begin{enumerate}[(\alph*)]
    \item $\quot{M}{K}$ is normal.
    \item $\forall \sigma \in \Aut(\quot{L}{K})$, $\sigma(M) \subset M$.
    \item $\forall \sigma \in \Aut(\quot{L}{K})$, $\sigma(M) = M$.
  \end{enumerate}

  \begin{proof}
    (a) $\Rightarrow$ (b): $\forall \alpha \in M$, $m_{\alpha, K}(\sigma(\alpha)) = \sigma(m_{\alpha, K}(\alpha)) = 0$.
    So $\sigma(\alpha)$ is a root of $m_{\alpha, K}$. Since $\quot{M}{K}$ normal, $m_{\alpha, K}$ splits in $M$
    and thus every root of $m_{\alpha, K}$ is in $M$, hence $\forall m, \, \sigma(m) \in M \implies \sigma(M) \subset M$.

    (b) $\Rightarrow$ (c): Since $L/K$ algebraic and $\sigma$ 1-1, by a homework problem, $\sigma$ onto.

    (c) $\Rightarrow$ (a): For any $\alpha \in M$, let $\beta \in L$ be a root of $m_{\alpha, K}$.
    By theorem~\ref{thm:splitting-field-iff-finite-normal}, we could assume $L$ is a splitting field of $f$ over $K$.
    Consider the following diagram,
    \[
      \begin{tikzcd}
        L \arrow[rr, "\sim", "\sigma"'] & & L \\
        K(\alpha) \arrow[u, hookrightarrow] \arrow[rr, "\sim", "\tau"'] & & K(\beta) \arrow[u, hookrightarrow] \\
        & K \arrow[ur, hookrightarrow] \arrow[ul, hookrightarrow] &
      \end{tikzcd}
    \]
    Where isomorphism $\tau$ with $\tau(\alpha) = \beta$ exists since $\alpha, \beta$ share the same minimal polynomial,
    and $\sigma$ with $\sigma\big|_K = \tau$ exists by theorem~\ref{thm:two-splitting-field-are-isom}.
    Since $\sigma \in \Aut(\quot{L}{K})$, $\beta = \sigma(\alpha) \in M$, thus $M/K$ normal.
  \end{proof}
\end{prop}

\begin{definition}
  Let $L/K$ is called a \emph{Galois}\index{Extension!Galois extension} extension if $L/K$ is finite, normal and separable.
  That is, $L$ is a splitting field of some separable polynomial over $K$.
\end{definition}

\begin{theorem}
  If $L/K$ is Galois, then $\abs{\Aut(L/K)} = [L:K]$. Otherwise, $\abs{\Aut(L/K)} < [L:K]$.

  \begin{proof}
    Since $L/K$ is normal, for any $\alpha$, $m_{\alpha, K}$ splits over $L$.
    Since $L/K$ is separable, $m_{\alpha, K}$ has no multiple roots. So there are exactly $[L:K]$
    extensions $\sigma:: L \to L$ of $\text{id}_K$.
  \end{proof}
\end{theorem}

\begin{definition}
  Given a field $L$, define the {\bf fixed field}\index{fixed field} of $G$ by
  $L^G \triangleq \{ \alpha \in L \mid \sigma(\alpha) = \alpha, \, \forall \sigma \in G \}$.
\end{definition}

\begin{theorem} \label{thm:extension-of-fix-field-is-galois}
  If $G$ is a subgroup of $\Aut(L)$ with $\abs{G} < \infty$, then $\abs{G} = [L: L^G]$, $G = \Aut(L / L^G)$
  and $L / L^G$ is Galois.

  \begin{proof}
    First we prove that $[L: L^G] \leq \abs{G}$. Assume not, then $\abs{G} < [L: L^G]$.
    Let $G = \{\sigma_1, \sigma_2, \dots, \sigma_n\}$ and $\alpha_1, \alpha_2, \dots, \alpha_{n+1} \in L$.
    Which $\alpha_i$ is linear independent over $L^G$.

    Consider the equations
    \[
      \left\{
        \begin{array}{ccc}
          \sigma_1(\alpha_1) x_1 + \dots + \sigma_1(\alpha_{n+1}) x_{n+1} &=& 0 \\
          \sigma_2(\alpha_1) x_1 + \dots + \sigma_2(\alpha_{n+1}) x_{n+1} &=& 0 \\
          \vdots &\vdots& \vdots \\
          \sigma_n(\alpha_1) x_1 + \dots + \sigma_n(\alpha_{n+1}) x_{n+1} &=& 0 \\
        \end{array}
      \right.
    \]
    Since the number of variables is more than the number of equations, there is a
    non-trivial solution. Choose one solution $(a_1, \dots, a_{n+1})$ having the
    least amount of nonzero element. By reordering, we could assume
    the solution is $(a_1, a_2, \dots, a_m, 0, 0, \dots, 0)$ and it is no harm to assume $\sigma_1 = 1_G$.
    If $m = 1$, then and $\sigma_1(\alpha_1) a_1 = \alpha_1 a_1 = 0 \implies a_1 = 0$,
    which is a contradiction.

    So assume that $m > 1$, we have
    \[
      \left\{
        \begin{array}{ccc}
          \sigma_1(\alpha_1) a_1 + \dots + \sigma_1(\alpha_{m}) a_{m} &=& 0 \\
          \sigma_2(\alpha_1) a_1 + \dots + \sigma_2(\alpha_{m}) a_{m} &=& 0 \\
          \vdots &\vdots& \vdots \\
          \sigma_n(\alpha_1) a_1 + \dots + \sigma_n(\alpha_{m}) a_{m} &=& 0 \\
        \end{array}
      \right.
    \]
    By multipling $a_m^{-1}$, we could assume $a_m = 1$. The equation about $\sigma_1$ gives
    $\alpha_1 a_1 + \dots + \alpha_m a_m = 0$, since $\alpha_i$ is linear independent,
    one of the $a_i$s, say $a_k$ is not in $L^G$, and thus there exists $t$ so that $\sigma_t(a_k) \neq a_k$.
    Apply $\sigma_t$ to each equation, we have
    \[ \sigma_t\sigma_i(\alpha_1) \sigma_t(a_1) + \dots + \sigma_t \sigma_i(\alpha_m) \sigma_t(a_m) = 0,\quad
      \forall 1 \leq i \leq n \]

    But since $\{\sigma_t \sigma_1, \dots, \sigma_t \sigma_n\} = \{\sigma_1, \dots, \sigma_n\}$,
    $(\sigma_t(a_1), \sigma_t(a_2), \dots, \sigma_t(a_m), 0, \dots, 0)$ is a solution
    and thus $(a_1 - \sigma_t(a_1), \dots, a_m - \sigma_t(a_m), 0, \dots)$ is also a solution to the equations.
    Since $\sigma_t(a_k) \neq a_k$, the solution is not trivial, and because $a_m = 1$, $a_m - \sigma_t(a_m) = 0$.
    Hence this solution has $m-1$ nonzero element, which contradict the minimality of the original solution.
    Thus $[L: L^G] \leq \Aut(L/L^G)$.

    Finally, $\abs{\Aut(L/L^g)} \leq [L: L^G]$ by theorem~\ref{thm:separable-elements-form-separable-field},
    thus $\abs{G} \leq \abs{\Aut(L/L^G)} \leq [L: L^G] \leq \abs{G}$, hence they are all equal.
  \end{proof}
\end{theorem}

\begin{definition}
  Let $f(x) \in K[x]$ and $L$ be a splitting field of $f(x)$ over $K$. We call $\text{Gal}(L/K)$
  the {\bf Galois group}\index{Galois group} of $f(x)$.
\end{definition}

\begin{prop}
  Let $f(x) \in \Qb[x]$ be irreducible polynomial of degree $p$ where $p$ is a prime.
  If $f(x)$ has exactly $p-2$ roots and $2$ complex roots, then the Galois group of $f(x)$ is $S_p$.

  \begin{proof}
    Let $L$ be a splitting field of $f$ over $\Qb$ and $R = \{ \alpha_1, \alpha_2, \dots, \alpha_p \}$ be
    the set of all roots of $f(x)$. Since $f(x)$ is irreducible, $f(x) / a_p = m_{\alpha_i, \Qb}, \, \forall i$.
    By lemma~\ref{lemma:extension-exists-condition}, for any $\sigma \in \Gal(L/\Qb)$, $\sigma$ sends
    $\alpha_i$ to another root $\alpha_j$. Also, $\{\alpha_i\}$ generates $L$ so
    $G \triangleq \Gal(L/\Qb) \leq S_p$.

    Now, we define an equivalence relation on $R$ such that $\alpha_i \sim \alpha_j \iff
    \cycle{\alpha_i, \alpha_j} \in G$, that is, $\exists \sigma \in G$ such that $\sigma(\alpha_i) = \alpha_j,
    \sigma(\alpha_j) = \alpha_i$ and $\sigma(\alpha_t) = \alpha_t, \, \forall t \neq i, j$.
 
    We claim that each equivalence class has the same size. Let $[\alpha_i], [\alpha_j]$ be
    two equivalence classes. Since $\alpha_i, \alpha_j$ share the same minimal polynomial,
    by lemma~\ref{lemma:extension-exists-condition}, $\exists \sigma,\, \sigma(\alpha_i) = \alpha_j$,
    and $\sigma$ sends $[\alpha_i]$ to $[\alpha_j]$, since if $\alpha_k \in [\alpha_i]$,
    $\cycle{\alpha_i, \alpha_k} \in G$ and thus $\sigma \cycle{\alpha_i, \alpha_k} \sigma^{-1}
    = \cycle{\alpha_j, \sigma(\alpha_k)} \in G$. Since $\sigma$ 1-1, $\big| [\alpha_i] \big| \leq \big| [\alpha_j] \big|$,
    and by symmetry we have $\big| [\alpha_i] \big| = \big| [\alpha_j] \big|$.

    But then if $[\alpha_i] = n$, $p = \abs{R} = \sum \big| [\alpha_j] \big| = kn$,
    so either there are $p$ equivalence classes each with size $1$, which is impossible since
    the two complex root are equivalent by conjugation, or there are only one equivalence classes,
    which means that any 2 cycle are in $G$, and thus $G = S_p$.
  \end{proof}
\end{prop}

\subsection{Fundamental theorem of Galois theory}
\begin{theorem}[Main theorem]
  Let $L/K$ be a Galois extension, where $L$ be a splitting field of a separable polynomial $f$,
  and let $G = \Gal(L/K)$. Then:

  \begin{enumerate}[(\arabic*)]
    \item There is a 1-1 correspond from the set of intermediate field to the set of subgroup given by
      \[
        \begin{array}{ccc}
          \{ M : K \subseteq M \subseteq L \} & \xleftrightarrow{\quad\quad} & \{ H : H \leq G \} \\
          M & \xmapsto{\quad\quad} & \Gal(L/M) \\
          L^H & \reflectbox{\ensuremath{\xmapsto{\quad\quad}}} & H
        \end{array}
      \]

      \begin{proof}
        We check these two mapping are the inverse of each other.

        By theorem~\ref{thm:extension-of-fix-field-is-galois}, $\Gal(L/L^H) = H$.

        Now we have $M \subseteq L^{\Gal(L/M)}$. Since $L/M$ is galois, $[L: M] = \abs{\Gal(L/M)}$.
        By theorem~\ref{thm:extension-of-fix-field-is-galois} again,
        $\abs{\Gal(L/M)} = [L: L^{\Gal(L/M)}]$, thus $[L: M] = [L: L^{\Gal(L/M)}] \implies M = L^{\Gal(L/M)}$.
      \end{proof}

    \item If $M_1 = L^{H_1}, M_2 = L^{H_2}$, then $M_1 \subseteq M_2 \iff H_2 \leq H_1$.
      \begin{proof}
        Obvious.
      \end{proof}
    \item If $M = L^H$, then $M/K$ is normal if and only if $H \lhd G$.
    \begin{proof}
      If $\sigma \in G$, then
      \begin{align*}
        \tau \in \Gal(L/\sigma(M)) &\iff \tau(\sigma(x)) = \sigma(x), \, \forall x \in M \\
        &\iff \sigma^{-1} \tau \sigma(x) = x, \, \forall x \in M \\
        &\iff \sigma^{-1} \tau \sigma \in \Gal(L/M) \\
        &\iff \tau \in \sigma \Gal(L/M) \sigma^{-1}
      \end{align*}
      So by prop~\ref{prop:TFAE-of-normal-extension}, $M/K$ is normal if and only if for all $\sigma \in G$,
      $\sigma(M) = M$ and thus if and only if $\Gal(L/M) = \Gal(L/\sigma(M))$.
      But $\Gal(L/\sigma(M)) = \sigma \Gal(L/M) \sigma^{-1} = \sigma H \sigma^{-1}$ by the discussion
      above, hence it is equivalent to $H = \sigma H \sigma^{-1}$ which is same as $H \lhd G$.
    \end{proof}

    \item If $H \lhd G$, then $G / H \cong \Gal(M/K)$.
      \begin{proof}
        Since $H \lhd G$, by (3) we know that $M/K$ is Galois. Define $\varphi = \sigma :: \Gal(L/K)
        \mapsto \sigma\big|_M :: \Gal(M/K)$. The mapping is well defined
        since by prop~\ref{prop:TFAE-of-normal-extension}, $\sigma(M) = M$.
        Also, this map is onto since by corollary~\ref{coro:extension-exists-splitting-field},
        each $\tau \in \Gal(M/K)$ could be extend to $\sigma \in \Gal(L/K)$ because
        $\bar\tau(f) = f$. Finally notice that $\ker \varphi = H$, thus by the
        first isomorphism theorem, $G/H \cong \Gal(M/K)$.
      \end{proof}

    \item If $M_1 = L^{H_1}, M_2 = L^{H_2}$, then $M_1 \cup M_2 = L^{\langle H_1, H_2 \rangle}$ and
      $M_1 M_2 = L^{H_1 \cup H_2}$.
  \end{enumerate}
\end{theorem}

\begin{theorem}
  Let $L/K$ be Galois, and $N/K$ be any extension, then $LN / N$ is galois and
  $\Gal(\quot{LN}{N}) \cong \Gal(\quot{L}{L \cap N})$ by the isomorphism
  $\varphi = \sigma \mapsto \sigma\big|_L$.

  \begin{proof}
    Let $L$ be a splitting field of the separable polynomial $f(x)$ over $K$,
    say $L = K(\alpha_1, \dots, \alpha_n)$. Then $LN = N(\alpha_1, \dots, \alpha_n)$,
    which can be regarded as a splitting field of $f(x)$ over $N$.
    Thus by theorem~\ref{thm:splitting-field-iff-finite-normal}, $LN/N$ is Galois.

    Now we check that $\varphi$ is well defined, notice that $f(\sigma(\alpha_i))
    = \sigma(f(\alpha_i)) = 0$ since $\sigma$ fixes $K$, and thus $f$ sends $\alpha_i$
    to some $\alpha_j$. Also, $\{ \alpha_i \}$ generate $L$ over $K$, thus $\sigma\big|_L(L) = L$.

    If $\sigma\big|_L = \text{id}_L$, then $\sigma(\alpha_i) = \alpha_i, \, \forall i$.
    Since $\{\alpha_i\}$ generate $LN$ over $N$, $\sigma = \text{id}_{LN}$. Thus $\varphi$ 1-1.

    Finally, let $H = \Image \varphi$, we claim that $L^H = L \cap N$, since
      \begin{align*}
        \alpha \in L^H &\iff \alpha \in L \text{ and } \forall \sigma \in \Gal(LN/N),\, \sigma\big|_L(\alpha) = \alpha \\
        &\iff \alpha \in L \text{ and } \forall \sigma \in \Gal(LN/N),\, \sigma(\alpha) = \alpha \\
        &\iff \alpha \in L \text{ and } \alpha \in (LN)^{\Gal(LN/N)} \\
        &\iff \alpha \in L \text{ and } \alpha \in N \iff \alpha \in L \cap N
      \end{align*}
  \end{proof}
\end{theorem}

\begin{remark}
  If $L/K$ is Galois and $N/K$ is finite, then $[NL: K] = [N: K][L: K] / [N \cap L: K]$.
  \begin{proof}
    \[ [NL: K]/[N:K] = [NL: L] = \Gal(NL/L) = \Gal(L / N \cap L) = [L: N \cup L] = [L: K]/[N \cap L: K] \]
    and the proof is completed.
  \end{proof}
\end{remark}
