%! TEX root=../main.tex
\subsection{Finite field}

\begin{definition}
  A polynomial $f(x) \in K[x]$ is said to be \emph{separable}\index{seperable}
  if its irreducible factors have no multiple roots in a splitting field $L$.
\end{definition}

\begin{definition}
  If $f(x) = a_n x^n + \dots + a_1 x + a_0$, then define $f'(x) \triangleq n a_n x^{n-1} + \dots + 2a_2 x + a_1$.
\end{definition}

\begin{theorem} \label{thm:multiple-root-condition}
  Let $f(x) \in K[x]$ be monic, irreducible of positive degree, then all the roots of $f(x)$
  in a splitting field are simple if and only if $\gcd(f(x), f'(x)) = 1$.

  \begin{proof}
    ``$\Rightarrow$'': We can write $f(x) = (x - \alpha_1) (x - \alpha_2) \dotsm (x - \alpha_n)$ where
    $\alpha_i$ are distinct roots of $f$. Then $f'(x) = \sum_{i = 1}^n f(x) / (x - \alpha_i)$
    and we have $(x - \alpha_i) \nmid f(x)$ for all $i$.

    ``$\Leftarrow$'': Assume $f(x) = (x - \alpha)^k g(x)$ with $k \geq 2$.
    Then $f'(x) = k (x - \alpha)^{k-1} g(x) + (x - \alpha)^k g'(x)$ which implies $(x - \alpha) \mid f(x)$.
    So $(x - \alpha) \mid \gcd(f(x), f'(x))$ and thus $\gcd(f(x), f'(x)) \neq 1$.
  \end{proof}
\end{theorem}

\begin{remark}
  The following are equivalent:
  \begin{enumerate}
    \item $\alpha$ is a multiple root of $f(x)$.
    \item $\alpha$ is a common root of $f(x)$ and $f'(x)$.
    \item $m_{\alpha, K} \mid f(x)$ and $m_{\alpha, K} \mid f'(x)$.
  \end{enumerate}
\end{remark}

\begin{theorem} \label{thm:size-of-finite-field}
  There is a finite field $K$ with $\abs{K} = q \iff q = p^n$ for some prime $p$ and $n \in \Nb$.
  In this situation, $K$ is unique up to isomorphism, denote by $\Fb_{p^n}$.

  \begin{proof}
    ``$\Rightarrow$": Let $p = \Char K$ and $[K : \Zb/ p\Zb] = n$, then $\abs{K} = p^n$.

    ``$\Leftarrow$": Let $K$ be a splitting field for $f(x) = x^{p^n} - x$ over $\Fb_p$.
    We claim that the set of all roots of $f(x)$ forms a field. Since if $\alpha, \beta$ are
    two roots of $f$, obviously $\alpha \beta, \alpha \beta^{-1}$ are also roots,
    and by $(\alpha \pm \beta)^{p^n} = \alpha^{p^n} \pm \beta^{p^n} = \alpha \pm \beta$ because
    $\Char K = p$.
    $\alpha \pm \beta$ are also roots, hence the roots form a field. By definition, $K$
    is the smallest field containing $\Fb_p$ and roots of $f(x)$, so
    $K$ is exactly the set of roots of $f(x)$.

    Also, $f'(x) = -1$ has no root, so $f(x)$ has no multiple root which implies $\abs{K} = p^n$.

    Moreover, if $K'$ is another finite field with $\abs{K'} = p^n$, then for all $\alpha \in K'$,
    $\alpha^{p^n} = \alpha$, so $\alpha$ is a root of $f(x)$, which implies that $K'$ is
    a splitting field for $f(x)$ over $\Fb_p$. By theorem~\ref{thm:two-splitting-field-are-isom},
    $K \cong K'$.
  \end{proof}
\end{theorem}

\begin{theorem} \label{thm:aut-of-finite-field}
  Let $n \in \Nb$ and $\Fb_q$ be a finite field. Then there exists
  a unique extension $\Fb_{q^n} / \Fb_q$ s.t. $[\Fb_{q^n} : \Fb_q] = n$, and
  $\Aut \Big( \Fb_{q^n} / \Fb_q \Big) = \gen{\sigma_q}$ with
  $\sigma_q = \alpha :: \Fb_{q^n} \mapsto \alpha^q :: \Fb_{q^n}$.
  $\sigma_q$ is called the \emph{\Index{Frobenius homomorphism}}.

  \begin{proof}
    By theorem~\ref{thm:size-of-finite-field},
    $q = p^r$ for some prime $p$ and $r \in \Nb$, so $q^n = p^{nr}$ which is a power of
    a prime. Again by theorem~\ref{thm:size-of-finite-field},
    $\Fb_{q^n}$ is the splitting field for $x^{p^{nr}} - x$ over $\Fb_p$.
    Since $x^q - x \mid x^{q^n} - x$, $\Fb_q \subseteq \Fb_{q^n}$ and thus $[\Fb_{q^n}: \Fb_{q}] = n$.

    Then we proof that $\sigma_q$ is indeed in $\Aut \Big( \quot{\Fb_{q^n}}{\Fb_q} \Big)$.
    We check that
    \[
      \begin{aligned}
        \sigma_q(\alpha+\beta) &= (\alpha+\beta)^q = \alpha^q + \beta^q = \sigma_q(\alpha) + \sigma_q(\beta) \\
        \sigma_q(\alpha\beta) &= (\alpha\beta)^q = \alpha^q \beta^q = \sigma_q(\alpha) \sigma_q(\beta)
      \end{aligned}
    \]
    Now $\sigma_q$ is nontrivial since $\sigma_q$ send $1$ to $1$, so $\sigma_q$ is 1-1 and hence an
    isomorphism since $\Fb_q$ is finite. Also, for all $\alpha \in \Fb_q$, $\sigma_q(\alpha)
    = \alpha^{q} = \alpha$, hence $\sigma_q$ fixes $\Fb_q$.

    Finally we prove that the order of $\sigma_q$ is $n$. Assume not, so $\ord(\sigma_q) = m < n$.
    Then $\sigma_q^m = \Id \implies x^{q^m} - x=0$ for each $x \in \Fb_{q^n}$.
    But $x^{q^m} - x = 0$ has at most $q^m < q^n$ roots, which leads to a contradiction.
  \end{proof}
\end{theorem}

\begin{remark}
  By theorem~\ref{thm:finite-subgroup-of-field-is-cyclic}, the multiplication group
  of $\Fb_{q^n}$ is cyclic, so $\Fb_{q^n}^\times = \gen{ \alpha }
  \subseteq \Fb_q(\alpha) \setminus \{0\} \subseteq \Fb_{q^n} \setminus \{0\}$,
  hence $\Fb_{q^n} = \Fb_q(\alpha)$.
\end{remark}

\begin{lemma}
  Every irreducible polynomial $f(x)$ in $\Fb_{p^n}[x]$ is separable.

  \begin{proof}
    Without lost of generality, assume $f(x)$ is monic.

    Since $\sigma_p$ is an isomorphism, $\Fb_{p^n} = \Fb_{p^n}^p = \{ \alpha^p \mid \alpha \in \Fb_{p^n}\}$.
    Now assume $f(x)$ has a multiple root $\alpha$, then $m_{\alpha, \Fb_p} = f(x)$ since $f$
    is irreducible. By theorem~\ref{thm:multiple-root-condition} we also have
    $f(x) = m_{\alpha, \Fb_p} \mid f'(x)$, but $\deg f'(x) < \deg f(x)$ so we must have $f'(x) \equiv 0$.

    Write $f(x) = a_n x^n + \ldots + a_1 x + a_0$, then $f'(x) \equiv 0$ implies $k a_k = 0_{\Fb_p}$ for each $k$,
    which means that if $a_k \neq 0 \implies p \mid k$. So
    \[ f(x) = a_{mp} x^{mp} + a_{(m-1)p} x^{(m-1)p} + \dots + a_p x^p + a_0 =
    (a_{mp} x^m + \ldots + a_p x + a_0)^p. \]
    But this implies $f(x)$ is reducible, which is a contradiction.
  \end{proof}
\end{lemma}

\begin{theorem}
  $x^{p^n} - x$ equals the product of all monic irreducible polynomials in
  $\Fb_p[x]$ of degree $d$ where $d$ runs through all divisors of $n$. i.e.

  \begin{proof}
    By lemma, each irreducible polynomial is separable, and if $f(x), g(x) \in \text{RHS}$,
    and $f(\alpha) = g(\alpha) = 0$, then $f = m_{\alpha, \Fb_p} = g$. Thus RHS is separable.
    LHS is separable since $f' = 1$, so we could prove the equality by checking that
    they have same roots.

    $\text{LHS} \mid \text{RHS}$: $\forall \alpha \in \Fb_{p^n}$,
    $[\Fb_p(\alpha): \Fb_p] \mid [\Fb_{p^n}: \Fb_p] = n$, thus $\deg m_{\alpha, \Fb_p} \mid n$
    and hence $m_{\alpha, \Fb_p}$ appears in RHS.

    $\text{RHS} \mid \text{LHS}$: Assume $\deg m_{\alpha, \Fb_p} = d \mid n$, then
    $[\Fb_p(\alpha): \Fb_p] = d$, so $\alpha^{p^d} = \alpha$, and hence
    $\alpha = \alpha^{p^d} = \alpha^{p^{2d}} = \dots = \alpha^{p^n}$.
  \end{proof}
\end{theorem}

\begin{definition}
  The {\bf M\"{o}bius $\mu$-function} \index{M\"{o}bius $\mu$-function}
  is defined as
  \[ \mu(n) =
    \begin{cases}
      1, & \text{ if } n = 1 \\
      0, & \text{ if } n \text{ has a square factor } \\
      (-1)^r, & \text{ if } n \text{ is a product of $r$ distinct primes} \\
    \end{cases}
  \]
\end{definition}


\begin{theorem}[M\"{o}bius inversion formula]
  If $f(n) = \sum\limits_{d \mid n} g(d)$, then
  $g(n) = \sum\limits_{d \mid n} \mu(d) f\left(\frac{n}{d}\right)$.
\end{theorem}

\begin{remark}
  Let $\psi_q(d)$ denote the number of monic irreducible polynomials of degree $d$
  in $\Fb_q$, then $q^n = \sum_{d \mid n} d \psi_q(d)$.

  Using the convolution notation, we have $(n \mapsto q^n) = \mathbbm{1} * (n \mapsto n \psi_q(n))$.
  Where $\mathbbm{1} \triangleq (n \mapsto 1)$. It could be seen that $\mathbbm{1}^{-1} = \mu$.
  Thus $n \psi_q(n) = \sum_{d \mid n} \mu(d) q^{n/d}$.
\end{remark}
