\documentclass[a4paper,titlepage]{article}

%%%%%%%%%%%%%%%%%%page size%%%%%%%%%%%%%%%%%%
% \paperwidth=65cm
% \paperheight=160cm

%%%%%%%%%%%%%%%%%%%Package%%%%%%%%%%%%%%%%%%%
\usepackage[margin=3cm]{geometry}
\usepackage{mathtools,amsthm,amssymb}
\usepackage{centernot}
\usepackage{yhmath}
\usepackage{mathalfa}
\usepackage{graphicx}
\usepackage{fontspec}
\usepackage{titlesec}
\usepackage{titling}
\usepackage{fancyhdr}
\usepackage{tabularx}
\usepackage[square, comma, numbers, super, sort&compress]{natbib}
\usepackage[usenames, dvipsnames]{color}
\usepackage[shortlabels, inline]{enumitem}
\usepackage{xpatch}
\usepackage{imakeidx}
\usepackage{hvindex}
%%% fix bugs in hvindex .................
\def\IndexXXii#1!#2@#3@#4\IndexNIL{%
  \ifx\relax#3\relax            %               no @ in last arg
    \def\hvTemp{#2}%
    \ifx\hvTemp\hvEncap\index{#1!{#2}}#2\else
      \ifx\hvIDXfont\hvIDXfontDefault\index{#1!{#2}}#2% <--- Here a % and #1 were missing
      \else\index{#1!#2@\hvIDXfont{#2}}\hvIDXfont{#2}\fi\fi%
  \else\index{#1!\protect#2@#3}#3\fi}

\def\IndexXXiii#1!#2!#3@#4@#5\IndexNIL{%
  \ifx\relax#4\relax 		% 		no @ in last arg
    \def\hvTemp{#3}%
    \ifx\hvTemp\hvEncap\index{#1!#2!{#3}}#3\else
      \ifx\hvIDXfont\hvIDXfontDefault\index{#1!#2!{#3}}#3
      \else\index{#1!#2!#3@\hvIDXfont{#3}}\hvIDXfont{#3}\fi\fi%
  \else\index{#1!#2!\protect#3@#4}#4\fi}
\usepackage[unicode, pdfborder={0 0 0}, bookmarksdepth=-1]{hyperref}
\hypersetup{
    colorlinks,
    linkcolor=red,
  }
\makeindex[columns=2, options= -s index_style.ist]

%\usepackage{tabto}     
%\usepackage{soul}      
%\usepackage{ulem}      
%\usepackage{wrapfig}   
%\usepackage{floatflt}  
\usepackage{float}     
\usepackage{caption}   
\usepackage{subcaption}
%\usepackage{setspace}  
\usepackage{mdframed}  
%\usepackage{multicol}  
%\usepackage[abbreviations]{siunitx}
%\usepackage{dsfont}   

%%%%%%%%%%%%%%%%%%%TikZ%%%%%%%%%%%%%%%%%%%%%%
\usepackage{tikz}
\usepackage{tikz-cd}
%\usepackage{circuitikz}
\usetikzlibrary{calc}
\usetikzlibrary{arrows}
\usetikzlibrary{shapes}
\usetikzlibrary{positioning}

\tikzstyle{every picture}+=[remember picture]

%%%%%%%%%%%%%%中文 Environment%%%%%%%%%%%%%%%
\usepackage[CheckSingle, CJKmath]{xeCJK}  % xelatex 中文
\usepackage{CJKulem}	% 中文字裝飾
\setCJKmainfont[BoldFont=cwTeX Q Hei]{cwTeX Q Ming}
\setCJKsansfont[BoldFont=cwTeX Q Hei]{cwTeX Q Ming}
\setCJKmonofont[BoldFont=cwTeX Q Hei]{cwTeX Q Ming}

%%%%%%%%%%%%%%%%%font size%%%%%%%%%%%%%%%%%%%
\renewcommand{\baselinestretch}{1.1}
%\def\normalsize{\fontsize{10}{15}\selectfont}
%\def\large{\fontsize{12}{18}\selectfont}
%\def\Large{\fontsize{14}{21}\selectfont}
%\def\LARGE{\fontsize{16}{24}\selectfont}
%\def\huge{\fontsize{18}{27}\selectfont}
%\def\Huge{\fontsize{20}{30}\selectfont}

%%%%%%%%%%%%%%%Theme Input%%%%%%%%%%%%%%%%%%%
%\input{themes/chapter/neat}
%\input{themes/env/problist}

%%%%%%%%%%%titlesec settings%%%%%%%%%%%%%%%%%
%\titleformat{\chapter}{\bf\Huge}
            %{\arabic{section}}{0em}{}
%\titleformat{\section}{\centering\Large}
            %{\arabic{section}}{0em}{}
%\titleformat{\subsection}{\large}
            %{\arabic{subsection}}{0em}{}
%\titleformat{\subsubsection}{\bf\normalsize}
            %{\arabic{subsubsection}}{0em}{}
%\titleformat{command}[shape]{format}{label}
            %{gutter}{before}[after]

%%%%%%%%%%%%variable settings%%%%%%%%%%%%%%%%
%\numberwithin{equation}{section}
%\setcounter{secnumdepth}{4}
%\setcounter{tocdepth}{1}
%\setcounter{section}{0}
%\graphicspath{{images/}}

%%%%%%%%%%%%%%%page settings%%%%%%%%%%%%%%%%%
\newcolumntype{C}[1]{>{\centering\arraybackslash}p{#1}}
\setlength{\headheight}{15pt}  % with titling
\setlength{\droptitle}{-2.5cm}
%\posttitle{\par\end{center}}  % distance between title and content
\parindent=0pt % indent size
\parskip=1ex    % line space
%\pagestyle{empty}  % empty: no page number
%\pagestyle{fancy}  % fancy: fancyhdr

% use with fancygdr
%\lhead{\leftmark}
%\chead{}
%\rhead{}
%\lfoot{}
%\cfoot{}
%\rfoot{\thepage}
%\renewcommand{\headrulewidth}{0.4pt}
%\renewcommand{\footrulewidth}{0.4pt}

%\fancypagestyle{firststyle}
%{
  %\fancyhf{}
  %\fancyfoot[C]{\footnotesize Page \thepage\ of \pageref{LastPage}}
  %\renewcommand{\headrule}{\rule{\textwidth}{\headrulewidth}}
%}

\setlist{itemsep=0em, topsep=0.2em}

%%%%%%%%%%%%%%%renew command%%%%%%%%%%%%%%%%%
% \renewcommand{\contentsname}{Table of Content}
% \renewcommand{\refname}{Reference}
\renewcommand{\abstractname}{\LARGE Abstract}

%%%%%%%%symbol and function settings%%%%%%%%%
% adjust \exists and \forall spacing
\let\existstemp\exists
\let\foralltemp\forall
\renewcommand*{\exists}{\existstemp\mkern4mu}
\renewcommand*{\forall}{\foralltemp\mkern4mu}

\DeclarePairedDelimiter{\abs}{\lvert}{\rvert}
\DeclarePairedDelimiter{\norm}{\lVert}{\rVert}
\DeclarePairedDelimiter{\inpd}{\langle}{\rangle} % inner product
\DeclarePairedDelimiter{\ceil}{\lceil}{\rceil}
\DeclarePairedDelimiter{\floor}{\lfloor}{\rfloor}
\DeclareMathOperator{\adj}{adj}
\DeclareMathOperator{\sech}{sech}
\DeclareMathOperator{\csch}{csch}
\DeclareMathOperator{\arcsec}{arcsec}
\DeclareMathOperator{\arccot}{arccot}
\DeclareMathOperator{\arccsc}{arccsc}
\DeclareMathOperator{\arccosh}{arccosh}
\DeclareMathOperator{\arcsinh}{arcsinh}
\DeclareMathOperator{\arctanh}{arctanh}
\DeclareMathOperator{\arcsech}{arcsech}
\DeclareMathOperator{\arccsch}{arccsch}
\DeclareMathOperator{\arccoth}{arccoth}
\newcommand{\np}[1]{\\[{#1}] \indent}
\newcommand{\tr}[1]{{#1}^\mathrm{t}}
%%%% Geometry Symbol %%%%
\newcommand{\degree}{^\circ}
\newcommand{\Arc}[1]{\wideparen{{#1}}}
\newcommand{\Line}[1]{\overleftrightarrow{{#1}}}
\newcommand{\Ray}[1]{\overrightarrow{{#1}}}
\newcommand{\Segment}[1]{\overline{{#1}}}
%%%% Math symbol %%%%
\newcommand{\defeq}{\vcentcolon=}
\newcommand{\Nb}{\mathbb{N}}
\newcommand{\Zb}{\mathbb{Z}}
\newcommand{\Qb}{\mathbb{Q}}
\newcommand{\Rb}{\mathbb{R}}
\newcommand{\Cb}{\mathbb{C}}
\newcommand{\Hb}{\mathbb{H}}
\newcommand{\Fb}{\mathbb{F}}
\newcommand{\Fbx}{\mathbb{F}^\times}
\newcommand{\Qbx}{\mathbb{Q}^\times}
\newcommand{\Rbx}{\mathbb{R}^\times}
\newcommand{\Cbx}{\mathbb{C}^\times}
\newcommand{\Hbx}{\mathbb{H}^\times}
\newcommand{\Ic}{\mathcal{I}}
\newcommand{\Vc}{\mathcal{V}}
\newcommand{\Gc}{\mathcal{G}}
\newcommand{\Fc}{\mathcal{F}}
\newcommand{\Cc}{\mathcal{C}}
\newcommand{\Dc}{\mathcal{D}}
\newcommand{\sT}{{\sf T}}
\newcommand{\sI}{{\sf I}}

\newcommand{\Mf}{\mathfrak{M}}
\newcommand{\Grf}{\mathfrak{Gr}}

\newcommand{\bigOp}{\raisebox{0.3ex}{$\bigoplus$}}
\newcommand{\bigOt}{\raisebox{0.3ex}{$\bigotimes$}}
\newcommand{\Op}{\oplus}
\newcommand{\Ot}{\otimes}

\DeclareMathOperator{\Sym}{Sym}
\DeclareMathOperator{\Alt}{Alt}
\DeclareMathOperator{\diag}{diag}
\DeclareMathOperator{\sgn}{sgn}
\DeclareMathOperator{\lcm}{lcm}
\DeclareMathOperator{\Image}{Im}
\DeclareMathOperator{\Char}{char}
\DeclareMathOperator{\Fix}{Fix}
\DeclareMathOperator{\Inn}{Inn}
\DeclareMathOperator{\Aut}{Aut}
\DeclareMathOperator{\Isom}{Isom}
\DeclareMathOperator{\Tor}{Tor}
\DeclareMathOperator{\Exp}{Exp}
\DeclareMathOperator{\Syl}{Syl}
\DeclareMathOperator{\Gal}{Gal}

% multilinear
\DeclareMathOperator{\Hom}{Hom}
\DeclareMathOperator{\Lrad}{lrad}
\DeclareMathOperator{\Rrad}{rrad}
\DeclareMathOperator{\rank}{rank}
\DeclareMathOperator{\trace}{Tr}

% extensions
\DeclareMathOperator{\Stab}{Stab}
\DeclareMathOperator{\Der}{Der}
\DeclareMathOperator{\PDer}{PDer}
\DeclareMathOperator{\Ext}{Ext}
%
\DeclareMathOperator{\Jac}{Jac}
\DeclareMathOperator{\Spec}{Spec}

\newcommand{\ob}{\overline}
\DeclareMathOperator{\ord}{ord}
\DeclarePairedDelimiter{\gen}{\langle}{\rangle} % generator
%\newcommand*\quot[2]{{^{\textstyle #1}\Big/_{\textstyle #2}}}
\newcommand*\quot[2]{{#1}/{#2}}
\newcommand\bij{\lhook\joinrel\twoheadrightarrow}
\newcommand\toone{\hookrightarrow}
\newcommand\onto{\twoheadrightarrow}
\newcommand\acts{\curvearrowright}
\newcommand\revacts{\curvearrowleft}
\newcommand\isoto{\xrightarrow{\sim}}
\newcommand\deffunc[5]{\ensuremath{
  \arraycolsep=1pt
  \begin{array}{rcl}
    #1: & #2 & \to #3 \\
       & #4 & \mapsto #5
  \end{array}
}}

% just to make sure it exists
\providecommand\given{}
% can be useful to refer to this outside \Set
\newcommand*\SetSymbol[1][]{%
  \nonscript\:#1\vert
  \allowbreak
  \nonscript\:
\mathopen{}}
\DeclarePairedDelimiterX\Set[1]\{\}{%
  \renewcommand\given{\SetSymbol[\delimsize]}
  \,#1\,
}

\DeclarePairedDelimiterX\Gen[1]{\langle}{\rangle}{%
  \renewcommand\given{\SetSymbol[\delimsize]}
  \,#1\,
}

% cycle group \cycle{1,2,3} => (1 2 3)
\ExplSyntaxOn
\NewDocumentCommand{\cycle}{ O{\;} m }
 {
  (
  \alec_cycle:nn { #1 } { #2 }
  )
 }

\seq_new:N \l_alec_cycle_seq
\cs_new_protected:Npn \alec_cycle:nn #1 #2
 {
  \seq_set_split:Nnn \l_alec_cycle_seq { , } { #2 }
  \seq_use:Nn \l_alec_cycle_seq { #1 }
 }
\ExplSyntaxOff

\newcommand\Div{\mathrel{\big|}}
\newcommand\nDiv{\mathrel{\not\big|}}
\newcommand\relmiddle[1]{\mathrel{}\middle#1\mathrel{}}
\newcommand{\RNum}[1]{\uppercase\expandafter{\romannumeral #1\relax}}

%%%%%%%%%%%%%%%%%%%%%%%%%%%%%%%%%%%%%%%%%%%%
%\renewcommand{\proofname}{\bf pf:}
\newtheoremstyle{mystyle}% custom style
  {6pt}{15pt}%      top and bottom margin
  {}%               content style
  {}%               indent
  {\bf}%            head style
  {.}%              after head
  {1em}%            distance between head and content
  {}%               Theorem head spec (can be left empty, meaning 'normal')

\theoremstyle{mystyle}
\newtheorem{theorem}{Theorem}
\newtheorem{formula}{Formula}
\newtheorem{conclusion}{Conclusion}
\newtheorem{lemma}{Lemma}
\newtheorem{remark}{Remark}
\newtheorem{observation}{Observation}
\newtheorem*{observation*}{Observation}
\newtheorem{definition}{Def}
\newtheorem{exercise}{Ex}[section]
\newtheorem{example}{Eg}[section]
\newtheorem{fact}{Fact}[section]
\newtheorem{prop}{Prop}[section]
\newtheorem{coro}{Coro}[section]

%%%%%%%%%%%%%%Title information%%%%%%%%%%%%%%
\title{Algebra Homework}
\author{}
\date{\today}

\begin{document}
\maketitle
% \thispagestyle{empty}
% \thispagestyle{fancy}
% \tableofcontents
%%%%%%%%%%%%%include file here%%%%%%%%%%%%%%%

%! TEX root=../main.tex
\section{Fields}

\subsection{Algebraic extensions}

\begin{definition} \hfill
  \begin{itemize}
    \item $L / K$ is called an field extension if $L$ is a field and $K$ is a subfield of $L$.
    \item $L / K$ is called an algebraic extension if $\forall \alpha \in L, \exists f(x) \in K[x]$
      such that $f(\alpha) = 0$.
    \item $K(\alpha_1, \alpha_2, \dots, \alpha_n) \triangleq \big\{ P(\alpha_1, \dots, \alpha_n)
      / Q(\alpha_1, \dots, \alpha_n) : P, Q \in K[x_1, x_2, \dots, x_n] \text{ and } Q \neq 0 \big\}$
  \end{itemize}
\end{definition}

\begin{theorem}[Eisenstein criterion] \mbox{} \\
  Let $f(x) = a_n x^n + \dots + a_1 x + a_0 \in \Zb[x]$ with $\gcd(a_0, a_1, \cdots, a_n) = 1$.
  Assume that there exists a prime $p$ s.t. $p \nmid a_n$ but $p \mid a_i$ for other $i \neq n$,
  and $p^2 \nmid a_0$, then $f$ is irreducible.

  \begin{proof}
    Consider $\bar{f}(x)$, by assumption, $\bar{f}(x) = \bar{a}_n x^n$. So if $f(x) = g(x) h(x)$
    with $\deg g, \deg h \geq 1$, let $g(x) = b_r x^r + \dots + b_0, h(x) = c_{n-r} x^{n-r} + \dots + c_0$,
    then $\bar{g}(x) = \bar{b}_r x^r, \bar{h}(x) = \bar{c}_{n-r} x^{n-r}$ for some
    $r$. But then we would find out that $\bar{b}_0 = \bar{c}_0 = 0$, and thus $p^2 \mid a_0$,
    which is a contradiction, hence $f$ is irreducible.
  \end{proof}
\end{theorem}

\begin{prop}
  Given $L/K$ and $\alpha \in L$, if $\alpha$ is algebraic over $K$, then
  there exists a unique monic irreducible polynomial $m_{\alpha, K}(x) \in K[x]$
  of minimal degree s.t. $m_{\alpha, K}(\alpha) = 0$ and for any other
  $f(x) \in K[x]$ with $f(\alpha) = 0$, we have $m_{\alpha, K} \mid f$.
  We call $m_{\alpha, K}$ the {\bf minimal polynomial} of $\alpha$ over $K$.

  \begin{proof}
    %Consider the evaluation map on $\alpha$:
    %\[ \deffunc{\text{ev}_\alpha}{K[x]}{K[\alpha]}{f(x)}{f(\alpha)} \]
    %The map is

    Let $I$ be the set of all polynomials such that $f(\alpha) = 0$, since $\alpha$ algebraic,
    $I \neq \varnothing$, so pick a monic polynomial $g(x)$ of minimal degree in $I$.
    For any other $f(x) \in I$, write $f(x) = g(x) q(x) + r(x)$ with $\deg r < \deg g$.
    If $r(x) \neq 0$, then $r(\alpha) = f(\alpha) - q(\alpha) g(\alpha)$.
    But then $r(\alpha) = f(\alpha) - q(\alpha) g(\alpha) = 0$ with $\deg r < \deg g$,
    which contradicts the minimality of $g$, thus $r = 0$, and hence $g \mid f$.

    Finally, if $g(x) = h_1(x) h_2(x)$ with $\deg h_1, \deg h_2 < \deg g$,
    then one of them, say $h_1(\alpha) = 0$ again contradicts the minimality of $g$,
    hence $g$ is irreducible.
  \end{proof}
\end{prop}

\begin{prop}
  Let $L/K$ be an extension and $\alpha \in L$, the following are equivalent:
  \begin{enumerate}[(\arabic*)]
    \item $\alpha$ is algebraic over $K$.
    \item $K[\alpha] = K(\alpha)$.
    \item $[K(\alpha): K] < \infty$.
  \end{enumerate}

  \begin{proof}
    (1) $\Rightarrow$ (2): ``$\subset$'' trivial. \\
    ``$\supset$'': For all $\beta \in K(\alpha), \beta = g(\alpha) / h(\alpha)$ with $h(\alpha) \neq 0$.
    So $m_{\alpha, K} \nmid h$. Since $m_{\alpha, K}$ is irreducible, $\gcd(m_{\alpha, K}, h) = 1$,
    hence there exists $a(x), b(x) \in K[x]$ such that $1 = a(x) h(x) + b(x) m_{\alpha, K}(x)$
    Subsitute $\alpha$ and we get $1/h(\alpha) = a(\alpha)$, hence $\beta = g(\alpha) a(\alpha) \in K[\alpha]$.

    (2) $\Rightarrow$ (1): Since $1 / \alpha \in K[\alpha]$, thus $1 / \alpha = f(\alpha)$ for some
    polynomial $f$, hence if $g(x) = xf(x) - 1, g(\alpha) = 0$ which implies $\alpha$ is algebraic.

    (1) $\Rightarrow$ (3): Assume that $\deg m_{\alpha, K} = n$, it is easy to see that
    $K[\alpha] = \gen{ 1, \alpha, \dots, \alpha^{n-1} }_K$. Since (1) $\implies$ (2),
    we have $[K(\alpha): K] = [K[\alpha], K] = n$.

    (3) $\Rightarrow$ (1): Since $[K(\alpha): K] = n$, consider $1, \alpha, \alpha^2, \dots, \alpha^n$.
    Some of these $n+1$ elements may be coincident, but nevertheless these elements are linearly dependent.
    Hence there exists $a_0, \dots, a_n$ not all zero in $K$ s.t.
    $a_0 + a_1 \alpha + \dots + a_n \alpha^n = 0 \implies \alpha$ is algebraic.
  \end{proof}
\end{prop}

\begin{prop}
  Given $M/L$ and $L/K$, $[M: K] = [M: L] [L: K]$.

  \begin{proof}
    If $[M:L] = m < \infty$ and $[L:K] = n < \infty$, then $L \cong K^{\oplus n}, M \cong L^{\oplus m}$.
    So $M \cong \left( K^{\oplus n} \right)^{\oplus m} \cong K^{\oplus mn}$, thus $[M: K] = mn$.

    Now if $[M: K] = l < \infty$, then there exists a basis $\{ z_1, z_2, \dots, z_l \}$
    which is a basis for $M$ over $K$. Then $M = K z_1 + \dots + K z_l \subset L z_1 +
    \dots + L z_l \subset M \implies M = L z_1 + \dots + L z_l$. Hence $[M: L] < \infty$.
    Also, since $L$ is a $K$-linear subspace of $M$, $[L: K] \leq l \implies [L: K] < \infty$.
    Thus if $[M: L] = \infty$ or $[L: K] = \infty$, then $[M: K] = \infty$.
  \end{proof}
\end{prop}

\begin{prop} \label{prop:alg-elements-form-a-field}
  Given $L/K$, define $L^{\text{alg}} \triangleq \{ \alpha \in L \mid \alpha \text{ is algebraic over } K \}$,
  then $L^{\text{alg}}$ is a subfield of $L$.

  \begin{proof}
    Notice that if $\alpha, \beta \in L^{\text{alg}}$, then $\beta$ is algebraic over $K$
    implies that $\beta$ is algebraic over $K(\alpha)$. Thus
    \[ [K(\alpha, \beta): K] = [K(\alpha)(\beta): K(\alpha)] [K(\alpha): K]
      < \infty \]

    Also, since $K(\alpha + \beta), K(\alpha - \beta), K(\alpha \beta), K(\alpha / \beta)$ are
    all contained in $K(\alpha, \beta)$, they are all algebraic over $K$, thus
    these elements are all algebraic, and hence $L^{\text{alg}}$ is a subfield.
  \end{proof}
\end{prop}

\begin{prop}
  $[L: K] < \infty$ if and only if $L = K(\alpha_1, \alpha_2, \dots, \alpha_n)$ with each $\alpha_i$
  algebraic over $K$. In this case, $L / K$ is algebraic (but the other side may not hold).

  \begin{proof}
    ``$\Rightarrow$'': Let $[L: K] = n$, so there is a basis $\{ \alpha_1, \alpha_2, \dots, \alpha_n \}$
    for $L$ over $K$. It is easy to see that $L = K(\alpha_1, \dots, \alpha_n)$.
    Also $[K(\alpha_i): K] \leq [L: K] < \infty$, thus $\alpha_i$ is algebraic.

    ``$\Leftarrow$'': Since $\alpha_i$ is algebraic over $K$, $\alpha_i$ is algebraic over $K(\alpha_1, \dots, \alpha_{i-1})$.
    Thus
    \[ [L: K] = [K(\alpha_1, \dots, \alpha_n): K(\alpha_1, \dots, \alpha_{n-1})]
      [K(\alpha_1, \dots, \alpha_{n-1}): K(\alpha_1, \dots, \alpha_{n-2})] \dots [K(\alpha_1): K] < \infty \]
    Moreover, $\forall \alpha \in L, [K(\alpha): K] \leq [L: K] < \infty$, so $\alpha$ is algebraic over $K$.
  \end{proof}
\end{prop}

\begin{coro}
  Given $L/K$, and $S$ a subset of $L$, if $\forall \alpha \in S$, $\alpha$ is
  algebraic over $K$, then $K(S) / K$ is algebraic.

  \begin{proof}
    If $\beta \in K(S)$, by definition we know that there exists
    $\alpha_1, \dots, \alpha_n$ such that $\beta \in K(\alpha_1, \dots, \alpha_n)$.
    Thus $\beta$ is algebraic over $K$.
  \end{proof}
\end{coro}

\begin{prop} \label{prop:alg-tower-implies-alg}
  If $M/L$ and $L/K$ are algebraic, then $M/K$ is algebraic.

  \begin{proof}
    For all $\alpha \in M$, since $\alpha$ is algebraic over $L$,
    there exists $a_{n-1}, \dots, a_0$ so that $\alpha^n + a_{n-1} \alpha^{n-1} + \dots + a_0 = 0$,
    that is, $\alpha$ is algebraic over $K(a_0, \dots, a_{n-1})$.

    So $[K(a_0, \dots, a_{n-1}, \alpha): K] = [K(a_0, \dots, a_{n-1})(\alpha): K(a_0, \dots, a_{n-1})]
    [K(a_0, \dots, a_{n-1}): K] < \infty$, thus $\alpha$ is algebraic over $K$.
  \end{proof}
\end{prop}

\begin{definition}
  Given $L/L_1$ and $L/L_2$, $L_1 L_2$ is defined as the smallest subfield of $L$
  containing both $L_1$ and $L_2$.
\end{definition}

\begin{prop}
  Let $[L_1: K] = m$ and $[L_2: K] = n$.
  \begin{enumerate}[(\arabic*)]
    \item $[L_1 L_2: K] \leq mn$.
    \item If $\gcd(m, n) = 1$, then $[L_1 L_2: K] = mn$.
  \end{enumerate}

  \begin{proof}
    (1): Assume $L_1 = K(\alpha_1, \dots, \alpha_m), L_2 = K(\beta_1, \dots, \beta_n)$.
    We could find that $L_1 L_2 = K(\alpha_1, \dots, \alpha_m, \beta_1, \dots, \beta_n)$.
    Notice that $[K(\beta_1, \dots, \beta_m)(\alpha_i): K(\beta_1, \dots, \beta_m)] \leq [K(\alpha_i): K]$,
    and thus $[L_1 L_2: K] = [K(\alpha_1, \dots, \alpha_m, \beta_1, \dots, \beta_n)
    : K(\beta_1, \dots, \beta_n)] [K(\beta_1, \dots, \beta_m): K] \leq [K(\alpha_i, \dots, \alpha_n): K]
    [K(\beta_1, \dots, \beta_n): K] = [L_1: K][L_2: K]$.

    (2): Notice that $[L_i: K] \mid [L_1 L_2: K]$, so $mn \mid [L_1 L_2: K]$. By
    (1), $[L_1 L_2: K] \leq nm$, hence $[L_1 L_2: K] = nm$.
  \end{proof}
\end{prop}

\begin{definition}
  Let $R$ be a commutative ring with $1$, and $I$ be an ideal of $R$, then
  \begin{itemize}
    \item $I$ is called a {\bf maximal ideal}\index{Ideal!maximal ideal} if for any ideal $J$ satisfying
      $I \subseteq J$ we have $J = I \text{ or } J = R$.
    \item $I$ is called a {\bf prime ideal}\index{Ideal!prime ideal}
      if $I \neq R$ and $ab \in I \implies a \in I \text{ or } b \in I$.
  \end{itemize}
\end{definition}

\begin{prop} \label{prop:max-prime-to-field-int-domain}
  Suppose $R$ is a ring and $I \subsetneq R$ is an ideal, then
  \begin{enumerate}
    \item $I$ is maximal $\iff$ $R / I$ is a field.
    \item $I$ is a prime ideal $\iff$ $R / I$ is an integral domain.
  \end{enumerate}

  \begin{proof} \hfill
    \begin{enumerate}
      \item ``$\Rightarrow$'': For any $\bar{r} \in R/I$ with $\bar{r} \neq 0$, then $r \not\in I$.
        Consider $\gen{ r } + I$ which contains $I$ and is not equal to $I$ because $r \not\in I$.
        Since $I$ is maximal, $\gen{ r } + I = R$, and thus $\exists x \in R, y \in I$ such that
        $xr + y = 1$, so $\bar{x} \bar{r} = \bar{1}$. Hence every non-zero element has multiply inverse
        and $R / I$ is a field.

      ``$\Leftarrow$'': If $J$ is an ideal such that $I \subsetneq J$, pick $x \in J \setminus I$,
      then $\bar{x} \neq 0$, so $\exists r \in J$ such that $\bar{x} \bar{r} = 1$. Then
      $xr + I = 1 + I \implies \exists y \in I \text{ s.t. } xr + y = 1$. So $1 \in J$, and
      because $J$ is an ideal, $J = R$.

      \item By the fact that $(ab \in I \implies a \in I \text{ or } b \in I) \iff
        (\bar{a}\bar{b} = 0 \implies \bar{a} = 0 \text{ or } \bar{b} = 0)$ the proof is complete.
        \qedhere
    \end{enumerate}
  \end{proof}
\end{prop}

\begin{prop} \label{prop:irr-to-max-ideal}
  If $f(x) \in K[x]$ is irreducible, where $K$ is a field, then $\gen{ f(x) }$ is maximal ideal.
  \begin{proof}
    We know that $K[x]$ is a principle ideal domain, so if $\gen{ f(x) } \subseteq J$, then
    $J$ is generated by a element, say $g(x)$. Since $f(x) \in J$, we could write $f(x) = g(x) h(x)$.
    By the fact that $f(x)$ is irreducible, either $g(x)$ is an unit then $J = R$, or $h(x)$ is
    an unit then $J = \gen{ f(x) }$.
  \end{proof}
\end{prop}

\begin{example}
  $f(x) = x^2 + 1$ has roots $\alpha = \pm \sqrt{-1}$, so $\Rb(\sqrt{-1}) \cong \Rb[x] / \gen{ x^2 + 1 }$.
\end{example}

\begin{theorem} \label{thm:field-ext-1}
  Let $f(x) \in K[x]$ be monic, irreducible and of degree $n$. Then there exists
  $L / K$ and $\alpha \in L$ s.t. $f(\alpha) = 0, L = K(\alpha)$ and $[L: K] = n$.
  \begin{proof}
    Since $f(x)$ is irreducible, by prop. \ref{prop:irr-to-max-ideal}
    $\gen{ f(x) }$ is a maximal ideal. Then by prop.
    \ref{prop:max-prime-to-field-int-domain} $L = K[x] / \gen{ f(x) }$ is a field,
    and $K$ is a subfield of $L$ by the inclusion map $\alpha \mapsto \bar\alpha$.
    The map is 1-1 since $\bar{1} \neq 0$ and a field homomorphism is either a
    1-1 map or a zero(全洪)map.

    Notice that $L \cong K[\bar{x}]$, where $\bar{x}$ is the coset $x + \gen{ f(x) }$.
    Now let $\alpha = \bar{x}$, and it is easy to see that $f(\alpha) = f(x) + \gen{ f(x) } = 0$.
    Also $L \cong K[\bar{x}] \cong K(\alpha)$. Finally, $m_{\alpha, K} \mid f$ and by the fact that
    $f$ is monic and irreducible, $m_{\alpha, K} = f$ and thus $[L: K] = \deg m_{\alpha, K} = \deg f = n$.
  \end{proof}
\end{theorem}

\begin{theorem}
  Let $f(x) \in K[x]$ be of degree $n > 0$. Then there exists $L/K$ s.t. $f$
  splits over $L$, that is,
  \[ f(x) = \lambda (x - \alpha_1) (x - \alpha_2) \dotsm (x - \alpha_n) \text{ with }
    \alpha_1, \alpha_2, \dots, \alpha_n \in L,\, \lambda \in K \]
  In fact, $L$ can be chosen to be the smallest field over which $f$ splits and in this case $[L : K] \leq n!$.\\
  $L$ is called a \emph{splitting field} \index{splitting field} for $f$ over $K$.
\end{theorem}

\begin{proof}
  By induction on $n$, $n = 1$ is trivial, simply pick $L = K$.

  For $n > 1$, let $p(x)$ be an monic irreducible factor of $f(x)$.
  By theorem \ref{thm:field-ext-1}, there exists an extension $K(\alpha_1)$ s.t. $p(\alpha_1) = 0$.
  By division algorithm, $f(x) = (x - \alpha_1) f_1(x)$ where $f_1(x) \in K(\alpha_1)[x]$
  and $\deg f_1 = n - 1$. Using the induction hypothesis, we know that there exists $L$,
  which is an extension of $K(\alpha_1)$, s.t. $f_1$
  splits over $L$. Hence $\exists \alpha_2, \alpha_3, \dots, \alpha_n \in L$ s.t.
  $ f_1(x) = \lambda (x - \alpha_2) \dots (x - \alpha_n)$,
  thus $f(x) = \lambda (x - \alpha_1) (x - \alpha_2) \dots (x - \alpha_n)$. Compare the
  coefficient of $x^n$ we know that $\lambda \in K$.

  More over, observe that $K(\alpha_1, \dots, \alpha_n)$ is the smallest field containing $K$ and
  $\{ \alpha_1, \dots, \alpha_n\}$. So if we choose $L = K(\alpha_1, \alpha_2, \dots, \alpha_n)$,
  then
  \[ [L: K] = [K(\alpha_1, \alpha_2, \dots, \alpha_n): K(\alpha_1, \alpha_2, \dots, \alpha_{n-1})]
    \dotsm
    [K(\alpha_1): K] \leq n! \]
  Since $[K(\alpha_1, \alpha_2, \dots, \alpha_k): K(\alpha_1, \alpha_2, \dots, \alpha_{k-1})]
  = [K(\alpha_1, \alpha_2, \dots \alpha_{k-1})(\alpha_k): K(\alpha_1, \alpha_2, \dots, \alpha_{k-1})]$
  and $\alpha_k$ is a root of $p(x) \in K(\alpha_1, \alpha_2, \dots, \alpha_{k-1})[x]$
  where $f(x) = (x - \alpha_1)(x - \alpha_2) \dotsm (x - \alpha_{k-1}) p(x)$.
\end{proof}

\begin{example}
  Find a splitting field $L$ for $x^8 - 2$ over $\Qb$ and determine $[L : \Qb]$.
\end{example}

\begin{remark}
  $\quot{\Qb[x]}{\gen{x^8 - 2}} = \Qb(\bar{x}) \cong \Qb(\sqrt[8]{2})
  \cong \Qb(\sqrt[8]{2} \zeta)$
\end{remark}

\begin{prop} \label{prop:after-homo-still-irr}
  Let $K, L$ be two fields and $\tau: K \to L$ be a nontrivial homomorphism.
  We define $\bar\tau : K[x] \to \tau(K)[x] \subseteq L[x]$ by
  \[ a_n x^n + \dots + a_0 \mapsto \bar\tau(f) \triangleq \tau(a_n)x^n + \dots + \tau(a_0) \]
  which is an isomorphism. Also, $f$ is irreducible implies $\bar\tau(f)$ is irreducible in $\tau(K)[x]$.
\end{prop}

\begin{lemma} \label{lemma:extension-exists-condition}
  Let $K(\alpha) / K$ be algebraic and $\tau: K \to L$ be a nontrivial homo,
  then there exists an extension $\sigma$ of $\tau$ from $K(\alpha)$ to $L$ if
  and only if $\exists \beta \in L$ s.t. $\bar\tau(m_{\alpha, K})(\beta) = 0$.

  In this case $m_{\beta, \tau(K)} = \bar\tau(m_{\alpha, K})$.

\begin{proof}
  ``$\Rightarrow$'': Let $\beta = \sigma(\alpha)$ and $m_{\alpha, K} = x^n + a_{n-1} x^{n-1} + \dots + a_0$.
  Then $\bar\tau(m_{\alpha, K})(\beta) = \beta^n + \tau(a_{n-1})\beta^{n-1} + \dots + \tau(a_0)
  = \tau(\alpha^n + a_{n-1} \alpha^{n-1} + \dots + a_0) = 0$

  ``$\Leftarrow$'': Observe that $m_{\beta, \tau(K)} = \bar\tau(m_{\alpha, K})$ since
  $\bar\tau(m_{\alpha, K})(\beta) = 0$ and $\bar\tau(m_{\alpha, K})$ is monic and irreducible
  by prop \ref{prop:after-homo-still-irr}. $\sigma$ is then given by the following diagram.
  \[
    \begin{tikzcd}
      & K[x] \arrow[r, "\sim", "\bar\tau"'] \arrow[d, two heads]
      & \tau(K)[x] \arrow[d, two heads] \\
      K(\alpha) \arrow[r, Leftrightarrow, "\cong"]
      & \raisebox{.1em}{$K[x]$} \Big/ \raisebox{-.1em}{$\gen{ m_{\alpha, K} }$}
      \arrow[r, "\sim", "\sigma"']
      & \raisebox{.1em}{$\tau(K)[x]$} \Big/ \raisebox{-.1em}{$\gen{ m_{\beta, \tau(K)} }$}
      \arrow[r, Leftrightarrow, "\cong"]
      & \tau(K)(\beta) \subseteq L
    \end{tikzcd}
  \]
\end{proof}
\end{lemma}

\begin{coro} \label{coro:num-of-extensions}
  Let $K(\alpha)/K$ be an algebraic extension and $\tau: K \hookrightarrow L$.
  If $\bar\tau(m_{\alpha, K})$ has $r$ distinct roots in $L$, then there are exactly $r$ extensions of $\tau$.
\end{coro}

\begin{theorem} \label{thm:two-splitting-field-are-isom}
  Let $\tau: K \to K'$ be an isomorphism of fields.
  If $L$ is a splitting field for $f$ over $K$ and $L'$ is a splitting field for $\bar\tau(f)$
  over $K'$, then $L \cong L'$

\begin{proof} \label{coro:extension-exists-splitting-field}
  By induction on $n = \deg f$. When $n = 1$, $L = K, L' = K'$, so $L \cong L'$.

  Now if $n > 1$, assume $f(\alpha) = 0$ for $\alpha \in L$. Then
  $\bar\tau(m_{\alpha, K}) \mid \bar\tau(f)$ and by the fact that $L'$ is a
  splitting field for $\bar\tau(f)$, $\exists \beta \in L'$ s.t. $\bar\tau(m_{\alpha, K})(\beta) = 0$.
  By lemma \ref{lemma:extension-exists-condition}, $\exists \tau_{\circ}:
  K(\alpha) \xrightarrow\sim K'(\beta)$ with $\tau_\circ \big|_K = \tau$.

  Now, write $f = (x - \alpha) f_\circ$, then $\bar\tau(f) = \bar\tau_\circ(f) = (x - \tau_\circ(\alpha))
  \bar\tau_\circ(f_\circ) = (x - \beta) \bar\tau_\circ(f_\circ)$. Then $L$ and $L'$ is a splitting
  field for $f_{\circ}$ over $K(\alpha)$ and $\bar\tau_\circ(f_\circ)$ over $K(\beta)$ respectively.
  By induction hypothesis, $L \cong L'$.
\end{proof}
\end{theorem}

\begin{coro}
  Let $\tau: K \to K'$ be an isomorphism of fields, and
  $L$ is a splitting field of $f$ over $K$, $L'$ is a splitting field of $\bar\tau(f)$ over $K'$.
  Then $\tau$ could be extend to $\sigma: L \to L'$ such that $\sigma\big|_K = \tau$.
\end{coro}

\begin{example}
  $L = \Qb(\sqrt{2}, \sqrt{3})$.
\end{example}

%! TEX root=../main.tex
\subsection{Week 2}

\begin{definition}
  A polynomial $f(x) \in K[x]$ is said to be \emph{separable}\index{seperable}
  if its irreducible factors have no multiple roots in a splitting field $L$.
\end{definition}

\begin{definition}
  If $f(x) = a_n x^n + \dots + a_1 x + a_0$, then define $f'(x) \triangleq n a_n x^{n-1} + \dots + 2a_2 x + a_1$.
\end{definition}

\begin{theorem} \label{thm:multiple-root-condition}
  Let $f(x) \in K[x]$ be monic, irreducible of positive degree, then all the roots of $f(x)$
  in a splitting field are simple if and only if $\gcd(f(x), f'(x)) = 1$.

  \begin{proof}
    ``$\Rightarrow$'': We can write $f(x) = (x - \alpha_1) (x - \alpha_2) \dotsm (x - \alpha_n)$ where
    $\alpha_i$ are distinct roots of $f$. Then $f'(x) = \sum_{i = 1}^n f(x) / (x - \alpha_i)$
    and we have $(x - \alpha_i) \nmid f(x)$ for all $i$.

    ``$\Leftarrow$'': Assume $f(x) = (x - \alpha)^k g(x)$ with $k \geq 2$.
    Then $f'(x) = k (x - \alpha)^{k-1} g(x) + (x - \alpha)^k g'(x)$ which implies $(x - \alpha) \mid f(x)$.
    So $(x - \alpha) \mid \gcd(f(x), f'(x))$ and thus $\gcd(f(x), f'(x)) \neq 1$.
  \end{proof}
\end{theorem}

\begin{remark}
  The following are equivalent:
  \begin{enumerate}
    \item $\alpha$ is a multiple root of $f(x)$.
    \item $\alpha$ is a common root of $f(x)$ and $f'(x)$.
    \item $m_{\alpha, K} \mid f(x)$ and $m_{\alpha, K} \mid f'(x)$.
  \end{enumerate}
\end{remark}

\begin{theorem} \label{thm:size-of-finite-field}
  There is a finite field $K$ with $\abs{K} = q \iff q = p^n$ for some prime $p$ and $n \in \Nb$.
  In this situation, $K$ is unique up to isomorphism, denote by $\Fb_{p^n}$.

  \begin{proof}
    ``$\Rightarrow$": Let $p = \Char K$ and $[K : \Zb/ p\Zb] = n$, then $\abs{K} = p^n$.

    ``$\Leftarrow$": Let $K$ be a splitting field for $f(x) = x^{p^n} - x$ over $\Fb_p$.
    We claim that the set of all roots of $f(x)$ forms a field. Since if $\alpha, \beta$ are
    two roots of $f$, obviously $\alpha \beta, \alpha \beta^{-1}$ are also roots,
    and by $(\alpha \pm \beta)^{p^n} = \alpha^{p^n} \pm \beta^{p^n} = \alpha \pm \beta$ because
    $\Char K = p$.
    $\alpha \pm \beta$ are also roots, hence the roots form a field. By definition, $K$
    is the smallest field containing $\Fb_p$ and roots of $f(x)$, so
    $K$ is exactly the set of roots of $f(x)$.

    Also, $f'(x) = -1$ has no root, so $f(x)$ has no multiple root which implies $\abs{K} = p^n$.

    Moreover, if $K'$ is another finite field with $\abs{K'} = p^n$, then for all $\alpha \in K'$,
    $\alpha^{p^n} = \alpha$, so $\alpha$ is a root of $f(x)$, which implies that $K'$ is
    a splitting field for $f(x)$ over $\Fb_p$. By theorem~\ref{thm:two-splitting-field-are-isom},
    $K \cong K'$.
  \end{proof}
\end{theorem}

\begin{theorem} \label{thm:aut-of-finite-field}
  Let $n \in \Nb$ and $\Fb_q$ be a finite field. Then there exists
  a unique extension $\Fb_{q^n} / \Fb_q$ s.t. $[\Fb_{q^n} : \Fb_q] = n$, and
  $\Aut \Big( \Fb_{q^n} / \Fb_q \Big) = \gen{\sigma_q}$ with
  $\sigma_q = \alpha :: \Fb_{q^n} \mapsto \alpha^q :: \Fb_{q^n}$.
  $\sigma_q$ is called the \emph{\Index{Frobenius homomorphism}}.

  \begin{proof}
    By theorem~\ref{thm:size-of-finite-field},
    $q = p^r$ for some prime $p$ and $r \in \Nb$, so $q^n = p^{nr}$ which is a power of
    a prime. Again by theorem~\ref{thm:size-of-finite-field},
    $\Fb_{q^n}$ is the splitting field for $x^{p^{nr}} - x$ over $\Fb_p$.
    Since $x^q - x \mid x^{q^n} - x$, $\Fb_q \subseteq \Fb_{q^n}$ and thus $[\Fb_{q^n}: \Fb_{q}] = n$.

    Then we proof that $\sigma_q$ is indeed in $\Aut \Big( \quot{\Fb_{q^n}}{\Fb_q} \Big)$.
    We check that
    \[
      \begin{aligned}
        \sigma_q(\alpha+\beta) &= (\alpha+\beta)^q = \alpha^q + \beta^q = \sigma_q(\alpha) + \sigma_q(\beta) \\
        \sigma_q(\alpha\beta) &= (\alpha\beta)^q = \alpha^q \beta^q = \sigma_q(\alpha) \sigma_q(\beta)
      \end{aligned}
    \]
    Now $\sigma_q$ is nontrivial since $\sigma_q$ send $1$ to $1$, so $\sigma_q$ is 1-1 and hence an
    isomorphism since $\Fb_q$ is finite. Also, for all $\alpha \in \Fb_q$, $\sigma_q(\alpha)
    = \alpha^{q} = \alpha$, hence $\sigma_q$ fixes $\Fb_q$.

    Finally we prove that the order of $\sigma_q$ is $n$. Assume not, so $\ord(\sigma_q) = m < n$.
    Then $\sigma_q^m = \Id \implies x^{q^m} - x=0$ for each $x \in \Fb_{q^n}$.
    But $x^{q^m} - x = 0$ has at most $q^m < q^n$ roots, which leads to a contradiction.
  \end{proof}
\end{theorem}

\begin{remark}
  By theorem~\ref{thm:finite-subgroup-of-field-is-cyclic}, the multiplication group
  of $\Fb_{q^n}$ is cyclic, so $\Fb_{q^n}^\times = \gen{ \alpha }
  \subseteq \Fb_q(\alpha) \setminus \{0\} \subseteq \Fb_{q^n} \setminus \{0\}$,
  hence $\Fb_{q^n} = \Fb_q(\alpha)$.
\end{remark}

\begin{lemma}
  Every irreducible polynomial $f(x)$ in $\Fb_{p^n}[x]$ is separable.

  \begin{proof}
    Without lost of generality, assume $f(x)$ is monic.

    Since $\sigma_p$ is an isomorphism, $\Fb_{p^n} = \Fb_{p^n}^p = \{ \alpha^p \mid \alpha \in \Fb_{p^n}\}$.
    Now assume $f(x)$ has a multiple root $\alpha$, then $m_{\alpha, \Fb_p} = f(x)$ since $f$
    is irreducible. By theorem~\ref{thm:multiple-root-condition} we also have
    $f(x) = m_{\alpha, \Fb_p} \mid f'(x)$, but $\deg f'(x) < \deg f(x)$ so we must have $f'(x) \equiv 0$.

    Write $f(x) = a_n x^n + \ldots + a_1 x + a_0$, then $f'(x) \equiv 0$ implies $k a_k = 0_{\Fb_p}$ for each $k$,
    which means that if $a_k \neq 0 \implies p \mid k$. So
    \[ f(x) = a_{mp} x^{mp} + a_{(m-1)p} x^{(m-1)p} + \dots + a_p x^p + a_0 =
    (a_{mp} x^m + \ldots + a_p x + a_0)^p. \]
    But this implies $f(x)$ is reducible, which is a contradiction.
  \end{proof}
\end{lemma}

\begin{theorem}
  $x^{p^n} - x$ equals the product of all monic irreducible polynomials in
  $\Fb_p[x]$ of degree $d$ where $d$ runs through all divisors of $n$. i.e.

  \begin{proof}
    By lemma, each irreducible polynomial is separable, and if $f(x), g(x) \in \text{RHS}$,
    and $f(\alpha) = g(\alpha) = 0$, then $f = m_{\alpha, \Fb_p} = g$. Thus RHS is separable.
    LHS is separable since $f' = 1$, so we could prove the equality by checking that
    they have same roots.

    $\text{LHS} \mid \text{RHS}$: $\forall \alpha \in \Fb_{p^n}$,
    $[\Fb_p(\alpha): \Fb_p] \mid [\Fb_{p^n}: \Fb_p] = n$, thus $\deg m_{\alpha, \Fb_p} \mid n$
    and hence $m_{\alpha, \Fb_p}$ appears in RHS.

    $\text{RHS} \mid \text{LHS}$: Assume $\deg m_{\alpha, \Fb_p} = d \mid n$, then
    $[\Fb_p(\alpha): \Fb_p] = d$, so $\alpha^{p^d} = \alpha$, and hence
    $\alpha = \alpha^{p^d} = \alpha^{p^{2d}} = \dots = \alpha^{p^n}$.
  \end{proof}
\end{theorem}

\begin{definition}
  M\"{o}bius $\mu$-function:
  Let $d = p_1^{k_1} p_2^{k_2} \dotsm p_n^{k_n}$, then
  \[
    \mu(d) = \begin{cases}
      1 & \text{if $n$ is even and all $k_i = 1$} \\
      -1 & \text{if $n$ is odd and all $k_i = 1$} \\
      0 & \text{otherwise}
    \end{cases}
  \]
\end{definition}

\begin{theorem}[M\"{o}bius inversion formula]
  If $f(n) = \sum\limits_{d \mid n} g(d)$, then
  $g(n) = \sum\limits_{d \mid n} \mu(d) f\left(\frac{n}{d}\right)$.
\end{theorem}

%! TEX root=../main.tex
\subsection{Week 3}
\subsubsection{Coset and Quotient Group}
Let $f: G_1 \to G_2$ be a group homo. Define $\Image f \defeq f(G_1)$.

Notice that $\Image f \le G_2$.
\begin{proof}
  Let $z_1 = f(a_1), z_2 = f(a_2)$, then
  $z_1z_2^{-1} = f(a_1)f(a_2)^{-1} = f(a_1)f(a_2^{-1}) = f(a_1a_2^{-1}) \in
  \Image f$.
\end{proof}

\begin{definition}
  $\ker f \defeq \{\, x \in G_1 \mid f(x) = 1 \,\} \le G_1$.
\end{definition}

\begin{fact} \mbox{}
  \begin{enumerate}
    \item $x \in (\ker f) a \iff f(x) = f(a)$.
    \item $\ker f = \{ 1 \} \iff f$ is 1-1.
  \end{enumerate}
\end{fact}

\begin{definition}
  Let $H \le G$, $\forall a \in G, Ha$ is called a {\bf right coset} of $H$
  in $G$.
\end{definition}

\begin{fact} \mbox{}
  \begin{enumerate}
    \item For 2 right cosets $Ha, Hb$, either $Ha = Hb$ or $Ha \cap Hb = \phi$
      must hold.
    \item $\{\, Ha : a\in G \,\}$ forms a partition of $G$.
  \end{enumerate}
\end{fact}

\begin{theorem}[Lagrange]
  Let $\abs{G} < \infty$ and $H \le G$, $\abs{H} \mid \abs{G}$.
  \begin{proof}
  \end{proof}
\end{theorem}

\begin{remark}
  $r$ is called the {\bf index} of $H$ in $G$, denoted by $[G:H]$.
  (The concept of index can be extended to infinite $G, H$.)
\end{remark}

\begin{exercise}
  no subgroup of $A_4$ has order $6$.
  (converse of Lagrange thm. is false.)
\end{exercise}

\begin{coro}
  If $\abs{G} = p$ is a prime in $\Zb$, then $G$ is cyclic.
  \begin{proof}
  \end{proof}
\end{coro}

\begin{coro}
  If $\abs{G} < \infty, a \in G$, then $a^{\abs{G}} = 1$.
  \begin{proof}
  \end{proof}
\end{coro}

\begin{remark} \mbox{}
  \begin{enumerate}
    \item Let $H \le G, a \in G$, $aH$ is called a {\bf left coset}.
    \item $\{ \text{right cosets of $H$} \} \leftrightarrow
      \{ \text{right cosets of $H$} \}$ by $Ha \mapsto a^{-1}H$.
  \end{enumerate}
\end{remark}

\underline{Ques}: How to make $\{\, aH : a \in G \,\}$ to be a group?
For $aH, bH$, we must have $(aH)(bH) = abH$.

In general, $(aH)(bH) = abH$ is not well-defined.

\begin{example}
  Let $H = \gen{\cycle{1,2}} \le S_3$. $a_1 = \cycle{1,3}, a_2 = \cycle{1,2,3},
b_1 = \cycle{1,3,2}, b_2 = \cycle{2,3}$. 出慘點
\end{example}

If we hope $a_1b_1H = a_2b_2H$, then we need $(a_1b_1)^{-1}a_2b_2 \in H$.
\[
  b_1^{-1}a_1^{-1}a_2b_2 = b_1^{-1}b_2b_2^{-1}a_1^{-1}a_2b_2
\]
Notice that $b_1^{-1}b_2, a_1^{-1}a_2 \in H$, so we need
$b_2^{-1}a_1^{-1}a_2b_2 \in H$.

\begin{definition}
  Let $H \le G$. $H$ is said to be {\bf normal subgroup} of $G$ if
  $\forall g \in G, h \in H, g^{-1}hg \in H \quad
  (\text{or~} g^{-1}Hg \subseteq H)$, denoted by $H \lhd G$.
\end{definition}

\begin{definition}
  Let $H \lhd G$. The set $\{\, aH \mid a \in G \,\}$ forms a group under
  $(aH)(bH) = abH, a,b \in G$. We call it the {\bf quotient group}
  of $G$ by $H$, denoted by $\quot{G}{H}$.

  (Note: The indentity is $H = hH$ and $(aH)^{-1} = a^{-1}H$.)
\end{definition}

\begin{remark}
  Define $q: G \to \quot{G}{H}, a \mapsto aH$, called the quotient homomorphism.
\end{remark}

\begin{exercise}
  Let $H \le G$. Then TFAE
  \begin{enumerate}[(a)]
    \item $H \lhd G$.
    \item $\forall x \in G, xHx^{-1} = H$.
    \item $\forall x \in G, xH = Hx$. \label{eq:xh=hx}
    \item $\forall x, y \in G, (xH)(yH) = (xy)H$.
  \end{enumerate}
\end{exercise}

\underline{Ques}: How to find a normal subgroup of $G$?

\begin{prop} \mbox{}
  \begin{enumerate}
    \item If $G$ is abelian, then $\forall H \le G \leadsto H \lhd G$.
      (done by \ref{eq:xh=hx})
    \item If $H \le G$ with $[G:H] = 2$, then $H \lhd G$.
      \begin{example}
        $n \le 3, [S_n:A_n] = 2 \implies A_n \lhd S_n$.
      \end{example}
      \begin{proof}
        We can write $G = H \cup Ha = H \cup aH \implies aH = Ha,
        \forall a \not\in H$.
      \end{proof}
  \end{enumerate}
\end{prop}

\begin{definition}
  Define the center of $G$ to be $Z_G = \{\, a \in G
  \mid ax = xa, \forall x \in G \,\} \le G$.
\end{definition}

\begin{prop} \mbox{}
  \begin{enumerate}
    \item $Z_G \lhd G$. (by \ref{eq:xh=hx} and def.)
    \item If $\quot{G}{Z_G}$ is cyclic, then $G$ is abelian.
      \begin{proof}
        Let $\quot{G}{Z_G} = \gen{aZ_G}$, (let $\ob{a} \defeq aZ_G$) for some
        $a \in G$.
        For $x_1, x_2 \in G$, let $x_1 = a^{k_1}z_1, x_2 = a^{k_2}z_2$, then
        $x_1x_2 = a^{k_1+k_2}z_1z_2 = x_2x_1$. ($z_i$ 可以各種交換)
      \end{proof}
  \end{enumerate}
\end{prop}

\begin{definition}
  The commutator of $G$ is define to be $[G,G] = \gen{xyx^{-1}y^{-1} \mid
  x,y \in G}$.
\end{definition}

\begin{prop}
  $[G,G] \lhd G$ ; $[G,G] = 1 \iff G$ is abelian.
  \begin{proof}
    $\forall x \in G, a \in [G,G], xax^{-1} = xax^{-1}a^{-1}a$ and
    $xax^{-1}a^{-1}, a \in [G,G]$.
  \end{proof}
\end{prop}

\begin{exercise} \mbox{}
  \begin{enumerate}
    \item If $H \le S_n$ and $\exists \sigma \in H$ is odd, then $[H:H\cap A_n] = 2$.
    \item For $n \ge 3$, $[S_n, S_n] = A_n$.
  \end{enumerate}
\end{exercise}

\begin{exercise}
  Let $H \le G$. Then  $H \lhd G$ and $\quot{G}{H}$ is abelian $\iff$
  $[G,G] \le H$.
  (hint: $\quot{G}{[G,G]}$ is "max" among all abelian quotient groups)
\end{exercise}


\subsubsection{Isomorphism theorems \& Factor theorem}
\begin{theorem}[1st isomorphism theorem]
  Let $f: G_1 \to G_2$ be a group homo. Then $\quot{G_1}{\ker f} \cong \Image f$.
  \begin{proof}
    Define $\varphi: a \ker f \mapsto f(a)$.
    \begin{itemize}
      \item well-defined: $a \ker f = b \ker f \implies a^{-1}b \in \ker f
        \implies f(a^{-1}b) = 1 \implies f(a)^{-1}f(b) = 1 \implies f(a)=f(b)$.
      \item group homo: $\varphi\left((a \ker f)(b\ker f)\right) = 
        \varphi(ab \ker f) = f(ab) = f(a)f(b) =
        \varphi(a\ker f)\varphi(b\ker f)$.
      \item onto: by def. of $\Image f$.
      \item 1-1: $f(a) = f(b) \implies a \ker f = b \ker f$ (easy).
      \end{itemize}
  \end{proof}
\end{theorem}

\begin{theorem}[Factor theorem]
  Let $f: G_1 \to G_2$ be a group homo. and $H \lhd G_1, H \le \ker f$. Then
  $\exists$ a group homo. $\varphi: \quot{G}{H} \to G_2$ s.t.
  \[
    \begin{tikzcd}
      G_1 \arrow{r}{q} \arrow[swap]{dr}{f} & \quot{G}{H} \arrow{d}{\varphi} \\
      & G_2
    \end{tikzcd}
  \]
\end{theorem}

\begin{example}
  Let $G = \gen{a}$ with $\ord(a) = n$. Then $G \cong \quot{\Zb}{n\Zb}$.
  (1st isom. thm.)
\end{example}

\begin{example}
  $\varphi: \Zb \to \quot{\Zb}{2\Zb}, 4\Zb \le 2\Zb$, so by factor thm.,
  $\quot{\Zb}{4\Zb} \to \quot{\Zb}{2\Zb}$.
\end{example}

\begin{example}
  $\det: \text{GL}(n, \Fb) \to \Fbx \implies
  \quot{\text{GL}(n, \Fb)}{\text{SL}(n, \Fb)} \cong \Fbx$
\end{example}

\begin{example}
  $\sgn: S_n \to \{ \pm 1 \} \implies \quot{S_n}{A_n} \cong \{ \pm 1 \}$
\end{example}

\begin{theorem}[2nd isomorphism theorem]
  Let $H \le G, K \lhd G$. Then $\quot{HK}{K} \cong \quot{H}{H\cap K}$.
  \begin{proof}
    First, $\begin{cases}H\le G \\ K \lhd G\end{cases} \implies HK = KH
      \implies HK \le G$ ; $K \lhd G \implies K \lhd HK$.

    Define $\varphi: H \to \quot{HK}{K}, h \mapsto hK$. which is a group homo.
    \begin{itemize}
      \item onto: $\forall (hk) K, hkK = hK$, so $\varphi(h) = hK = hkK$.
      \item Find $\ker \varphi$: $a \in \ker \varphi \iff \begin{cases}
          a \in H \\
          aK = K
        \end{cases} \iff a \in H \cap K$, so $\ker \varphi = H\cap K$.
    \end{itemize}
    Then by 1st isom. thm.
  \end{proof}
\end{theorem}

\begin{example}
  $G = \text{GL}(2, \Cb), H = \text{SL}(2, \Cb), K = \Cbx I_2 = Z_G \lhd G$.

  By 2nd isom. thm., $\quot{G}{K} \cong \quot{H}{\{\pm I_2\}}$.
  ($G = HK, \{\pm I_2 \} = H \cap K$)

  projective linear group: $\text{PGL}(2, \Cb) = \quot{G}{K}$.

  projective special linear group: $\text{PSL}(2, \Cb) = \quot{H}{H\cap K}$.
\end{example}

齊次座標...OTL

\begin{exercise} \mbox{}
  \begin{enumerate}
    \item Let $H_1 \lhd G_1, H_2 \lhd G_2$. Then $(H_1 \times H_2) \lhd
      (G_1 \times G_2)$ and $\quot{G_1\times G_2}{H_1\times H_2} \cong
      \quot{G_1}{H_1} \times \quot{G_2}{H_2}$.
    \item Let $H \lhd G, K \lhd G$ s.t. $G = HK$. Then
      $\quot{G}{H\cap K} \cong \quot{G}{H} \times \quot{G}{K}$.
  \end{enumerate}
\end{exercise}

\begin{exercise}
  Let $H \lhd G$ with $[G:H] = p$, which is a prime in $\Zb$. Then
  $\forall K \le G$, either \begin{enumerate*}[(1)]
    \item $K \le H$ or
    \item $G = HK$ and $[K:K\cap H] = p$.
  \end{enumerate*}
\end{exercise}

\begin{theorem}[3rd isomorphism theorem]
  Let $K \lhd G$.
  \begin{enumerate}
    \item There is a 1-1 correspondence between $\{\, H \le G \mid K \le H \,\}$
      and $\{ \text{subgroups of $\quot{G}{K}$} \}$. ($H \lhd G$ ... normal)
      \begin{proof}
        Define $\varphi: H \mapsto \quot{H}{K}$. ($\quot{H}{K} \le \quot{G}{K}$)
        \begin{itemize}
          \item 1-1: Assume $\quot{H_1}{K} = \quot{H_2}{K}$.
            For $a \in H_1$, $aK \in \quot{H_1}{K} = \quot{H_2}{K}$.
            so $\exists b \in H_2$ s.t. $aK = bK \implies b^{-1}a \in K \le H_2
            \implies a \in b H_2 = H_2$. So $H_1 \le H_2$. By symmetry,
            $H_2 \le H_1$, and thus $H_1 = H_2$.
          \item onto: Given a subgroup $Q$ of $\quot{G}{K}$, consider
            $H = q^{-1}(Q)$ where $q: G\to \quot{G}{K}$.
            \begin{itemize}
              \item $H \le G$: $\forall a, b \in H, q(a), q(b) \in Q \implies
                q(a)q(b)^{-1} \in Q \implies q(ab^{-1}) \in Q \implies
                ab^{-1} \in H \implies H \le G$.
              \item $K \le H$: $\forall a \in K, q(a) = aK = K \in Q \implies
                a \in H \implies K \le H$.
              \item $Q = \quot{H}{K}$: $\forall aK \in Q, aK = q(a) \implies
                a \in H \implies aK \in \quot{H}{K} \implies
                Q \subseteq \quot{H}{K}$.
                And $\forall aK \in \quot{H}{K} (a \in H), q(a) \in Q \implies
                  \quot{H}{K} \subseteq Q$. So $Q = \quot{H}{K}$.
            \end{itemize}
          \item $H \lhd G, K \le H \iff \forall g\in G, gHg^{-1} = H, K \le H
            \iff \forall \ob{g} \in \quot{G}{K}, \ob{g}(\quot{H}{K})\ob{g}^{-1}
            = \quot{H}{K} \iff \quot{H}{K} \lhd \quot{G}{K}$. \qedhere
        \end{itemize}
      \end{proof}
    \item If $H \lhd G$ with $K \le H$, then $\quot{(\quot{G}{K})}{(\quot{H}{K})}
      \cong \quot{G}{H}$.
      \begin{proof}
        Define $\varphi: G \to\quot{(\quot{G}{K})}{(\quot{H}{K})}$ with
        $\varphi: a \mapsto aK(\quot{H}{K})$.
        \begin{itemize}
          \item onto: ... easy.
          \item Find $\ker \varphi$: $a \in \ker \varphi \iff aK(\quot{H}{K})
            = \quot{H}{K} \iff aK \in \quot{H}{K} \iff a \in H$.
        \end{itemize}
        By 1st isom. thm., $\quot{(\quot{G}{K})}{(\quot{H}{K})} \cong
        \quot{G}{H}$.
      \end{proof}
  \end{enumerate}
\end{theorem}

\begin{example}
  $\quot{m\Zb + n\Zb}{m\Zb} \cong \quot{n\Zb}{m\Zb \cap n\Zb}$.
  ($m\Zb + n\Zb = \gcd(m,n)\Zb, m\Zb \cap n\Zb = \lcm(m,n)\Zb$)
\end{example}

\underline{Ques}: $\quot{G}{K} \cong \quot{G'}{K'}$ and $K \cong K'
\centernot\implies G \cong G'$.

\begin{example}
  $Q_8$ and $D_4$
  交給陳力
\end{example}

Extension problem: given two groups $A, B$, how to find $G$ and $K \lhd G$,
s.t. $K \cong A, \quot{G}{K} \cong B$?
($1 \to H \to G \to \quot{G}{H} \to 1$, short exact sequence)

 (e.g. $G = A \times B, K = A \times \{1\}$)

%! TEX root=../main.tex
\subsection{Week 4}
\subsubsection{Universal property and direct sum \& product}
In general, let $f_1: G_1 \to G, f_2: G_2 \to G$ are group homo.
$f_1 \times f_2: G_1 \times G_2 \to G, (a, b) \mapsto f_1(a)f_2(b)$.
But we have $(a, b) = (a, 1)(1, b) = (1, b)(a, 1)$, so
$f_1(a)f_2(b) = f_2(b)f_1(a) \implies $ need $G$ to be abelian.

So we intend to define the direct sum in the category of abelian group.

\underline{Notation}: For abelian groups, we use ``$+$'' to denote the group
operation and ``$0$'' to denote the identity.

\begin{definition}
  Given a non-empty family of abelian groups $\{\, G_s \mid s \in \Lambda \,\}$,
  a (external) direct sum of $\{\, G_s \mid s \in \Lambda \,\}$ is an
  abelian group $\bigoplus_{s\in \Lambda} G_s$ with the embedding mappings
  $i_{s_0}: G_{s_0} \to \bigoplus_{s\in \Lambda} G_s,
  \forall s_0 \in \Lambda$ satisfying the universal property:

  for any abelian group $H$ and group homo. $\varphi_s: G_s \to H
  \forall s \in \Lambda, \quad \exists!$ group homo. $\varphi:
  \bigoplus_{s\in \Lambda} G_s \to H$ s.t. 又一個ㄛ圖
\end{definition}

\begin{theorem}
  $\bigoplus_{s\in \Lambda} G_s$ exists and is unique up to isomorphisms.

  \begin{proof}
    Existence: $\bigoplus_{s\in \Lambda} G_s = \{\, (g_s)_{s\in \Lambda}
      \mid g_s \in G_s, \text{~almost all of the $g_s$' are $0$} \,\}$ and
      \[ i_{s_0}: G_{s_0} \to \bigoplus_{s\in \Lambda} G_s,
        a_{s_0} \mapsto (g_{s_0})_{s\in \Lambda} \text{~with~}
        g_{s_0} = a_{s_0}, g_s = 0, \forall s \ne s_0. \]
        group operaion: $(g_s)_{s \in \Lambda} + (g_s')_{s \in \Lambda}
        \defeq (g_s + g_s')_{s \in \Lambda} \in
        \bigoplus_{s\in \Lambda} G_s$.
        這邊也一個ㄛ圖

    Uniqueness: Assume $\exists$ another $G$ satisfies the universal property,
    一個大ㄛ圖 ($G, \bigoplus_{s\in \Lambda} G_s$ 互相有唯一個映射可以
    keep $i_{s_0}$, $\varphi \circ \psi = \text{id}_{G}, \psi \circ \varphi
    = \text{id}_{\bigoplus_{s\in \Lambda} G_s}$)
  \end{proof}
\end{theorem}

\begin{definition}
  Given a non-empty family of groups $\{\, G_s \mid s \in \Lambda \,\}$,
  a direct product of $\{\, G_s \mid s \in \Lambda \,\}$ is a group
  $\prod_{s\in \Lambda} G_s$ with projections
  $p_{s_0}: \prod_{s\in \Lambda} G_s \to G_{s_0}, \forall s_0 \in \Lambda$
  satifsfying the following universal property:

  for any group $H$ with group homo.
  $\varphi_s: H \to G_s, \forall s \in \Lambda$, $\exists! \varphi:
  H \to \prod_{s\in \Lambda} G_s$ s.t. 又一個ㄛ圖
\end{definition}

\begin{theorem}
  $\prod_{s\in \Lambda} G_s$ exists and is unique up to isomorphisms.

  \begin{proof}
    Existence: $\prod_{s\in \Lambda} G_s = \{\, (g_s)_{s\in \Lambda}
      \mid g_s \in G_s \,\}$ and
      \[ p_{s_0}: \prod_{s\in \Lambda} G_s \to G_{s_0},
        (g_{s_0})_{s\in \Lambda} \mapsto g_{s_0}, \forall s_0 \in \Lambda \]
      \begin{itemize}
        \item group operaion: $(g_s)_{s \in \Lambda} \cdot (g_s')_{s \in \Lambda}
          \defeq (g_s g_s')_{s \in \Lambda} \in \prod_{s\in \Lambda} G_s$.
        \item Define $\varphi$:
          這邊也一個ㄛ圖
          which is uniquely defined.
      \end{itemize}

    Uniqueness: Assume $\exists$ another $G$ satisfies the universal property,
    一個大ㄛ圖 ($G, \prod_{s\in \Lambda} G_s$ 互相有唯一個映射可以
    keep $i_{s_0}$, $\varphi \circ \psi = \text{id}_{G}, \psi \circ \varphi
    = \text{id}_{\prod_{s\in \Lambda} G_s}$)
  \end{proof}
\end{theorem}

\begin{exercise}
  Google the definition of the {\bf direct limit} and show the existence and
  uniqueness.
\end{exercise}

\begin{exercise}
  Google the definition of the {\bf inverse limit} and show the existence and
  uniqueness.
\end{exercise}

\underline{Motivation}: $\zeta_m$ is called an $m$-th root of unity if
$\zeta_m^m = 1$.
\[ \varinjlim\limits_n \quot{\Zb}{2^n\Zb} \cong
\{\, \text{$2^n$-th roots of unity} : n \in \Nb \,\} \]


\[ \varinjlim\limits_n \quot{\Zb}{2^n\Zb} =
  \quot{(\bigoplus_{n\in\Nb} \quot{\Zb}{2^n\Zb})}{
  \gen{ i_k(a) - i_j(f_{kj}(a)) \mid k \le j, a \in \quot{\Zb}{2^k\Zb} }}
\]
where $f_{kj}: \quot{\Zb}{2^k\Zb} \to \quot{\Zb}{2^j\Zb}$.

Inverse limit:
\[
  \varprojlim \quot{\Zb}{2^n\Zb} = \left\{\,
    (n_1, n_2, \dots ) \in \prod_n \quot{\Zb}{2^n\Zb} \middle|
    \forall i < j, n_i \equiv n_j \pmod 2^{i+1}   \,\right\}
\]

\subsubsection{Rings and fields}

\begin{definition}
  A {\bf ring} is sa non-empty set $R$ with two operations $R\times R \to R$
  \[
    (a, b) \mapsto a + b \quad \text{and} \quad (a, b) \mapsto ab
  \]
  satisfying
  \begin{enumerate}
    \item $(R, +, 0)$ is an abelian group.
    \item $(R, \cdot)$ is a semigroup. (if it is a monoid, then it is called
      ``a ring with 1.'')
    \item (Distributive laws) $\forall a, b, c \in \Rb, \begin{cases}
      a(b + c) = ab + ac\\ (b + c)a = ba + ca\end{cases}$
  \end{enumerate}
\end{definition}

\begin{example}
  $\Zb, \Rb, \Cb, \quot{\Zb}{n\Zb}, M_{n\times n}(\Fb)$
\end{example}

\begin{example}
  Let $G$ be an abelian group.
  Define (endomorphism, automorphism)
  \[
    \text{End}(G) \defeq \{\, \text{group homo.~} G \to G \,\} \quad
    \text{Aut}(G) \defeq \{\, \text{group isom.~} G \to G \,\}
  \]
  A natural ring structure on $\text{End}(G)$ is:
  \[
    \forall a \in G, \begin{cases}
      (f+g)(a) \defeq f(a)g(a) \\
      (f\cdot g)(a) \defeq f(g(a))
    \end{cases}
  \]
\end{example}

\begin{example}
  $\Zb\left[\sqrt{2}\right] = \left\{\,
  a + b\sqrt{2} \relmiddle| a, b \in \Zb \,\right\} \subset \Rb$.
\end{example}

\begin{definition}
  Let $R$ be a ring with $1$.
  \begin{enumerate}[(a)]
    \item $\forall a \in R, a \ne 0$, a in called a unit if
      $\exists a^{-1} \in R$.
  \item $\left(R^\times = \{\text{units in $R$}\}, \cdot, 1)\right)$ forms
    a group.
  \item $R$ is called a division ring if $R \setminus \{0\} = R^\times$.
  \item $R$ is said to be commutative if $ab = ba, \forall a, b \in R$.
  \item $R$ is a field if $R$ is a commutative division ring.
  \item $a \ne 0$ is called a left zero divisor if $\exists b \in R, b \ne 0$
    s.t. $ab = 0$.
  \item $a$ is called a zero divisor if $a$ is either a left or right zero
    divisor.
  \item $R$ is called an integral domain if $R$ is a commutative ring without
    zero divisors.
  \end{enumerate}
\end{definition}

\underline{Fact}:
\begin{enumerate}
  \item fields $\implies$ integral domains.
  \item finite + integral domain $\implies$ fields.
    \begin{proof}
      Let $R = \{ 0, a_1, \dots, a_n \}$, for $a \in R, a \ne 0$,
      $aa_i = aa_j \implies a(a_i - a_j) = 0 \implies i = j$.
      So $\{0, aa_1, \dots, aa_n \} = R \implies \exists a_i$ s.t. $aa_i = 1$.
    \end{proof}
\end{enumerate}

\begin{prop}
  TFAE
  \begin{enumerate}
    \item $\quot{\Zb}{n\Zb}$ is an integral domain.
    \item $\quot{\Zb}{n\Zb}$ is a field.
    \item $n = p$ is a prime.
  \end{enumerate}
  easy to prove.
\end{prop}

\begin{definition} \mbox{}
  \begin{itemize}
    \item $f: R_1 \to R_2$ is called a ring homomorphism if
      $\forall a, b \in R, \begin{cases}
        f(a+b) &= f(a) + f(b) \\
        f(ab) &= f(a)f(b)
      \end{cases}$.
    \item $\Image f$ is a subring of $R_2$.
    \item $\Ker f = \{\, x \in R_1 \mid f(x) = 0 \,\}$ is an additive group of
      $R_1$ and $\forall r \in R_1, x \in \Ker f, f(rx) = f(r)f(x) = f(r)0 = 0
      \implies rx \in \Ker f, xr \in \Ker f$.
    \item $\quot{R_1}{\Ker f}$ is an additive group and
      $\quot{R_1}{\Ker f} \cong \Image f$ (additive isomorphism).
  \end{itemize}
\end{definition}

\begin{definition}
  Let $I$ be an additive subgroup of $R$.
  $I$ is called an ideal if $\forall r \in R, x \in I, rx \in I, xr \in I$.

  $\left(\quot{R}{I}, +, \cdot \right)$ forms a quotient ring under
  \[ \forall r_1, r_2 \in R, (r_1+I)(r_2+I) = r_1r_2 + I \]
  well-defined: easy to show.
\end{definition}

\begin{exercise}
  State and show the isomorphism theorems and the factor theorem.
\end{exercise}

\begin{prop}
  If $R$ is a ring with $1$, then $\exists!$ ring homo. $\varphi: \Zb \to R$
  s.t. $\varphi(1) = 1$.
  \begin{proof}
    $\varphi(n) = \varphi(1) + \dots + \varphi(1) = nr$, so $\varphi$ is
    well-defined.

    Original:
    Consider $\varphi_r: \Zb \to R, 1 \mapsto r$, for $r \in R$.
    ($\varphi(n) = \varphi(1) + \dots + \varphi(1) = nr$)

    If $\varphi_r$ is a rong homo., then $\varphi_r(nm) = nmr$ and
    $\varphi_r(n)\varphi_r(m) = nrmr = nmr^2$.
    So $nmr = nmr^2 \implies r = r^2 \implies r = 1$ (if $r \ne 0$).
  \end{proof}
  \label{prop:phi1e1}
\end{prop}

\begin{definition}
  In Prop \ref{prop:phi1e1}, $\Ker \varphi = m\Zb$ for some $m > 0$.
  We call $m$ the characteristic of $R$, denoted by $\Char R = m$.
\end{definition}

\begin{prop} \mbox{}
  \begin{enumerate}
    \item If $R$ is an integral domain, then $\Char R = 0 \text{~or~} p$,
      where $p$ is a prime. (try to prove this)
    \item In the case of $\Char R = p$,
      $\forall a, b \in R, (a + b)^p = a^p + b^p$.
      \begin{proof}
        \[ (a+b)^p = a^p + \binom{p}{1}a^{p-1}b + \dots + b^p = a^p + b^p \]
        because $p \mid \binom{p}{1} \implies \binom{p}{i}a^{p-i}b^{i} = 0$.
      \end{proof}
  \end{enumerate}
\end{prop}

\begin{exercise}
  Let $F$ be a field. Show that
  \begin{enumerate}
    \item if $\Char F = 0$, then $\Qb \hookrightarrow \text{subfield of~} F$.
    \item if $\Char F = p$, then
      $\quot{\Zb}{p\Zb} \hookrightarrow \text{subfield of~} F$.
  \end{enumerate}
  \label{ex:4-4}
\end{exercise}

\begin{theorem}
  If $F$ is a finite field, then $\abs{F} = p^n$ for some $n \in \Nb$ and
  $p$ is a prime.
  \begin{proof}
    By Ex. \ref{ex:4-4}, $\Char F = p$, $p$ is a prime and $\quot{\Zb}{p\Zb}
    \hookrightarrow F$.

    We have $\quot{\Zb}{p\Zb} \times F \to F, (r, v) \mapsto rv$.
    $F$ can be rearded as a vector space over $\quot{\Zb}{p\Zb}$.

    Let $\dim_{\quot{\Zb}{p\Zb}} F = n$, then $F \cong
    \left(\quot{\Zb}{p\Zb}\right)^n \implies \abs{F} = n$.
  \end{proof}
\end{theorem}

\begin{theorem}
  Let $F$ be a field. Then any finite subgroup $G$ of $(F^\times, \cdot, 1)$
  is cyclic.

  \begin{proof}
    Let $\abs{G} = n$. Define $h$ to be the max order of an element in $G$,
    say $a^h = 1$.

    If $h = n$, then $\abs{\gen{a}} = h = n = \abs{G}$ and $\gen{a} \subseteq G$,
    so $G = \gen{a}$.

    Otherwise, $h < n$. We know that $x^h - 1$ has at most $h$ roots.
    So $\exists b \in G$ is not a root of $x^h - 1$.
    Let $\ord(b) = h'$, so $h' \mid n$ and $h' \not\mid h$.
    So $\exists$ a prime $p$ s.t. $p^r \mid h'$ but $p^r \not\mid h$.

    Write $h = mp^s, s < r$ and $\gcd(m, p) = 1 \implies
    \ord \left( a^{p^s} \right) = m$.

    Write $h' = qp^r \implies \ord \left( b^q \right) = p^r$.

    Since $\gcd(m, p^r) = 1, \ord\left(a^{p^s} b^q \right) = mp^r > mp^s = h$,
    which is a contradiction.
  \end{proof}
\end{theorem}

\begin{exercise} \mbox{}
  \begin{enumerate}
    \item Let $a, b \in G$ with $ab = ba$ and $\ord(a) = m, \ord(b) = n$.
      If $\gcd(m, n) = 1$, then $\ord(ab) = mn$.
      In general, is the order of $ab$ equal to $\lcm(m, n)$?
    \item Let $G$ be a finite group and $H, K \le G$. Then
      $\abs{HK} = \frac{\abs{H}\abs{K}}{\abs{H \cap K}}$.
  \end{enumerate}
\end{exercise}

%! TEX root=./hw.tex

\section{Week 5}

\begin{exercise} \mbox{}
  \begin{enumerate}
    \item Let $p$ be an odd prime with $p \nmid m$.
      Suppose $a \in \Zb$ s.t. $\Phi_m(a) \equiv 0 \pmod p$. then
      $\ord(a) = m$ in $\left(\quot{\Zb}{p\Zb}\right)^\times$.
      (hint: $x^m - 1 = \prod_{d\mid m} \Phi_d(x)$)
    \item Let $a\in \Zb$. Show that if $p$ is an odd prime dividing $\Phi_m(a)$,
      then either $p \mid m$ or $p \equiv 1 \pmod m$.
  \end{enumerate}
\end{exercise}

\begin{exercise} \mbox{}
  \begin{enumerate}
    \item Show that $\left[\Qb\left(\zeta_n + \frac{1}{\zeta_n}\right) : \Qb\right]
      = \frac{\varphi(n)}{2}$.
    \item Find $\Phi_8, \Phi_9$.
    \item Show that $x^{16} + 1$ is irreducible in $\Qb[x]$ and is reducible
      in $\Fb_7[x]$ as a product of $4$ quartic polynomials.
  \end{enumerate}
\end{exercise}

\begin{exercise}
  show that $p$: odd prime, $\left(\quot{\Zb}{p^e\Zb}\right)^\times$ is cyclic
  of order $p^{e-1}(p-1)$ and $\left(\quot{\Zb}{2^e\Zb}\right)^\times \cong
  \quot{\Zb}{2\Zb} \times \quot{\Zb}{2^{e-2}\Zb}, e\ge 2$.
  
  Hints:
  \begin{enumerate}
    \item Check $(1+p)^{p^{e-1}} \equiv 1 \pmod{p^e}$ but
      $(1+p)^{p^{e-2}} \not\equiv 1 \pmod{p^e}$. And for $e \ge 3$,
      $(1+2^2)^{2^{e-2}} \equiv 1 \pmod{2^e}$ but
      $(1+2^2)^{2^{e-3}} \equiv 1 \pmod{2^e}$.
    \item If each Sylow $p$-subgroup of $g$ is normal, then $G$ is isomorphic
      to the product of all sylow $p$-subgroups.
      $(1+p)^{p^{e-2}} \not\equiv 1 \pmod{p^e}$.
  \end{enumerate}
\end{exercise}

\begin{exercise} \mbox{}
  \begin{enumerate}
    \item Let $\Cb(t)$ be the field of rational functions over $\Cb$ and
      $L$ be a splitting field of $x^n - t$ over $\Cb(t)$.
      Find $\Gal(L / \Cb(t))$.
    \item Let $\Fb_p(t)$ be the field of rational functions over $\Fb_p$ and
      $L$ be a splitting field of $x^n - t$ over $\Fb_p(t)$.
      Find $\Gal(L / \Fb_p(t))$.
  \end{enumerate}
\end{exercise}

\begin{exercise}
  Let $\Char K \ne 2, 3$ and $f(x) = x^4 + px^2 + qx + r$ be irr. and separable
  with roots $\alpha_1, \dots, \alpha_4$. Let $L = K(\alpha_1, \alpha_2, \alpha_3, \alpha_4)$
  and $G_f = \Gal(L/K) \le S_4$. Set
  $\beta_1 = \alpha_1 \alpha_2 + \alpha_3 \alpha_4,
  \beta_2 = \alpha_1 \alpha_3 + \alpha_2 \alpha_4,
  \beta_3 = \alpha_1 \alpha_4 + \alpha_2 \alpha_3$.
  \begin{enumerate}
    \item Show that $L^{G_f \cap V} = K(\beta_1, \beta_2, \beta_3)$ and
      $\Gal\left(\quot{K(\beta_1, \beta_2, \beta_3)}{K}\right) \cong \quot{G_f}{G_f \cap V}$
      where $V = \Set{1, \cycle{1,2}\cycle{3,4}, \cycle{1,3}\cycle{2,4},
      \cycle{1,4}\cycle{2,3}} \le S_4$.
    \item Show that there exists $i$ s.t. $\beta_i \in K \iff G_f \subseteq D_4$.
    \item Let $h(x) = (x-\beta_1)(x-\beta_2)(x-\beta_3) \in K[x]$ with discriminant
      $D(h)$, Show that
      \begin{enumerate}
        \item If $h(x)$ is irr. and $D(h) \not\in K^2$, then $G_f \cong S_4$.
        \item If $h(x)$ is irr. and $D(h) \in K^2$, then $G_f \cong A_4$.
        \item If $h(x)$ splits completely in $K[x]$, then $G_f \cong V$.
        \item Let $h(x)$ has one root in $K$. Then
          \begin{enumerate}
            \item If $f(x)$ is irr. over $K(\beta_1, \beta_2, \beta_3)$, then
              $G_f \cong D_4$.
            \item If $f(x)$ is reducible over $K(\beta_1, \beta_2, \beta_3)$, then
              $G_f \cong C_4$.
          \end{enumerate}
      \end{enumerate}
  \end{enumerate}
\end{exercise}

%1 TEX root=../main.tex
\subsection{Week 6}
\subsubsection{Group actions \RNum{2}}

\begin{definition}
  Let $G \acts X$ and $\abs{X} < \infty$.
  Write $\Fix G \defeq \{\, x \in X \mid gx = x \quad \forall g \in G \,\}$.
\end{definition}

\begin{itemize}
  \item $x \in \Fix G$, $Gx = \{ x \}$.
  \item $x \not\in \Fix G$, $\abs{Gx} = [G : G_x]$.
\end{itemize}

Let $\{ G_{x_1}, \dots, G_{x_n} \}$ be the set of distinct orbits.
After rearrangement, assume $x_1, \dots, x_r \in \Fix G,
x_{r+1}, \dots, x_n \not\in \Fix G$. Then
\[
  \abs{X} = \abs{\Fix G} + \sum_{i=r+1}^{n} [G : G_{x_i}]
\]

\begin{theorem}[class equation]
  Let $\abs{G} < \infty$. Then either $G = Z_G$ or
  $\exists a_1, \dots, a_m \in G \setminus Z_G$ s.t.
  \[
    \abs{G} = \abs{Z_G} + \sum_{i=1}^{n} [G : G_{a_i}]
  \]
  \begin{proof}
    Consider the action $(g, x) \mapsto gxg^{-1}$, then
    \[
      \Fix G = \{\, x \in G \mid gxg^{-1} = x \quad \forall g \in G \,\}
      = Z_G
    \]
    It follows from the above argument.
  \end{proof}
\end{theorem}

\begin{definition}
  $G$ is called a $p$-group if $\abs{G} = p^n$, where $p$ is a prime,
  $n \in \Nb$.
\end{definition}

\begin{prop}
  If $G$ is a $p$-group, then $Z_G \ne \{ 1 \}$.
  \begin{proof}
    Let $\abs{G} = p^n$. If $G = Z_G$, then done.
    Otherwise, by the class equation (use action by conjugation),
    $\abs{G} = \abs{Z_G} + \sum_{i=1}^{n} [G : G_{a_i}], \quad a_i \not\in Z_G$.

    $G_{a_i} = Z_G(a_i)$, so $a_i \not\in Z_G \implies Z_G(a_i) \lneq G
    \implies p \mid [G:Z_G(a_i)] = \frac{\abs{G}}{\abs{Z_G(a_i)}}$.

    So $\abs{Z_G} = \abs{G} - \sum_{i=1}^{n} [G : Z_G(a_i)]
    \implies p \mid \abs{Z_G} \implies Z_G \ne \{ 1 \}$.
  \end{proof}
  \label{prop:pgroup}
\end{prop}

\begin{prop}
  If $\abs{G} = p^2$, then $G$ is abelian.
  ($\quot{\Zb}{p\Zb} \times \quot{\Zb}{p\Zb}$ and $\quot{\Zb}{p^2\Zb}$)
  \begin{proof}
    Assume that $G$ is not abelian.
    By prop \ref{prop:pgroup}, $\abs{Z_G} = p \implies \abs{\quot{G}{Z_G}} = p
    \implies \quot{G}{Z_G}$ is cyclic $\implies G$ is abelian. (contradiction)
  \end{proof}
\end{prop}

\begin{prop}
  If $\abs{G} = p^3$ and $G$ is not abelian, then $\abs{Z_G} = p$.

  (Abelian: $\quot{\Zb}{p\Zb} \times \quot{\Zb}{p\Zb} \times \quot{\Zb}{p\Zb},
  \quot{\Zb}{p^2\Zb} \times \quot{\Zb}{p\Zb}, \quot{\Zb}{p^3\Zb}$)

  \label{prop:w6p3}
\end{prop}

\begin{prop}
  Let $\abs{G} = p^n$. Then $\forall 0 \le k \le n, \exists G_k \lhd G$ s.t.
  $\abs{G_k} = p^k$ and $G_i \lneq G_{i+1}$.

  In general, for a finite group $G$, $\exists {\{1\}} =
  G_r \lhd G_{r-1} \lhd \dots \lhd G_1 \lhd G_0 = G$ s.t. $\quot{G_i}{G_{i+1}}$
  is cyclic.

  we call $G$ a solvable group.

  \begin{proof}
    By induction on $n$, $n = 1$ is trivial.
    For $n > 1$, assume that the statement a holds for $n-1$.
    By prop \ref{prop:pgroup}, $Z_G \ne \{1\}$. $\exists a \in Z_G, a \ne 1$.
    Let $\ord(a) = p^l$, then $\ord(a^{p^{l-1}}) = p$.
    $\implies$ in any case, $\exists a \in Z_G$ with $\ord(a) = p$.

    Now $\abs*{\quot{G}{\gen{a}}} = p^{n-1}$, so by induction hypothesis,
    $\forall 0 \le k \le n - 1, \exists \ob{G_k} \lhd \quot{G}{\gen{a}}$ s.t.
    $\abs*{\ob{G_k}} = p^k, \ob{G_i} \lneq \ob{G_{i+1}}$.

    By 3rd isom. thm., $\exists G_{k+1} \lhd G$ s.t. $\ob{G_k} =
    \quot{G_{k+1}}{\gen{a}}, G_j \lneq G_{j+1}$ and $\abs{G_{k+1}} = p^{k+1}$.

  \end{proof}
\end{prop}

\begin{prop}
  Let a $p$-group $G \acts X$ with $\abs{X} < \infty$.
  Then $\abs{X} \equiv \abs{\Fix G} \pmod p$.
  \label{prop:useful}
\end{prop}

\begin{theorem}[Cauchy theorem]
  Let $p \Div \abs{G}$. Then $\exists a \in G$ s.t. $\ord(a) = p$. Consider
  \[ X = \{\, (a_1, \dots, a_p) \mid a_i \in G, a_1a_2\dots a_p = 1\,\} \]
  and the action $\quot{\Zb}{p\Zb} \times X \to X$:
  \[
    (\ob{k}, (a_1, \dots, a_p)) \mapsto (a_{k+1}, \dots, a_p, a_1, \dots, a_k)
  \]
  (This is well-defined since $ab = 1 \implies ba = 1$ in a group.)
  We find that $(a_1, \dots, a_p) \in \Fix \quot{\Zb}{p\Zb} \iff a_1 = a_2
  \dots a_p$.
  By prop \ref{prop:useful}, $\abs*{\Fix \quot{\Zb}{p\Zb}} \equiv \abs{X}
  \pmod p$. And $\abs{X} = \abs{G}^{p-1} \equiv 0 \pmod p$.
  Since $(1, \dots, 1) \in \Fix \quot{\Zb}{p\Zb}, \abs*{\quot{\Zb}{p\Zb}} \ne 0
  \implies \abs*{\quot{\Zb}{p\Zb}} \ge p$.

  So $\exists (a, \dots, a) \in \Fix \quot{\Zb}{p\Zb} \implies a^p = 1$.
\end{theorem}

\underline{Application}: Let $\abs{G} = p^3$ and $G$ be non-abelian
($p$ is odd).
By prop \ref{prop:w6p3}, $\abs*{\quot{G}{Z_G}} = p^2$. Since $G$ is non-abelian,
we have $\quot{G}{Z_G} \cong \quot{\Zb}{p\Zb} \times \quot{\Zb}{p\Zb}$.
That is, $\forall a \in G, a^p \in Z_G$.

So,
\[
  \exists \varphi: G \to Z_G \cong C_p \text{~with~}
  \varphi: a \mapsto a^p
\]

Since $\quot{G}{Z_G}$ is abelian, $[G,G] \le Z_G$. And
\[
  \begin{cases}
    \abs{[G,G]} \Div \abs{Z_G} = p \\
    G \text{~is non-abelian}
  \end{cases}
  \implies [G,G] = Z_G
\]

\begin{definition}
  $[x, y] = x^{-1}y^{-1}xy \in [G,G], [x,y]^p = 1$.
\end{definition}

So $a^p b^p = a^p b^p [b, a]^p$ ... 換換換 總共需要 $p(p-1)/2$
\[ a^p b^p = (ab)^p [b,a]^{\frac{p(p-1)}{2}} = (ab)^p \]

So $\varphi$ is a group homo.

Now if $\ker \varphi = G \quad (\forall a \in G, a^p = 1)$,
i.e. $\varphi$ is trivial, then $\varphi$ is useless.
Else, $\exists a \in G$ s.t. $\ord(a) = p^2$, then
$H = \gen{a} \lhd G$. ($[G:H] = p$ is the smallest prime dividing $\abs{G}$)

Also, in this case, $\varphi: G \onto Z_G \implies
\quot{G}{\ker \varphi} \cong Z_G$. Let $E = \ker \varphi$, $\abs{E} = p^2$.
By the def. of $\ker \varphi$, $E \cong \quot{\Zb}{p\Zb} \times
\quot{\Zb}{p\Zb}$.

We find that $H \cap E = \gen{a^p}$. Pick $b \in E \setminus H$ and let
$K = \gen{b} \implies \abs{K} = p, H \cap K = \{ 1 \}, HK = G$.

\subsubsection{Semidirect product}

\begin{fact}
$K \lhd G, H \lhd G, K \cap H = \{1\} \implies KH = K \times H$ \\
($\forall k\in K, h \in H, khk^{-1} h^{-1} \in H \cap K = \{1\}, \implies kh=hk$)
\end{fact}

\begin{fact}
Let $K, H$ be two groups, and $G=K \times H \implies K \times \{1\} \lhd K \times H, \{1\} \times H \lhd K \times H$
\end{fact}

\begin{observation}
$K \leq G, H \lhd G, K \cap H = \{1\}$ (K 慘 H 好,簡稱慘好集) \\
$\implies$ elements in $KH$ has unique representation ? 好事喔\\
$KH \iff K \times H$ 1-1 corresp, $(kh) \leftrightarrow (k, h)$
\end{observation}

Group operation :
$\forall k_1, k_2 \in K, h_1, h_2 \in H, (k_1 h_1) (k_2 h_2) = k_1 k_2 (k_2^{-1} h_1 k_2) h_2$ \\
Let $\tau : K \to \Aut(H), k \mapsto (\tau(k) : h\mapsto khk^{-1})$ (類似 $\in \Inn(H)$ )

\begin{definition}[Semi-Direct Product (慘好積)]
  $K \times_{\tau} H = \{(k, h) | k\in K, h \in H\}$ with group operation :
$(k_1, h_1)(k_2, h_2) = (k_1 k_2, \tau(k_2^{-1})(h_1)(h_2))$
where $\tau : K \to \Aut(H)$  (need not to be inner homomorphism)
\end{definition}

Properties:
\begin{itemize}
  \item Associativity: Good, ex
  \item The identity = $(1, 1)$
  \item Inverse : $(k, h)^{-1} = (k^{-1}, \tau(k)(h^{-1}))$
  \item $K \cong K \times \{1\} \leq K \times_{\tau} H$ :
    $(k_1, 1)(k_2, 1) = (k_1 k_2, \tau(k_2^{-1})(1)1) = (k_1 k_2, 1) \in K \times \{1\}$
    $H \cong \{1\} \times H \leq K \times{\tau} H : (1, h+1), (1, h_2) = (1, \tau(1^{-1})(h_1)h_2) = (1, h_1 h_2) \in \{1\} \times K $
  \item $H \lhd K \times_t H : (k, h) (1, h')(k, h)^{-1} = (k, hh')(k^{-1}, \tau(k)(h^{-1}))
    = (1, \tau(k)(hh')\tau(k)(h^{-1})) \in H$

  \item $\tau(k)(h) = khk^{-1}$ :
    $(k, 1)(1, h)(k^{-1}, 1) = (k, h)(k^{-1}, 1) = (1, \tau(k)(h))$
  \item If $\tau$ is trivial $\implies K \times_t H \cong K \times H$
\end{itemize}

\begin{remark}
Some definition swaps the order of $H$ and $K$, i.e. $(h_1, k_1) (h_2, k_2) = (h_1 \phi(k_1)(h_2), k_1 k_2)$
\end{remark}

\begin{exercise}
Show that $H \rtimes_\phi K$ is a group and satisfies the above properties.
\end{exercise}

\begin{example}
Construct a non-abelian group of order 21.
\end{example}

\begin{fact}
  $\Aut(\quot{\Zb}{p\Zb}) \cong (\quot{\Zb}{p\Zb})^\times \cong C_{p-1}$
\end{fact}
Sol : $\phi_k: \quot{\Zb}{p\Zb} \to \quot{\Zb}{p\Zb}, \bar{1} \mapsto \bar{k}$

$\phi_{k_2} o \phi_{k_1} (T) = \phi_{k_2}(\bar{k_1}) = \phi_{k_2}(T+\cdots+T)
= \bar{k_2} + \cdots \bar{k_2} = \ob{k_1 k_2}$

Let $K = C_3, H = C_7$, define $\tau : C_3 \to \Aut(C_7) \cong C_6, a \mapsto \phi_2$

$\phi_k : b \mapsto b^k$

$G = \gen{a, b | a^3=1, b^7=1, aba^{-1}=b^2}$

\begin{example}
  p : odd, $\abs{G} = p^3$, $G$ is non-abelian.
\end{example}
(sol)
$\phi: G \to Z(G), a \mapsto a^p$ non trivial
case $\exists a \in G $ with $\ord(a) = p^2$.
Let $H = \gen{a}$ here $\phi$ is onto and $E = \ker \phi \cong \quot{\Zb}{p\Zb} \times \quot{\Zb}{p\Zb}$
And $\abs{H \cap E} = p$
$H \lhd G$ because $[G: H]=p$
Pick $b \in E \setminus H$ and let $K = \gen{b} \implies \abs{K} = p, K \cap H = \{1\}$
so $\abs{G} = \abs{KH} = p^3$

\begin{fact}
  $\Aut(\quot{\Zb}{p^2 \Zb}) \cong (\quot{\Zb}{p^2 \Zb})^\times$
\end{fact}
Sol : $\phi_k: \quot{\Zb}{p^2\Zb} \to \quot{\Zb}{p^2\Zb}, \bar{1} \mapsto \bar{k}, \gcd(k, p) = 1$

Find a group homo $\tau : K \implies \Aut(H)$
because $(1+p)^p \equiv 1 \mod p^2$, $\ord\left(\ob{1+p}\right) \in
\quot{\Zb}{p^2\Zb}$
Let $P = \gen*{\ob{1+p}}$ is the only subgroup of order $p$.
(if $\exists |Q| = p$, $P \neq Q$ then $P \cap Q = 1$, $|PQ| = p^2$, miserable.)
So let $\tau : b \mapsto (\phi_{1+p} : a \mapsto a^{1+p})$
so $G = \gen{a, b | a^{p^2}=1, b^p = 1, bab^{-1} = a^{1+p}}$ is a non-abelian group of order $p^3$.

\begin{example}
Isometry of $R^n$
\end{example}

\begin{definition}[Isometry]
An isometry of $R^n$ is a function $h: R^n \to R^n$ that preserves the distance between vectors.
\end{definition}
$h = t \circ k$ where $t$ is translation, $k$ is an isometry fixing the origin, i.e. $k \in O(n)$.
Let $T$ be the group of translations on $R^n$, $T \cong (R^n, +, 0), t \mapsto t(0)$.

Let $\tau : O(n) \to \Aut(T), A \mapsto L_A : R^n \to R^n, v \mapsto Av$

$\implies \Isom(R^n) = O(n) \times_\tau R^n$

\begin{example}
Quaternium $Q_8 = \{\pm 1, \pm i, \pm j, \pm k\}$ is not a semi-deriect product of any two proper subgroups.
\end{example}
pf: since $\{\pm 1\}$ is contained in any non-trivial subgroups, can't find $H \cap K = \{1\}$.

\begin{example}
  $A_4$, $V_4 = \{1, (12)(34), (14)(23), (13)(24)\} \lhd A_4, V_4 \cong \quot{\Zb}{2\Zb} \times \quot{\Zb}{2\Zb}$
\end{example}
Let $H = \gen{(123)} \cong C_3$, define $\tau : H \to \Aut(V_4) \cong GL_2(\quot{\Zb}{2\Zb})$
$(123) \mapsto (\bar{0} \bar{1}; \bar{1} \bar{1})$
so $A_4 \cong C_3 \times_\tau V_4$.

\begin{exercise}
  Construct $D_n$ as a semi-direct product of $\quot{\Zb}{n\Zb}$ and
  $\quot{\Zb}{2\Zb}$.
\end{exercise}

\begin{exercise} \mbox{}
  \begin{enumerate}
    \item Show that $S_4$ is a semi-direct product of $V_4$ and $H = \{\, \sigma \in S_4 | \sigma(4) = 4 \,\} \sim S_3$.
    \item Show that $S_n$ is a semi-direct product of $A_n$ and $H = \gen{(12)}$.
  \end{enumerate}
\end{exercise}

\begin{remark} \mbox{}
  \begin{itemize}
    \item $\Aut(\quot{\Zb}{p\Zb} \times \quot{\Zb}{p\Zb}) \cong GL_2(\quot{\Zb}{p\Zb})$
      (regarded as a vector space over $\quot{\Zb}{p\Zb}$)
    \item $\Aut(\quot{\Zb}{p\Zb} \times \quot{\Zb}{p\Zb}) \cong
      \Aut(\quot{\Zb}{p\Zb}) \times \Aut(\quot{\Zb}{q\Zb}) \cong
      C_{p-1} \times C_{q-1}$
  \end{itemize}
\end{remark}

%! TEX root=../main.tex
\subsection{Calculation of Galois groups}
Let $f(x)$ be separable in $K[x]$ and $L = K(\alpha_1, \dots, \alpha_n)$ be a splitting field of
$f$ over $K$. The goal is to find $\Gal(L/K)$ which is in $S_n$.

Define $\theta \triangleq y_1 \alpha_1 + \dots + y_n \alpha_n$. For each $\sigma \in S_n$,
define $\sigma_y(\theta) \triangleq y_{\sigma(1)} \alpha_1 + \dots + y_{\sigma(n)} \alpha_n$ and
$\sigma_{\alpha}(\theta) = y_1 \alpha_{\sigma(1)} + \dots + y_n \alpha_{\sigma(n)}$.
It is easy to see that $\sigma_y^{-1} = \sigma_\alpha$. 

In $L(x, y_1, \dots, y_n)$, we consider $F(x, \bm{y})
= \prod\limits_{\sigma \in S_n} (x - \sigma_y(\theta))
= \prod\limits_{\sigma^{-1} \in S_n} (x - \sigma_{\alpha}(\theta))
= \prod\limits_{\sigma \in S_n} (x - \sigma_\alpha(\theta))$.
Since each coefficient of $F$ is a symmetric polynomial of $\alpha_1, \dots, \alpha_n$,
by the fundamental theorem of symmetric polynomials, these symmetric polynomials are
polynomials of the elementary symmetric polynomials. Thus $F(x, y) \in K[x, y_1, \dots, y_n]$.

Decompose $F$ into irreducible factors in $K[x, y_1, \dots, y_n]$, say $F = F_1 F_2 \dotsm F_r$.
Notice that for any $\sigma \in S_n$, $F = \sigma_y F = \sigma_y F_1 \cdot \sigma_y F_2 \dotsm \sigma_y F_r$.
And each $F_i$ is map to some $F_j$, thus $\sigma$ induces a permutation of $F_1, F_2, \dots, F_r$.

For convenience, assume $(x - \theta) \mid F_1$. We have the following lemma:

\begin{lemma}
  \[ Q \triangleq \Set{ \sigma : \sigma_y F_1 = F_1 } = \Set{ \sigma : \sigma_y(x - \theta) \mid F_1} \]
  \begin{proof}
    ``$\subseteq$'': Since $x - \theta \mid F_1$, so $\sigma_y (x - \theta) \mid \sigma_y F_1 = F_1$.

    ``$\supseteq$'': $\sigma_y(x - \theta) = x - \sigma_y(\theta) \mid \sigma_y(F_1)$, so $\sigma_y(F_1)$
    and $F_1$ has a common factor. Since $F$ is separable, $\sigma_y(F_1) = F_1$.
  \end{proof}
\end{lemma}

\begin{prop}
  $\Gal(L/K) = Q$.

  \begin{proof}
    ``$\subseteq$'': For each $\sigma \in \Gal(L/K) \hookrightarrow S_n$, extend $\sigma$ to
    \[
      \arraycolsep=1pt
      \begin{array}{rccc}
        \tilde\sigma: & L(y_1, \dots, y_n) & \to & L(y_1, \dots, y_n) \\
        & \alpha \in L & \mapsto & \sigma(\alpha) \\
        & y_i & \mapsto & y_i
      \end{array}
    \]
    The automorphism fixes $K(y_1, \dots, y_n)$, so
    $\tilde\sigma(\theta) = \sigma_\alpha(\theta)$ and $\theta$ share the
    same minimal polynomial over $K(y_1, \dots, y_n)$.
    By Gauss's lemma, $F_1$ is irreducible in $K[y_1, \dots, y_n][x] \implies F_1$
    is irreducible in $K(y_1, \dots, y_n)[x]$, thus
    $F_1 = m_{\theta, K(y_1, \dots, y_n)}
    = m_{\sigma_\alpha(\theta), K(y_1, \dots, y_n)}$, which implies
    $(x - \sigma_\alpha(\theta)) \mid F_1$.
    So $\sigma_y^{-1} F_1 = F_1 \implies \sigma^{-1} \in Q \implies \sigma \in Q$.

    ``$\supseteq$'': For any $\sigma \in Q$, $F_1 = m_{\theta, K(y_1, \dots, y_n)}
    = m_{\sigma^{-1}_\alpha(\theta), K(y_1, \dots, y_n)}$, so there exists
    $\tau \in \Aut(L(\bm{y}) / K(\bm{y}))$ satisfying
    $\tau(\theta) = \sigma_\alpha^{-1}(\theta) = \sigma_y(\theta)$.
    Since $L/K$ normal, $\tau(L) = L$ and thus $\tau\big|_L \in \Gal(L/K)$ with
    $\tau\big|_L(\alpha_i) = \alpha_{\sigma^{-1}(i)}$, which implies that
    $\sigma^{-1} \in \Gal(L/K) \implies \sigma \in \Gal(L/K)$.
  \end{proof}
\end{prop}

\begin{theorem}
  Let $f(x)$ be monic, separable, in $\Zb[x]$. Assume $p \nmid D = \prod_{i < j} (\alpha_i - \alpha_j)^2$,
  then the Galois group of $\bar{f}(x)$ in $\Fb_p[x]$ is a subgroup of the Galois group of $f(x)$.

  \begin{proof}
    Since $f$ is separable, $D \neq 0$. The discriminant could be calculate
    by $D = (-1)^{n(n+1)/2} R(f, f')$ which only depends on the coefficients,
    so $\bar{D} \neq 0$ in $\Fb_p$ since $p \nmid D$. Thus $f$ separable.

    As above, let $F = F_1 F_2 \dotsm F_r$ in $\Zb[x, \bm{y}]$.
    Assume $f(x) = x^n + a_{n-1} x^{n-1} + \dots + a_0$, then $\bar{f}(x) =
    x^n + \bar{a}_{n-1}x^{n-1} + \dots + \bar{a}_0$.  Let $\alpha_1, \dots, \alpha_n$
    and $\beta_1, \dots, \beta_n$ be their roots, respectively.
    Define $\theta_p \triangleq y_1 \beta_1 + \dots + y_n \beta_n$.
    Since the coefficients of $F$ are symmetric polynomials of
    $\alpha_1, \dots, \alpha_n$, which only depends on the coefficients of $f$,
    and so is $F_p(x, y) = \prod_{\sigma \in S_n}(x - \sigma_y(\theta_p))$,
    we know that $F_p(x, y) = \bar{F}(x, y)$.

    Now $\bar{F} = \bar{F}_1 \bar{F}_2 \dotsm \bar{F}_r
    = (G_{1, 1} \dotsm G_{1, q_1})(G_{2, 1} \dotsm G_{2, q_2}) \dotsm (G_{r, 1}
    \dotsm G_{r, q_r})$

    The Galois group of $\bar{f}$ is
    \[ \Set{ \sigma \in S_n : \sigma_y G_{1, j} = G_{1, j}, \, \forall j }
      \subseteq \Set{ \sigma \in S_n : \sigma_y \bar{F}_1 = \bar{F}_1 } =
    \Set{\sigma \in S_n : \sigma_y F_1 = F_1} \]
    Where the equality holds because $\sigma_y \bar{F}_1 = \bar{F}_1 \iff
    (x - \sigma_y(\theta_p)) \mid \bar{F}_1 \iff
    (x - \sigma_y(\theta)) \mid F_1 \iff \sigma_y F_1 = F_1$. Thus the galois group
    of $\bar{f}$ is a subgroup of $f$.
  \end{proof}
\end{theorem}

\begin{fact} \hfill
  \begin{itemize}
    \item Every finite extension of $\Fb_p$ is cyclic, so the Galois group of
      $\bar{f}(x)$ in $\Fb_p[x]$ is cyclic.
    \item If $\bar{f}$ is irreducible, then the Galois group of $\bar{f}$ is
      transitive on its roots, thus the only possibililty is a cycle of length
      $n = \deg \bar{f}$ in $S_n$.
    \item If $\bar{f} = \bar{f}_1 \dotsm \bar{f}_r$, with each $\bar{f}_i$ irreducible.
      Let the Galois group be $\gen{\sigma}$, then $\sigma$ should be
      transitive on the roots of each $\bar{f}_i$. The only possibility of $\sigma$
      is a permutation composited by cycles of length
      $\deg \bar{f}_1, \dots, \deg \bar{f}_r$.
      That is, $\sigma = \cycle{\alpha_{1,1}, \dots, \alpha_{1, m_1}} \dotsm
      \cycle{\alpha_{r,1}, \dots, \alpha_{r, m_r}}$ where $m_i \triangleq
      \deg \bar{f}_i$.
  \end{itemize}
\end{fact}

%! TEX root=../main.tex
\subsection{Week 8}
\subsubsection{Fundamental theorem of finitely generated abelian groups}

\begin{theorem}[Structure theorem of finitely generated module over a PID]
  Let $R$ be a PID and $M$ be a finitely generated $R$-module.
  Then $M \cong \quot{R}{d_1 R} \oplus \dots \oplus \quot{R}{d_l R} \oplus R^s,
  d_i \in R$ with $d_i \Div d_{i+1} \quad \forall i = 1, \dots, l-1$
  for some $s \in \Zb^{\ge 0}$.

  \begin{proof}
    Let $M = \gen{x_1, \dots, x_n}_R$ and consider
    \[
      \arraycolsep=1pt
      \begin{array}{rcl}
        \varphi: & R^n & \onto M \\
                 & e_i & \to x_i
      \end{array}
    \]
    By 1st isom. thm., $\quot{R^n}{\ker \varphi} \cong M$.

    We know $\ker \varphi \cong R^m$ ($e_i' \mapsto f_i, e_i' \in R^m$)
    for some $m \le n$ and
    $\forall x \in \ker\varphi \quad \exists! x_1, \dots, x_m \in R$ s.t.
    $x = \sum_{i=1}^m x_i f_i$.

    Note that $\ker \varphi \subseteq R^n$. So we can write
    $f_i = \sum_{j=1}^n a_{ji}e_j \quad \forall i = 1,\dots, m$.
    Then $x = \sum x_i \sum a_{ji}e_j =
    \sum \left(\sum a_{ji}x_i\right) e_j$.

    $R$ is a PID $\implies \exists P \in \text{GL}_n(R), Q \in \text{GL}_m(R)$
    s.t.
    \[
      PAQ = \begin{pmatrix}
      d_1 \\
      & \ddots \\
      & & d_r \\
      & & & 0 \\
      & & & & \ddots
      \end{pmatrix}
      \quad \text{with} \quad
      d_i \Div d_{i+1} \quad \forall i = 1, \dots, r-1
    \]
    So consider $[w_i] = Q e_i$. Since $P, Q$ invertible, $R^n = \bigoplus R w_i, \ker \varphi = \bigoplus d_i R w_i$
    Hence
    \[ M \simeq R / ker \varphi = \bigoplus R w_i / \bigoplus d_i R w_i = \bigoplus R / d_i R \]
  \end{proof}

  $\arraycolsep=3pt
  \begin{array}{rcl}
    R & \onto & \quot{Rw_i}{Rd_i'w_i} \\
    1 & \to & \ob{w_i} \\
    r & \to & \ob{rw_i}
  \end{array}
  $
\end{theorem}

\begin{remark}
  If $R$ is commutative, then ``$R^n \cong R^m \implies n = m$.''
\end{remark}

\begin{theorem}
  Let $G$ be a finitely generated abelian group. Then
  Then $G \cong \quot{\Zb}{d_1 \Zb} \oplus \dots \oplus \quot{\Zb}{d_l \Zb}
  \oplus R^s, d_i \in \Zb$ with $d_i \Div d_{i+1} \quad
  \forall i = 1, \dots, l-1$ for some $s \in \Zb^{\ge 0}$.

  Since $G$ can be regarded as a f.g. $\Zb$-module and $\Zb$ is a PID,
  it follows from the main theorem.

  $\Tor(G) = \quot{\Zb}{d_1 \Zb} \oplus \dots \oplus \quot{\Zb}{d_l \Zb}
  \le G$ and $\quot{G}{\Tor(G)} \cong \Zb^s$ (free part of $G$).
\end{theorem}

\begin{fact}
  If $d = p_1^{m_1}p_2^{m_2}\dots p_s^{m_s}$, then
  $\quot{\Zb}{d\Zb} \cong \quot{\Zb}{p_1^{m_1}\Zb} \oplus
  \quot{\Zb}{p_2^{m_2}\Zb} \oplus \dots \oplus \quot{\Zb}{p_s^{m_s}\Zb}$.
\end{fact}

\begin{theorem}[Chinese Remainder theorem]
  Let $R$ be a commutative ring with $1$ and $I_1, \dots, I_n$ be ideals of $R$.
  Then
  \[
    \arraycolsep=3pt
    \begin{array}{rccl}
      \varphi: & R & \to & \quot{R}{I_1} \times \dots \times \quot{R}{I_n} \\
               & r & \mapsto & (\ob{r}, \dots, \ob{r})
    \end{array}
    \text{~is a ring homo.}
  \]
  and
  \begin{enumerate}[(1)]
    \item if $I_i, I_j$ are coprime $\forall i \ne j$, then
      $I_1I_2\dots I_n = I_1 \cap I_2 \cap \dots \cap I_n$.
    \item $\varphi$ is surjective $\iff$ $I_i, I_j$ are coprime
      $\forall i \ne j$.
    \item $\varphi$ is injective $\iff$ $I_1 \cap I_2 \cap \dots \cap I_n
      = \{ 0 \}$.
  \end{enumerate}
  So if $I_i, I_j$ are coprime $\forall i \ne j$, then
  \[
    \quot{R}{I_1I_2\dots I_n} \cong
    \quot{R}{I_1} \times \dots \times \quot{R}{I_n}.
  \]

  $I_i, I_j$ are coprime $\iff$ $I_i + I_j = R$.

  \begin{proof}
    we only need to prove (1), (2).

    \begin{enumerate}[(1)]
      \item By induction on $n$. $n = 2$, need $I_1\cap I_2 \subseteq I_1I_2$.
        Indeed, $I_1\cap I_2 = (I_1\cap I_2)R = (I_1\cap I_2)(I_1+I_2)
        \subseteq I_1I_2$.

        For $n > 2$, since $I_i + I_n = R \quad \forall i = 1,\dots, n-1$,
        $\exists x_i \in I_i, y_i \in I_n$ s.t. $x_i + y_i = 1 \quad
        \forall i = 1,\dots, n-1$.

        So $x_1x_2\dots x_{n-1} = (1-y_1)(1-y_2)\dots(1-y_{n-1}) = 1 - y,
        y \in I_n
        \implies I_1I_2\dots I_{n-1} + I_n = R$.

        Now, $I_1I_2\dots I_n = (I_1\dots I_{n-1})I_n =
        (I_1\dots I_{n-1})\cap I_n = I_1\cap \dots \cap I_n$.
      \item ``$\Rightarrow$'': WLOG, we may let $I_i = I_1, I_j = I_2$.
        We have $x \in R$ s.t.
        \[
          \varphi(x) = (\ob{1}, \ob{0}, \dots, \ob{0})
          \quad \text{i.e.~}
          \ob{x} = \ob{1} \text{~in~} \quot{R}{I_1}
        \]
        Write $x \equiv 1 \pmod {I_1}$.
        Since $1 - x \in I_1, x \in I_2$ and $(1 - x) + x = 1$, $I_1 + I_2 = R$.

        ``$\Leftarrow$'': $\forall y \in \text{RHS}$,
        $y = (\ob{r_1}, \dots, \ob{r_n})$.
        If we may find that $x_i \in R$ s.t.
        $\varphi(x_i) = (\ob{0}, \dots, \ob{1}, \ob{0}, \dots, \ob{0})$,
        then
        \[
          \varphi\left(\sum_{i=1}^n r_ix_i \right) = y
        \]

        It is enough to show, for example, $\exists x \in R$ s.t.
        $\varphi(x) = (\ob{1}, \ob{0}, \dots, \ob{0})$.

        Since $I_1 + I_i = R \quad \forall i = 2, \dots, n$,
        $\exists x_i \in I_1, y_i \in I_i$ s.t. $x_i + y_i = 1
        \forall i = 2, \dots, n$.

        So let $x = y_2\dots y_n = (1-x_2)\dots(1-x_n)$.
        We have $x \in I_2, \dots, I_n$ and $x \equiv 1 \pmod {I_1}$.
    \end{enumerate}
  \end{proof}
\end{theorem}

\begin{example}
  $\abs{G} = 72$ and $G$ is abelian:
  \[
    72 = 2 \times 36 = 3 \times 24 = 2 \times 2 \times 18
    = 6 \times 12 = 2 \times 6 \times 6
  \]
  Invariant factors

  Elementary divisors
\end{example}

\begin{definition}
  The exponent of $G$ with $\abs{G} < \infty$ is
  \[
    \Exp(G) \defeq \min \left\{
      m \in \Nb \middle| g^m = 1 \quad \forall g \in G
    \right\}
  \]
\end{definition}

\begin{exercise} \mbox{}
  \begin{enumerate}
    \item Let $G$ be abelian with $\abs{G} = n$. Show that if $d \Div n$, then
      $\exists H \le G$ s.t. $\abs{H} = d$.
    \item If $n=540, d=90$, then construct all possible $G$ and
      corresponding $H$.
  \end{enumerate}
\end{exercise}

\begin{exercise}
  Let $G$ be abelian with $\abs{G} < \infty$. Show that $G$ is cyclic
  $\iff \Exp(G) = \abs{G}$.
\end{exercise}

\begin{exercise}
  Let $f_i(x) \in \Zb[x], i = 1, \dots, k$ with $\deg f_i = d$ and
  $p_1, \dots, p_k$ be distinct primes.
  Show that $\exists f(x) \in \Zb[x]$ with $\deg f = d$ s.t.
  $\ob{f}(x) = \ob{f_i}(x)$ in 
  $\quot{\Zb}{p_i\Zb}[x] \quad \forall i = 1, \dots k$.

  $f(x) = a_d x^d + \dots + a_0,
  \ob{f}(x) = \ob{a_d} x^d + \dots + \ob{a_0}$
\end{exercise}

\subsubsection{Sylow theorems}

\begin{definition}
  Let $\abs{G} = p^\alpha r$ with $p \nmid r$.
  \begin{enumerate}
    \item If $H \le G$ with $\abs{H} = p^\alpha$, then we call $H$ a Sylow
      $p$-subgroup of $G$.
    \item $\text{Syl}_p(G) =$ the set of all Sylow $p$-subgroups of $G$.
    \item $n_p = \abs{\text{Syl}_p(G)}$.
  \end{enumerate}
\end{definition}

\begin{lemma}[Key lemma]
  Let $P \in \text{Syl}_p(G)$ and $Q$ be a $p$-subgroup of $G$. Then
  $Q \cap N_G(P) = Q \cap P$.

  \begin{proof}
    By Lagrange theorem, $H = Q \cap N_G(P)$ is also a $p$-subgroup of
    $N_G(P)$ since $\abs{H} \Div \abs{Q}$.

    Since $\begin{cases}
      P \lhd N_G(P) \\
      H \le N_G(P)
    \end{cases} \implies HP \le N_G(P)$, we have
    \[
      \abs{HP} = \frac{\abs{H}\abs{P}}{\abs{H\cap P}} = p^{\alpha+k-s}
    \]
    where $\abs{H\cap P} = p^s, s \le k$. Then
    $p^{\alpha+k-s} \Div \abs*{N_G(P)} \Div \abs{G} = p^\alpha r$.

    So $k = s \implies H = H \cap P \implies H \le P \cap Q$.
  \end{proof}
\end{lemma}

\begin{theorem}[Sylow \RNum{1}]
  $\forall 0 \le k \le \alpha$, $\exists H \le G$ s.t. $\abs{H} = p^k$.
  In particular, $\text{Syl}_p(G) \ne \phi$.

  \begin{proof}
    By induction on $\abs{G}$. If $\abs{G} = 1$, then $k = 0$, $H = \{1\}$.

    Assume $\abs{G} > 1, k \ge 1, \alpha \ge 1$.

    \begin{description}
      \item[case 1:] $p \Div \abs{Z_G}$. By Cauchy theorem,
        $\exists a \in Z_G$ with $\ord(a) = p$.
        Then $\gen{a} \lhd G$ and $\abs*{\quot{G}{\gen{a}}} = p^{\alpha-1} r
        \le \abs{G}$.
        If $k=1$, then $H = \gen{a}$.
        Otherwise, we may assume that $1\le k-1\le \alpha-1$. By induction
        hypothesis, $\exists H' = \quot{G}{\gen{a}}$ s.t. $\abs{H'} = p^{k-1}$.
        By 3rd isom. thm., we can write $H' = \quot{H}{\gen{a}}$ and thus
        $\abs{H} = p^k$.
      \item[case 2:] $p \nDiv \abs{Z_G}$. By the class equation,
        $\abs{G} = \abs{Z_G} + \sum_{i=1}^m \frac{\abs{G}}{\abs{Z_G(a_i)}},
        a_i \in Z_G$.

        In this cases, $\exists a_j$ s.t.
        $p \nDiv \frac{\abs{G}}{\abs{Z_G(a_j)}} \implies
        p^\alpha \Div \abs{Z_G(a_j)}$. And $Z_G(a_j) \lneq G$ since
        $a_j \not\in Z_G$.
        By induction hypothesis, $\exists H \le Z_G(a_j) \le G$ s.t.
        $\abs{H} = p^k$. \qedhere
    \end{description}
  \end{proof}
\end{theorem}

\begin{theorem}[Sylow \RNum{2}]
  Let $P \in \text{Syl}_p(G)$ and $Q$ be a $p$-subgroup of $G$. Then
  $\exists a \in G$ s.t. $Q \le aPa^{-1}$.
  In particular, $\forall P_1, P_2 \in \text{Syl}_p(G), \exists a \in G$
  s.t. $P_2 = aP_1a^{-1}$.
  \begin{proof}
    Let $X = \{\, \text{left cosets of $P$} \,\}$ and consider
    $\arraycolsep=1pt \begin{array}{rcl}
      Q \times X & \to & X \\
      (a, xP) & \mapsto & axP
    \end{array}$.

    Observe that $xP \in \Fix Q \iff axP = xP \quad \forall a \in Q \iff
    x^{-1}axP = P \quad \forall a \in Q \iff
    x^{-1}ax \in P \quad \forall a \in Q \iff
    a \in xPx^{-1} \quad \forall a \in Q$.

    We know $\abs{\Fix Q} \equiv \abs{X} \pmod p$ and $p \nmid r \implies$
    $\abs{\Fix Q} \ne 0 \iff \exists a \in G, Q \le aPa^{-1}$.

    In particular, $\begin{cases}
      P_2 \le aP_1a^{-1} \\
      \abs{P_2} = \abs{aP_1a^{-1}}
    \end{cases} \implies P_2 = aP_1a^{-1}$.
  \end{proof}
\end{theorem}

\begin{theorem}[Sylow \RNum{3}]
  $n_p \equiv 1 \pmod p$ and $n_p \mid r$.
  \begin{proof}
    \begin{itemize}
      \item Consider $\arraycolsep=1pt \begin{array}{ccrcl}
          &P \times &\text{Syl}_p(G) & \to & \text{Syl}_p(G) \\
          (&a, &Q) & \mapsto & aQa^{-1}
        \end{array}$ where $P \in \text{Syl}_p(G)$.

        $P' \in \Fix P \iff aP'a^{-1} = P' \quad \forall a \in P
        \iff P \le N_G(P') \cap P = P' \cap P \iff P' = P$.

        So $\Fix P = \{ P \} \implies n_p \equiv \abs{\Fix P} = 1 \pmod p$.

      \item Consider $\arraycolsep=1pt \begin{array}{ccrcl}
          &G \times &\text{Syl}_p(G) & \to & \text{Syl}_p(G) \\
          (&a, &Q) & \mapsto & aQa^{-1}
        \end{array} \implies$ There is only one orbit $\text{Syl}_p(G)$.
        
        We know $\abs{\text{Syl}_p(G)} = \frac{\abs{G}}{\abs{G_Q}}$
        and $G_Q = N_G(Q)$. Then $n_p = \frac{\abs{G}}{\abs{G_Q}} \Div \abs{G}$.
        So $n_p \Div p^\alpha r \implies n_p \Div r$.
    \end{itemize}
  \end{proof}
\end{theorem}

\begin{prop}
  Let $\abs{G} = pq$ where $p, q$ are primes with $\begin{cases}
    p < q \\
    q \not\equiv 1 \pmod p.
  \end{cases}$
  Then $G \cong C_{pq}$.
  \begin{proof}
    $n_p = 1+kp \mid q \implies n_p = 1$ i.e.
    $H \in \text{Syl}_p(G) \implies H \lhd G$.

    $n_q = 1+kq \mid p \implies n_q = 1$ i.e.
    $K \in \text{Syl}_q(G) \implies K \lhd G$.

    Since $\gcd(p, q) = 1$, $H \cap K = 1$.
    Hence $G = H \times K \cong C_p \times C_q \cong C_{pq}$.
  \end{proof}
\end{prop}

\begin{example}
  Consider $\abs{G} = 255 = 3 \times 5 \times 17$.
  \begin{enumerate}
    \item 找兩個 normal subgroup (17, 5 or 3)
    \item quot 掉後發現剩下的是 abelian $\leadsto$ $[G, G]$ 在裡面
    \item $[G, G] = 1$
    \item 唱 f.g. xxx thm. 得到 $G \cong \Zb_3 \times \Zb_5 \times \Zb_{17}$.
    \item 中國剩飯定理 $G \cong C_{255}$.
  \end{enumerate}
\end{example}

\begin{exercise}
  If $\abs{G} = 7 \times 11 \times 19$, then $G$ is abelian.
\end{exercise}

\begin{example}
  No group $G$ of order $48 = 2^4 \times 3$ is simple.
  \begin{enumerate}
    \item $n_2 = 1 + 2k \Div 3 \leadsto n_2 = 1 \text{~or~} 3$.
    \item $n_2 = 1$ then OK.
    \item Assume $n_2 = 3$. Let $P \in \text{Syl}_2(G),
      X = \{\, \text{left cosets of $P$} \,\}$ ($\abs{X} = 3$).
    \item Consider $\arraycolsep=1pt \begin{array}{ccrcl}
        &G \times &X & \to & X \\
        (&a, &xP) & \mapsto & axP
      \end{array} \leadsto \varphi: G \to S_3$.
    \item 考慮 $\ker\varphi$.
  \end{enumerate}
\end{example}

\begin{exercise}
  No group $G$ of order $36$ is simple.
\end{exercise}

\begin{exercise}
  No group $G$ of order $30$ is simple.
\end{exercise}

\begin{exercise}
  Let $\abs{G} = 385$. Show that $\exists P \in \text{Syl}_7(G)$ s.t.
  $P \le Z_G$.
\end{exercise}

%! TEX root=../main.tex
\section{Commutative Algebra}

\subsection{ED, PID and UFD (week 9)}

We shall consider $R$ to be a integral domain below.
\begin{definition}
  A function $N: R \to \Nb$ with $N(0) = 0$ is called a norm on $R$.
\end{definition}

\begin{definition}
  $R$ is called a Euclidean domain if exists a norm $N$ on $R$
  satisfying
  \[ \forall a, b \in R, \ \exists q, r \in R \text{ s.t. }
  a = qb + r \text{ with } r = 0 \text{ or } N(r) < N(b) \]
\end{definition}

\begin{example} \hfill
  \begin{itemize}
    \item $\Zb$ is a ED with $N(n) = \abs{n}$.
    \item $K[x]$ is a ED with $N(f) = \deg f, \, \forall f \in K[x]$.
  \end{itemize}
\end{example}

\begin{definition}
  $A_d$ is defined to be the ring of integers in the quadratic field $\Qb(\sqrt{d})$
  with $d \neq 1$ and $d$ is square-free. That is,
  \[ A_d \triangleq \Set{ \alpha \in \Qb(\sqrt{d}) \mid \alpha \text{ is integral over } \Zb} \]
\end{definition}

\begin{theorem} \hfill
  \begin{itemize}
    \item If $d \equiv 1 \pmod{4}$, then
      \[ A_d = \big\{ a + b \frac{1 + \sqrt{d}}{2} : a, b \in \Zb \big\} \]
    \item Else, $d \equiv 2, 3 \pmod{4}$, then
      \[ A_d = \big\{ a + b \sqrt{d} : a, b \in \Zb \big\} \]
  \end{itemize}
  \begin{proof}
    %TODO
  \end{proof}
\end{theorem}

\begin{theorem}
  $A_d$ is a ED if $d = 2, 3, 5, -1, -2, -3, -7, -11$. Hence $A_d$ is also PID and UFD.
  \begin{proof}
    %TODO
  \end{proof}
\end{theorem}

\begin{example}
  $A_{-5}$ is not a ED.

  \begin{proof}
    Consider $6 = 2 \cdot 3 = (1 + \sqrt{-5})(1 - \sqrt{-5})$.
    Notice that $1 + \sqrt{-5}$ is irreducible, since if $1 + \sqrt{-5} = \alpha \beta$,
    then $6 = N(1 + \sqrt{-5}) = N(\alpha) N(\beta)$. But there is
    $a^2 + 5b^2 = 2 \text{ or } 3$ has no integer solution.
    Also $1 + \sqrt{-5} \nmid 2, 3$. Since if $(1 + \sqrt{-5}) \alpha = 2$,
    then $N(1 + \sqrt{-5}) N(\alpha) = N(2)$, but $N(1 + \sqrt{-5}) = 6$.
  \end{proof}
\end{example}

\subsubsection{$A_{-1}$ and $A_{-3}$}
First, $\alpha$ is a unit $\iff$ $N(\alpha) = 1$.
so we have:
\begin{itemize}
  \item $A_{-1}$: $\pm 1, \pm \mathrm{i}$.
  \item $A_{-3}$: $\pm 1, \pm \omega, \pm \omega^2$.
\end{itemize}

If $\alpha$ is a prime in $A_{-1}$ or $A_{-3}$, then $N(\alpha) = p \text{ or } p^2$ for some prime integer $p$.

Let $N(\alpha)  = \alpha \bar\alpha = p_1 \dotsm p_n$ in $\Zb$

\begin{definition}
  If $p$ is add and $a \not\equiv 0 \pmod{p}$, then
  \begin{itemize}
    \item If $x^2 \equiv a \pmod{p}$ is solvable, then define $\left( \frac{a}{p} \right) = 1$.
    \item Else $x^2 \equiv a \pmod{p}$ is not solvable and define $\left( \frac{a}{p} \right) = -1$.
  \end{itemize}
\end{definition}

\begin{prop} \hfill
  \begin{itemize}
    \item $a \equiv b \pmod{p} \implies \left( \frac{a}{p} \right) = \left( \frac{b}{p} \right)$.
  \end{itemize}
  % TODO missing something...
\end{prop}
% TODO missing something...

\subsection{Primary decomposition}
\begin{definition} \hfill
  \begin{itemize}
    \item The radical of an ideal $I$ is defined by $\sqrt{I} =
      \Set{ a \in R \mid a^n \in I \text{ for some } n \in \Nb}$.
    \item $I$ is radical if $\sqrt{I} = I$.
  \end{itemize}
\end{definition}

\begin{definition}
  The {\bf nilradical}\index{nilradical} is defined as $\sqrt{\gen{0}} \triangleq
  \Set{ a \in R \mid a^n = 0 \text{ for some } n \in \Nb}$.
  Elements in it are called nilpotent.
\end{definition}

\begin{prop}
  $\sqrt{ \gen{0} } = \bigcap\limits_{P \in \Spec R} P$, where $\Spec R$ is the
  set of prime ideals in $R$.

  \begin{proof}
    ``$\subset$'': Notice that $a^n = 0 \in P$ for any prime ideal $P$. By the definition of
    prime ideal, either $a \in P$ or $a^{n-1} \in P$. No matter which, eventually we would get
    $a \in P$.

    ``$\supset$'':
    Let $\Sc \triangleq \Set{ I : \text{ ideal in } R \given a^n \notin I, \, \forall n \in \Nb}$.
    By the routine argument of Zorn's lemma, exists maximal element $Q$ in $\Sc$.
    We claim that $\Sc$ is a prime ideal.

    For each $x, y \notin Q$, we have $Q + Rx \supsetneq Q$ and $Q + Ry \supsetneq Q$.
    By the maximality of $Q$, these two ideals are not in $\Sc$.
    So exists $n, m$ such that $a^n \in Q + Rx,\, a^m \in Q + Ry$ which implies
    $a^{n+m} \in Q + Rxy$, so $Q + Rxy \notin \Sc$, thus $xy \notin Q$,
    hence $Q$ is prime.
  \end{proof}
\end{prop}

\begin{coro} \label{coro:equation-of-sqrt-ideal}
  \[ \sqrt{I} = \bigcap_{\substack{P \supset I \\ P \in \Spec R}} P \]

  \begin{proof}
    Notice that $\Spec \quot{R}{I} = \Set{P \in \Spec R \mid R \subset I}$.
    By the proposition above,
    \[
      \sqrt{\langle \bar0 \rangle} = \bigcap_{\bar{P} \in \Spec \quot{R}{I}} \bar{P}
      \quad \implies \quad \sqrt{I} = \bigcap_{\substack{P \supset I \\ P \in \Spec R}} P
      \qedhere
    \]
    \end{proof}
\end{coro}

\begin{definition}
  An ideal $q$ of $R$ is called primary if $q \neq R$ and ``$xy \in q$ and $x \notin q$''
  implies $y^n \in q$ for some $n \in \Nb$.
\end{definition}

\begin{prop} \hfill
  \begin{itemize}
    \item $\text{prime} \implies \text{primary}$.
    \item $\sqrt\text{primary} \implies \text{prime}$. Also, if $q$ is primary, then $p = \sqrt{q}$
      is the smallest prime ideal containing $q$, we say $q$ is $p$-primary.
  \end{itemize}

  \begin{proof}
    The first one is obvious.

    If $q$ is primary and $\sqrt{q} = p$. For any $xy \in p$ and $x \notin p$,
    there exists $n$ so that $x^n y^n \in q$, and for this $n$, $x^n \notin q$.
    Thus $(y^n)^m \in q$ for some $m$, hence $y \in p$. We conclude that $p$ is a prime ideal.

    Finally, by corollary~\ref{coro:equation-of-sqrt-ideal},
    \[ p = \sqrt{q} = \bigcap_{\substack{P \supset q \\ P \in \Spec R}} P \subset P,
    \quad \forall P \text{ prime }, \]
    thus $p$ is indeed the smallest.
  \end{proof}
\end{prop}

\begin{example}
  The primary ideals in $\Zb$ are $\langle 0 \rangle$ and $\langle p^m \rangle$
  where $p$ is a prime.

  \begin{proof}
    If $q = \langle a \rangle$ is primary, then $\sqrt{q} = \langle p \rangle$ is
    prime, and $p^n \in \langle a \rangle$. So $ab = p^n$ which implies $a = p^m$
    for some $m$.
  \end{proof}
\end{example}

\begin{definition}
  An ideal $I$ is said to be {\bf irreducible} \index{Ideal!irreducible}
  if $I = q_1 \cap q_2 \implies I = q_1 \lor I = q_2$.
\end{definition}

\begin{definition}
  Define $(I: x) = \Set{ a \in R \mid ax \in I}$.
\end{definition}

\begin{theorem} \label{thm:noeth-irr-ideal-is-primary}
  In a Noetherian ring $R$, every irreducible ideal $I$ is primary.

  \begin{proof}
    Let $xy \in I$ and $x \notin I$. Consider $(I : y) \subseteq (I: y^2) \subseteq \dotsm$.
    Since $R$ is Noetherian, exists $n$ such that $(I: y^n) = (I: y^m)$ for any $m \geq n$.

    We claim that $I = (I + Ry^n) \cap (I + Rx)$.
    \begin{itemize}
      \item ``$\subset$'': Obvious.
      \item ``$\supset$'': For any $b \in (I + ry^n) \cap (I + Rx)$,
        write $b = a_1 + r_1 y^{n} = a_2 + r_2 x$. Then
        $r_1 y^{n+1} = a_2 y - a_1 y + r_2 x y \in I$ since $a_1, a_2, xy \in I$.
        So $r_1 \in (I: y^{n+1}) = (I: y_n) \implies r_1 y^n \in I$.
        Thus $b = a_1 + r_1 y^n \in I$.
    \end{itemize}

    Now by the fact that $I$ is irreducible and $I \neq I + Rx$ since $x \notin I$,
    thus $I = I + Ry^n \implies y^n \in I$.
  \end{proof}
\end{theorem}

\begin{theorem} \label{thm:noeth-ideal-is-finite-intersection}
  In a Noetherian ring $R$, every ideal is a finite intersection of irreducible ideals.

  \begin{proof}
    If not, let $\Ic \triangleq \Set{I: \text{ ideal in } R \mid I \text{ is not a finite intersection
        of irreducible ideals }}$ and $\Ic$ is not an empty set.
    Since $R$ is Noetherian, the set has a maximal element $I_0$. Then $I_0$ is not
    irreducible (or else it is an intersection of itself, which is irreducible).
    Write $I_0 = I_1 \cap I_2$, with $I_1, I_2 \neq I_0$. Then $I_1, I_2 \notin \Ic$,
    so these two ideals could be written as a finite intersection of irreducible ideals,
    implying that $I_0$ could also be written as a finite intersection of irreducible ideals,
    which is an contradiction.
  \end{proof}
\end{theorem}

\begin{prop} \label{prop:primary-divide-by-element}
  Let $q$ be a $p$-primary ideal and $x \in R$.
  \begin{enumerate}
    \item If $x \in q$, then $(q: x) = R$.
      \begin{proof}
        In this case $1 \in (q: x)$, thus $(q: x) = R$.
      \end{proof}
    \item If $x \notin q$, then $(q: x)$ is $p$-primary.
      \begin{proof}
        For any $y \in (q: x)$, $xy \in q$ but $x \notin q$, thus $y^n \in q \implies y \in p$.
        Hence
        \[ q \subset (q: x) \subset p \implies p = \sqrt{q} \subset \sqrt{(q: x)} \subset \sqrt{p} = p \]
        and thus $(q: x)$ is $p$-primary.

        For any $y, z$ with $yz \in (q: x)$ but $y \notin (q: x)$, which is equivalent
        to $xyz \in q$ but $xy \notin q$. Since $q$ primary, $z^n \in q \subset (q: x)$.
      \end{proof}
    \item If $x \notin p$, then $(q: x) = q$.
      \begin{proof}
        \[
          \left\{ \begin{array}{l}
            y \in (q: x) \\
            x \notin p \\
          \end{array} \right. \implies
          \left\{ \begin{array}{l}
            xy \in (q: x) \\
            x^n \notin q, \ \forall n \in \Nb \\
          \end{array} \right. \implies y \in q
          \qedhere
        \]
      \end{proof}
  \end{enumerate}
\end{prop}

\begin{prop}  \label{prop:intersection-of-primary-is-primary}
  If each $q_i$ are $p$-primary, then $q \triangleq \cap_{i = 1}^n q_i$ is $p$-primary.

  \begin{proof}
    We check that $\sqrt{q} = \bigcap_{i = 1}^n \sqrt{q_i} = \bigcap_{i = 1}^n p = p$.

    Also, if $xy \in q$ with $x \notin q$, then $x \notin q_k$ for some $k$.
    But $xy \in q_k$, thus $y^n \in q_k$. Since $\sqrt{q} = q_k$, $(y^n)^{m'} = y^m \in p \subset q$,
    thus $q$ is $p$-primary.
  \end{proof}
\end{prop}

\begin{definition}
  A {\bf primary decomposition} of $I = q_1 \cap \dots \cap q_n$ is {\bf minimal} if $\sqrt{q_1}, \dots, \sqrt{q_n}$
  are distinct and $q_i \not\supseteq \bigcap_{j \neq i} q_j$.
\end{definition}

A minimal primary decomposition of an ideal always exists in Noetherian ring since by
theorem~\ref{thm:noeth-ideal-is-finite-intersection}, the ideal could be written
as a finite intersection of irreducible ideals, and then by theorem~\ref{thm:noeth-irr-ideal-is-primary},
these ideals are primary. Now If $\sqrt{q_i} = \sqrt{q_j}$ happen in these ideal,
we could remove these two ideals and add $q' = \sqrt{q_i} \cap \sqrt{q_j}$.
By proposition~\ref{prop:intersection-of-primary-is-primary}, $q'$ is also primary.
And if $q_i \subseteq \bigcap_{j \neq i} q_j$, we could simply remove $q_i$.

\medskip

\begin{theorem}[Uniqueness of primary decomposition]
  Let $I = \cap_{i = 1}^n q_i$ be a minimal decomposition of $I$.
  If $p_i = \sqrt{q_i}, \, \forall i$, then we have
  \[ \Set{p_i} = \Set[\Big ]{ \sqrt{(I: x)} \given x \in R \land \sqrt{(I: x)} \in \Spec R } \]
  which is independent of the decomposition.

  \begin{proof}
    ``$\supset$'': Let $x \in R \setminus I$, then $(I: x) = \big( \bigcap_{i=1}^n q_i : x \big)
    = \bigcap_{i = 1}^n (q_i: x)$. By proposition~\ref{prop:primary-divide-by-element},
    we have $\sqrt{(I: x)} = \bigcap \sqrt{(q_i: x)} = \bigcap_{x \notin q_i} p_i$.

    Now, we have the following observation. ``If $p \in \Spec R$ with $p = \bigcap_{i=1}^n J_i$,
    then $p = J_j$ for some $j$.'' If not, then $J_i \not\subset p$ for all $i$,
    so we could pick $x_i \in J_i \setminus p$.
    But then $x_1 x_2 \dotsm x_n \in \cap J_i \in p$ since $J_i$ are ideals,
    which leads to a contradiction since $p$ is prime.

    So if $\sqrt{(I: x)}$ is a prime, then it is equal to some $p_i$.

    ``$\subset$'': By assumption, $q_i \not\subseteq \bigcap_{j \neq i} q_j$ for each $i$,
    thus we could pick $x \in \bigcap_{j \neq i} q_j \setminus q_i$,
    then $\sqrt{(I: x)} = \bigcap_j \sqrt{(q_j: x)} = \sqrt{(q_i: x)} = p_i$.
  \end{proof}
\end{theorem}

\begin{definition}
  If $\Set{p_i}$ is the unique prime ideals from the minimal primary decomposition of $I$.
  \begin{itemize}
    \item $\Set{p_i}$ is said to be associated with $I$ or to belong to $I$.
    \item The minimal elements in $\Set{p_i}$ are called isolated primes.
    \item The other are called embedded primes.
  \end{itemize}
\end{definition}

\begin{example}
  Let $R = k[x, y]$ and $I = \gen{ x^2, xy }$. If $P_1 = \gen{x},
  P_2 = \gen{ x, y }$, then $I = P_1 \cap P_2^2$.
  $P_1$ is isolated, while $P_2$ is embedded.
\end{example}

%! TEX root=../main.tex
\subsection{The equivalence of algebra and geometry (week 10)}

In the following, $k$ will be an algebraically closed field.
\begin{definition}
  The category of affine algebraic sets $\Gc$ and its objects and morphisms are defined as following:

  \begin{description}[leftmargin=0cm]
    \item[objects:] The objects are affine algebraic sets in $k^n$.

    An {\bf affine algebraic set} is the common zero set of $\{ F_i \}_{i \in \Lambda} \subset k[x_1, \dots, x_n]$
    in $k^n$.
    We denote it by $V = \Vc(\Set{F_i}_{i \in \Lambda}) \subset k^n$.
    (In fact, $I = \langle F_i : i \in \Lambda \rangle$ is Noetherian, so
    $I = \langle F_1, \dots, F_n \rangle$ and $V = \Vc(I)$.)

    \item[morphisms:] The morphisms are the polynomial map from $k^n$ to $k^m$.

      A {\bf polynomial map} is a mapping as following:
      \[
        \begin{tikzcd}[column sep=1cm,row sep=0ex]
            k^n \arrow[r]& k^m \\
           \alpha \arrow[r, mapsto] & (F_1(\alpha), \dots, F_m(\alpha))
        \end{tikzcd}
      \]
      where each $F_i$ is a polynomial in $K[x_1, \dots, x_n]$.

      Given two affine algebraic sets $V \subset k^n$ and $W \subset k^m$, if a map $F: V \to W$ is
      the restriction of a polynomial map from $k^n$ to $k^m$, then $F$ is a morphism from $V$ to $W$.

      Moreover, if $F: V \to W$ and $G : W \to V$ satisfy $F \circ G = \Id$ and $G \circ F = \Id$,
      then we say $V \cong W$.
  \end{description}
\end{definition}

\begin{definition}
  The category of finitely generated reduced $k$-algebra $\Ac$
  and its objects and morphisms are defined as following:

  \begin{description}
    \item[objects:] The objects are the reduced finitely generated $k$-algebra $R$.

    A finitely generated $k$-algebra $R$ is reduced if $R$ has no non-zero nilpotent elements.

    \item[morphisms:] The morphisms are the $k$-algebra homomorphisms.
  \end{description}
\end{definition}

\begin{example}
  It is easy to see that $\Vc(0) = k^n$ and $\Vc(1) = \varnothing$.
\end{example}

\subsubsection{One-one correspondence between affine algebraic sets and radical ideals}

\begin{definition}
  Define $\Ic(V) = \Set{ f \in k[x_1, \dots, x_n] \mid f(\alpha) = 0, \, \forall \alpha \in V}$.
\end{definition}

The one-one correspondence is given by
  \[
    \begin{tikzcd}[column sep=0.8cm,row sep=0ex]
      \Set{ \text{affine algebraic sets in } \Ab^n_k }
      \arrow[r, leftrightarrow] & \Set{ \text{ radical ideals in } k[x_1, \dots, x_n]} \\
      V \arrow[r, mapsto] & \Ic(V) \\
      \Vc(I) \arrow[r, mapsfrom] & I
    \end{tikzcd}
  \]

\begin{prop} \hfill
\begin{itemize}
  \item $\sqrt{\Ic(V)} = \Ic(V)$.
    \begin{proof}
      For all $f^n \in \Ic(V)$,
      $f^n(\alpha) = 0, \forall \alpha \in V \implies f(\alpha) = 0, \forall \alpha \in V$.
      Thus $f \in \Ic(V)$.
    \end{proof}
  \item If $V$ is an affine set, then $\Vc(\Ic(V)) = V$.
    \begin{proof}
      ``$\supset$'': $\forall \alpha \in V,\, f \in \Ic(V)$, $f(\alpha) = 0 \implies \alpha \in \Vc(\Ic(V))$.

      ``$\subset$'': Since $V$ is an affine set, $V = \Vc(I)$, then $I \subset \Ic(V)$,
      so $\Vc(\Ic(V)) \subset \Vc(I) = V$.
    \end{proof}
\end{itemize}
\end{prop}

\begin{lemma} \label{lemma:module-finite-ring-finite}
  Given $T/S/R$, a tower of rings.
  If $R$ is Noetherian, $T/S$ is module finite and $T/R$ is ring finite, then
  $S/R$ is ring finite.

  \begin{proof}
    Let $T = R[a_1, \dots, a_n] = S w_1 + \dots + S w_m$.
    Then $a_i = \sum r_{i, j} w_j$ for some $r_{i, j}$
    and $w_i w_j = \sum t_{i, j, k} w_k$ for some $t_{i, j, k}$.

    Let $S' = R\big[ \Set{r_{i, j}}, \Set{t_{i, j, k}} \big] \subseteq S$, which
    is Noetherian by the Hilbert basis theorem
    ($R$ Notherian $\implies$ $R[x]$ Notherian).
    Thus $T = S' \omega_1 + \dots + S' \omega_m$ is a Noetherian $S'$-module
    by the fact that finitely generated module over a Noetherian ring is a
    Noetherian module.

    Since $S \subset T$, $S$ is a finitely generated $S'$ submodule,
    so
    \[
      S = S' v_1 + \dots + S' v_r = R\big[ \Set{r_{i, k}}, \Set{t_{i, j, k}},
      \Set{v_i} \big].
      \qedhere
    \]
  \end{proof}
\end{lemma}

\begin{lemma} \label{lemma:transcendental-is-not-ring-finite}
    If $S = k(z_1, \dots, z_p), \, p > 0$ with each $z_i$ transcendental,
    then $S/k$ is not ring finite.

  \begin{proof}
    If not, say $S = k[f_1, \dots, f_n]$ with $f_i = g_i / h_i$, $g_i, h_i \in k[z_1, \dots, z_p]$.
    Then for any irreducible polynomial $p$ such that $p \nmid h_i$ for each $h_i$
    (This polynomial exists since for each $h_i$ there are only finite degree $1$ factors).
    Then $1/p \notin k[f_1, \dots, f_n]$ by checking the divisibility of the denominator
    under addition and multiplication, which leads to a contradiction.
  \end{proof}
\end{lemma}

\begin{lemma} \label{lemma:ring-finite-is-alg}
  If $A/k$ is an extension of fields and ring finite, then $A/k$ is algebraic.

  \begin{proof}
    If $A/k$ is transcendental and let $\Set{z_1, \dots, z_t}$ be a transcendental base.
    Then $A/k(z_1, \dots, z_t)$ is algebraic, thus module finite
    (note that $A/k$ is ring finite).
    By lemma~\ref{lemma:module-finite-ring-finite}, $k(z_1, \dots, z_t)$ is ring finite,
    which contradicts with lemma~\ref{lemma:transcendental-is-not-ring-finite}.
  \end{proof}
\end{lemma}

\begin{theorem}[Weak form of Hilbert Nullstellensatz] \label{thm:weak-hilbert-null}
  \[ I \subsetneq k[x_1, \dots, x_n] \implies \Vc(I) \neq \varnothing \]

  \begin{proof}
    Since $I$ proper, by lemma~\ref{lemma:max-ideal-exists},
    there exists a maximal ideal $M$ such that $I \subseteq M$.
    Consider $K \triangleq k[x_1, \dots, x_n] / M = k[\bar{x}_1, \dots, \bar{x}_n]$.
    By proposition~\ref{prop:max-prime-to-field-int-domain}, $K$ is a field,
    and by lemma~\ref{lemma:ring-finite-is-alg}, $K/k$ is algebraic. Since $k$
    is already algebraically closed, $K = k$ and hence each $\bar{x}_i \in k$.
    Let $\alpha \triangleq (\bar{x}_1, \dots, \bar{x}_n) \in A_k^n$, then
    for any $f \in M$, $f(\alpha) = f(\bar{x}_1, \dots, \bar{x}_n) = \bar{f} = 0$,
    thus $\alpha \in \Vc(M) \subseteq \Vc(I)$.
  \end{proof}
\end{theorem}

\begin{theorem}[Strong form of Hilbert Nullstellensatz] \label{thm:strong-hilbert-null}
  $\Ic(\Vc(I)) = \sqrt{I}$

  \begin{proof}
    ``$\supset$'': $f \in \sqrt{I} \implies f^n \in I$, then
    $f^n(\alpha) = 0, \forall \alpha \in \Vc(I)
    \implies f(\alpha) = 0, \forall \alpha \in \Vc(I)$, thus $f \in \Ic(\Vc(I))$.

    ``$\subset$'': If $\Ic(\Vc(I)) = 0$, then $I \subseteq \sqrt{I} \subseteq \Ic(\Vc(I)) = 0$,
    thus $I = 0$.

    Otherwise, exists $0 \neq f \in \Ic(\Vc(I))$, Let $J = \langle I, ft-1 \rangle \subset k[x_1, \dots, x_n, t]$.
    If $(a_1, \dots, a_n, t_0)$ is a zero of $J$, then $ft-1 \in J \implies -1 = f(a_1, \dots, a_n) t_0 - 1 =
    0$, which is a contradiction, so by theorem~\ref{thm:weak-hilbert-null}, $J = k[x_1, \dots, x_n, t]$.

    Write $1 = \sum h_i f_i + s (ft-1)$, where each $f_i \in I$ and
    $h_i, s \in k[x_1, \dots, x_n, t]$.
    This is a equation of variables, so if we set $t = 1/f$, the equation still holds.
    Now each $h_i$ would be the form $\sum p_i / f^{k_i}$, so we could multiply each
    side by a suitable $f^\rho$ and get $f^\rho = \sum c_i f_i$ with each $c_i \in k[x_1, \dots, x_n]$.
    This implies $f^\rho \in I$, thus $f \in \sqrt{I}$.
  \end{proof}
\end{theorem}

\begin{definition}
  Let $V \in \Gc$, the coordinate ring of $V$ is $k[V] \triangleq k[x_1, \dots, x_n] / \Ic(V)$.
\end{definition}

\subsubsection{Equivalence of $\Gc$ and $\Ac$}
We define a functor $F$ from $\Gc$ to $\Ac$ by
  \[
    \begin{tikzcd}[column sep=0.8cm,row sep=0ex]
      F: &[-0.7cm] \Gc \arrow[r] & \Ac \\
      & V \arrow[r, mapsto] & k[V]
    \end{tikzcd}
  \]
  And For a polynomial map $f : V \to W$, define
  \[
    \begin{tikzcd}[column sep=0.8cm,row sep=0ex]
      F(f) = f^*: &[-0.7cm] k[W] \arrow[r] & k[V] \\
      & g \arrow[r, mapsto] & g \circ f
    \end{tikzcd}
  \]

  Conversely, define a functor $G$ by

  \[
    \begin{tikzcd}[column sep=0.8cm,row sep=0ex]
      G: &[-0.7cm] \Ac \arrow[r] & \Gc \\
      & k[x_1, \dots, x_n]/I \arrow[r, mapsto] & \Vc(I)
    \end{tikzcd}
  \]

  Then if
  \[
    \begin{tikzcd}[column sep=0.8cm,row sep=0ex]
      \varphi: &[-0.7cm] k[\dots]/I \arrow[r] & k[\dots]/J \\
      & \bar{x}_i \arrow[r, mapsto] & \bar{f}_i
    \end{tikzcd}
  \]
  Define
  \[
    \begin{tikzcd}[column sep=0.8cm,row sep=0ex]
      G(\varphi) = \psi: &[-0.7cm] \Vc(J) \arrow[r] & \Vc(I) \\
      & \alpha = (a_1, \dots, a_m) \arrow[r, mapsto] & (f_1(\alpha), \dots, f_n(\alpha))
    \end{tikzcd}
  \]


%\printindex
%%%%%%%%%%%%%%%%%%%%%%%%%%%%%%%%%%%%%%%%%%%%%
% \bibliographystyle{plain}
% \bibliography{journal.bib}
% \begin{thebibliography}{99}
% \bibitem[1]{ex}\url{http://www.example.com/}
% \end{thebibliography}
\end{document}
