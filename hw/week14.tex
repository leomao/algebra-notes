%! TEX root=./hw.tex

\section{Week 14}

\begin{exercise}
  If $0 \to M_1 \xrightarrow{\alpha} M_2 \xrightarrow{\beta} M_3 \to 0$
  is exact in $\Mod_R$, then for $M, N \in \Mod_R$,
  \begin{itemize}
    \item $0 \to \Hom_R(M, M_1) \xrightarrow{\alpha'}
      \Hom_R(M, M_2) \xrightarrow{\beta'} \Hom_R(M, M_3)$ is exact,
    \item $0 \to \Hom_R(M_3, N) \xrightarrow{\beta''}
      \Hom_R(M_2, N) \xrightarrow{\alpha''} \Hom_R(M_1, N)$ is exact,
    \item $M \otimes_R M_1 \xrightarrow{1_M \otimes \alpha}
      M\otimes_R M_2 \xrightarrow{1_M \otimes \beta} M\otimes_R M_3 \to 0$ is exact.
  \end{itemize}
\end{exercise}

\begin{exercise}
  Give examples to explain why $\Hom_R(M, \cdot), \Hom_R(\cdot, N), M\otimes_R \cdot$
  are not exact functors.
\end{exercise}

\begin{exercise}
  Show that $h(y) \in \Hom_\Zb(R, N_0), h \in \Hom_R(M_2, N)$ and $h \circ \alpha = f$.
\end{exercise}

\begin{exercise}[Snake lemma]
  Suppose the following diagram commutes in $\Mod_R$.
  (two exact sequences)
  \[
    \begin{tikzcd}
      {\color{red} 0} \ar[r, red] & M_1 \ar[r] \ar[d, "f_1"]
      & M_2 \ar[r] \ar[d, "f_2"] & M_3 \ar[r] \ar[d, "f_3"] & 0 \\
      0 \ar[r] & N_1 \ar[r] & N_2 \ar[r] & N_3 \ar[r, red] & {\color{red} 0}
    \end{tikzcd}
  \]
  Show that there exists the following exact sequence:
  (Prove that the red part in the diagram above implies the red part in
  the following sequence)
  \[
    \mathcolor{red}{0 \to} \ker f_1 \to \ker f_2 \to \ker f_3
    \to \coker f_1 \to \coker f_2 \to \coker f_3 \mathcolor{red}{\to 0}
  \]
\end{exercise}

\begin{exercise} \mbox{}
  \begin{enumerate}[(1)]
    \item State the property of being homotopic in the case of cochain complexes.
    \item Let $M \in \Mod_R$. Construct an injective resolution of $M$.
  \end{enumerate}
\end{exercise}

\begin{exercise}
  State and show the dual version of comparison theorem.
\end{exercise}

\begin{exercise}
  $0 \to C_\bullet \to \bar{C_\bullet} \to \tilde{C_\bullet} \to 0$ exact
  will induce a long exact sequence.
\end{exercise}
