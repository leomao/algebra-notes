%! TEX root=../main.tex
\subsection{Week 12}
\subsubsection{Tensor product \RNum{2}}

By universal property, we get
$\{ R\text{-bilinear maps~} M \times N \to L \} \leftrightarrow
\Hom_R(M \otimes_R N, L)$.

Similarly,
\begin{gather*}
  \Hom\left(\bigoplus_{s\in \Lambda} M_s, L\right) \cong
  \prod_{s\in \Lambda} \Hom\left(M_s, L\right) \\
  \Hom\left(N, \prod_{s\in \Lambda} M_s\right) \cong
  \prod_{s\in \Lambda} \Hom\left(N, M_s\right) \\
\end{gather*}

\begin{fact}
  $f\in \Hom_R(M, M'), g \in \Hom_R(N, N') \leadsto
  f\otimes g \in \Hom_R(M\otimes N, M'\otimes N')$ by
  $(f \otimes g)(x \otimes y) = f(x) \otimes g(y)$.
  \begin{proof}
    Define
    $\arraycolsep=1pt
    \begin{array}{rcl}
      h: M \times N & \to & M' \otimes_R N' \\
      (x, y) & \mapsto & f(x) \otimes g(y)
    \end{array}$
  \end{proof}
\end{fact}

Restrition and extension of scalars.

Let $f: R \to S$ be a ring homomorphism and $R, S$ be commutative with $1$.
Then $S$ can be regarded as an $R$-module.
    $\arraycolsep=1pt
    \left(\begin{array}{rcl}
      R \times S & \to & S \\
      (r, x) & \mapsto & f(r)x
    \end{array}\right)$.

If $M$ is a $S$-module, then $M$ is also an $R$-module.
    $\arraycolsep=1pt
    \left(\begin{array}{rcl}
      R \times M & \to & M \\
      (r, a) & \mapsto & f(r)a
    \end{array}\right)$.

If $N$ is an $R$-module, then $S \otimes_R N$ an $S$-module.
    $\arraycolsep=1pt
    \left(\begin{array}{rrcl}
      S & \times (S \otimes_R N) & \to & S \otimes_R N \\
      (r, & x\otimes a) & \mapsto & rx \otimes a
    \end{array}\right)$.

\begin{example}[Important example]
  Let $V$ be a real vector space. The complexification of $V$ is
  $V^\Cb \defeq \Cb \otimes_\Rb V$ which is a $\Cb$-vector space.
\end{example}

\begin{exercise}
  Let $K \subseteq L$ be an inclusion of fields and let $E$ be a vector space
  over $K$. Show that $E^L \defeq L \otimes_K E$ satisfies the following
  universal property: For any vector space $U$ over $L$ and any
  $K$-linear map $f: E \to U$, $\exists!$ $L$-linear map $\varphi$:
  \[
    \begin{tikzcd}[cramped, column sep=tiny]
      \varphi: 1\otimes x :: E^L \arrow{rr} & & f(x) :: U \\
      & x :: E \arrow{ul} \arrow{ur}{f} &
    \end{tikzcd}
    \qedhere
  \]
\end{exercise}

\begin{exercise}
  $E \to E^L$ is a covariant functor from the category of vector spaces over
  $K$ to the category of vector spaces over $L$.
\end{exercise}

\begin{example}
  $\Zb^n \cong \Zb^m \leadsto
  \Qb \otimes_\Zb \Zb^n \cong \Qb \otimes_\Zb \Zb^m \leadsto n = m$.
\end{example}

\begin{example}
  $G \cong \quot{\Zb}{d_1\Zb} \oplus \dots \oplus \quot{\Zb}{d_l\Zb}
  \oplus \Zb^s, \Qb \otimes_\Zb G = \Qb^s$.
\end{example}

Let $M, N$ and $U$ be $R$-module. Then
\[
  \Hom_R\left(M \otimes_R N, U\right) \cong
  \Hom_R\left(N, \Hom_R(M, U)\right)
\]

\begin{proof} \mbox{}
  \begin{itemize}
    \item For $f \in \Hom_R\left(M \otimes_R N, U\right)$ and $a \in N$,
      define $f_a = x :: M \mapsto f(x \otimes a) :: U$.
      \begin{itemize}
        \item linear: easy.
        \item $\ob{f}: a \mapsto f_a$ is an $R$-mod homo.: easy.
        \item $\tau: f \mapsto \ob{f}$ is an $R$-mod homo.:
          $\tau(rf+g)(a)(x) = (rf+g)_a(x) = (rf+g)(x\otimes a)
          = rf(x\otimes a) + g(x\otimes a) = \dots
          = r\tau(f)(a)(x) + \tau(g)(a)(x)$
      \end{itemize}
    \item For $g \in \Hom_R\left(N, \Hom_R(M, U)\right)$,
      define $g' = (x, a) :: M \times N \mapsto g(a)(x) :: U$.
      \begin{itemize}
        \item $g'$ is $R$-bilinear: easy.
        \item $\exists! \tilde{g}: x\otimes a \mapsto g(a)(x)$.
        \item $\sigma: g\mapsto \tilde{g}$ is an $R$-mod homo.: easy.
      \end{itemize}
    \item $\sigma \tau = \text{id}, \tau \sigma = \text{id}$: easy...
      \qedhere
  \end{itemize}
\end{proof}

\begin{exercise}
  $\Hom_R(M, \cdot), M \otimes_R \cdot$ are covariant functors from the
  category of $R$-modules to itself.
  (is an adjoint pair)
\end{exercise}

\begin{fact}
  $\Hom_R(R, M) \cong M$. By $f \mapsto f(1)$.
\end{fact}

\begin{definition}
  An exact sequence $A \xrightarrow{f_1} B \xrightarrow{f_2} \cdots$ is
  a sequence satisfied $\text{im}\; f_k = \ker f_{k+1}$.
\end{definition}

\begin{itemize}
  \item $0 \to \Zb \xrightarrow{2} \Zb$.
  \item $\Qb \to \quot{\Qb}{\Zb} \to 0$.
\end{itemize}

Let $V, W$ be vector spaces over $F$. Then
$V^* \otimes_F W \cong \Hom_F(V, W)$.
\begin{proof}
  Let $\alpha = \{e_1, \dots, e_n\}$ and $\beta = \{f_1,\dots, f_m\}$ be
  bases for $V$ and $W$ respectively.
  Via $\alpha, \beta$, $\Hom_F(V, W) \cong
  \gen*{E_{ij} \middle|
  \begin{aligned}i &= 1,\dots,m\\j &= 1,\dots,n\end{aligned}}_F$.
  $V^* \otimes W \cong
  \gen*{e_j^* \otimes f_i \middle|
  \begin{aligned}i &= 1,\dots,m\\j &= 1,\dots,n\end{aligned}}_F$.
\end{proof}

\subsubsection{Tensor algebra}
\begin{definition} \mbox{}
  \begin{itemize}
    \item Let $R$ be a commutative ring with $1$.
      An $R$-algebra is a ring $A$ which is also an $R$-module s.t. the
      multiplication map $A \times A \to A$ is $R$-bilinear.
      ( $r(ab) = (ra)b = a(rb)$ )
    \item Let $A$ be an $R$-algebra. A grading of $A$ is a collection of
      $R$-submodules $\{ A_n \}_{n=0}^\infty$ ($n$-th homogeneous part) s.t.
      \[
        A = \bigoplus_{n=0}^\infty A_n \quad \text{and} \quad
        A_nA_m \subseteq A_{n+m} \quad \forall n,m
      \]
    \item A graded $R$-algebra is an $R$-algebra with a chosen grading.
    \item $\mathfrak{M}_R$ is the category of $R$-modules.
    \item $\mathfrak{Gr}_R$ is the category of graded $R$-algebras.
      ($f: A \to A'$ with $f(A_n) \subseteq A_n'$)
  \end{itemize}
\end{definition}

\begin{example}
  $A = R[x], A_n = \gen{x^n}_R$. If $I = \gen{x+1}_A$, $I$ is not graded.
  $I = \gen{x^2}_A$ is graded.
\end{example}

\begin{definition}
  \color{red}
  An ideal $I$ is graded in a graded ring $A$ if and only if
  $I = \raisebox{0.3ex}{$\bigoplus$} I \cap A_n$.
  \footnote{This is not mentioned in class}
\end{definition}

\begin{exercise}
  TFAE
  \begin{enumerate}[(1)]
    \item $I$ is graded.
    \item $\forall a \in I$ write $a = a_{k_1} + a_{k_2} + \dots + a_{k_m},
      a_{k_i} \in A_{k_i} \implies a_{k_i} \in I$.
      ($a_{k_i}$ is the homogenuous component of $a$)
    \item $\quot{A}{I}$ is a graded ring with
      $\left(\quot{A}{I}\right)_n = \quot{(A_n + I)}{I}
      \cong \quot{A_n}{I \cap A_n}$.
  \end{enumerate}
\end{exercise}

\begin{exercise} \mbox{}
  \begin{enumerate}[(1)]
    \item If $I$ is a f.g. graded ideal, then $I$ has a finite system of
      generators consisting of homogeneous elements alone.
    \item $I, J$ are graded $\implies I+J, IJ, I \cap J$ are graded.
  \end{enumerate}
\end{exercise}

\underline{Observation}:
Let $\{ M_i \}_{i=1}^\infty$ be a collection of $R$-modules.
\begin{itemize}
  \item $M_1 \otimes_R M_2$ exists.
  \item $(M_1 \otimes_R M_2) \otimes_R M_3 \cong
    M_1 \otimes_R (M_2 \otimes_R M_3) \implies
    M_1 \otimes_R M_2 \otimes_R M_3$ is well-defined.
    Universal property: for any $R$-module $L$ and a $3$-multilinear map
    $f: M_1 \times M_2 \times M_3 \to L$. (拆括號囉)
  \item By induction, $M_1 \otimes \dots \otimes M_n$ is well-defined and
    satisfies the universal property. ($n$-multilinear map)
\end{itemize}

Goal: For a given $R$-module $M$, we intend to construct an graded $R$-algebra
$T(M)$ containing $M$ that is ``universal'' w.r.t. $R$-algebras containing $M$.

That is, a tensor algebra is a pair $(T(M), i)$ where $T(M)$ is an $R$-algebra
and $i :: M \to T(M)$, such that for any $R$-algebra $A$ containing $M$,
which is to say that exist a $R$-module homomorphism $\varphi: M \to A$,
then exists an $R$-algebra homomorphism $\psi :: T(M) \to A$ such
that $\varphi = \psi \circ i$. \\[.5em]

\underline{Construction}:
\begin{itemize}
  \item $\forall k \in \Nb$, $T^k(M) \defeq
    \underbrace{M\otimes \dots \otimes M}_{k \text{~times}}$, each
    $x_1\otimes x_2\otimes \dots \otimes x_k \in T^k(M)$ is called a $k$-tensor.

    $T^0(M) \defeq R$ and
    \[
      T(M) \defeq \bigoplus_{k=0}^\infty T^k(M) = R \oplus T^1(M) \oplus \dots
    \]
  \item define multiplication on $T(M)$ by:
    \[
      \begin{tikzcd}[cramped, row sep=tiny]
        T^i(M) \times T^j(M) \arrow[r] & T^{i+j}(M) \\
        (x_1\otimes \dots \otimes x_i, y_1\otimes \dots \otimes y_j)
        \arrow[r, mapsto]
        & x_1\otimes \dots \otimes x_i\otimes y_1\otimes \dots\otimes y_j \\
      \end{tikzcd}
    \]
  \item Distribution law: easy.
\end{itemize}

Proving the universal property:
For any $R$-algebra $A$ containing $M$ and an $R$-module homo.
$\varphi: M \to A$.
$\forall k \ge 2$, we define
$f_k: M \times \dots \times M \to A$
\[
  \arraycolsep=1pt
  \begin{array}{rcl}
    f_k: & M \times \dots \times M & \to A \\
         & (x_1, \dots, x_k) & \mapsto
    \varphi(x_1) \dots \varphi(x_k)
  \end{array}
\]
$f_k$ is $k$-multilinear $\leadsto$
\[
  \arraycolsep=1pt
  \begin{array}{rcl}
    \exists! \tilde{f_k}: & M \otimes \dots \otimes M & \to A \\
         & x_1 \otimes \dots\otimes x_k & \mapsto
    \varphi(x_1) \dots \varphi(x_k)
  \end{array}
\]
By the universal property of $\bigOp$, exists a unique $R$-module homo.
$\tilde\varphi :: T(M) \to A$ which make the following diagram commutes.
\[
  \begin{tikzcd}[cramped, column sep=tiny]
    \tilde\varphi: T(M) \arrow{rr} & & A \\
      & T^k(M) \arrow{ul}{i} \arrow{ur}{f_k} &
  \end{tikzcd}
\]

$\tilde\varphi$ is an $R$-algebra homomorphism.

\begin{definition}
  $T(M)$ is called the tensor algebra of $M$.
\end{definition}

\begin{exercise}
  $T$ is a covariant functor from $\Mf_R$ to $\Grf_R$.
\end{exercise}

\begin{prop}
  Let $V$ be a vector space over $F$ with a basis $\beta = \{
  v_1, \dots, v_n \}$. Then
  \[
    \left\{
      v_{i_1} \otimes \dots \otimes v_{i_k} \,\middle|\,
      \forall j = 1, \dots, k,\; i_j = 1, \dots, n
    \right\}
  \]
  forms a basis for $T^k(V)$. $\dim_F T^k(V) = n^k$.
\end{prop}

$T(V)$ can be regarded as a non-commutative polynomial algebra over $F$.
\\[.5em]
$\odot$ Symmetrization ($\Char F = 0$)
\[
  \begin{tikzcd}[cramped, row sep=tiny]
    V \times \dots \times V \arrow[r] & T^n(V) \\
    (x_1, \dots, x_n) \arrow[r, mapsto]
    & \displaystyle \frac{1}{n!}\sum_{\tau \in S_n}
    x_{\tau(1)}\otimes \dots \otimes x_{\tau(n)}
  \end{tikzcd}
\]
is $n$-multilinear.

The symmetrizer operator $\sigma: T^n(V) \to T^n(V)$,
$\tilde{S}^n(V) \defeq \sigma(T^n(V)) \subseteq T^n(V)$.

\underline{Claim}:
$T^n(V) = \tilde{S}^n(V) \oplus C^n(V)$ where
\[ C^n(V) = C(V) \cap T^n(V) \quad
C(V) = \gen{v\otimes w - w\otimes v \mid v, w\in V} \]
