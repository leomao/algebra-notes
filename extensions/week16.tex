%! TEX root=../main.tex
\section{Extensions of Groups}
\subsection{Week 16}
\subsubsection{Extensions of abelian groups}

\begin{definition}
  If a group $E$ contains a normal subgroup $N$ and $\quot{E}{N} \cong G$,
  then we call $E$ an extension of $N$ by $G$, denoted by
  $1 \to N \to E \to G \to 1$.
\end{definition}

\underline{Ques}: When $N$ and $G$ are given, how to obtain all extensions of
$N$ by $G$.

{\bf Now assume that $N$ is abelian.}

\begin{definition}
  $1 \to N \to E \xrightarrow{p} G \to 1$.
  $l: G \to E $ is a lifting if $p \circ l = \text{id}_G$ and $l(1) = 1$.
\end{definition}

\begin{remark}
  $G \cong \quot{E}{N} = \{\, xN \mid x \in E \,\}$,
  $p \circ l(\bar{x}) = \bar{x}$, $l(\bar{x})$ is a representative of $xN = \bar{x}$.
\end{remark}

\begin{prop} \mbox{}
  \begin{enumerate}
    \item $\forall \bar{x}\in G, \theta_{\bar{x}}: N \to N, a \mapsto l(\bar{x})al(\bar{x})^{-1}$.
      is independent of the choice of $l$.
    \item $\theta: G \to \Aut(N), \bar{x} \mapsto \theta_{\bar{x}}$ is a group homomorphism.
  \end{enumerate}
  \begin{proof} \mbox{}
    \begin{enumerate}
      \item Suppose $l': G \to E$ is another lifting. Then $l(\bar{x})N = l'(\bar{x})N$.
        So $l'(\bar{x}) = l(\bar{x})b$ for some $b \in N$.
        $\forall a \in N$, $l'(\bar{x})al'(\bar{x})^{-1} = l(\bar{x})bab^{-1}l(\bar{x})^{-1}
        = l(\bar{x})al(\bar{x})^{-1}$ since $N$ is abelian.
      \item $\theta_{\bar{x}\bar{y}}(a) = l(\bar{x}\bar{y})al(\bar{x}\bar{y})^{-1}$.
        \[
          \begin{cases}
            p \circ l(\bar{x}\bar{y}) = \bar{x}\bar{y} \\
            p \circ (l(\bar{x})l(\bar{y})) = \bar{x}\bar{y}
          \end{cases}
          \leadsto l(\bar{x}\bar{y}), l(\bar{x})l(\bar{y})
          \text{~are liftings of~} \bar{x}\bar{y}
          \qedhere
        \]
    \end{enumerate}
  \end{proof}
\end{prop}

\begin{definition}
  An extension $1\to N\to E\to G\to 1$ splits if $\exists$ a lifting
  $l: G\to E$ is a group homo.
\end{definition}

\begin{prop} TFAE
  \begin{enumerate}
    \item $1\to N\to E\to G\to 1$ splits.
    \item $\exists$ a subgroup $K \le E$ s.t. $K \cong G$ and
      $\begin{cases} K \cap N = \{1\} \\ NK = E\end{cases}
      \leadsto E \cong N \rtimes K (\cong N \rtimes G)$.
  \end{enumerate}
  \begin{proof}
    (1) $\Rightarrow$ (2): Let $K = \Image l$ which is a subgroup since $l$
    is a group homo.
    \begin{itemize}
      \item $l$ is an isomorphism from $G$ to $K$: If $l(\bar{x}) = l(\bar{y})$, then
        $p \circ l(\bar{x}) = p\circ l(\bar{y}) \leadsto \bar{x} = \bar{y}$.
        So $l$ is 1-1.
      \item $E = NK$: $\forall x \in E, \bar{x} = p(x) \leadsto
        y = l(\bar{x})$ and $p(x) = p(y) \leadsto \exists a \in N$ s.t.
        $x = ay$.
      \item $K \cap N = \{1\}$:
        $a = l(\bar{x}) \in K \cap N \leadsto 1 = p(a) = p(l(\bar{x})) = \bar{x}
        \leadsto a = l(1) = 1$.
    \end{itemize}

    (2) $\Rightarrow$ (1):
    \begin{itemize}
      \item $p \big|_K: K \to G$ is an isom.:
        onto: $p(K) = p(NK) = p(E) = G$, 1-1: $\ker(p\big|_K) = N \cap K = \{1\}$.
      \item $l = \left(p\big|_K\right)^{-1}$ is a group homo.

        \underline{Observation}: Let $l: G\to E$ be a lifting.
        Then $E = \bigcup_{\bar{x}\in G} Nl(\bar{x}), \forall x, y \in E$,
        write $x = al(\bar{x}), y = bl(\bar{y}), a, b\in N, \bar{x},\bar{y}\in G$.
        \[
          xy = (al(\bar{x})bl(\bar{y})) = al(\bar{x})bl(\bar{x})^{-1}l(\bar{x})l(\bar{y})
          = a \theta_{\bar{x}}(b)l(\bar{x})l(\bar{y})
        \]
        Notice that $l(\bar{x})l(\bar{y})$ and $l(\bar{x}\bar{y})$ are liftings,
        so we can write $l(\bar{x})l(\bar{y}) = f(\bar{x}, \bar{y})l(\bar{x}\bar{y})$
        for some $f(\bar{x}, \bar{y})\in N$.
        \qedhere
    \end{itemize}
  \end{proof}
\end{prop}

\begin{exercise}
  $B^2(G, N) \le Z^2(G, N)$.
\end{exercise}
\begin{exercise}
  Show that there are inequivalent  extensions of $N$ by $G$ with isomorphic
  middle groups.
  (Hint: $N=\quot{\Zb}{p\Zb}$ with $p$ is odd, $E=\quot{\Zb}{p^2\Zb}$,
  $a:: N \mapsto x^p::E$ and please give another morphism $N\to E$ by yourself.)
\end{exercise}

\begin{definition}
  Given $1\to N\to E\xrightarrow{p} G\to 1$ and $l: G\to E$, a factor set is
  a function $f: G\times G \to N$ s.t.
  $\forall \bar{x}, \bar{y}\in G, l(\bar{x})l(\bar{y})
  = f(\bar{x}, \bar{y})l(\bar{x}\bar{y})$.
\end{definition}

\begin{prop}
  Let $1\to N \to E\xrightarrow{p} G\to 1$ and $l: G\to E$.
  If $f$ is a factor set, then
  \begin{enumerate}[(1)]
    \item $f(x, 1) = 1 = f(1, y) \quad \forall x, y\in G$.
    \item (cocycle identity) $\forall x,y,z\in G,
      f(x,y)f(xy, z)
      = \theta_{x}(f(y, z))f(x, yz)$.

      (i.e. $f(x, y) + f(xy,z)
      = xf(y,z) + f(x, yz)$)
  \end{enumerate}
  \begin{proof} \mbox{}
    \begin{enumerate}[(1)]
      \item Trivial since $l(x)l(1) = l(1 \cdot x)$.
      \item By associativity. $(l(x)l(y))l(z) = l(x)(l(y)l(z))$. \\
        $(l(x)l(y))l(z) = f(x, y) l(xy) l(z) = f(x, y) f(xy, z) l(xyz)$, and \\
        $l(x)(l(y)l(z)) = l(x) f(y, z) l(yz) = l(x) f(y, z) l^{-1}(x) l(x) l(yz)
        = \theta_x(f(y, z)) f(x, yz) l(xyz)$.
        Thus $f(x, y) f(xy, z) = \theta_x(f(y, z)) f(x, yz)$.
        \qedhere
    \end{enumerate}
  \end{proof}
  \label{prop:factor-set}
\end{prop}

\begin{theorem}
  Let $\sigma: G \to \Aut(N), x\mapsto \sigma_x$  be a group homo. and
  $f: G\times G\to N$ satisfies (1),(2) in Prop. \ref{prop:factor-set}.
  Then $\exists 1\to N\to E\to G\to 1$ and $l: G\to E$ s.t. $\theta = \sigma$
  and $f$ is the corresponding factor set.

  \begin{proof}
    \begin{itemize}
      \item Define $E = N \times G$ equipped with the operation
        \[
          (a, x)(b, y) = (a \sigma_{x}(b)f(x,y),
          xy)
        \]
        \begin{itemize}
          \item associativity:
            \begin{align*}
              \big( (a, x) (b, y) \big) (c, z) &= (a \sigma_x(b) f(x, y), xy) (c, z) \\
              &= (a \sigma_x(b) f(x, y) \sigma_{xy}(c) f(xy, z), xyz) \\
              &= (a \sigma_x(b) \sigma_{xy}(c) f(x, y) f(xy, z), xyz) \quad (\because N \text{ abelian})
            \end{align*}
            and
            \begin{align*}
              (a, x) \big( (b, y) (c, z) \big) &= (a, x) (b \sigma_y(c) f(y, z)) \\ 
              &= \big( a \sigma_x \big( b \sigma_y(c) f(y, z) \big) f(x, yz), xyz \big) \\
              &= \big( a \sigma_x ( b ) \sigma_{xy}(c) \sigma_x(f(y, z)) f(x, yz), xyz \big) \\
              &= \big( a \sigma_x ( b ) \sigma_{xy}(c) f(x, y) f(xy, z), xyz \big) \\
            \end{align*}
          \item indentity: $(1, 1)$.
          \item inverse: $(a, x)^{-1} =
            (\sigma_{x^{-1}}(a^{-1}f(x,x^{-1})^{-1}), x^{-1})$.
        \end{itemize}
      \item $p: E\to G, (a, x) \mapsto x$ is a group homo by def.
      \item $i: N\to E, a \mapsto (a, 1)$ is a group homo.
        $(a, 1)(b, 1) = (a\sigma_1(b)f(1, 1), 1) = (ab, 1)$.
      \item $\ker p = \Image i$.
      \item $\Fix l: G\to E, a\in N, x\in G$, say $l(x) = (b, x)$.
        \begin{align*}
          l(x)(a, 1)l(x)^{-1}
          &= (b, x) (a, 1) (b, x)^{-1} = (b \sigma_x(a) , x)
          \big(\sigma_{x^{-1}}(a^{-1}f(x,x^{-1})^{-1}), x^{-1} \big) \\
          &= (b \sigma_{x}(a) \cdot (\sigma_x \circ \sigma_{x^{-1}}) \big( b^{-1} f(x, x^{-1})^{-1} \big)
          \cdot f(x, x^{-1}), 1) \\
          &= (\sigma_x(a), 1)
        \end{align*}
        So $\theta_{x} = \sigma_{x}$.
      \item Let $l: G\to E, x\mapsto (1, x)$.
        Check $l(x)l(y)l(xy)^{-1} = (f(x,y), 1)$.
        Then $f$ is the corresponding factor set.
        \qedhere
    \end{itemize}
  \end{proof}
\end{theorem}

\begin{prop}
  Let $1\to N\to E \xrightarrow{p}G \to 1$ with two liftings
  $l_1: G \to E,\ l_2: G\to E$ with $f_1: G\times G\to N,\ f_2: G\times G\to N$
  respectively.

  Then $\exists h: G\to N$ with $h(1) = 1$ and $\forall x,y\in G,
  f_2(x, y)f_1(x,y)^{-1} = \theta_{x}(h(y))h(xy)^{-1}h(x)$. 
  ($f_2(x,y) - f_1(x,y) = xh(y) - h(xy) + h(x)$)

  \begin{proof}
    For $x\in G$, $\exists h(x) \in N$ s.t.
    $l_2(x) = h(x)l_1(x)$.
    Since $l_1(1) = l_2(1) = 1$, $h(1) = 1$.

    Now, $l_2(x)l_2(y) = f_2(x, y) l_2(x, y) = f_2(x, y) h(xy) l_1(x, y)$. and
    \begin{align*}
      l_2(x) l_2(y) &= h(x) l_1(x) h(y) l_1(y) = h(x) l_1(x) h(y) l_1^{-1}(x) l_1(x) l_1(y) \\
      &= h(x) \theta_x(h(y)) l_1(x) l_1(y) = f_1(x, y) h(x) \theta_x(h(y)) l_1(x, y)
    \end{align*}
    So $f_2(x, y) f_1(x, y)^{-1} = \theta_x(h(y)) h(xy)^{-1} h(x)$.
  \end{proof}
\end{prop}

\begin{remark}
  A map which has the form $\tilde{h}: G\times G \to N, (x,y) \mapsto
  xh(y) - h(xy) + h(x)$ is called a coboundary map.
\end{remark}

\begin{definition}
  $Z^2(G, N) =$ the abelian group of all factor sets.

  $B^2(G, N) =$ the abelian group of all coboundary maps.

  $H^2(G, N) = \quot{Z^2(G, N)}{B^2(G, N)}$
\end{definition}

\begin{definition}
  Two extensions $\begin{cases}
    1\to N\to E\to G\to 1 \\
    1\to N\to E'\to G\to 1
  \end{cases}$
  are equivalent if exists an isomorphism $\varphi: E \xrightarrow{\sim} E'$ which let the
  following diagram comutes.
  \[
    \begin{tikzcd}
      1 \arrow[r]
      & N \arrow[d, "1_N"] \arrow[r]
      & E \arrow[d, "\rotatebox{90}{\(\sim\)}", "\varphi"'] \arrow[r]
      & G \arrow[d, "1_G"] \arrow[r]
      & 1 \\
      1 \arrow[r] & N \arrow[r] & E' \arrow[r] & G \arrow[r] & 1
    \end{tikzcd}
  \]
\end{definition}

\begin{theorem}
  Two extensions $\begin{cases}
    1\to N\to E\to G\to 1 \\
    1\to N\to E'\to G\to 1
  \end{cases}$ are equivalent $\iff$ \\
  Exists mappings $l: G\to E, l': G\to E'$ with two factor sets $f, f'$ respectively satisfies
  $f - f' \in B^2(G, N)$.

  \begin{proof}
    ``$\Rightarrow$'': Choose $l: G\to E$ which has a corresponding factor set $f: G\times G \to N$.
    Now define $l': G\to E'$ by $l' = \varphi \circ l$. Since 
    $p' \circ l' = p' \circ \varphi \circ l = p \circ l = 1$, $l'$ is a lifting.
    Let   $f': G\times G \to N$ be its factor set.
    
    Since $1_N = 1_N \circ \varphi$, $\varphi \big|_N = 1_N$. And
    \begin{align*}
      & l(x) l(y) = f(x, y) l(xy) \\
      \Rightarrow \ & \varphi(l(x) l(y)) = \varphi(f(x, y) l(xy)) \\
      \Rightarrow \ & l'(x) l'(y) = \varphi(f(x, y)) l'(xy) \\
      \Rightarrow \ & f'(x, y) = \varphi(f(x, y))
    \end{align*}
    But $f(x, y) \in N$, $\varphi(f(x, y)) = \varphi\big|_N(f(x, y)) = f(x, y)$. So $f(x, y) = f'(x, y)$,
    hence $f - f' = 0 \in B^2(G, N)$.
    \begin{exercise} \mbox{}
      \begin{enumerate}[(1)]
        \item Show that $f' - f \in B^2(G, N)$.
        \item ``$\Leftarrow$'': Show all details of the following steps:
          \begin{itemize}
            \item $\begin{cases}
                1\to N\to E \to G \to 1 \\
                1\to N\to E(N, G, f, \theta) \to G\to 1
              \end{cases}$ are equivalent.
            \item Similarly $\begin{cases}
                1\to N\to E' \to G\to 1 \\
                1\to N\to E(N, G, f', \theta')\to G\to 1
              \end{cases}$ are equivalent.
            \item $f'-f \leadsto h:G\to N$,
          \end{itemize}
          \qedhere
      \end{enumerate}
    \end{exercise}
  \end{proof}
\end{theorem}

\subsubsection{1st and 2nd group cohomology}
Let $N$ be an abelian group and $G$ be a group with a group homo
$\sigma: G \to \Aut(N)$ ($G \acts N$)
\begin{align*}
  & e(G, N) = \{ \text{equivalence classes of $N$ by $G$} \} \\
  & Z^2(G, N) = \{
    f: G\times G \to N \mid f(1, v) = f = f(u, 1),
    f(u, v) + f(uv, w) = uf(v, w) + f(u, vw) \quad u, v, w \in G \} \\
  & B^2(G, N) = \{
    f: G\times G \to N \mid \exists h: G\to N \text{~with~} h(1) = 1
    \text{~s.t.~}
  f(u, v) = uh(v) - h(uv) + h(u) \quad u, v \in G \} \\
  & H^2(G, N) = \quot{Z^2(G, N)}{B^2(G, N)}
\end{align*}
Then $e(G, N) \leftrightarrow H^2(G, N)$.

\begin{definition} \mbox{}
  \begin{itemize}
    \item $\varphi \in \Aut(E)$ stabilizes $1\to N\to E\to G\to 1$ if
      \[
        \begin{tikzcd}
          1 \arrow[r]
          & N \arrow[d, "1_N"] \arrow[r]
          & E \arrow[d, "\rotatebox{90}{\(\sim\)}", "\varphi"'] \arrow[r]
          & G \arrow[d, "1_G"] \arrow[r]
          & 1 \\
          1 \arrow[r] & N \arrow[r] & E \arrow[r] & G \arrow[r] & 1
        \end{tikzcd}
      \]
    \item $\Stab_E(G, N) = \{ \text{stabilizing automorphisms} \} \le \Aut(E)$
  \end{itemize}
\end{definition}

\begin{definition} \mbox{}
  \begin{itemize}
    \item A derivation is a function $d: G\to N$ s.t. $d(uv) = ud(v) + d(u)
      \quad \forall u, v \in G$.
    \item $\Der(G, N) = \{ \text{derivations} : G\to N \}$ is an abelian group
      with pointwise addition.
  \end{itemize}
\end{definition}

\begin{theorem}
  Let $1\to N\to E\to G\to 1$ with $\theta = \sigma$. Then
  $\Stab_E(G, N) \cong \Der(G, N)$. So $\Stab_E(G, N)$ is abelian.
  \begin{proof} \mbox{}
    \begin{itemize}
      \item Let $\varphi \in \text{LHS}$ and fix $l:G\to E$.
        \[
          \begin{tikzcd}
            1 \arrow[r]
            & N \arrow[d, "1_N"] \arrow[r]
            & E \arrow[d, "\rotatebox{90}{\(\sim\)}", "\varphi"'] \arrow[r]
            & G \arrow[d, "1_G"] \arrow[r] \arrow[l, bend left, "l"]
            & 1 \\
            1 \arrow[r] & N \arrow[r] & E \arrow[r] & G \arrow[r] & 1
          \end{tikzcd}
          \quad \varphi(al(u)) = \varphi(a)\varphi(l(u)) = a d(u) l(u)
        \]
        \begin{itemize}
          \item For another $l':G\to E$, say $l'(u) = g(u)l(u)$, where
            $g(u) \in N$,
            we have
            \[ d'(u) = \varphi(l'(u))(l'(u))^{-1}
            = \varphi(g(u)l(u))(g(u)l(u))^{-1})
            = g(u)\varphi(l(u)) l(u)^{-1} g(u)^{-1} = d(u).
            \]

          \item $d \in \text{RHS}$, 
            \begin{align*}
              d(uv) &= \varphi(l(uv))l(uv)^{-1} \\
                    &= \varphi(f(u, v)^{-1} l(u)l(v)) l(v)^{-1} l(u)^{-1} f(u, v) \\
                    &= f(u, v)^{-1} d(u)l(u) d(v)l(v) l(v)^{-1} l(u)^{-1} f(u, v) \\
                    &= f(u, v)^{-1} d(u) \big(l(u) d(v) l(u)^{-1}\big) f(u, v) \\
                    &= \big(u d(v)\big) d(u)
            \end{align*}
        \end{itemize}

      \item Conversely, \begin{exercise} proof it \end{exercise}
      \item group homo:
        $\varphi_2\circ\varphi_1(al(u)) = \varphi_2(ad_1(u)l(u))
        = ad_1(u)\varphi_2(l(u)) = a d_1(u) d_2(u) l(u)$.
        That is, $\varphi_2 \circ \varphi_1 \mapsto d_1 d_2$.
        \qedhere
    \end{itemize}
  \end{proof}
\end{theorem}

\begin{definition} \mbox{}
  \begin{itemize}
    \item $\Inn_E(G, N) = \{ \varphi \in \Stab_E(G, N) \mid
      \varphi: E \to E, x \mapsto a_0 x a_0^{-1} \text{~for some~} a_0 \in N \}$.
    \item $\PDer(G, N) = \{ d \in \Der(G, N) \mid
      d(u) = ua_1 - a_1 \text{~for some~} a_1 \in N \}$.
  \end{itemize}
\end{definition}

\begin{exercise}
  Show that $\Inn_E(G, N) \cong \PDer(G, N)$.
\end{exercise}

$\quot{\Stab_E(G, N)}{\Inn_E(G, N)} \cong \quot{\Der(G, N)}{\PDer(G, N)}
= H^1(G, N)$.

\begin{exercise}
  Fix $1\to N\to E\to G\to 1$. Show that if
  $H^2(G, N) = 0, H^1(G, N) = 0$, then for $\begin{aligned}
    l&:G\to E\\
    l'&:G\to E
  \end{aligned}$ with $\begin{aligned}
    K &= l(G) \\
    K' &= l'(G)
  \end{aligned}$, we get that $K$ and $K'$ are conjugate.
\end{exercise}

\begin{definition}
  Let $R$ be a commutative ring with $1$ and $G$ be a group.
  The group ring 
  \[ R[G] = \left\{ \sum_{g\in G} r_g g \,\middle|\,
    \text{only finitely many $r_g$'s $\ne 0$ in $R$}
  \right\} \]
  forms an $R$-algebra via
  \begin{gather*}
    \sum_{g\in G} r_g g + \sum_{g\in G} r_g' g
    = \sum_{g\in G} (r_g + r_g') g \\
    \left(\sum_{g\in G} r_g g\right)\left( \sum_{g'\in G} r_g' g' \right)
    = \sum_{g, g'\in G} (r_g r_g') g g' \\
    r\left(\sum_{g\in G} r_g g\right) = \sum_{g\in G} (rr_g) g
  \end{gather*}
\end{definition}

\begin{remark} \mbox{}
  \begin{enumerate}
    \item
      \begin{itemize}
        \item $\{ \rho: G\to \text{GL}(V) \} \leftrightarrow
          \{ V: \Cb[G]\text{-module} \}$.
        \item $\rho$: irr $\leftrightarrow$
          $V:$ simple $\Cb[G]$-module (i.e. no nontrivial proper submodule)
        \item $W \subset V$: $G$-invariant $\leftrightarrow$
          $W: \Cb[G]$-submodule.
      \end{itemize}
    \item $N$: abelian $\leadsto N: \Zb$-module and $G \acts N$. $\implies$
      $N: \Zb[G]$-module.
  \end{enumerate}
\end{remark}

\begin{definition}
  $G \acts \Zb$ trivially. i.e. $g\cdot n = n \quad \forall g \in G,
  n \in \Zb$, then $\Zb: \Zb[G]$-module.

  \begin{itemize}
    \item $B_0 = \Zb[G][~]$: the free $\Zb[G]$-module on the symbol $[~]$.
    \item $B_1 = \bigoplus_{u\in G} \Zb[G][u]$: the free $\Zb[G]$-module on the
      set $G$.
    \item $B_2 = \bigoplus_{u, v\in G} \Zb[G][u|v]$: the free
      $\Zb[G]$-module on the set $G \times G$.
    \item $B_3 = \bigoplus_{u, v, w\in G} \Zb[G][u|v|w]$: the free
      $\Zb[G]$-module on the set $G \times G \times G$.
  \end{itemize}
\end{definition}


...

Now apply $\Hom(\cdot, N)$ to it:

...

\begin{theorem}
  $\Ext^1_{\Zb[G]}(\Zb, N) \defeq \quot{\ker d_2^*}{\ker d_1^*}
  \cong \quot{\Der(G, N)}{\PDer(G, N)} = H^1(G, N)$.
  \begin{proof} \mbox{}
    \begin{itemize}
      \item $g\in \ker d_2^* \subseteq \Hom(B_1, N) \implies g\circ d_2 = 0$.
        ...
      \item ...
      \item Let $t\in \Hom(B_0, N)$, say $t([~]) = a_0 \in N$.
        \[ d_1^*(t)([u]) = t\circ d_1([u]) = t(u[~]-[~]) = ut([~])-t([~])
        = ua_0 - a_0 \]
        Then $d(u) \defeq d_1^*(t)([u]) \implies d \in \PDer(G, N)$.
      \item ...
    \end{itemize}
  \end{proof}
\end{theorem}

\begin{remark}
  $\Ext^2_{\Zb[G]}(\Zb, N) \cong H^2(G, N)$.
\end{remark}
