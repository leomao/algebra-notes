\documentclass[a4paper,titlepage]{article}

%%%%%%%%%%%%%%%%%%page size%%%%%%%%%%%%%%%%%%
% \paperwidth=65cm
% \paperheight=160cm

%%%%%%%%%%%%%%%%%%%Package%%%%%%%%%%%%%%%%%%%
\usepackage[margin=3cm]{geometry}
\usepackage{mathtools,amsthm,amssymb}
\usepackage{centernot}
\usepackage{yhmath}
\usepackage{graphicx}
\usepackage{fontspec}
\usepackage{titlesec}
\usepackage{titling}
\usepackage{fancyhdr}
\usepackage{tabularx}
\usepackage[square, comma, numbers, super, sort&compress]{natbib}
\usepackage[usenames, dvipsnames]{color}
\usepackage[shortlabels, inline]{enumitem}
\usepackage{xpatch}
\usepackage{imakeidx}
\usepackage{bm}
\usepackage{bbm}
\usepackage{hvindex}
\usepackage[normalem]{ulem}
%%% fix bugs in hvindex .................
\def\IndexXXii#1!#2@#3@#4\IndexNIL{%
  \ifx\relax#3\relax            %               no @ in last arg
    \def\hvTemp{#2}%
    \ifx\hvTemp\hvEncap\index{#1!{#2}}#2\else
      \ifx\hvIDXfont\hvIDXfontDefault\index{#1!{#2}}#2% <--- Here a % and #1 were missing
      \else\index{#1!#2@\hvIDXfont{#2}}\hvIDXfont{#2}\fi\fi%
  \else\index{#1!\protect#2@#3}#3\fi}

\def\IndexXXiii#1!#2!#3@#4@#5\IndexNIL{%
  \ifx\relax#4\relax 		% 		no @ in last arg
    \def\hvTemp{#3}%
    \ifx\hvTemp\hvEncap\index{#1!#2!{#3}}#3\else
      \ifx\hvIDXfont\hvIDXfontDefault\index{#1!#2!{#3}}#3
      \else\index{#1!#2!#3@\hvIDXfont{#3}}\hvIDXfont{#3}\fi\fi%
  \else\index{#1!#2!\protect#3@#4}#4\fi}
\usepackage[unicode, pdfborder={0 0 0}, bookmarksdepth=-1]{hyperref}
\hypersetup{
    colorlinks,
    linkcolor=red,
  }
\makeindex[columns=2, options= -s index_style.ist]

%\usepackage{tabto}     
%\usepackage{soul}      
%\usepackage{ulem}      
%\usepackage{wrapfig}   
%\usepackage{floatflt}  
\usepackage{float}     
\usepackage{caption}   
\usepackage{subcaption}
%\usepackage{setspace}  
\usepackage{mdframed}  
%\usepackage{multicol}  
%\usepackage[abbreviations]{siunitx}
%\usepackage{dsfont}   
\usepackage[makeroom]{cancel}
%%%%%%%%%%%%%%%%%%%TikZ%%%%%%%%%%%%%%%%%%%%%%
\usepackage{tikz}
\usepackage{tikz-cd}
%\usepackage{circuitikz}
\usetikzlibrary{calc}
\usetikzlibrary{arrows}
\usetikzlibrary{shapes}
\usetikzlibrary{positioning}

\tikzstyle{every picture}+=[remember picture]

%%%%%%%%%%%%%%中文 Environment%%%%%%%%%%%%%%%
\usepackage[CheckSingle, CJKmath]{xeCJK}  % xelatex 中文
\usepackage{CJKulem}	% 中文字裝飾
\setCJKmainfont[BoldFont=cwTeX Q Hei]{cwTeX Q Ming}
\setCJKsansfont[BoldFont=cwTeX Q Hei]{cwTeX Q Ming}
\setCJKmonofont[BoldFont=cwTeX Q Hei]{cwTeX Q Ming}

%%%%%%%%%%%%%%%%%font size%%%%%%%%%%%%%%%%%%%
\renewcommand{\baselinestretch}{1.1}
%\def\normalsize{\fontsize{10}{15}\selectfont}
%\def\large{\fontsize{12}{18}\selectfont}
%\def\Large{\fontsize{14}{21}\selectfont}
%\def\LARGE{\fontsize{16}{24}\selectfont}
%\def\huge{\fontsize{18}{27}\selectfont}
%\def\Huge{\fontsize{20}{30}\selectfont}

%%%%%%%%%%%%%%%Theme Input%%%%%%%%%%%%%%%%%%%
%\input{themes/chapter/neat}
%\input{themes/env/problist}

%%%%%%%%%%%titlesec settings%%%%%%%%%%%%%%%%%
%\titleformat{\chapter}{\bf\Huge}
            %{\arabic{section}}{0em}{}
%\titleformat{\section}{\centering\Large}
            %{\arabic{section}}{0em}{}
%\titleformat{\subsection}{\large}
            %{\arabic{subsection}}{0em}{}
%\titleformat{\subsubsection}{\bf\normalsize}
            %{\arabic{subsubsection}}{0em}{}
%\titleformat{command}[shape]{format}{label}
            %{gutter}{before}[after]

%%%%%%%%%%%%variable settings%%%%%%%%%%%%%%%%
%\numberwithin{equation}{section}
%\setcounter{secnumdepth}{4}
%\setcounter{tocdepth}{1}
%\setcounter{section}{0}
%\graphicspath{{images/}}

%%%%%%%%%%%%%%%page settings%%%%%%%%%%%%%%%%%
\newcolumntype{C}[1]{>{\centering\arraybackslash}p{#1}}
\setlength{\headheight}{15pt}  % with titling
\setlength{\droptitle}{-2.5cm}
%\posttitle{\par\end{center}}  % distance between title and content
\parindent=0pt % indent size
\parskip=1ex    % line space
%\pagestyle{empty}  % empty: no page number
%\pagestyle{fancy}  % fancy: fancyhdr

% use with fancygdr
%\lhead{\leftmark}
%\chead{}
%\rhead{}
%\lfoot{}
%\cfoot{}
%\rfoot{\thepage}
%\renewcommand{\headrulewidth}{0.4pt}
%\renewcommand{\footrulewidth}{0.4pt}

%\fancypagestyle{firststyle}
%{
  %\fancyhf{}
  %\fancyfoot[C]{\footnotesize Page \thepage\ of \pageref{LastPage}}
  %\renewcommand{\headrule}{\rule{\textwidth}{\headrulewidth}}
%}

\setlist{itemsep=0em, topsep=0.2em}

%%%%%%%%%%%%%%%renew command%%%%%%%%%%%%%%%%%
% \renewcommand{\contentsname}{Table of Content}
% \renewcommand{\refname}{Reference}
\renewcommand{\abstractname}{\LARGE Abstract}

%%%%%%%%symbol and function settings%%%%%%%%%
% adjust \exists and \forall spacing
\let\existstemp\exists
\let\foralltemp\forall
\renewcommand*{\exists}{\existstemp\mkern4mu}
\renewcommand*{\forall}{\foralltemp\mkern4mu}

\DeclarePairedDelimiter{\abs}{\lvert}{\rvert}
\DeclarePairedDelimiter{\norm}{\lVert}{\rVert}
\DeclarePairedDelimiter{\inpd}{\langle}{\rangle} % inner product
\DeclarePairedDelimiter{\ceil}{\lceil}{\rceil}
\DeclarePairedDelimiter{\floor}{\lfloor}{\rfloor}
\DeclareMathOperator{\adj}{adj}
\DeclareMathOperator{\sech}{sech}
\DeclareMathOperator{\csch}{csch}
\DeclareMathOperator{\arcsec}{arcsec}
\DeclareMathOperator{\arccot}{arccot}
\DeclareMathOperator{\arccsc}{arccsc}
\DeclareMathOperator{\arccosh}{arccosh}
\DeclareMathOperator{\arcsinh}{arcsinh}
\DeclareMathOperator{\arctanh}{arctanh}
\DeclareMathOperator{\arcsech}{arcsech}
\DeclareMathOperator{\arccsch}{arccsch}
\DeclareMathOperator{\arccoth}{arccoth}
\newcommand{\np}[1]{\\[{#1}] \indent}
\newcommand{\tr}[1]{{#1}^\mathrm{t}}
%%%% Geometry Symbol %%%%
\newcommand{\degree}{^\circ}
\newcommand{\Arc}[1]{\wideparen{{#1}}}
\newcommand{\Line}[1]{\overleftrightarrow{{#1}}}
\newcommand{\Ray}[1]{\overrightarrow{{#1}}}
\newcommand{\Segment}[1]{\overline{{#1}}}
%%%% Math symbol %%%%
\newcommand{\defeq}{\vcentcolon=}
\newcommand{\Nb}{\mathbb{N}}
\newcommand{\Zb}{\mathbb{Z}}
\newcommand{\Qb}{\mathbb{Q}}
\newcommand{\Rb}{\mathbb{R}}
\newcommand{\Cb}{\mathbb{C}}
\newcommand{\Hb}{\mathbb{H}}
\newcommand{\Fb}{\mathbb{F}}
\newcommand{\Fbx}{\mathbb{F}^\times}
\newcommand{\Qbx}{\mathbb{Q}^\times}
\newcommand{\Rbx}{\mathbb{R}^\times}
\newcommand{\Cbx}{\mathbb{C}^\times}
\newcommand{\Hbx}{\mathbb{H}^\times}
\newcommand{\Ic}{\mathcal{I}}
\newcommand{\Vc}{\mathcal{V}}
\newcommand{\Gc}{\mathcal{G}}
\newcommand{\Fc}{\mathcal{F}}
\newcommand{\Cc}{\mathcal{C}}
\newcommand{\Dc}{\mathcal{D}}
\newcommand{\sT}{{\sf T}}
\newcommand{\sI}{{\sf I}}
\newcommand{\Id}{{\rm Id}}

\newcommand{\Mf}{\mathfrak{M}}
\newcommand{\Grf}{\mathfrak{Gr}}

\newcommand{\bigOp}{\raisebox{0.3ex}{$\bigoplus$}}
\newcommand{\bigOt}{\raisebox{0.3ex}{$\bigotimes$}}
\newcommand{\Op}{\oplus}
\newcommand{\Ot}{\otimes}

\DeclareMathOperator{\Sym}{Sym}
\DeclareMathOperator{\Alt}{Alt}
\DeclareMathOperator{\diag}{diag}
\DeclareMathOperator{\sgn}{sgn}
\DeclareMathOperator{\lcm}{lcm}
\DeclareMathOperator{\Image}{Im}
\DeclareMathOperator{\Char}{char}
\DeclareMathOperator{\Fix}{Fix}
\DeclareMathOperator{\Inn}{Inn}
\DeclareMathOperator{\Aut}{Aut}
\DeclareMathOperator{\Gal}{Gal}
\DeclareMathOperator{\Isom}{Isom}
\DeclareMathOperator{\Tor}{Tor}
\DeclareMathOperator{\Exp}{Exp}
\DeclareMathOperator{\Syl}{Syl}

% multilinear
\DeclareMathOperator{\Hom}{Hom}
\DeclareMathOperator{\Lrad}{lrad}
\DeclareMathOperator{\Rrad}{rrad}
\DeclareMathOperator{\rank}{rank}
\DeclareMathOperator{\trace}{Tr}

% extensions
\DeclareMathOperator{\Stab}{Stab}
\DeclareMathOperator{\Der}{Der}
\DeclareMathOperator{\PDer}{PDer}
\DeclareMathOperator{\Ext}{Ext}
\DeclareMathOperator{\Ann}{Ann}

\DeclareMathOperator{\trdeg}{\mathrm{tr} \deg}

% commutative
\DeclareMathOperator{\Spec}{Spec}
\DeclareMathOperator{\LT}{LT}
%
\newcommand{\Resf}[1]{\big|_{#1}}

\newcommand{\ob}{\overline}
\DeclareMathOperator{\ord}{ord}
\DeclarePairedDelimiter{\gen}{\langle}{\rangle} % generator
%\newcommand*\quot[2]{{^{\textstyle #1}\Big/_{\textstyle #2}}}
\newcommand*\quot[2]{{#1}/{#2}}
\newcommand\bij{\lhook\joinrel\twoheadrightarrow}
\newcommand\toone{\hookrightarrow}
\newcommand\onto{\twoheadrightarrow}
\newcommand\acts{\curvearrowright}
\newcommand\revacts{\curvearrowleft}
\newcommand\isoto{\xrightarrow{\sim}}
\newcommand\deffunc[5]{\ensuremath{
  \arraycolsep=1pt
  \begin{array}{rccc}
    #1: & #2 & \to & #3 \\
       & #4 & \mapsto & #5
  \end{array}
}}

% just to make sure it exists
\providecommand\given{}
% can be useful to refer to this outside \Set
\newcommand*\SetSymbol[1][]{%
  \nonscript\:#1\vert
  \allowbreak
  \nonscript\:
\mathopen{}}
\DeclarePairedDelimiterX\Set[1]\{\}{%
  \renewcommand\given{\SetSymbol[\delimsize]}
  \,#1\,
}

\DeclarePairedDelimiterX\Gen[1]{\langle}{\rangle}{%
  \renewcommand\given{\SetSymbol[\delimsize]}
  \,#1\,
}

% cycle group \cycle{1,2,3} => (1 2 3)
\ExplSyntaxOn
\NewDocumentCommand{\cycle}{ O{\;} m }
 {
  (
  \alec_cycle:nn { #1 } { #2 }
  )
 }

\seq_new:N \l_alec_cycle_seq
\cs_new_protected:Npn \alec_cycle:nn #1 #2
 {
  \seq_set_split:Nnn \l_alec_cycle_seq { , } { #2 }
  \seq_use:Nn \l_alec_cycle_seq { #1 }
 }
\ExplSyntaxOff

\newcommand\Div{\mathrel{\big|}}
\newcommand\nDiv{\mathrel{\not\big|}}
\newcommand\relmiddle[1]{\mathrel{}\middle#1\mathrel{}}
\newcommand{\RNum}[1]{\uppercase\expandafter{\romannumeral #1\relax}}

%%%%%%%%%%%%%%%%%%%%%%%%%%%%%%%%%%%%%%%%%%%%
%\renewcommand{\proofname}{\bf pf:}
\newtheoremstyle{mystyle}% custom style
  {6pt}{15pt}%      top and bottom margin
  {}%               content style
  {}%               indent
  {\bf}%            head style
  {.}%              after head
  {1em}%            distance between head and content
  {}%               Theorem head spec (can be left empty, meaning 'normal')

\theoremstyle{mystyle}
\newtheorem{theorem}{Theorem}
\newtheorem{formula}{Formula}
\newtheorem{conclusion}{Conclusion}
\newtheorem{lemma}{Lemma}
\newtheorem{remark}{Remark}
\newtheorem{observation}{Observation}
\newtheorem*{observation*}{Observation}
\newtheorem{definition}{Def}
\newtheorem{exercise}{Ex}[subsection]
\newtheorem{example}{Eg}[subsection]
\newtheorem{fact}{Fact}[subsection]
\newtheorem{prop}{Prop}[subsection]
\newtheorem{coro}{Coro}[subsection]

%%%%%%%%%%%%%%Title information%%%%%%%%%%%%%%
\title{Algebra}
\author{}
\date{\today}

\begin{document}
\maketitle
% \thispagestyle{empty}
% \thispagestyle{fancy}
% \tableofcontents
%%%%%%%%%%%%%include file here%%%%%%%%%%%%%%%
%! TEX root=../main.tex
\section{Group theory}

\subsection{Basic}

\begin{definition}
  A non-empty set $G$ with a binary function $f: G \times G \to G,
  (a, b) \mapsto ab$
  is a {\it group} if it satisfies
  \begin{enumerate}
    \item $(ab)c = a(bc)$.
    \item $\exists 1 \in G$ s.t. $1a = a1 = a, \forall a \in G$.
    \item $\exists a^{-1} \in G$ s.t. $aa^{-1} = a^{-1}a = 1$.
  \end{enumerate}
\end{definition}

CONCON

\begin{definition}
  Let $G$ be a group. Then $G$ is said to be {\it abelian} if
  $\forall a, b \in G, ab = ba$.
\end{definition}

\begin{exercise}
  Let $G$ be a semigroup. Then TFAE (the following are equivalent)
  \begin{enumerate}
    \item $G$ is a group.
    \item For all $a, b \in G$ and the equations $bx=a, yb=a$, each of them
      has a solution in $G$.
    \item $\exists e \in G$ s.t. $ae=a \; \forall a \in G$ and if we fix
      such $e$, then $\forall b \in G \; \exists b' \in G$ s.t. $bb' = e$.
  \end{enumerate}
\end{exercise}

\begin{exercise}
  Let $G$ be a group. Show that
  \begin{enumerate}
    \item $\forall a \in G, a^2 = 1$, then $G$ is abelian.
    \item $G$ is abelian $\iff \forall a, b \in G, (ab)^n = a^n b^n$ for three
      consecutive integer $n$.
  \end{enumerate}
\end{exercise}

\begin{definition}
  Let $G$ be a group and $H \subseteq G, H \ne \phi$.
  Then $H$ is said to be a subgroup of $G$, denoted by $H \le G$, if
  \begin{enumerate}
    \item $\forall a, b \in H, ab \in H$.
    \item $1 \in H$.
    \item $\forall a \in H, a^{-1} \in H$.
  \end{enumerate}
  \underline{useful criterion}:
  $H \le G \iff \forall a, b \in H, ab^{-1} \in H$.
  \begin{proof} \mbox{}
    \begin{description}[style=nextline]
      \item[$\Rightarrow$] $b \in H \implies b^{-1} \in H$, and $a \in H$, so
        $a b^{-1} \in H$.
      \item[$\Leftarrow$]
        \begin{enumerate}
          \item $H \ne \phi \implies \exists a \in H \implies
            aa^{-1} = 1 \in H$.
          \item $1, a \in H \implies 1 a^{-1} = a^{-1} \in H$.
          \item $a, b^{-1} \in H \implies a (b^{-1})^{-1} = ab \in H$.
            \qedhere
        \end{enumerate}
    \end{description}
  \end{proof}
\end{definition}

\begin{example}
  $(\Zb, +, 0) \le (\Qb, +, 0) \le (\Rb, +, 0) \le (\Cb, +, 0)$ ;
  $(\Qbx, \times, 1) \le (\Rbx, \times, 1) \le (\Cbx, \times, 1)$
\end{example}

\begin{example} \mbox{}
  \begin{itemize}
    \item Special linear group $\text{SL}(n, \Fb) = \{\, A \in
      \text{GL}(n, \Fb) \mid \det A = 1 \,\}$
    \item Orthogonal group $\text{O}(n) = \{\, A \in \text{GL}(n, \Rb)
      \mid \tr{A} A = I_n \,\}$
    \item Unitary group $\text{U}(n) = \{\, A \in \text{GL}(n, \Cb) \mid
      A^* A = I_n \,\}$
    \item Special orthogonal group $\text{SO}(n) = \text{SL}(n, \Rb) \cap
      \text{O}(n)$
    \item Special unitary group $\text{SU}(n) = \text{SL}(n, \Cb) \cap
      \text{U}(n)$
  \end{itemize}
\end{example}

\begin{definition}
  Let $f: G_1 \to G_2$. f is called an {\it isomorphism} if
  \begin{enumerate}
    \item $f$ is 1-1 and onto.
    \item $\forall a, b \in G_1, f(ab) = f(a)f(b)$. ({\it homomorphism})
  \end{enumerate}
  , denoted by $G_1 \cong G_2$.
\end{definition}

\begin{remark} (practice)
  \begin{enumerate}
    \item $f(1) = 1$.
    \item $f(a^{-1}) = f(a)^{-1}$.
    \item If $f$ is an isomorphism, then $\exists f^{-1}$ is also a
      homomorphism.
  \end{enumerate}
\end{remark}

\begin{example} \mbox{}
  \begin{itemize}
    \item $\text{U}(1) = \{\, z \in \Cbx \mid \bar{z}z = 1 \,\},
      z = \cos \theta + \sin \theta i$
    \item $\text{SO}(2) = \left\{ \begin{pmatrix}
        \cos \theta & - \sin\theta \\
        \sin \theta & \cos\theta
      \end{pmatrix} : \theta \in \Rb \right\}$
  \end{itemize}
  notice that $\text{U}(1) \cong \text{SO}(2)$.
  $S^1 = \{\, (a, b) \in \Rb^2 \mid a^2 + b^2 = 1 \,\}$, 可被賦予群的結構.
\end{example}

\begin{example}
  Let $A \in \text{SU}(2) \implies A = \begin{pmatrix}
    \alpha & \beta \\
    -\bar{\beta} & \bar{\alpha} \end{pmatrix},
    \alpha\bar{\alpha} + \beta\bar{\beta} = 1, \alpha, \beta \in \Cb$.
\end{example}

\underline{Quaternion}(四元數): $\Hb = \{\,
  a + bi + cj + dk \mid a, b, c, d \in \Rb \,\}$ with $i^2 = j^2 = k^2 = -1,
  ij = k, jk = i, ki = j (\implies ij = -ji)$.

Let $x = a + bi + cj + dk, \bar{x} = a - bi - cj - dk$, then
$N(x) = x\bar{x} = a^2 + b^2 + c^2 + d^2$, For $x \ne 0, N(x) \ne 0,
x^{-1} = \frac{1}{N(x)} \bar{x}$

Now, for $x = a + bi + cj + dk = (a + bi) + (c + di) j$.
So $\text{SU}(2) \cong \{\, x \in \Hbx \mid N(x) = 1 \,\}$.
$S^3 = \{\, (a, b, c, d) \in \Rb^4 \mid a^2 + b^2 + c^2 + d^2 = 1 \,\}$,
可被賦予群的結構.

$\bigstar$ The only spheres with continuous group law are $S^1, S^3$.

\begin{exercise}
  Find a way to regard $M_{n\times n}(\Hb)$ as a subset of
  $M_{2n \times 2n}(\Cb)$, which preserves addition and multiplication,
  and then there is a way to characterize $\text{GL}(n, \Hb)$.
\end{exercise}

\begin{definition}[symplectic group]
  $\text{Sp}(n, \Fb) = \{\, A \in \text{GL}(2n, \Fb) \mid \tr{A} JA = J \,\}$
  where $J = \begin{pmatrix} O & I_n \\ -I_n & O \end{pmatrix}$.
  ($\tr{A} JA = J$ preserving non-degenerate skew-symmetric forms)

  $\text{Sp}(n) = \{\, A \in \text{GL}(n, \Hb) \mid A^* A = I_n \,\}$.
\end{definition}

\begin{exercise}
  Show $\text{Sp}(n) \cong \text{U}(2n) \cap \text{Sp}(n, \Cb)$.
\end{exercise}

\underline{Ques}: Find the smallest subgroup of $\text{SU}(2)$ containing
$\begin{pmatrix}i & 0 \\ 0 & -i\end{pmatrix}$.


%! TEX root=../main.tex
\subsection{Week 2}
\subsubsection{Permutation groups and Dihedral groups}
\begin{definition}
  A permutation of a set $B$ is a 1-1 and onto function from $B$ to $B$.

  Let $S_B \defeq \text{the set of permutations of $B$}$. Then
  $(S_B, \cdot, {\rm Id}_B)$ forms a group.

  If $B = \{a_1, \dots, a_n\}$, then $S_B \cong S_{\{1,\dots,n\}}$ and write
  $S_n = S_{\{1,\dots,n\}}$, called the symmetric group of degree $n$.
\end{definition}

\begin{theorem}[Cayley theorem]
  Any group is isomorphic to a subgroup of some permutation group.
  
  (Hint): Let $G$ be a group. Set $B = G$. Consider $a \in G$ as
  $\sigma_a: G \to G, x \mapsto ax$.
  Then $\sigma_a \in S_G \implies G \le S_G$.
\end{theorem}

\begin{fact}
  $S_n$ is a finite group of order $n!$, i.e. $\abs{S_n} = n!$.
  \begin{proof}
    EASY =O
  \end{proof}
\end{fact}

\underline{Cyclic notation}: $\sigma \in S_5$, say $\sigma = \begin{pmatrix}
  1 & 2 & 3 & 4 & 5 \\
  4 & 3 & 5 & 1 & 2
\end{pmatrix}$.
Write $\sigma = \cycle{1, 4}\cycle{2, 3, 5}$.

$\Rightarrow$ Any permutation can be written as a product of disjoint cycles.

\begin{example}
  In $S_7$, $\sigma_1 = \cycle{1,2,3}\cycle{4,5,6}\cycle{7},
  \sigma_2 = \cycle{1,3,5,6}\cycle{2,4,7}$.

  Then $\sigma_1\sigma_2 = \cycle{2,5,4,7,3,6},
  \sigma_1^{-1} = \cycle{1,3,2}\cycle{4,6,5}$.
\end{example}

\begin{definition}
  A 2 cycle is called a {\bf transposition}.
\end{definition}

\begin{example}
  $\cycle{1,2,3} = \cycle{1,3}\cycle{1,2},
  \cycle{1,2,3,4,5} = \cycle{1,5}\cycle{1,4}\cycle{1,3}\cycle{1,2}$.

  Any permutation is a product of 2 cycles.
\end{example}

\underline{Useful formula}: $\sigma \in S_n$,
$\sigma \cycle{j_1, \dots, j_m} \sigma^{-1} =
\cycle{\sigma(j_1),\dots,\sigma(j_m)}$.

\begin{example}
  Let $\sigma = \cycle{1,2,3}\cycle{4,5,6,7}$,
  $\sigma \cycle{2,3,4} \sigma^{-1} = \cycle{3,1,5}$.
\end{example}

\begin{proof}
  Note that both sides are functions. For $i \in \{1,\dots,n\}$,
  \begin{enumerate}[\underline{Case \arabic*}:]
    \item $\exists k$ s.t.  $\sigma(j_k) = i$, CONCON
    \item Otherwise, CONCON
  \end{enumerate}
\end{proof}

\begin{fact}
  $S_n = \gen{\cycle{1,2},\dots,\cycle{1,n}}$.
  \begin{proof}
    $\cycle{1,i}^{-1} = \cycle{1,i}$ and
    $\cycle{i,j} = \cycle{1,i}\cycle{1,j}\cycle{1,i}^{-1}$.
  \end{proof}
\end{fact}

\begin{definition}
  Let $G$ be a group and $S \subset G$. The subgroup generated by $S$ defined
  to be the smallest subgroup of $G$ which contains $S$, denoted by
  $\gen{S}$.
\end{definition}

\begin{exercise} \mbox{}
  \begin{enumerate}
    \item $S_n = \gen{\cycle{1,2},\cycle{2,3},\dots,\cycle{n-1,n}}, \quad n \ge 2$.
    \item $S_n = \gen{\cycle{1,2},\cycle{1,2,\dots,n}}, \quad n \ge 2$.
  \end{enumerate}
\end{exercise}

\begin{definition}
  $A_n = \{ \text{even permutations of $S_n$} \} \le S_n,
  \abs{A_n} = \frac{n!}{2}$.
\end{definition}

\begin{exercise} \mbox{}
  \begin{enumerate}
    \item $A_n = \gen{\cycle{1,2,3},\cycle{1,2,4},\dots,\cycle{1,2,n}}, n \ge 3$.
    \item $A_n = \gen{\cycle{1,2,3},\cycle{2,3,4},\dots,\cycle{n-2,n-1,n}}, n \ge 3$.
  \end{enumerate}
\end{exercise}

\begin{remark}
  $\gen{S} = \bigcap\limits_{S \subseteq H \le G} H =
  \{ a_1a_2\dots a_k \mid k \in \Nb, a_i \in S \cup S^{-1}\} \cup \{1\}$
\end{remark}

The orthogonal transformations on $\Rb^2$: $\text{O}(2)$.

Let $A = \begin{pmatrix}a_1 & a_2 \\ b_1 & b_2\end{pmatrix} \in \text{O}(2)$.

略... (這邊討論旋轉和反射的矩陣)

\begin{enumerate}[\underline{Case \arabic*}:]
  \item $A = \begin{pmatrix}
      \cos\alpha & -\sin\alpha \\
      \sin\alpha & \cos\alpha
    \end{pmatrix}$ is counterclockwise roration w.r.t. $\alpha$.
  \item $A = \begin{pmatrix}
      \cos\alpha & \sin\alpha \\
      \sin\alpha & -\cos\alpha
    \end{pmatrix}$ is the reflection.
    $A^2 = I_2 \implies$ eigenvalues are $\pm 1$.

    Easy to show that $\text{L}_A(v) = v - 2 \inpd{v, v_2} v_2$.
\end{enumerate}

$\text{O}(2) = \{\text{rotations}\} \cup \{\text{reflections}\}$.

\begin{definition}
  The dihedral group $D_n$ is the group of symmetries of a regular $n$-gon.

  In general, $D_n = \gen{ {\rm T, R} \mid
  {\rm T}^n = 1, {\rm R}^2 = 1, {\rm TR} = {\rm RT}^{-1} } \le O(2)
  \le S_n, \abs{D_n} = 2n$.
\end{definition}

\begin{definition}
  Let $\rm T$ be a linear transformation from $\Rb^n \to \Rb^n$.
  \begin{itemize}
    \item $\rm T$ is called a rotation if $\exists$ a $\rm T$-invariant
      subspace $W \subseteq \Rb^n$ with $\dim W = 2$ s.t.
      $\begin{cases}
        {\rm T} \big|_W \text{~is a rotation} \\
        {\rm T} \big|_{W^\bot} = {\rm id}_{W^\bot}
      \end{cases}$
    \item $\rm T$ is called a reflection if $\exists$ a $\rm T$-invariant
      subspace $W \subseteq \Rb^n$ with $\dim W = 1$ s.t.
      $\begin{cases}
        {\rm T} \big|_W = -{\rm id}_W \\
        {\rm T} \big|_{W^\bot} = {\rm id}_{W^\bot}
      \end{cases}$
  \end{itemize}
\end{definition}

\underline{Main result}: the group of orthogonal transformations
$= \gen{ \text{rotations}, \text{reflections} }$.

\begin{prop}
  For ${\rm T}: \Rb^n \to \Rb^n$, $\exists$ a $\rm T$-invariant
  subspace $W \subseteq \Rb^n$ with $1 \le \dim W \le 2$.
  \begin{proof}
    Let $A=[{\rm T}]_\alpha \in M_{n\times n}(\Rb) \subseteq M_{n\times n}(\Cb)$.
    Consider $\widetilde{{\rm L}_A}: \Cb^n \to \Cb^n, v \mapsto Av$.

    Then $\exists$ an eigenvalue $\lambda \in \Cb$ and an eigenvector
    $v \in \Cb^n$ for $\widetilde{{\rm L}_A}$.
    Let $\lambda = \lambda_1 + \lambda_2 i, v = v_1 + v_2 i$. By definition,
    we have
    \[
      Av = \widetilde{{\rm L}_A}(v) = \lambda v =
      (\lambda_1 + \lambda_2 i)(v_1 + v_2 i)
      \implies \begin{cases}
        Av_1 = \lambda_1 v_1 - \lambda_2 v_2 \\
        Av_1 = \lambda_2 v_1 + \lambda_1 v_2
      \end{cases},
    \]
    so $W = \gen{ v_1, v_2 }$.
  \end{proof}
\end{prop}

\begin{exercise} \mbox{}
  \begin{enumerate}
    \item If $\rm T$ is orthogonal, then $W^\bot$ is also $\rm T$-invariant.
    \item Use induction on $n$ to show the main result.
  \end{enumerate}
\end{exercise}

For $n = 3, A \in \text{O}(3)$, we have $A \sim \begin{pmatrix}
  \cos\alpha & -\sin\alpha & \\
  \sin\alpha & \cos\alpha & \\
   & & \pm 1
\end{pmatrix}$.

\subsubsection{Cyclic groups and internal direct product}

\begin{definition}
  If $G = \gen{a} = \{ \dots, a^{-2}, a^{-1}, a, 1, a, a^2, \dots \}
  = \{\, a^n \mid n \in \Zb \,\}$, then $G$ is a cyclic group generated by $a$.
\end{definition}

\begin{example}
  $\Zb = \gen{1} = \gen{-1}$.
\end{example}

\begin{example}
  Let $A = \begin{pmatrix}
    \cos \frac{2\pi}{n} & -\sin \frac{2\pi}{n} \\ 
    \sin \frac{2\pi}{n} & \cos \frac{2\pi}{n}
  \end{pmatrix} \in \text{SO}(2)$. Then $\gen{A} =
  \{ I_2, A, A^2, \dots, A^{n-1} \}$ and $A^n = I_2, A^m = A^r$ where
  $m \equiv r \pmod n$.
\end{example}

\begin{example}
  $\quot{\Zb}{n\Zb} = \{ \ob{0}, \ob{1}, \dots, \ob{(n-1)} \}$ with
  $\ob{j} = \{\, m \in \Zb \mid m \equiv j \pmod n \,\}$.

  Define $\ob{i} + \ob{j} = \begin{cases}
    \ob{i+j} & \text{~if~} 0 \le i + j \le n \\
    \ob{i+j-n} & \text{~otherwise}
  \end{cases} \implies (\quot{\Zb}{n\Zb}, +, \ob{0})$ forms a group.
\end{example}

\begin{remark}
  $\ob{i} \times \ob{j} = \ob{i \times j}$.
  \begin{itemize}
    \item 略
    \item If $\gcd(j, n) = d, \exists h, k \in \Zb$ s.t. $hj + kn = d$.
  \end{itemize}
\end{remark}

\begin{definition}
  $\left( \quot{\Zb}{n\Zb} \right)^\times = \{\, j \in \quot{\Zb}{n\Zb} \mid
  \gcd(j,n) = 1 \,\} \implies \left(\left( \quot{\Zb}{n\Zb} \right)^\times,
  \times, \ob{1} \right)$ forms a group.
\end{definition}

\begin{example}
  略... 簡化剩餘系, 原根 (generator) ($1, 2, 4, p^k, 2p^k$, $p$ is an odd prime)
\end{example}

\begin{definition} \mbox{}
  \begin{itemize}
    \item The {\bf order} of a finite gorup $G$ is the number of elements in
      $G$, denoted by $\abs{G}$.
    \item Let $a \in G$, the order of $a$ is defined to be the least positive
      integer $n$ s.t. $a^n = 1$, denoted by $\ord(a) = n$.
    \item If $a^n \ne 1 \quad \forall n \in \Nb$, then we call
      ``$a$ has infinte order''.
  \end{itemize}
\end{definition}

\begin{prop}
  Let $G = \gen{a}$ with $\ord(a) = n$. Then
  \begin{enumerate}
    \item $a^m = 1 \iff n \mid m$.
      \begin{proof} \mbox{}
        \begin{description}
          \item[$\Leftarrow:$] Let $m = dn$, then $a^m = (a^n)^d = 1$.
          \item[$\Rightarrow:$] Let $m = qn + r, 0 \le r < n$.
            If $r \ne 0$, then $a^r = a^{m - qn} = (a^m)(a^n)^{-q} = 1$.
            But $r < n$, which is a contradiction.
            Hence $r = 0 \implies n \mid m$. \qedhere
        \end{description}
      \end{proof}
    \item $\ord(a^r) = n / \gcd(r, n)$.
      \begin{proof}
        Let $\gcd(r, n) = d, n = dn', r = dr'$ with $\gcd(n', r') = 1$.
        Plan to show ``$\ord(a^r) = n'$.''
        \begin{itemize}
          \item $(a^r)^{n'} = a^{r'dn'} = (a^n)^{r'} = 1 \implies \ord(a^r) \mid n'$.
          \item $1 = (a^r)^{\ord(a^r)} = a^{r \ord(a^r)} \implies
            n \mid r \ord(a^r) \implies n' \mid r' \ord(a^r) \implies
            n' \mid \ord(a^r)$.
        \end{itemize}
      \end{proof}
  \end{enumerate}
\end{prop}

\begin{prop}
  Any subgroup of a cyclic group is cyclic.
  \begin{proof}
    Let $G = \gen{a}$ and $H \le G$. If $H = \{1\}$, then
    $H = \gen{1}$, done!

    Otherwise, $d = \min \{ m \in \Nb \mid a^m \in H \}$, by well-ordering
    axiom. Claim $H = \gen{a^d}$.
    \begin{description}
      \item[$\supset:$] $a^d \in H$ by the definition of $d$.
      \item[$\subset:$] $\forall a^m \in H$, write $m = qd + r, 0 \le r < d$.
        If $r \ne 0$, then $a^r = a^{m - qd} = a^m (a^d)^{-q} \in H$, which
        is a contradiction. Hence $r = 0 \implies d \mid m$.
    \end{description}
  \end{proof}
\end{prop}

\begin{exercise} \mbox{}
  \begin{enumerate}
    \item $\ord(a) = \ord(a^{-1}) = n$.
    \item $\gen{a^r} = \gen{a^{\gcd(n, r)}}$.
    \item $\gen{a^{r_1}} = \gen{a^{r_2}} \iff \gcd(n, r_1) = \gcd(n, r_2)$.
    \item $\forall m \mid n, \exists! H \le \gen{a}$ s.t.
      $\abs{H} = m$. Conversely, if $H \le \gen{a}$, then $\abs{H} \mid n$.
  \end{enumerate}
\end{exercise}

\begin{prop}
  Let $G = \gen{a}$. Then
  \begin{enumerate}
    \item $\ord(a) = n \implies G \cong \quot{\Zb}{n\Zb}$
    \item $\ord(a) = \infty \implies G \cong \Zb$
  \end{enumerate}
  \label{prop:eqzg}
  \begin{exercise}
    Show Prop \ref{prop:eqzg}.
  \end{exercise}
\end{prop}

\begin{definition}
  Let $G_1, G_2 \le G$. $G$ is the internal direct product of $G_1, G_2$ if
  $G_1 \times G_2 \to G, (g_1, g_2) \mapsto g_1g_2$ is an isom.
\end{definition}

\begin{remark}
  In this case, we find that
  \begin{itemize}
    \item $G = G_1 G_2 = \{\, g_1 g_2 \mid g_1 \in G_1, g_2 \in G_2 \,\}$.
    \item $G_1 \cap G_2 = \{ 1 \}$. (consider $a \ne 1 \in G_1 \cap G_2$, then
      $(1, a) \mapsto a, (a, 1) \mapsto a$, but the function is 1-1, which
      is a contradiction.)
    \item If $a \in G$ with $a = g_1g_2 = g_1'g_2'$, then
      $(g_1')^{-1}g_1 = (g_2')g_2^{-1} \in G_1 \cap G_2 = \{ 1 \} \implies
      \begin{cases} g_1 = g_1' \\ g_2 = g_2'\end{cases}$.
    \item For $g_1 \in G_1, g_2 \in G_2, (g_1, g_2) = (g_1, 1)(1, g_2) =
      (1, g_2)(g_1, 1) \implies g_1g_2 = g_2g_1$.
  \end{itemize}
\end{remark}

\begin{exercise} TFAE
  \begin{enumerate}
    \item $G$ is the internal direct product of $G_1, G_2$.
    \item $\forall a \in G, \exists! g_1 \in G_1, g_2 \in G_2$ s.t.
      $a = g_1g_2$ ; $\forall g_1 \in G_1, g_2 \in G_2, g_1g_2 = g_2g_1$.
    \item $G_1 \cap G_2 = \{ 1 \}$ ; $G = G_1G_2$ ;
      $\forall g_1 \in G_1, g_2 \in G_2, g_1g_2 = g_2g_1$.
  \end{enumerate}
\end{exercise}

\begin{example} \mbox{}
  \begin{enumerate}
    \item $G = \quot{\Zb}{6\Zb} = \{ \ob{0}, \ob{1}, \ob{2}, \ob{3}, \ob{4}, \ob{5} \},
      G_1 = \{ \ob{0}, \ob{3} \}, G_2 = \{ \ob{0}, \ob{2}, \ob{4} \}$.
      We have $G \cong G_1 \times G_2$.
    \item $G = S_3, G_1 = \gen{\cycle{1,2}}, G_2 = \gen{\cycle{1,2,3}}$.
      We have $G_1 \times G_2 \not\cong G$ since $\cycle{1,2}\cycle{1,2,3} \ne
      \cycle{1,2,3}\cycle{1,2}$.
  \end{enumerate}
\end{example}

\begin{example}
  $G = S_3, G_1 = \gen{\cycle{1,2}}, G_2 = \gen{\cycle{2,3}},
  G_1G_2 = \{ 1, \cycle{1,2}, \cycle{2,3}, \cycle{1,2,3} \} \not\le G$
  since $\cycle{1,3,2} = \cycle{1,2,3}^{-1} \not\in G_1G_2$.
\end{example}

\begin{prop}
  Let $H, K \le G$. Then $HK \le G \iff HK = KH$.
  \begin{proof} \mbox{}
    \begin{description}
      \item[$\Rightarrow:$] $\begin{cases} H \le HK \\ K \le HK \end{cases}
          \implies KH \subseteq HK$ ;
          $\forall hk \in HK, \exists h'k' \in HK$ s.t. $(hk)(h'k') = 1 \implies
          hk = (k')^{-1}(h')^{-1} \in KH \implies HK \subseteq KH$.
      \item[$\Leftarrow:$] For $h_1k_1, h_2k_2 \in HK$, $(h_1k_1)(h_2k_2)^{-1}
        = h_1k_1k_2^{-1}h_2^{-1} = h_1h'k' \in HK$.
    \end{description}
  \end{proof}
\end{prop}


%! TEX root=../main.tex
\subsection{Week 3}
\subsubsection{Coset and Quotient Group}
Let $f: G_1 \to G_2$ be a group homo. Define $\Image f \defeq f(G_1)$.

Notice that $\Image f \le G_2$.
\begin{proof}
  Let $z_1 = f(a_1), z_2 = f(a_2)$, then
  $z_1z_2^{-1} = f(a_1)f(a_2)^{-1} = f(a_1)f(a_2^{-1}) = f(a_1a_2^{-1}) \in
  \Image f$.
\end{proof}

\begin{definition}
  $\ker f \defeq \{\, x \in G_1 \mid f(x) = 1 \,\} \le G_1$.
\end{definition}

\begin{fact} \mbox{}
  \begin{enumerate}
    \item $x \in (\ker f) a \iff f(x) = f(a)$.
    \item $\ker f = \{ 1 \} \iff f$ is 1-1.
  \end{enumerate}
\end{fact}

\begin{definition}
  Let $H \le G$, $\forall a \in G, Ha$ is called a {\bf right coset} of $H$
  in $G$.
\end{definition}

\begin{fact} \mbox{}
  \begin{enumerate}
    \item For 2 right cosets $Ha, Hb$, either $Ha = Hb$ or $Ha \cap Hb = \phi$
      must hold.
    \item $\{\, Ha : a\in G \,\}$ forms a partition of $G$.
  \end{enumerate}
\end{fact}

\begin{theorem}[Lagrange]
  Let $\abs{G} < \infty$ and $H \le G$, $\abs{H} \mid \abs{G}$.
  \begin{proof}
  \end{proof}
\end{theorem}

\begin{remark}
  $r$ is called the {\bf index} of $H$ in $G$, denoted by $[G:H]$.
  (The concept of index can be extended to infinite $G, H$.)
\end{remark}

\begin{exercise}
  no subgroup of $A_4$ has order $6$.
  (converse of Lagrange thm. is false.)
\end{exercise}

\begin{coro}
  If $\abs{G} = p$ is a prime in $\Zb$, then $G$ is cyclic.
  \begin{proof}
  \end{proof}
\end{coro}

\begin{coro}
  If $\abs{G} < \infty, a \in G$, then $a^{\abs{G}} = 1$.
  \begin{proof}
  \end{proof}
\end{coro}

\begin{remark} \mbox{}
  \begin{enumerate}
    \item Let $H \le G, a \in G$, $aH$ is called a {\bf left coset}.
    \item $\{ \text{right cosets of $H$} \} \leftrightarrow
      \{ \text{right cosets of $H$} \}$ by $Ha \mapsto a^{-1}H$.
  \end{enumerate}
\end{remark}

\underline{Ques}: How to make $\{\, aH : a \in G \,\}$ to be a group?
For $aH, bH$, we must have $(aH)(bH) = abH$.

In general, $(aH)(bH) = abH$ is not well-defined.

\begin{example}
  Let $H = \gen{\cycle{1,2}} \le S_3$. $a_1 = \cycle{1,3}, a_2 = \cycle{1,2,3},
b_1 = \cycle{1,3,2}, b_2 = \cycle{2,3}$. 出慘點
\end{example}

If we hope $a_1b_1H = a_2b_2H$, then we need $(a_1b_1)^{-1}a_2b_2 \in H$.
\[
  b_1^{-1}a_1^{-1}a_2b_2 = b_1^{-1}b_2b_2^{-1}a_1^{-1}a_2b_2
\]
Notice that $b_1^{-1}b_2, a_1^{-1}a_2 \in H$, so we need
$b_2^{-1}a_1^{-1}a_2b_2 \in H$.

\begin{definition}
  Let $H \le G$. $H$ is said to be {\bf normal subgroup} of $G$ if
  $\forall g \in G, h \in H, g^{-1}hg \in H \quad
  (\text{or~} g^{-1}Hg \subseteq H)$, denoted by $H \lhd G$.
\end{definition}

\begin{definition}
  Let $H \lhd G$. The set $\{\, aH \mid a \in G \,\}$ forms a group under
  $(aH)(bH) = abH, a,b \in G$. We call it the {\bf quotient group}
  of $G$ by $H$, denoted by $\quot{G}{H}$.

  (Note: The indentity is $H = hH$ and $(aH)^{-1} = a^{-1}H$.)
\end{definition}

\begin{remark}
  Define $q: G \to \quot{G}{H}, a \mapsto aH$, called the quotient homomorphism.
\end{remark}

\begin{exercise}
  Let $H \le G$. Then TFAE
  \begin{enumerate}[(a)]
    \item $H \lhd G$.
    \item $\forall x \in G, xHx^{-1} = H$.
    \item $\forall x \in G, xH = Hx$. \label{eq:xh=hx}
    \item $\forall x, y \in G, (xH)(yH) = (xy)H$.
  \end{enumerate}
\end{exercise}

\underline{Ques}: How to find a normal subgroup of $G$?

\begin{prop} \mbox{}
  \begin{enumerate}
    \item If $G$ is abelian, then $\forall H \le G \leadsto H \lhd G$.
      (done by \ref{eq:xh=hx})
    \item If $H \le G$ with $[G:H] = 2$, then $H \lhd G$.
      \begin{example}
        $n \le 3, [S_n:A_n] = 2 \implies A_n \lhd S_n$.
      \end{example}
      \begin{proof}
        We can write $G = H \cup Ha = H \cup aH \implies aH = Ha,
        \forall a \not\in H$.
      \end{proof}
  \end{enumerate}
\end{prop}

\begin{definition}
  Define the center of $G$ to be $Z_G = \{\, a \in G
  \mid ax = xa, \forall x \in G \,\} \le G$.
\end{definition}

\begin{prop} \mbox{}
  \begin{enumerate}
    \item $Z_G \lhd G$. (by \ref{eq:xh=hx} and def.)
    \item If $\quot{G}{Z_G}$ is cyclic, then $G$ is abelian.
      \begin{proof}
        Let $\quot{G}{Z_G} = \gen{aZ_G}$, (let $\ob{a} \defeq aZ_G$) for some
        $a \in G$.
        For $x_1, x_2 \in G$, let $x_1 = a^{k_1}z_1, x_2 = a^{k_2}z_2$, then
        $x_1x_2 = a^{k_1+k_2}z_1z_2 = x_2x_1$. ($z_i$ 可以各種交換)
      \end{proof}
  \end{enumerate}
\end{prop}

\begin{definition}
  The commutator of $G$ is define to be $[G,G] = \gen{xyx^{-1}y^{-1} \mid
  x,y \in G}$.
\end{definition}

\begin{prop}
  $[G,G] \lhd G$ ; $[G,G] = 1 \iff G$ is abelian.
  \begin{proof}
    $\forall x \in G, a \in [G,G], xax^{-1} = xax^{-1}a^{-1}a$ and
    $xax^{-1}a^{-1}, a \in [G,G]$.
  \end{proof}
\end{prop}

\begin{exercise} \mbox{}
  \begin{enumerate}
    \item If $H \le S_n$ and $\exists \sigma \in H$ is odd, then $[H:H\cap A_n] = 2$.
    \item For $n \ge 3$, $[S_n, S_n] = A_n$.
  \end{enumerate}
\end{exercise}

\begin{exercise}
  Let $H \le G$. Then  $H \lhd G$ and $\quot{G}{H}$ is abelian $\iff$
  $[G,G] \le H$.
  (hint: $\quot{G}{[G,G]}$ is "max" among all abelian quotient groups)
\end{exercise}


\subsubsection{Isomorphism theorems \& Factor theorem}
\begin{theorem}[1st isomorphism theorem]
  Let $f: G_1 \to G_2$ be a group homo. Then $\quot{G_1}{\ker f} \cong \Image f$.
  \begin{proof}
    Define $\varphi: a \ker f \mapsto f(a)$.
    \begin{itemize}
      \item well-defined: $a \ker f = b \ker f \implies a^{-1}b \in \ker f
        \implies f(a^{-1}b) = 1 \implies f(a)^{-1}f(b) = 1 \implies f(a)=f(b)$.
      \item group homo: $\varphi\left((a \ker f)(b\ker f)\right) = 
        \varphi(ab \ker f) = f(ab) = f(a)f(b) =
        \varphi(a\ker f)\varphi(b\ker f)$.
      \item onto: by def. of $\Image f$.
      \item 1-1: $f(a) = f(b) \implies a \ker f = b \ker f$ (easy).
      \end{itemize}
  \end{proof}
\end{theorem}

\begin{theorem}[Factor theorem]
  Let $f: G_1 \to G_2$ be a group homo. and $H \lhd G_1, H \le \ker f$. Then
  $\exists$ a group homo. $\varphi: \quot{G}{H} \to G_2$ s.t.
  \[
    \begin{tikzcd}
      G_1 \arrow{r}{q} \arrow[swap]{dr}{f} & \quot{G}{H} \arrow{d}{\varphi} \\
      & G_2
    \end{tikzcd}
  \]
\end{theorem}

\begin{example}
  Let $G = \gen{a}$ with $\ord(a) = n$. Then $G \cong \quot{\Zb}{n\Zb}$.
  (1st isom. thm.)
\end{example}

\begin{example}
  $\varphi: \Zb \to \quot{\Zb}{2\Zb}, 4\Zb \le 2\Zb$, so by factor thm.,
  $\quot{\Zb}{4\Zb} \to \quot{\Zb}{2\Zb}$.
\end{example}

\begin{example}
  $\det: \text{GL}(n, \Fb) \to \Fbx \implies
  \quot{\text{GL}(n, \Fb)}{\text{SL}(n, \Fb)} \cong \Fbx$
\end{example}

\begin{example}
  $\sgn: S_n \to \{ \pm 1 \} \implies \quot{S_n}{A_n} \cong \{ \pm 1 \}$
\end{example}

\begin{theorem}[2nd isomorphism theorem]
  Let $H \le G, K \lhd G$. Then $\quot{HK}{K} \cong \quot{H}{H\cap K}$.
  \begin{proof}
    First, $\begin{cases}H\le G \\ K \lhd G\end{cases} \implies HK = KH
      \implies HK \le G$ ; $K \lhd G \implies K \lhd HK$.

    Define $\varphi: H \to \quot{HK}{K}, h \mapsto hK$. which is a group homo.
    \begin{itemize}
      \item onto: $\forall (hk) K, hkK = hK$, so $\varphi(h) = hK = hkK$.
      \item Find $\ker \varphi$: $a \in \ker \varphi \iff \begin{cases}
          a \in H \\
          aK = K
        \end{cases} \iff a \in H \cap K$, so $\ker \varphi = H\cap K$.
    \end{itemize}
    Then by 1st isom. thm.
  \end{proof}
\end{theorem}

\begin{example}
  $G = \text{GL}(2, \Cb), H = \text{SL}(2, \Cb), K = \Cbx I_2 = Z_G \lhd G$.

  By 2nd isom. thm., $\quot{G}{K} \cong \quot{H}{\{\pm I_2\}}$.
  ($G = HK, \{\pm I_2 \} = H \cap K$)

  projective linear group: $\text{PGL}(2, \Cb) = \quot{G}{K}$.

  projective special linear group: $\text{PSL}(2, \Cb) = \quot{H}{H\cap K}$.
\end{example}

齊次座標...OTL

\begin{exercise} \mbox{}
  \begin{enumerate}
    \item Let $H_1 \lhd G_1, H_2 \lhd G_2$. Then $(H_1 \times H_2) \lhd
      (G_1 \times G_2)$ and $\quot{G_1\times G_2}{H_1\times H_2} \cong
      \quot{G_1}{H_1} \times \quot{G_2}{H_2}$.
    \item Let $H \lhd G, K \lhd G$ s.t. $G = HK$. Then
      $\quot{G}{H\cap K} \cong \quot{G}{H} \times \quot{G}{K}$.
  \end{enumerate}
\end{exercise}

\begin{exercise}
  Let $H \lhd G$ with $[G:H] = p$, which is a prime in $\Zb$. Then
  $\forall K \le G$, either \begin{enumerate*}[(1)]
    \item $K \le H$ or
    \item $G = HK$ and $[K:K\cap H] = p$.
  \end{enumerate*}
\end{exercise}

\begin{theorem}[3rd isomorphism theorem]
  Let $K \lhd G$.
  \begin{enumerate}
    \item There is a 1-1 correspondence between $\{\, H \le G \mid K \le H \,\}$
      and $\{ \text{subgroups of $\quot{G}{K}$} \}$. ($H \lhd G$ ... normal)
      \begin{proof}
        Define $\varphi: H \mapsto \quot{H}{K}$. ($\quot{H}{K} \le \quot{G}{K}$)
        \begin{itemize}
          \item 1-1: Assume $\quot{H_1}{K} = \quot{H_2}{K}$.
            For $a \in H_1$, $aK \in \quot{H_1}{K} = \quot{H_2}{K}$.
            so $\exists b \in H_2$ s.t. $aK = bK \implies b^{-1}a \in K \le H_2
            \implies a \in b H_2 = H_2$. So $H_1 \le H_2$. By symmetry,
            $H_2 \le H_1$, and thus $H_1 = H_2$.
          \item onto: Given a subgroup $Q$ of $\quot{G}{K}$, consider
            $H = q^{-1}(Q)$ where $q: G\to \quot{G}{K}$.
            \begin{itemize}
              \item $H \le G$: $\forall a, b \in H, q(a), q(b) \in Q \implies
                q(a)q(b)^{-1} \in Q \implies q(ab^{-1}) \in Q \implies
                ab^{-1} \in H \implies H \le G$.
              \item $K \le H$: $\forall a \in K, q(a) = aK = K \in Q \implies
                a \in H \implies K \le H$.
              \item $Q = \quot{H}{K}$: $\forall aK \in Q, aK = q(a) \implies
                a \in H \implies aK \in \quot{H}{K} \implies
                Q \subseteq \quot{H}{K}$.
                And $\forall aK \in \quot{H}{K} (a \in H), q(a) \in Q \implies
                  \quot{H}{K} \subseteq Q$. So $Q = \quot{H}{K}$.
            \end{itemize}
          \item $H \lhd G, K \le H \iff \forall g\in G, gHg^{-1} = H, K \le H
            \iff \forall \ob{g} \in \quot{G}{K}, \ob{g}(\quot{H}{K})\ob{g}^{-1}
            = \quot{H}{K} \iff \quot{H}{K} \lhd \quot{G}{K}$. \qedhere
        \end{itemize}
      \end{proof}
    \item If $H \lhd G$ with $K \le H$, then $\quot{(\quot{G}{K})}{(\quot{H}{K})}
      \cong \quot{G}{H}$.
      \begin{proof}
        Define $\varphi: G \to\quot{(\quot{G}{K})}{(\quot{H}{K})}$ with
        $\varphi: a \mapsto aK(\quot{H}{K})$.
        \begin{itemize}
          \item onto: ... easy.
          \item Find $\ker \varphi$: $a \in \ker \varphi \iff aK(\quot{H}{K})
            = \quot{H}{K} \iff aK \in \quot{H}{K} \iff a \in H$.
        \end{itemize}
        By 1st isom. thm., $\quot{(\quot{G}{K})}{(\quot{H}{K})} \cong
        \quot{G}{H}$.
      \end{proof}
  \end{enumerate}
\end{theorem}

\begin{example}
  $\quot{m\Zb + n\Zb}{m\Zb} \cong \quot{n\Zb}{m\Zb \cap n\Zb}$.
  ($m\Zb + n\Zb = \gcd(m,n)\Zb, m\Zb \cap n\Zb = \lcm(m,n)\Zb$)
\end{example}

\underline{Ques}: $\quot{G}{K} \cong \quot{G'}{K'}$ and $K \cong K'
\centernot\implies G \cong G'$.

\begin{example}
  $Q_8$ and $D_4$
  交給陳力
\end{example}

Extension problem: given two groups $A, B$, how to find $G$ and $K \lhd G$,
s.t. $K \cong A, \quot{G}{K} \cong B$?
($1 \to H \to G \to \quot{G}{H} \to 1$, short exact sequence)

 (e.g. $G = A \times B, K = A \times \{1\}$)

%! TEX root=../main.tex
\subsection{Week 4}
\subsubsection{Universal property and direct sum \& product}
In general, let $f_1: G_1 \to G, f_2: G_2 \to G$ are group homo.
$f_1 \times f_2: G_1 \times G_2 \to G, (a, b) \mapsto f_1(a)f_2(b)$.
But we have $(a, b) = (a, 1)(1, b) = (1, b)(a, 1)$, so
$f_1(a)f_2(b) = f_2(b)f_1(a) \implies $ need $G$ to be abelian.

So we intend to define the direct sum in the category of abelian group.

\underline{Notation}: For abelian groups, we use ``$+$'' to denote the group
operation and ``$0$'' to denote the identity.

\begin{definition}
  Given a non-empty family of abelian groups $\{\, G_s \mid s \in \Lambda \,\}$,
  a (external) direct sum of $\{\, G_s \mid s \in \Lambda \,\}$ is an
  abelian group $\bigoplus_{s\in \Lambda} G_s$ with the embedding mappings
  $i_{s_0}: G_{s_0} \to \bigoplus_{s\in \Lambda} G_s,
  \forall s_0 \in \Lambda$ satisfying the universal property:

  for any abelian group $H$ and group homo. $\varphi_s: G_s \to H
  \forall s \in \Lambda, \quad \exists!$ group homo. $\varphi:
  \bigoplus_{s\in \Lambda} G_s \to H$ s.t. 又一個ㄛ圖
\end{definition}

\begin{theorem}
  $\bigoplus_{s\in \Lambda} G_s$ exists and is unique up to isomorphisms.

  \begin{proof}
    Existence: $\bigoplus_{s\in \Lambda} G_s = \{\, (g_s)_{s\in \Lambda}
      \mid g_s \in G_s, \text{~almost all of the $g_s$' are $0$} \,\}$ and
      \[ i_{s_0}: G_{s_0} \to \bigoplus_{s\in \Lambda} G_s,
        a_{s_0} \mapsto (g_{s_0})_{s\in \Lambda} \text{~with~}
        g_{s_0} = a_{s_0}, g_s = 0, \forall s \ne s_0. \]
        group operaion: $(g_s)_{s \in \Lambda} + (g_s')_{s \in \Lambda}
        \defeq (g_s + g_s')_{s \in \Lambda} \in
        \bigoplus_{s\in \Lambda} G_s$.
        這邊也一個ㄛ圖

    Uniqueness: Assume $\exists$ another $G$ satisfies the universal property,
    一個大ㄛ圖 ($G, \bigoplus_{s\in \Lambda} G_s$ 互相有唯一個映射可以
    keep $i_{s_0}$, $\varphi \circ \psi = \text{id}_{G}, \psi \circ \varphi
    = \text{id}_{\bigoplus_{s\in \Lambda} G_s}$)
  \end{proof}
\end{theorem}

\begin{definition}
  Given a non-empty family of groups $\{\, G_s \mid s \in \Lambda \,\}$,
  a direct product of $\{\, G_s \mid s \in \Lambda \,\}$ is a group
  $\prod_{s\in \Lambda} G_s$ with projections
  $p_{s_0}: \prod_{s\in \Lambda} G_s \to G_{s_0}, \forall s_0 \in \Lambda$
  satifsfying the following universal property:

  for any group $H$ with group homo.
  $\varphi_s: H \to G_s, \forall s \in \Lambda$, $\exists! \varphi:
  H \to \prod_{s\in \Lambda} G_s$ s.t. 又一個ㄛ圖
\end{definition}

\begin{theorem}
  $\prod_{s\in \Lambda} G_s$ exists and is unique up to isomorphisms.

  \begin{proof}
    Existence: $\prod_{s\in \Lambda} G_s = \{\, (g_s)_{s\in \Lambda}
      \mid g_s \in G_s \,\}$ and
      \[ p_{s_0}: \prod_{s\in \Lambda} G_s \to G_{s_0},
        (g_{s_0})_{s\in \Lambda} \mapsto g_{s_0}, \forall s_0 \in \Lambda \]
      \begin{itemize}
        \item group operaion: $(g_s)_{s \in \Lambda} \cdot (g_s')_{s \in \Lambda}
          \defeq (g_s g_s')_{s \in \Lambda} \in \prod_{s\in \Lambda} G_s$.
        \item Define $\varphi$:
          這邊也一個ㄛ圖
          which is uniquely defined.
      \end{itemize}

    Uniqueness: Assume $\exists$ another $G$ satisfies the universal property,
    一個大ㄛ圖 ($G, \prod_{s\in \Lambda} G_s$ 互相有唯一個映射可以
    keep $i_{s_0}$, $\varphi \circ \psi = \text{id}_{G}, \psi \circ \varphi
    = \text{id}_{\prod_{s\in \Lambda} G_s}$)
  \end{proof}
\end{theorem}

\begin{exercise}
  Google the definition of the {\bf direct limit} and show the existence and
  uniqueness.
\end{exercise}

\begin{exercise}
  Google the definition of the {\bf inverse limit} and show the existence and
  uniqueness.
\end{exercise}

\underline{Motivation}: $\zeta_m$ is called an $m$-th root of unity if
$\zeta_m^m = 1$.
\[ \varinjlim\limits_n \quot{\Zb}{2^n\Zb} \cong
\{\, \text{$2^n$-th roots of unity} : n \in \Nb \,\} \]


\[ \varinjlim\limits_n \quot{\Zb}{2^n\Zb} =
  \quot{(\bigoplus_{n\in\Nb} \quot{\Zb}{2^n\Zb})}{
  \gen{ i_k(a) - i_j(f_{kj}(a)) \mid k \le j, a \in \quot{\Zb}{2^k\Zb} }}
\]
where $f_{kj}: \quot{\Zb}{2^k\Zb} \to \quot{\Zb}{2^j\Zb}$.

Inverse limit:
\[
  \varprojlim \quot{\Zb}{2^n\Zb} = \left\{\,
    (n_1, n_2, \dots ) \in \prod_n \quot{\Zb}{2^n\Zb} \middle|
    \forall i < j, n_i \equiv n_j \pmod 2^{i+1}   \,\right\}
\]

\subsubsection{Rings and fields}

\begin{definition}
  A {\bf ring} is sa non-empty set $R$ with two operations $R\times R \to R$
  \[
    (a, b) \mapsto a + b \quad \text{and} \quad (a, b) \mapsto ab
  \]
  satisfying
  \begin{enumerate}
    \item $(R, +, 0)$ is an abelian group.
    \item $(R, \cdot)$ is a semigroup. (if it is a monoid, then it is called
      ``a ring with 1.'')
    \item (Distributive laws) $\forall a, b, c \in \Rb, \begin{cases}
      a(b + c) = ab + ac\\ (b + c)a = ba + ca\end{cases}$
  \end{enumerate}
\end{definition}

\begin{example}
  $\Zb, \Rb, \Cb, \quot{\Zb}{n\Zb}, M_{n\times n}(\Fb)$
\end{example}

\begin{example}
  Let $G$ be an abelian group.
  Define (endomorphism, automorphism)
  \[
    \text{End}(G) \defeq \{\, \text{group homo.~} G \to G \,\} \quad
    \text{Aut}(G) \defeq \{\, \text{group isom.~} G \to G \,\}
  \]
  A natural ring structure on $\text{End}(G)$ is:
  \[
    \forall a \in G, \begin{cases}
      (f+g)(a) \defeq f(a)g(a) \\
      (f\cdot g)(a) \defeq f(g(a))
    \end{cases}
  \]
\end{example}

\begin{example}
  $\Zb\left[\sqrt{2}\right] = \left\{\,
  a + b\sqrt{2} \relmiddle| a, b \in \Zb \,\right\} \subset \Rb$.
\end{example}

\begin{definition}
  Let $R$ be a ring with $1$.
  \begin{enumerate}[(a)]
    \item $\forall a \in R, a \ne 0$, a in called a unit if
      $\exists a^{-1} \in R$.
  \item $\left(R^\times = \{\text{units in $R$}\}, \cdot, 1)\right)$ forms
    a group.
  \item $R$ is called a division ring if $R \setminus \{0\} = R^\times$.
  \item $R$ is said to be commutative if $ab = ba, \forall a, b \in R$.
  \item $R$ is a field if $R$ is a commutative division ring.
  \item $a \ne 0$ is called a left zero divisor if $\exists b \in R, b \ne 0$
    s.t. $ab = 0$.
  \item $a$ is called a zero divisor if $a$ is either a left or right zero
    divisor.
  \item $R$ is called an integral domain if $R$ is a commutative ring without
    zero divisors.
  \end{enumerate}
\end{definition}

\underline{Fact}:
\begin{enumerate}
  \item fields $\implies$ integral domains.
  \item finite + integral domain $\implies$ fields.
    \begin{proof}
      Let $R = \{ 0, a_1, \dots, a_n \}$, for $a \in R, a \ne 0$,
      $aa_i = aa_j \implies a(a_i - a_j) = 0 \implies i = j$.
      So $\{0, aa_1, \dots, aa_n \} = R \implies \exists a_i$ s.t. $aa_i = 1$.
    \end{proof}
\end{enumerate}

\begin{prop}
  TFAE
  \begin{enumerate}
    \item $\quot{\Zb}{n\Zb}$ is an integral domain.
    \item $\quot{\Zb}{n\Zb}$ is a field.
    \item $n = p$ is a prime.
  \end{enumerate}
  easy to prove.
\end{prop}

\begin{definition} \mbox{}
  \begin{itemize}
    \item $f: R_1 \to R_2$ is called a ring homomorphism if
      $\forall a, b \in R, \begin{cases}
        f(a+b) &= f(a) + f(b) \\
        f(ab) &= f(a)f(b)
      \end{cases}$.
    \item $\Image f$ is a subring of $R_2$.
    \item $\Ker f = \{\, x \in R_1 \mid f(x) = 0 \,\}$ is an additive group of
      $R_1$ and $\forall r \in R_1, x \in \Ker f, f(rx) = f(r)f(x) = f(r)0 = 0
      \implies rx \in \Ker f, xr \in \Ker f$.
    \item $\quot{R_1}{\Ker f}$ is an additive group and
      $\quot{R_1}{\Ker f} \cong \Image f$ (additive isomorphism).
  \end{itemize}
\end{definition}

\begin{definition}
  Let $I$ be an additive subgroup of $R$.
  $I$ is called an ideal if $\forall r \in R, x \in I, rx \in I, xr \in I$.

  $\left(\quot{R}{I}, +, \cdot \right)$ forms a quotient ring under
  \[ \forall r_1, r_2 \in R, (r_1+I)(r_2+I) = r_1r_2 + I \]
  well-defined: easy to show.
\end{definition}

\begin{exercise}
  State and show the isomorphism theorems and the factor theorem.
\end{exercise}

\begin{prop}
  If $R$ is a ring with $1$, then $\exists!$ ring homo. $\varphi: \Zb \to R$
  s.t. $\varphi(1) = 1$.
  \begin{proof}
    $\varphi(n) = \varphi(1) + \dots + \varphi(1) = nr$, so $\varphi$ is
    well-defined.

    Original:
    Consider $\varphi_r: \Zb \to R, 1 \mapsto r$, for $r \in R$.
    ($\varphi(n) = \varphi(1) + \dots + \varphi(1) = nr$)

    If $\varphi_r$ is a rong homo., then $\varphi_r(nm) = nmr$ and
    $\varphi_r(n)\varphi_r(m) = nrmr = nmr^2$.
    So $nmr = nmr^2 \implies r = r^2 \implies r = 1$ (if $r \ne 0$).
  \end{proof}
  \label{prop:phi1e1}
\end{prop}

\begin{definition}
  In Prop \ref{prop:phi1e1}, $\Ker \varphi = m\Zb$ for some $m > 0$.
  We call $m$ the characteristic of $R$, denoted by $\Char R = m$.
\end{definition}

\begin{prop} \mbox{}
  \begin{enumerate}
    \item If $R$ is an integral domain, then $\Char R = 0 \text{~or~} p$,
      where $p$ is a prime. (try to prove this)
    \item In the case of $\Char R = p$,
      $\forall a, b \in R, (a + b)^p = a^p + b^p$.
      \begin{proof}
        \[ (a+b)^p = a^p + \binom{p}{1}a^{p-1}b + \dots + b^p = a^p + b^p \]
        because $p \mid \binom{p}{1} \implies \binom{p}{i}a^{p-i}b^{i} = 0$.
      \end{proof}
  \end{enumerate}
\end{prop}

\begin{exercise}
  Let $F$ be a field. Show that
  \begin{enumerate}
    \item if $\Char F = 0$, then $\Qb \hookrightarrow \text{subfield of~} F$.
    \item if $\Char F = p$, then
      $\quot{\Zb}{p\Zb} \hookrightarrow \text{subfield of~} F$.
  \end{enumerate}
  \label{ex:4-4}
\end{exercise}

\begin{theorem}
  If $F$ is a finite field, then $\abs{F} = p^n$ for some $n \in \Nb$ and
  $p$ is a prime.
  \begin{proof}
    By Ex. \ref{ex:4-4}, $\Char F = p$, $p$ is a prime and $\quot{\Zb}{p\Zb}
    \hookrightarrow F$.

    We have $\quot{\Zb}{p\Zb} \times F \to F, (r, v) \mapsto rv$.
    $F$ can be rearded as a vector space over $\quot{\Zb}{p\Zb}$.

    Let $\dim_{\quot{\Zb}{p\Zb}} F = n$, then $F \cong
    \left(\quot{\Zb}{p\Zb}\right)^n \implies \abs{F} = n$.
  \end{proof}
\end{theorem}

\begin{theorem}
  Let $F$ be a field. Then any finite subgroup $G$ of $(F^\times, \cdot, 1)$
  is cyclic.

  \begin{proof}
    Let $\abs{G} = n$. Define $h$ to be the max order of an element in $G$,
    say $a^h = 1$.

    If $h = n$, then $\abs{\gen{a}} = h = n = \abs{G}$ and $\gen{a} \subseteq G$,
    so $G = \gen{a}$.

    Otherwise, $h < n$. We know that $x^h - 1$ has at most $h$ roots.
    So $\exists b \in G$ is not a root of $x^h - 1$.
    Let $\ord(b) = h'$, so $h' \mid n$ and $h' \not\mid h$.
    So $\exists$ a prime $p$ s.t. $p^r \mid h'$ but $p^r \not\mid h$.

    Write $h = mp^s, s < r$ and $\gcd(m, p) = 1 \implies
    \ord \left( a^{p^s} \right) = m$.

    Write $h' = qp^r \implies \ord \left( b^q \right) = p^r$.

    Since $\gcd(m, p^r) = 1, \ord\left(a^{p^s} b^q \right) = mp^r > mp^s = h$,
    which is a contradiction.
  \end{proof}
\end{theorem}

\begin{exercise} \mbox{}
  \begin{enumerate}
    \item Let $a, b \in G$ with $ab = ba$ and $\ord(a) = m, \ord(b) = n$.
      If $\gcd(m, n) = 1$, then $\ord(ab) = mn$.
      In general, is the order of $ab$ equal to $\lcm(m, n)$?
    \item Let $G$ be a finite group and $H, K \le G$. Then
      $\abs{HK} = \frac{\abs{H}\abs{K}}{\abs{H \cap K}}$.
  \end{enumerate}
\end{exercise}

%! TEX root=../main.tex
\subsection{Week 5}
\subsubsection{Group actions \RNum{1}}

\begin{definition}
  A group $G$ is said to act on a nonempty set $X$ if $\exists$ a map
  $G \times X \to X$ with $(g, x) \mapsto gx$ s.t.
  \begin{enumerate}
    \item $1x = x$
    \item $(g_1g_2)x = g_1(g_2 x) \quad \forall g_1, g_2 \in G$
  \end{enumerate}
\end{definition}

\begin{prop}
  $\{ \text{actions of $G$} \} \leftrightarrow
  \{ \text{group homo.~} G \to S_X \}$
  \begin{proof}
    Given an action $(g, x) \mapsto gx$, consider $\varphi: G \to S_X$ s.t.
    $\varphi: g \mapsto (\tau_g: x \mapsto gx)$.
    \begin{itemize}
      \item 1-1: $gx = gy \implies g^{-1}(gx) = y \implies x = y$.
      \item onto: $\forall y \in X$, let $x = g^{-1}y$, then $y = gx$.
      \item group homo.: $\varphi(gg') = (\tau_{gg'}: x \mapsto gg'x)
        = \tau_g \circ \tau_g' = \varphi(g)\varphi(g')$.
    \end{itemize}
    Conversely, given a group homo. $\varphi: G \to S_X$, consider
    $(g, x) \mapsto \varphi(g)(x)$.
    \begin{itemize}
      \item $1x = \varphi(1)(x) = \text{Id}(x) = x$.
      \item $g_1g_2x = \varphi(g_1g_2)(x) = \varphi(g_1) \circ \varphi(g_2)(x)
        = g_1(g_2x)$. \qedhere
    \end{itemize}
  \end{proof}
\end{prop}

\begin{definition}
  A representation of $G$ on a vector space $V$ is a group action of $G$ on
  $V$ linearly. i.e. $\exists$ group homo. $\varphi: G \to \text{GL}(V)$.
\end{definition}

\begin{example}
  \[
    \quot{\Zb}{m\Zb} \to \text{SO}(2), \quad
    \ob{k} \mapsto \begin{pmatrix}
      \cos \frac{2k\pi}{m} & -\sin \frac{2k\pi}{m} \\
      \sin \frac{2k\pi}{m} & \cos \frac{2k\pi}{m}
    \end{pmatrix}
  \]
\end{example}

\begin{example}
  \[
    S_n \to \text{GL}(n, \Rb), \quad
    \sigma \mapsto (\tau_\sigma: e_i \mapsto e_{\sigma(i)})
  \]
\end{example}

\begin{remark} \mbox{}
  \begin{enumerate}
    \item An action $G \times X \to X$ is said to be faithful if the
      corresponding group homo. $\varphi: G \toone S_X$, denoted by
      $G \acts X$.
    \item In general, $\ker \varphi = \{\, g \in G \mid gx = x \quad
      \forall x \in X \,\} = \bigcap_{x \in X} \{\, g \mid gx = x \,\}$.

      Define $G_x = \{\, g \mid gx = x \,\} \le G$ is the isotropy subgroup
      of $G$ at $x$. (the stabilizer of $G$ at $x$)
    \item $\varphi: G \to S_X \implies \quot{G}{\ker \varphi} \toone S_X$.
      So $\quot{G}{\ker \varphi} \times X \to X$ is faithful.
    \item Let $\mathcal{C}(X) = \{\, f: X \to \Cb \,\}$. If $G \acts X$, then
      $G \acts \mathcal{C}(X)$ by
      $G \times \mathcal{C}(X) \to \mathcal{C}(X)$ with
      $(g, f) \mapsto g f(x) = f(g^{-1}x)$.

      The reason: $(g_1g_2)f(x) = f((g_1g_2)^{-1}x) = f(g_2^{-1}g_1^{-1}x)
      = g_1(g_2 f)(x)$.
  \end{enumerate}
\end{remark}

\begin{definition}
  Let $G \acts X$ and $x \in X$.
  \begin{itemize}
    \item The {\bf orbit} of $x$ is defined to be
      $Gx = \{\, gx \mid g \in G \,\}$.
    \item $G \acts X$ is said to be transitive if $\exists$ only one orbit.
      i.e. $\forall x, y \in X, \exists g \in G$ s.t. $y = gx$.
  \end{itemize}
  The set of orbits forms a partition: $x \sim y \iff \exists g \in G
  \text{~s.t.~} y = gx$.
\end{definition}

\begin{prop}
  Let $G \acts X$ and $x \in X$. Then $\abs{Gx} = [G : G_x]$.

  In particular, $\abs{G} < \infty \implies \abs{G} = \abs{Gx}\abs{G_x}
  \quad \forall x \in X$.
  \begin{proof}
    Define $\psi: Gx \to \{ \text{left coset of~} G_x \}$ as
    $\psi: gx \mapsto g G_x$.
    \begin{itemize}
      \item well-defined and 1-1:
        $g_1x = g_2x \iff g_2^{-1}g_1x = x \iff g_2^{-1}g_1 \in G_x \iff
        g_2^{-1}g_1G_x = G_x \iff g_1G_x = g_2G_x$.
      \item onto: $\forall g \in G, \psi(gx) = gG_x$. \qedhere
    \end{itemize}
  \end{proof}
\end{prop}

\subsubsection{Action by left multiplication}
\begin{itemize}
  \item The action $G \times G \to G, \quad (g, x) \mapsto gx$ is associated
    with $\varphi: G \toone S_G$.
    It is faithful (Cayley theorem) and transitive.
  \item Let $H \le G$ and $X \defeq \{ \text{left coset of~} H \}$.
    The group action $(g, xH) \mapsto gxH \leadsto \varphi: G \to S_X$.
    \[
      \ker \varphi = \bigcap_{x \in G} \tikz[anchor=base, baseline]{
      \node (conj-of-H) {$\underbrace{xHx^{-1}}$}; } \le H
    \]
    \begin{tikzpicture}[overlay]
      \node[inner sep=0pt,outer sep=0pt] (t) at ($(conj-of-H) + (2, -0.5)$) {
         \footnotesize a conjugate of $H$};
       \path[->,shorten >= 6pt] ($(conj-of-H.base) + (0, -0.3)$) edge
         [bend right=10] (t.west) ;
    \end{tikzpicture}
    which is the largest normal subgroup in $G$ contained in $H$.
    \begin{proof}
      If $\begin{cases} N \lhd G \\ N \le H\end{cases}, \forall x \in G,
        xNx^{-1} \le xHx^{-1} \implies N = N(xx^{-1}) = xNx^{-1} \le xHx^{-1}$.
    \end{proof}
\end{itemize}

\begin{prop}
  Let $H \le G$ with $[G:H] = p$ being the smallest prime dividing $\abs{G}$.
  Then $H \lhd G$.
  \begin{proof}
    Let $X = \{ a_1H, \dots, a_pH \}$ (all left coests of $H$) and
    $\varphi: G \to S_p$ be the associated group homo. for the group action
    $(g, a_iH) \mapsto ga_iH$.

    By the 1st isom. thm., $\quot{G}{\ker\varphi} \toone S_p$.

    By Lagrange thm. $\abs*{\quot{G}{\ker\varphi}} \Div \abs{S_p} = p!$ and
    $\abs*{\quot{G}{\ker\varphi}} \Div \abs{G} \implies
    \abs*{\quot{G}{\ker\varphi}} \Div p$.

    So $\abs*{\quot{G}{\ker\varphi}} = 1 \text{~or~} p$.

    If $\abs*{\quot{G}{\ker\varphi}} = 1 \implies G = \ker\varphi \le H \lneq G$,
    which is a contradiction.

    So $\abs*{\quot{G}{\ker\varphi}} = p \implies [G:\ker\varphi] = p
    \implies [G:H][H:\ker\varphi] = p \implies [H:\ker\varphi] = 1
    \implies H = \ker\varphi \lhd G$.
  \end{proof}
\end{prop}

\subsubsection{Action by conjugation}
\begin{itemize}
  \item The action $G \times G \to G \quad (g,x) \mapsto gxg^{-1}$ is
    associated with the group homo. $\varphi: G \to S_G \quad g \mapsto
    (\tau_g: x \mapsto gxg^{-1})$.
    \[
      \Inn(G) \defeq \{\, \tau_g \mid g \in G \,\}
    \]

    \begin{fact} 
      $\tau_g$ is an automorphism. (isom. $G \to G$)
    \end{fact}

    So $\varphi: G \onto \Inn(G) \le \Aut(G) \le S_G$.

    $\ker\varphi = \{\, g\in G \mid gxg^{-1} = x \quad \forall x \in G \,\}
    = Z_G$.

    By the 1st isom. thm., $\quot{G}{\ker\varphi} \cong \Inn(G)$.
    \begin{itemize}
      \item The conjugacy class:
        $Gx = \{\, gxg^{-1} \mid g \in G \,\} = \text{Cl}(x)$.
      \item The centralizer of $x$ in $G$:
        $G_x = \{\, g \in G \mid gxg^{-1} = x \,\} = Z_G(x)$.
    \end{itemize}
    \[
      \abs{\text{Cl}(x)} = [G:Z_G(x)], \text{~if~} \abs{G} < \infty, 
      \abs{G} = \abs{\text{Cl}(x)}\abs{Z_G(x)}
    \]
  \item For $H \lhd G$, define $G \times H \to H \quad (g, h) \mapsto ghg^{-1}$
    with the group homo. $\varphi: G \to \Aut(H)$.
    \[
      \ker\varphi = \{\, g\in G \mid gxg^{-1} = x \quad \forall x \in H \,\}
      = Z_G(H)
      \implies \quot{G}{Z_G(H)} \le \Aut(H)
    \]
  \item The normalizer of $H$ in $G$:
    $N_G(H) = \{\, g\in G \mid gHg^{-1} \le H \,\}$
\end{itemize}

%1 TEX root=../main.tex
\subsection{Week 6}
\subsubsection{Group actions \RNum{2}}

\begin{definition}
  Let $G \acts X$ and $\abs{X} < \infty$.
  Write $\Fix G \defeq \{\, x \in X \mid gx = x \quad \forall g \in G \,\}$.
\end{definition}

\begin{itemize}
  \item $x \in \Fix G$, $Gx = \{ x \}$.
  \item $x \not\in \Fix G$, $\abs{Gx} = [G : G_x]$.
\end{itemize}

Let $\{ G_{x_1}, \dots, G_{x_n} \}$ be the set of distinct orbits.
After rearrangement, assume $x_1, \dots, x_r \in \Fix G,
x_{r+1}, \dots, x_n \not\in \Fix G$. Then
\[
  \abs{X} = \abs{\Fix G} + \sum_{i=r+1}^{n} [G : G_{x_i}]
\]

\begin{theorem}[class equation]
  Let $\abs{G} < \infty$. Then either $G = Z_G$ or
  $\exists a_1, \dots, a_m \in G \setminus Z_G$ s.t.
  \[
    \abs{G} = \abs{Z_G} + \sum_{i=1}^{n} [G : G_{a_i}]
  \]
  \begin{proof}
    Consider the action $(g, x) \mapsto gxg^{-1}$, then
    \[
      \Fix G = \{\, x \in G \mid gxg^{-1} = x \quad \forall g \in G \,\}
      = Z_G
    \]
    It follows from the above argument.
  \end{proof}
\end{theorem}

\begin{definition}
  $G$ is called a $p$-group if $\abs{G} = p^n$, where $p$ is a prime,
  $n \in \Nb$.
\end{definition}

\begin{prop}
  If $G$ is a $p$-group, then $Z_G \ne \{ 1 \}$.
  \begin{proof}
    Let $\abs{G} = p^n$. If $G = Z_G$, then done.
    Otherwise, by the class equation (use action by conjugation),
    $\abs{G} = \abs{Z_G} + \sum_{i=1}^{n} [G : G_{a_i}], \quad a_i \not\in Z_G$.

    $G_{a_i} = Z_G(a_i)$, so $a_i \not\in Z_G \implies Z_G(a_i) \lneq G
    \implies p \mid [G:Z_G(a_i)] = \frac{\abs{G}}{\abs{Z_G(a_i)}}$.

    So $\abs{Z_G} = \abs{G} - \sum_{i=1}^{n} [G : Z_G(a_i)]
    \implies p \mid \abs{Z_G} \implies Z_G \ne \{ 1 \}$.
  \end{proof}
  \label{prop:pgroup}
\end{prop}

\begin{prop}
  If $\abs{G} = p^2$, then $G$ is abelian.
  ($\quot{\Zb}{p\Zb} \times \quot{\Zb}{p\Zb}$ and $\quot{\Zb}{p^2\Zb}$)
  \begin{proof}
    Assume that $G$ is not abelian.
    By prop \ref{prop:pgroup}, $\abs{Z_G} = p \implies \abs{\quot{G}{Z_G}} = p
    \implies \quot{G}{Z_G}$ is cyclic $\implies G$ is abelian. (contradiction)
  \end{proof}
\end{prop}

\begin{prop}
  If $\abs{G} = p^3$ and $G$ is not abelian, then $\abs{Z_G} = p$.

  (Abelian: $\quot{\Zb}{p\Zb} \times \quot{\Zb}{p\Zb} \times \quot{\Zb}{p\Zb},
  \quot{\Zb}{p^2\Zb} \times \quot{\Zb}{p\Zb}, \quot{\Zb}{p^3\Zb}$)

  \label{prop:w6p3}
\end{prop}

\begin{prop}
  Let $\abs{G} = p^n$. Then $\forall 0 \le k \le n, \exists G_k \lhd G$ s.t.
  $\abs{G_k} = p^k$ and $G_i \lneq G_{i+1}$.

  In general, for a finite group $G$, $\exists {\{1\}} =
  G_r \lhd G_{r-1} \lhd \dots \lhd G_1 \lhd G_0 = G$ s.t. $\quot{G_i}{G_{i+1}}$
  is cyclic.

  we call $G$ a solvable group.

  \begin{proof}
    By induction on $n$, $n = 1$ is trivial.
    For $n > 1$, assume that the statement a holds for $n-1$.
    By prop \ref{prop:pgroup}, $Z_G \ne \{1\}$. $\exists a \in Z_G, a \ne 1$.
    Let $\ord(a) = p^l$, then $\ord(a^{p^{l-1}}) = p$.
    $\implies$ in any case, $\exists a \in Z_G$ with $\ord(a) = p$.

    Now $\abs*{\quot{G}{\gen{a}}} = p^{n-1}$, so by induction hypothesis,
    $\forall 0 \le k \le n - 1, \exists \ob{G_k} \lhd \quot{G}{\gen{a}}$ s.t.
    $\abs*{\ob{G_k}} = p^k, \ob{G_i} \lneq \ob{G_{i+1}}$.

    By 3rd isom. thm., $\exists G_{k+1} \lhd G$ s.t. $\ob{G_k} =
    \quot{G_{k+1}}{\gen{a}}, G_j \lneq G_{j+1}$ and $\abs{G_{k+1}} = p^{k+1}$.

  \end{proof}
\end{prop}

\begin{prop}
  Let a $p$-group $G \acts X$ with $\abs{X} < \infty$.
  Then $\abs{X} \equiv \abs{\Fix G} \pmod p$.
  \label{prop:useful}
\end{prop}

\begin{theorem}[Cauchy theorem]
  Let $p \Div \abs{G}$. Then $\exists a \in G$ s.t. $\ord(a) = p$. Consider
  \[ X = \{\, (a_1, \dots, a_p) \mid a_i \in G, a_1a_2\dots a_p = 1\,\} \]
  and the action $\quot{\Zb}{p\Zb} \times X \to X$:
  \[
    (\ob{k}, (a_1, \dots, a_p)) \mapsto (a_{k+1}, \dots, a_p, a_1, \dots, a_k)
  \]
  (This is well-defined since $ab = 1 \implies ba = 1$ in a group.)
  We find that $(a_1, \dots, a_p) \in \Fix \quot{\Zb}{p\Zb} \iff a_1 = a_2
  \dots a_p$.
  By prop \ref{prop:useful}, $\abs*{\Fix \quot{\Zb}{p\Zb}} \equiv \abs{X}
  \pmod p$. And $\abs{X} = \abs{G}^{p-1} \equiv 0 \pmod p$.
  Since $(1, \dots, 1) \in \Fix \quot{\Zb}{p\Zb}, \abs*{\quot{\Zb}{p\Zb}} \ne 0
  \implies \abs*{\quot{\Zb}{p\Zb}} \ge p$.

  So $\exists (a, \dots, a) \in \Fix \quot{\Zb}{p\Zb} \implies a^p = 1$.
\end{theorem}

\underline{Application}: Let $\abs{G} = p^3$ and $G$ be non-abelian
($p$ is odd).
By prop \ref{prop:w6p3}, $\abs*{\quot{G}{Z_G}} = p^2$. Since $G$ is non-abelian,
we have $\quot{G}{Z_G} \cong \quot{\Zb}{p\Zb} \times \quot{\Zb}{p\Zb}$.
That is, $\forall a \in G, a^p \in Z_G$.

So,
\[
  \exists \varphi: G \to Z_G \cong C_p \text{~with~}
  \varphi: a \mapsto a^p
\]

Since $\quot{G}{Z_G}$ is abelian, $[G,G] \le Z_G$. And
\[
  \begin{cases}
    \abs{[G,G]} \Div \abs{Z_G} = p \\
    G \text{~is non-abelian}
  \end{cases}
  \implies [G,G] = Z_G
\]

\begin{definition}
  $[x, y] = x^{-1}y^{-1}xy \in [G,G], [x,y]^p = 1$.
\end{definition}

So $a^p b^p = a^p b^p [b, a]^p$ ... 換換換 總共需要 $p(p-1)/2$
\[ a^p b^p = (ab)^p [b,a]^{\frac{p(p-1)}{2}} = (ab)^p \]

So $\varphi$ is a group homo.

Now if $\ker \varphi = G \quad (\forall a \in G, a^p = 1)$,
i.e. $\varphi$ is trivial, then $\varphi$ is useless.
Else, $\exists a \in G$ s.t. $\ord(a) = p^2$, then
$H = \gen{a} \lhd G$. ($[G:H] = p$ is the smallest prime dividing $\abs{G}$)

Also, in this case, $\varphi: G \onto Z_G \implies
\quot{G}{\ker \varphi} \cong Z_G$. Let $E = \ker \varphi$, $\abs{E} = p^2$.
By the def. of $\ker \varphi$, $E \cong \quot{\Zb}{p\Zb} \times
\quot{\Zb}{p\Zb}$.

We find that $H \cap E = \gen{a^p}$. Pick $b \in E \setminus H$ and let
$K = \gen{b} \implies \abs{K} = p, H \cap K = \{ 1 \}, HK = G$.

\subsubsection{Semidirect product}

\begin{fact}
$K \lhd G, H \lhd G, K \cap H = \{1\} \implies KH = K \times H$ \\
($\forall k\in K, h \in H, khk^{-1} h^{-1} \in H \cap K = \{1\}, \implies kh=hk$)
\end{fact}

\begin{fact}
Let $K, H$ be two groups, and $G=K \times H \implies K \times \{1\} \lhd K \times H, \{1\} \times H \lhd K \times H$
\end{fact}

\begin{observation}
$K \leq G, H \lhd G, K \cap H = \{1\}$ (K 慘 H 好,簡稱慘好集) \\
$\implies$ elements in $KH$ has unique representation ? 好事喔\\
$KH \iff K \times H$ 1-1 corresp, $(kh) \leftrightarrow (k, h)$
\end{observation}

Group operation :
$\forall k_1, k_2 \in K, h_1, h_2 \in H, (k_1 h_1) (k_2 h_2) = k_1 k_2 (k_2^{-1} h_1 k_2) h_2$ \\
Let $\tau : K \to \Aut(H), k \mapsto (\tau(k) : h\mapsto khk^{-1})$ (類似 $\in \Inn(H)$ )

\begin{definition}[Semi-Direct Product (慘好積)]
  $K \times_{\tau} H = \{(k, h) | k\in K, h \in H\}$ with group operation :
$(k_1, h_1)(k_2, h_2) = (k_1 k_2, \tau(k_2^{-1})(h_1)(h_2))$
where $\tau : K \to \Aut(H)$  (need not to be inner homomorphism)
\end{definition}

Properties:
\begin{itemize}
  \item Associativity: Good, ex
  \item The identity = $(1, 1)$
  \item Inverse : $(k, h)^{-1} = (k^{-1}, \tau(k)(h^{-1}))$
  \item $K \cong K \times \{1\} \leq K \times_{\tau} H$ :
    $(k_1, 1)(k_2, 1) = (k_1 k_2, \tau(k_2^{-1})(1)1) = (k_1 k_2, 1) \in K \times \{1\}$
    $H \cong \{1\} \times H \leq K \times{\tau} H : (1, h+1), (1, h_2) = (1, \tau(1^{-1})(h_1)h_2) = (1, h_1 h_2) \in \{1\} \times K $
  \item $H \lhd K \times_t H : (k, h) (1, h')(k, h)^{-1} = (k, hh')(k^{-1}, \tau(k)(h^{-1}))
    = (1, \tau(k)(hh')\tau(k)(h^{-1})) \in H$

  \item $\tau(k)(h) = khk^{-1}$ :
    $(k, 1)(1, h)(k^{-1}, 1) = (k, h)(k^{-1}, 1) = (1, \tau(k)(h))$
  \item If $\tau$ is trivial $\implies K \times_t H \cong K \times H$
\end{itemize}

\begin{remark}
Some definition swaps the order of $H$ and $K$, i.e. $(h_1, k_1) (h_2, k_2) = (h_1 \phi(k_1)(h_2), k_1 k_2)$
\end{remark}

\begin{exercise}
Show that $H \rtimes_\phi K$ is a group and satisfies the above properties.
\end{exercise}

\begin{example}
Construct a non-abelian group of order 21.
\end{example}

\begin{fact}
  $\Aut(\quot{\Zb}{p\Zb}) \cong (\quot{\Zb}{p\Zb})^\times \cong C_{p-1}$
\end{fact}
Sol : $\phi_k: \quot{\Zb}{p\Zb} \to \quot{\Zb}{p\Zb}, \bar{1} \mapsto \bar{k}$

$\phi_{k_2} o \phi_{k_1} (T) = \phi_{k_2}(\bar{k_1}) = \phi_{k_2}(T+\cdots+T)
= \bar{k_2} + \cdots \bar{k_2} = \ob{k_1 k_2}$

Let $K = C_3, H = C_7$, define $\tau : C_3 \to \Aut(C_7) \cong C_6, a \mapsto \phi_2$

$\phi_k : b \mapsto b^k$

$G = \gen{a, b | a^3=1, b^7=1, aba^{-1}=b^2}$

\begin{example}
  p : odd, $\abs{G} = p^3$, $G$ is non-abelian.
\end{example}
(sol)
$\phi: G \to Z(G), a \mapsto a^p$ non trivial
case $\exists a \in G $ with $\ord(a) = p^2$.
Let $H = \gen{a}$ here $\phi$ is onto and $E = \ker \phi \cong \quot{\Zb}{p\Zb} \times \quot{\Zb}{p\Zb}$
And $\abs{H \cap E} = p$
$H \lhd G$ because $[G: H]=p$
Pick $b \in E \setminus H$ and let $K = \gen{b} \implies \abs{K} = p, K \cap H = \{1\}$
so $\abs{G} = \abs{KH} = p^3$

\begin{fact}
  $\Aut(\quot{\Zb}{p^2 \Zb}) \cong (\quot{\Zb}{p^2 \Zb})^\times$
\end{fact}
Sol : $\phi_k: \quot{\Zb}{p^2\Zb} \to \quot{\Zb}{p^2\Zb}, \bar{1} \mapsto \bar{k}, \gcd(k, p) = 1$

Find a group homo $\tau : K \implies \Aut(H)$
because $(1+p)^p \equiv 1 \mod p^2$, $\ord\left(\ob{1+p}\right) \in
\quot{\Zb}{p^2\Zb}$
Let $P = \gen*{\ob{1+p}}$ is the only subgroup of order $p$.
(if $\exists |Q| = p$, $P \neq Q$ then $P \cap Q = 1$, $|PQ| = p^2$, miserable.)
So let $\tau : b \mapsto (\phi_{1+p} : a \mapsto a^{1+p})$
so $G = \gen{a, b | a^{p^2}=1, b^p = 1, bab^{-1} = a^{1+p}}$ is a non-abelian group of order $p^3$.

\begin{example}
Isometry of $R^n$
\end{example}

\begin{definition}[Isometry]
An isometry of $R^n$ is a function $h: R^n \to R^n$ that preserves the distance between vectors.
\end{definition}
$h = t \circ k$ where $t$ is translation, $k$ is an isometry fixing the origin, i.e. $k \in O(n)$.
Let $T$ be the group of translations on $R^n$, $T \cong (R^n, +, 0), t \mapsto t(0)$.

Let $\tau : O(n) \to \Aut(T), A \mapsto L_A : R^n \to R^n, v \mapsto Av$

$\implies \Isom(R^n) = O(n) \times_\tau R^n$

\begin{example}
Quaternium $Q_8 = \{\pm 1, \pm i, \pm j, \pm k\}$ is not a semi-deriect product of any two proper subgroups.
\end{example}
pf: since $\{\pm 1\}$ is contained in any non-trivial subgroups, can't find $H \cap K = \{1\}$.

\begin{example}
  $A_4$, $V_4 = \{1, (12)(34), (14)(23), (13)(24)\} \lhd A_4, V_4 \cong \quot{\Zb}{2\Zb} \times \quot{\Zb}{2\Zb}$
\end{example}
Let $H = \gen{(123)} \cong C_3$, define $\tau : H \to \Aut(V_4) \cong GL_2(\quot{\Zb}{2\Zb})$
$(123) \mapsto (\bar{0} \bar{1}; \bar{1} \bar{1})$
so $A_4 \cong C_3 \times_\tau V_4$.

\begin{exercise}
  Construct $D_n$ as a semi-direct product of $\quot{\Zb}{n\Zb}$ and
  $\quot{\Zb}{2\Zb}$.
\end{exercise}

\begin{exercise} \mbox{}
  \begin{enumerate}
    \item Show that $S_4$ is a semi-direct product of $V_4$ and $H = \{\, \sigma \in S_4 | \sigma(4) = 4 \,\} \sim S_3$.
    \item Show that $S_n$ is a semi-direct product of $A_n$ and $H = \gen{(12)}$.
  \end{enumerate}
\end{exercise}

\begin{remark} \mbox{}
  \begin{itemize}
    \item $\Aut(\quot{\Zb}{p\Zb} \times \quot{\Zb}{p\Zb}) \cong GL_2(\quot{\Zb}{p\Zb})$
      (regarded as a vector space over $\quot{\Zb}{p\Zb}$)
    \item $\Aut(\quot{\Zb}{p\Zb} \times \quot{\Zb}{p\Zb}) \cong
      \Aut(\quot{\Zb}{p\Zb}) \times \Aut(\quot{\Zb}{q\Zb}) \cong
      C_{p-1} \times C_{q-1}$
  \end{itemize}
\end{remark}

%1 TEX root=../main.tex
\subsection{Week 7}
\subsubsection{Composition series}
\underline{Ques}: How to simplify a finite group $G$?

\underline{Strategy}:
\begin{itemize}
  \item If $G = \{1\}$, then done.
  \item Otherwise, check whether $G$ has a nontrivial proper normal subgroup.
  \item If no, then $G$ is said to be a simple group.
  \item Otherwise, find a normal subgroup $G_1$ as large as possible s.t.
    $\quot{G}{G_1}$ is simple.
  \item If $G_1$ is simple, then done.
  \item Otherwse, repeat above on $G_1$ and get $G_2, \dots, G_n$ s.t.
    \[ G_n = \{1\} \lhd G_1 \lhd \dots \lhd G_1 \lhd G_0 = G \quad
      \tikz[anchor=base, baseline]{
    \node(n-comp-fac) {$\quot{G_i}{G_{i+1}}$}; } \text{~is simple} \]
    \begin{tikzpicture}[overlay]
      \node[inner sep=0pt,outer sep=0pt] (t) at ($(n-comp-fac) + (2, -0.5)$) {
         \footnotesize composition factors};
       \path[->,shorten >= 6pt] ($(n-comp-fac.base) + (0, -0.15)$) edge
         [bend right=10] (t.west) ;
    \end{tikzpicture}
    Say ``it is a composition series'' with $\text{length}(G) = n$.
\end{itemize}

Hence simple groups can be regarded as basic building blocks of groups.

The classification of all finite simple groups is given as follows:
\begin{enumerate}
  \item $\quot{\Zb}{p\Zb}$, $p$ is a prime.
  \item $A_n, n \ge 5$.
  \item simple groups of Lie type.
  \item $26$ sporadic simple groups.
\end{enumerate}

\begin{example}
  $G = S_4, G_1 = A_4, G_2 = V_4, G_3 = \gen{\cycle{1,2}\cycle{3,4}},
  G_4 = \{1\} \leadsto \text{length}(S_4) = 4$.

  factors: $C_2, C_3, C_2, C_2$.
\end{example}

\begin{example}
  $G = \quot{\Zb}{12\Zb} = \gen{\bar{1}}$.
  \begin{itemize}
    \item $G_1 = \gen{\bar{2}}, G_2 = \gen{\bar{4}}, G_3 = \gen{\bar{0}}
      \leadsto \text{length}(3)$, factors: $C_2, C_2, C_3$.
    \item $G_1' = \gen{\bar{2}}, G_2' = \gen{\bar{6}}, G_3' = \gen{\bar{0}}
      \leadsto \text{length}(3)$, factors: $C_2, C_3, C_2$.
    \item $G_1'' = \gen{\bar{3}}, G_2'' = \gen{\bar{6}}, G_3'' = \gen{\bar{0}}
      \leadsto \text{length}(3)$, factors: $C_3, C_2, C_2$.
  \end{itemize}
\end{example}

\begin{example}
  Let $\abs{G} = p^n$. We know $\forall 0 \le k \le n$, $\exists G_k \lhd G$
  with $\abs{G_k} = p^k$ and $G_i \lneq G_{i+1}$.

  $\text{length}(G) = n$, factors: $C_p, \dots, C_p$. ($n$ times)
\end{example}

\begin{theorem}[Jorden-H\"older theorem]
  If $G$ has a composition series, then any two composition series have the
  same length and the same factors up to permutation.
\end{theorem}

\begin{lemma}[Zassenhaus lemma]
  Let $H' \lhd H \le G, K' \lhd K \le G$. Then 
  $(H\cap K')H' \lhd (H\cap K)H', (H'\cap K)K' \lhd (H\cap K)K'$ and
  \[
    \quot{(H\cap K)H'}{(H\cap K')H'} \cong \quot{(H\cap K)K'}{(H'\cap K)K'}.
  \]
\end{lemma}

\begin{theorem}[Schreier theorem]
  Any two normal series of $G$ have equivalent refinements.

  refinements: inserting a finite number of subgroups into the normal series.
  \begin{proof}
    For two normal series:
    \begin{align*}
      & \{1\} = H_r \lhd H_{r-1} \lhd \dots \lhd H_1 \lhd H_0 = G \\
      & \{1\} = K_s \lhd K_{r-1} \lhd \dots \lhd K_1 \lhd K_0 = G
    \end{align*}
    We define
    \begin{align*}
      H_{ij} = (H_i \cap K_j)H_{i+1} \\
      K_{ji} = (H_i \cap K_j)K_{j+1}.
    \end{align*}
    Then we have
    \begin{align*}
      & \{1\} = H_{(r-1)s} \lhd H_{(r-1)(s-1)} \lhd \dots \lhd
      H_{(r-1)0} = H_{r-1} = H_{(r-2)s} \lhd \dots \lhd
      H_{10} = H_1 = H_{0s} \lhd \dots \lhd H_{00} = G \\
      & \{1\} = K_{(s-1)r} \lhd K_{(s-1)(r-1)} \lhd \dots \lhd
      K_{(s-1)0} = K_{s-1} = K_{(s-2)r} \lhd \dots \lhd
      K_{10} = K_1 = K_{0r} \lhd \dots \lhd K_{00} = G
    \end{align*}
    Both have size $= rs$. By lemma,
    $\quot{H_{ij}}{H_{i(j+1)}} \cong \quot{K_{ji}}{K_{j(i+1)}}$.
    Note that if $H_{ij} = H_{i(j+1)}$, then $K_{ji} = K_{j(i+1)}$.
  \end{proof}
\end{theorem}

\begin{proof}[proof of Jorden-H\"older theorem]
  Let 
  \[
  \begin{cases}
    \{1\} = G_n \lhd \dots \lhd G_1 \lhd G_0 = G  & (*)\\
    \{1\} = G_m' \lhd \dots \lhd G_1' \lhd G_0' = G & (**)
  \end{cases}
  \]
  be two composition series.

  By Schreier theorem, we get two refined equivalent series $(*)', (**)'$.
  Since $(*), (**)$ are already composition series, $(*)=(*)', (**)=(**)'$
  So $(*), (**)$ are equivalent.
\end{proof}

\begin{proof}[proof of lemma]
  First prove $(H\cap K')H' \lhd (H\cap K)H'$.
  \begin{itemize}
    \item $\forall g \in H \cap K, g K'g^{-1} = K' \leadsto
      (gHg^{-1}) \cap (gK'g^{-1} = H \cap K'$ and $gH'g^{-1} = H'$. So
      \[ g(H\cap K')H'g^{-1} = (H\cap K')H' \]
    \item $\forall g \in H', ab \in (H\cap K')H'$, 
  \end{itemize}

  To prove
  \[
    \quot{(H\cap K)H'}{(H\cap K')H'} \cong \quot{(H\cap K)K'}{(H'\cap K)K'}.
  \]
  \begin{align*}
    \quot{(H\cap K)H'}{(H\cap K')H'} &\cong
    \quot{(H\cap K)(H\cap K')H'}{(H\cap K')H'} \\
    &\cong \quot{(H\cap K)}{(H\cap K)\cap(H\cap K')H'} \\
    &\cong \quot{(H\cap K)}{K\cap(H\cap K')H'} \\
    &\cong \quot{(H\cap K)}{(H'\cap K)(H\cap K')}
  \end{align*}
  ($K\cap(H\cap K')H' = (H'\cap K)(H\cap K')$, tricky)
  By symmetry, 
  \[
    \quot{(H\cap K)K'}{(H'\cap K')K'} \cong
    \quot{(H\cap K)}{(H'\cap K)(H\cap K')}
  \]
\end{proof}

\begin{prop}
  Let $\abs{G} < \infty$. Then $G$ is solvable $\iff$ all composition factors
  are cyclic of prime order.
  \begin{proof}
    ``$\Leftarrow$'': by def.

    ``$\Rightarrow$'': If $\quot{G_i}{G_{i+1}} \cong C_n$ with
    $n = p_1^{m_1} p_2^{m_2} \dots p_r^{m_r}$.
  \end{proof}
\end{prop}

\begin{observation*}
  Let $K \lhd G$. 把 $K, \quot{G}{K}$ 拆成兩個 composition series 的話,
  就可以把兩串接起來,長度就是加起來。
\end{observation*}

\begin{exercise}
  Let $\{1\} = G_n \lhd G_{n-1} \lhd \dots \lhd G_1 \lhd G_0 = G$ be a
  composition series of $G$ and $K \lhd G$.

  Then after we eliminate equalities,
  \begin{enumerate}
    \item $\{1\} = (K \cap G_n) \lhd (K \cap G_{n-1}) \lhd \dots \lhd
      (K \cap G_1) \lhd (K \cap G_0) = K$ is a composition series of $K$.
    \item $\{\bar{1}\} = \quot{KG_n}{K} \lhd \quot{KG_{n-1}}{K} \lhd \dots \lhd
      \quot{KG_1}{K} \lhd \quot{KG_0}{K} = \quot{G}{K}$ is a composition
      series of $\quot{G}{K}$.
  \end{enumerate}
\end{exercise}

\begin{exercise}
  Let $\begin{cases}
    H \lhd G \\
    K \lhd G
  \end{cases}$ with $H \ne K$ s.t. $\quot{G}{H}, \quot{G}{K}$ are simple.
  Then $\quot{H}{H\cap K}, \quot{K}{K \cap H}$ are simple too.
\end{exercise}

\begin{exercise}
  Let $\{1\} = G_n \lhd G_{n-1} \lhd \dots \lhd G_1 \lhd G_0 = G$ be a
  composition series of length $n$.

  Show by induction on $n$ that for every composition series of $G$:
  \[
    \{1\} = H_m \lhd H_{n-1} \lhd \dots \lhd H_1 \lhd H_0 = G,
  \]
  we have $m = n$ and
  \[
    \left\{
      \quot{H_{n-1}}{H_n}, \dots, \quot{H_0}{H_1}
    \right\} =
    \left\{
      \quot{G_{n-1}}{G_n}, \dots, \quot{G_0}{G_1}
    \right\}
  \]

\end{exercise}

\begin{exercise}
  Exhibit all composition series for
  $Q_8, D_4, \quot{\Zb}{8\Zb}, \quot{\Zb}{4\Zb} \oplus \quot{\Zb}{2\Zb},
  \quot{\Zb}{2\Zb} \oplus \quot{\Zb}{2\Zb} \oplus \quot{\Zb}{2\Zb}$
  respectively.
\end{exercise}

\subsubsection{Modules over a PID}

\begin{definition}
  Let $R$ be a ring with $1$. A $R$-moduule is an abelian group $M$
  (wriiten additively) on which $R$ acts linearly.
  $R \times M \to M \quad (r, x) \mapsto rx$
  \begin{enumerate}
    \item $r(x + y) = rx + ry \quad r \in R, x, y \in M$
    \item $(r_1 + r_2)x = r_1x + r_2x \quad r_1, r_2 \in R, x \in M$
    \item $(r_1r_2)x = r_1(r_2x) \quad r_1, r_2 \in R, x \in M$
    \item $1x = x \quad x \in M$
  \end{enumerate}
\end{definition}

\begin{example}
  A $k$-vector space is a $k$-module.
\end{example}

\begin{example}
  An abelian group $G$ can be regarded as a $\Zb$-module.
  \[
    \begin{aligned}
      \Zb \times G &\to G\\
      (n, a) &\mapsto na
    \end{aligned}
    \quad \text{by} \quad
    na = \begin{cases}
      \underbrace{a + \dots + a}_{n\text{~times}} & \text{if~} n \ge 0 \\
      \underbrace{(-a) + \dots + (-a)}_{n\text{~times}} & \text{if~} n < 0
    \end{cases}
  \]
\end{example}

\begin{example}
  Let $I$ be an ideal of $R$. Then $I$ can be regarded as an $R$-module
  since $\forall r \in R, a \in I, \quad ra \in I$.
\end{example}

\begin{definition}
  A ubmodule $N$ of $M$ is an additive subgroup of $M$ s.t.
  $\forall r \in R, a \in N, \quad ra \in N$.
\end{definition}

\begin{prop}
  Let $\phi \ne S \subseteq M$. The submodule generated by $S$ is defined to be
  \begin{align*}
    \gen{S}_R = \left\{
      \sum_{\text{finite}} r_i x_i \middle| x_i \in S, r_i \in R
    \right\} &= \text{the least submodule containg $S$} \\
             &= \bigcap_{S \subset N \subset M} N
  \end{align*}
\end{prop}

\begin{definition}
  An $R$-module $M$ is said to be finitely generated if
  $\exists x_1, \dots, x_n \in M$ s.t.
  $M = \gen{x_1, \dots, x_n}_R = Rx_1 + Rx_2 + \dots Rx_n$
\end{definition}

\begin{example}
  $R$ is generated by $1$ as an $R$-module.
\end{example}

\begin{definition}
  An additive group homo. $\varphi: M_1 \to M_2$ is called an $R$-module
  homo. if
  \[ \varphi(rx) = r\varphi(x) \quad \forall r \in R, x \in M_1 \]
\end{definition}

\begin{definition}
  An integral domain $R$ is called a principal ideal domain (PID)
  if $\forall I$ ieal in $R$, $\exists a \in R$ s.t. $I = \gen{a}_R$.
\end{definition}

\begin{example}
  $\Zb$ is a PID.

  For $I \subseteq \Zb$, $I$ is an additive subgroup, so
  $I = m\Zb = \gen{m}_\Zb$.
\end{example}

\begin{definition}
  $M$ is said to be a free module of rank $n$ if
  $M \cong R^n = R \oplus \dots \oplus R$ (or $R \times \dots \times R$)
\end{definition}

\begin{theorem}
  If $R$ is a PID, then any submodule of $R^n$ is free of rank $\le n$.
  \begin{proof}
    By induction on $n$, $n = 1$, $\forall I \subseteq R, \exists a \in R$ s.t.
    $I = \gen{a}_R = Ra \cong R$ ({\bf as a $R$-module}).

    Let $n > 1$ and $N$ be a submodule of $R^n$.
    Consider
    \[\arraycolsep=1pt
      \pi_1:
      \begin{array}{ccl}
        R^n & \to & R \\
        (r_1, \dots, r_n) & \mapsto & r_1
      \end{array}
      \quad \text{and} \quad
      \pi = \pi_1\Big|_N: N \to R
    \]
    \begin{description}
      \item[case 1:] $\Image \pi = \{0\}$. In this case,
        $N \subseteq \ker\pi_1 \cong R^{n-1}$.
        By induction hypothesis, $N$ is free of rank $\le n-1 < n$.
      \item[case 2:] $\Image \pi = \gen{a}$, say $\pi(x) = a$.
        Claim: $N = Rx \oplus \ker\pi,
        \ker \pi \subseteq \ker \pi_1 \cong R^{n-1}$.
        \begin{itemize}
          \item $Rx \cap \ker\pi = \{0\}$:
            $rx \in Rx \cap \ker\pi \implies \pi(rx) = 0 \implies r = 0
            \implies rx = 0$
          \item $\forall y \in N, \pi(y) = r_0a $ for some $r_0 \in R$,
            $\pi(y - r_0x) = 0 \implies y - r_0x \in \ker\pi$.
            So $N \subseteq Rx \oplus \ker\pi$. \qedhere
        \end{itemize}
    \end{description}
  \end{proof}
\end{theorem}

Recall that the elementary matrices are
\begin{itemize}
  \item $D_i(u) = \diag(1,\dots,1,u,1, \dots,1)$.
    $D_i(u) \in \text{GL}(n, R)$ if $u$ is a unit.
  \item $B_{ij}(a) = I_n + ae_{ij}, a\in R, i \ne j$.
    $B_{ij}(a)^{-1} = B_{ij}(-a) \implies B_{ij}(a) \in \text{GL}(n, R)$.
  \item $P_{ij} = I_n - e_{ii} - e{jj} + e_{ij} + e_{ji}$.
\end{itemize}

\begin{fact}
  If $R$ is a PID and $\gen{a, b}_R = \gen{d}_R$, then $d = \gcd(a, b)$.
  \begin{proof} \mbox{}
    \begin{itemize}
      \item
        $a \in \gen{d}_R \implies a = rd$ for some $r \in R \implies d \Div a$.
        $v \in \gen{d}_R \implies d \Div b$.
      \item Let $c \Div a, c \Div b$, say $a = k_1c, b = k_2c$.
      $d \in \gen{a, b}_R \implies d = x_1a + x_2b$ for some $x_1, x_2 \in R$.
      So $d = x_1k_1c + x_2k_2c = (x_1k_1 + x_2k_2)c \implies c \Div d$.
      \qedhere
    \end{itemize}
  \end{proof}
\end{fact}

\begin{theorem}
  Let $R$ be a PID and $A \in M_{n\times m}(R)$. Then
  $\exists P \in \text{GL}_n(R)$ and $Q \in \text{GL}_m(R)$ s.t.
  \[
    PAQ = \begin{pmatrix}
    d_1 \\
    & d_2 \\
    & & \ddots \\
    & & & d_r \\
    & & & & 0 \\
    & & & & & \ddots \\
    & & & & & & 0
    \end{pmatrix}
    \quad \text{with} \quad
    d_i \Div d_{i+1} \quad \forall i = 1, \dots, r-1
  \]
  \begin{proof}
    Define the length $l(a)$ of $a \ne 0$ to be $r$ if $a = p_1p_2 \dots p_r$
    where $p_1, \dots, p_r$ are prime elements.

    prime elements: $p \Div ab \implies p \Div a \text{~or~} p \Div b$.

    \begin{enumerate}
      \item We may assume $a_{11} \ne 0$ and $l(a_{11}) \le l(a_{ij})
        \forall a_{ij} \ne 0$. (換一換就上去了...XD)
      \item We may assume $\begin{cases}
          a_{11} \Div a_{1k} & \forall k = 2, \dots, m \\
          a_{11} \Div a_{k1} & \forall k = 2, \dots, n
        \end{cases}$.
        If $a_{11} \nmid a_{1k}$, then we can interchange 2nd and $k$th
        columns to assume $a = a_{11} \nmid a_{12} = b$.

        Let $d = \gcd(a, b) \implies \begin{cases}
          l(d) < l(a) \\
          d = ax + by \text{~for some~} x, y \in R
        \end{cases} \implies 1 = \frac{a}{d}x + \frac{b}{d}y$.
        &
        Write $b' = \frac{b}{d}, a' = -\frac{a}{d}$.
        Then
        \[
          \begin{pmatrix} -a' & b' \\ y & -x \end{pmatrix}
          \begin{pmatrix} x & b' \\ y & a' \end{pmatrix}
          = I_2
        \]
        反正就是移一下減掉, length 會一直變小 $\implies$ 這個操作會停.
      \item 有這個 $\begin{cases}
          a_{11} \Div a_{1k} & \forall k = 2, \dots, m \\
          a_{11} \Div a_{k1} & \forall k = 2, \dots, n
        \end{cases}$ 就可以全部消掉變成
        \[
          \begin{pmatrix}
            a_{11} & 0 & \dots & 0 \\
            0 & b_{22} & \dots & b_{2m}\\
            \vdots & \vdots & \ddots & \vdots \\
            0 & b_{n2} & \dots & b_{nm}
          \end{pmatrix}
        \]
      \item May assume $a_{11} \Div b_{kl} \quad \forall k, l$.
        不是的話就把該 row 往第一 row 加上去,重複前面的操作,
        $l(a_{11})$ 總是變小,因此會停.
      \item 遞迴下去...
    \end{enumerate}
    最後就弄出想要的矩陣了.
  \end{proof}
\end{theorem}


%! TEX root=../main.tex
\subsection{Week 8}
\subsubsection{Fundamental theorem of finitely generated abelian groups}

\begin{theorem}[Structure theorem of finitely generated module over a PID]
  Let $R$ be a PID and $M$ be a finitely generated $R$-module.
  Then $M \cong \quot{R}{d_1 R} \oplus \dots \oplus \quot{R}{d_l R} \oplus R^s,
  d_i \in R$ with $d_i \Div d_{i+1} \quad \forall i = 1, \dots, l-1$
  for some $s \in \Zb^{\ge 0}$.

  \begin{proof}
    Let $M = \gen{x_1, \dots, x_n}_R$ and consider
    \[
      \arraycolsep=1pt
      \begin{array}{rcl}
        \varphi: & R^n & \onto M \\
                 & e_i & \to x_i
      \end{array}
    \]
    By 1st isom. thm., $\quot{R^n}{\ker \varphi} \cong M$.

    We know $\ker \varphi \cong R^m$ ($e_i' \mapsto f_i, e_i' \in R^m$)
    for some $m \le n$ and
    $\forall x \in \ker\varphi \quad \exists! x_1, \dots, x_m \in R$ s.t.
    $x = \sum_{i=1}^m x_i f_i$.

    Note that $\ker \varphi \subseteq R^n$. So we can write
    $f_i = \sum_{j=1}^n a_{ji}e_j \quad \forall i = 1,\dots, m$.
    Then $x = \sum x_i \sum a_{ji}e_j =
    \sum \left(\sum a_{ji}x_i\right) e_j$.

    $R$ is a PID $\implies \exists P \in \text{GL}_n(R), Q \in \text{GL}_m(R)$
    s.t.
    \[
      PAQ = \begin{pmatrix}
      d_1 \\
      & \ddots \\
      & & d_r \\
      & & & 0 \\
      & & & & \ddots
      \end{pmatrix}
      \quad \text{with} \quad
      d_i \Div d_{i+1} \quad \forall i = 1, \dots, r-1
    \]
    So consider $[w_i] = Q e_i$. Since $P, Q$ invertible, $R^n = \bigoplus R w_i, \ker \varphi = \bigoplus d_i R w_i$
    Hence
    \[ M \simeq R / ker \varphi = \bigoplus R w_i / \bigoplus d_i R w_i = \bigoplus R / d_i R \]
  \end{proof}

  $\arraycolsep=3pt
  \begin{array}{rcl}
    R & \onto & \quot{Rw_i}{Rd_i'w_i} \\
    1 & \to & \ob{w_i} \\
    r & \to & \ob{rw_i}
  \end{array}
  $
\end{theorem}

\begin{remark}
  If $R$ is commutative, then ``$R^n \cong R^m \implies n = m$.''
\end{remark}

\begin{theorem}
  Let $G$ be a finitely generated abelian group. Then
  Then $G \cong \quot{\Zb}{d_1 \Zb} \oplus \dots \oplus \quot{\Zb}{d_l \Zb}
  \oplus R^s, d_i \in \Zb$ with $d_i \Div d_{i+1} \quad
  \forall i = 1, \dots, l-1$ for some $s \in \Zb^{\ge 0}$.

  Since $G$ can be regarded as a f.g. $\Zb$-module and $\Zb$ is a PID,
  it follows from the main theorem.

  $\Tor(G) = \quot{\Zb}{d_1 \Zb} \oplus \dots \oplus \quot{\Zb}{d_l \Zb}
  \le G$ and $\quot{G}{\Tor(G)} \cong \Zb^s$ (free part of $G$).
\end{theorem}

\begin{fact}
  If $d = p_1^{m_1}p_2^{m_2}\dots p_s^{m_s}$, then
  $\quot{\Zb}{d\Zb} \cong \quot{\Zb}{p_1^{m_1}\Zb} \oplus
  \quot{\Zb}{p_2^{m_2}\Zb} \oplus \dots \oplus \quot{\Zb}{p_s^{m_s}\Zb}$.
\end{fact}

\begin{theorem}[Chinese Remainder theorem]
  Let $R$ be a commutative ring with $1$ and $I_1, \dots, I_n$ be ideals of $R$.
  Then
  \[
    \arraycolsep=3pt
    \begin{array}{rccl}
      \varphi: & R & \to & \quot{R}{I_1} \times \dots \times \quot{R}{I_n} \\
               & r & \mapsto & (\ob{r}, \dots, \ob{r})
    \end{array}
    \text{~is a ring homo.}
  \]
  and
  \begin{enumerate}[(1)]
    \item if $I_i, I_j$ are coprime $\forall i \ne j$, then
      $I_1I_2\dots I_n = I_1 \cap I_2 \cap \dots \cap I_n$.
    \item $\varphi$ is surjective $\iff$ $I_i, I_j$ are coprime
      $\forall i \ne j$.
    \item $\varphi$ is injective $\iff$ $I_1 \cap I_2 \cap \dots \cap I_n
      = \{ 0 \}$.
  \end{enumerate}
  So if $I_i, I_j$ are coprime $\forall i \ne j$, then
  \[
    \quot{R}{I_1I_2\dots I_n} \cong
    \quot{R}{I_1} \times \dots \times \quot{R}{I_n}.
  \]

  $I_i, I_j$ are coprime $\iff$ $I_i + I_j = R$.

  \begin{proof}
    we only need to prove (1), (2).

    \begin{enumerate}[(1)]
      \item By induction on $n$. $n = 2$, need $I_1\cap I_2 \subseteq I_1I_2$.
        Indeed, $I_1\cap I_2 = (I_1\cap I_2)R = (I_1\cap I_2)(I_1+I_2)
        \subseteq I_1I_2$.

        For $n > 2$, since $I_i + I_n = R \quad \forall i = 1,\dots, n-1$,
        $\exists x_i \in I_i, y_i \in I_n$ s.t. $x_i + y_i = 1 \quad
        \forall i = 1,\dots, n-1$.

        So $x_1x_2\dots x_{n-1} = (1-y_1)(1-y_2)\dots(1-y_{n-1}) = 1 - y,
        y \in I_n
        \implies I_1I_2\dots I_{n-1} + I_n = R$.

        Now, $I_1I_2\dots I_n = (I_1\dots I_{n-1})I_n =
        (I_1\dots I_{n-1})\cap I_n = I_1\cap \dots \cap I_n$.
      \item ``$\Rightarrow$'': WLOG, we may let $I_i = I_1, I_j = I_2$.
        We have $x \in R$ s.t.
        \[
          \varphi(x) = (\ob{1}, \ob{0}, \dots, \ob{0})
          \quad \text{i.e.~}
          \ob{x} = \ob{1} \text{~in~} \quot{R}{I_1}
        \]
        Write $x \equiv 1 \pmod {I_1}$.
        Since $1 - x \in I_1, x \in I_2$ and $(1 - x) + x = 1$, $I_1 + I_2 = R$.

        ``$\Leftarrow$'': $\forall y \in \text{RHS}$,
        $y = (\ob{r_1}, \dots, \ob{r_n})$.
        If we may find that $x_i \in R$ s.t.
        $\varphi(x_i) = (\ob{0}, \dots, \ob{1}, \ob{0}, \dots, \ob{0})$,
        then
        \[
          \varphi\left(\sum_{i=1}^n r_ix_i \right) = y
        \]

        It is enough to show, for example, $\exists x \in R$ s.t.
        $\varphi(x) = (\ob{1}, \ob{0}, \dots, \ob{0})$.

        Since $I_1 + I_i = R \quad \forall i = 2, \dots, n$,
        $\exists x_i \in I_1, y_i \in I_i$ s.t. $x_i + y_i = 1
        \forall i = 2, \dots, n$.

        So let $x = y_2\dots y_n = (1-x_2)\dots(1-x_n)$.
        We have $x \in I_2, \dots, I_n$ and $x \equiv 1 \pmod {I_1}$.
    \end{enumerate}
  \end{proof}
\end{theorem}

\begin{example}
  $\abs{G} = 72$ and $G$ is abelian:
  \[
    72 = 2 \times 36 = 3 \times 24 = 2 \times 2 \times 18
    = 6 \times 12 = 2 \times 6 \times 6
  \]
  Invariant factors

  Elementary divisors
\end{example}

\begin{definition}
  The exponent of $G$ with $\abs{G} < \infty$ is
  \[
    \Exp(G) \defeq \min \left\{
      m \in \Nb \middle| g^m = 1 \quad \forall g \in G
    \right\}
  \]
\end{definition}

\begin{exercise} \mbox{}
  \begin{enumerate}
    \item Let $G$ be abelian with $\abs{G} = n$. Show that if $d \Div n$, then
      $\exists H \le G$ s.t. $\abs{H} = d$.
    \item If $n=540, d=90$, then construct all possible $G$ and
      corresponding $H$.
  \end{enumerate}
\end{exercise}

\begin{exercise}
  Let $G$ be abelian with $\abs{G} < \infty$. Show that $G$ is cyclic
  $\iff \Exp(G) = \abs{G}$.
\end{exercise}

\begin{exercise}
  Let $f_i(x) \in \Zb[x], i = 1, \dots, k$ with $\deg f_i = d$ and
  $p_1, \dots, p_k$ be distinct primes.
  Show that $\exists f(x) \in \Zb[x]$ with $\deg f = d$ s.t.
  $\ob{f}(x) = \ob{f_i}(x)$ in 
  $\quot{\Zb}{p_i\Zb}[x] \quad \forall i = 1, \dots k$.

  $f(x) = a_d x^d + \dots + a_0,
  \ob{f}(x) = \ob{a_d} x^d + \dots + \ob{a_0}$
\end{exercise}

\subsubsection{Sylow theorems}

\begin{definition}
  Let $\abs{G} = p^\alpha r$ with $p \nmid r$.
  \begin{enumerate}
    \item If $H \le G$ with $\abs{H} = p^\alpha$, then we call $H$ a Sylow
      $p$-subgroup of $G$.
    \item $\text{Syl}_p(G) =$ the set of all Sylow $p$-subgroups of $G$.
    \item $n_p = \abs{\text{Syl}_p(G)}$.
  \end{enumerate}
\end{definition}

\begin{lemma}[Key lemma]
  Let $P \in \text{Syl}_p(G)$ and $Q$ be a $p$-subgroup of $G$. Then
  $Q \cap N_G(P) = Q \cap P$.

  \begin{proof}
    By Lagrange theorem, $H = Q \cap N_G(P)$ is also a $p$-subgroup of
    $N_G(P)$ since $\abs{H} \Div \abs{Q}$.

    Since $\begin{cases}
      P \lhd N_G(P) \\
      H \le N_G(P)
    \end{cases} \implies HP \le N_G(P)$, we have
    \[
      \abs{HP} = \frac{\abs{H}\abs{P}}{\abs{H\cap P}} = p^{\alpha+k-s}
    \]
    where $\abs{H\cap P} = p^s, s \le k$. Then
    $p^{\alpha+k-s} \Div \abs*{N_G(P)} \Div \abs{G} = p^\alpha r$.

    So $k = s \implies H = H \cap P \implies H \le P \cap Q$.
  \end{proof}
\end{lemma}

\begin{theorem}[Sylow \RNum{1}]
  $\forall 0 \le k \le \alpha$, $\exists H \le G$ s.t. $\abs{H} = p^k$.
  In particular, $\text{Syl}_p(G) \ne \phi$.

  \begin{proof}
    By induction on $\abs{G}$. If $\abs{G} = 1$, then $k = 0$, $H = \{1\}$.

    Assume $\abs{G} > 1, k \ge 1, \alpha \ge 1$.

    \begin{description}
      \item[case 1:] $p \Div \abs{Z_G}$. By Cauchy theorem,
        $\exists a \in Z_G$ with $\ord(a) = p$.
        Then $\gen{a} \lhd G$ and $\abs*{\quot{G}{\gen{a}}} = p^{\alpha-1} r
        \le \abs{G}$.
        If $k=1$, then $H = \gen{a}$.
        Otherwise, we may assume that $1\le k-1\le \alpha-1$. By induction
        hypothesis, $\exists H' = \quot{G}{\gen{a}}$ s.t. $\abs{H'} = p^{k-1}$.
        By 3rd isom. thm., we can write $H' = \quot{H}{\gen{a}}$ and thus
        $\abs{H} = p^k$.
      \item[case 2:] $p \nDiv \abs{Z_G}$. By the class equation,
        $\abs{G} = \abs{Z_G} + \sum_{i=1}^m \frac{\abs{G}}{\abs{Z_G(a_i)}},
        a_i \in Z_G$.

        In this cases, $\exists a_j$ s.t.
        $p \nDiv \frac{\abs{G}}{\abs{Z_G(a_j)}} \implies
        p^\alpha \Div \abs{Z_G(a_j)}$. And $Z_G(a_j) \lneq G$ since
        $a_j \not\in Z_G$.
        By induction hypothesis, $\exists H \le Z_G(a_j) \le G$ s.t.
        $\abs{H} = p^k$. \qedhere
    \end{description}
  \end{proof}
\end{theorem}

\begin{theorem}[Sylow \RNum{2}]
  Let $P \in \text{Syl}_p(G)$ and $Q$ be a $p$-subgroup of $G$. Then
  $\exists a \in G$ s.t. $Q \le aPa^{-1}$.
  In particular, $\forall P_1, P_2 \in \text{Syl}_p(G), \exists a \in G$
  s.t. $P_2 = aP_1a^{-1}$.
  \begin{proof}
    Let $X = \{\, \text{left cosets of $P$} \,\}$ and consider
    $\arraycolsep=1pt \begin{array}{rcl}
      Q \times X & \to & X \\
      (a, xP) & \mapsto & axP
    \end{array}$.

    Observe that $xP \in \Fix Q \iff axP = xP \quad \forall a \in Q \iff
    x^{-1}axP = P \quad \forall a \in Q \iff
    x^{-1}ax \in P \quad \forall a \in Q \iff
    a \in xPx^{-1} \quad \forall a \in Q$.

    We know $\abs{\Fix Q} \equiv \abs{X} \pmod p$ and $p \nmid r \implies$
    $\abs{\Fix Q} \ne 0 \iff \exists a \in G, Q \le aPa^{-1}$.

    In particular, $\begin{cases}
      P_2 \le aP_1a^{-1} \\
      \abs{P_2} = \abs{aP_1a^{-1}}
    \end{cases} \implies P_2 = aP_1a^{-1}$.
  \end{proof}
\end{theorem}

\begin{theorem}[Sylow \RNum{3}]
  $n_p \equiv 1 \pmod p$ and $n_p \mid r$.
  \begin{proof}
    \begin{itemize}
      \item Consider $\arraycolsep=1pt \begin{array}{ccrcl}
          &P \times &\text{Syl}_p(G) & \to & \text{Syl}_p(G) \\
          (&a, &Q) & \mapsto & aQa^{-1}
        \end{array}$ where $P \in \text{Syl}_p(G)$.

        $P' \in \Fix P \iff aP'a^{-1} = P' \quad \forall a \in P
        \iff P \le N_G(P') \cap P = P' \cap P \iff P' = P$.

        So $\Fix P = \{ P \} \implies n_p \equiv \abs{\Fix P} = 1 \pmod p$.

      \item Consider $\arraycolsep=1pt \begin{array}{ccrcl}
          &G \times &\text{Syl}_p(G) & \to & \text{Syl}_p(G) \\
          (&a, &Q) & \mapsto & aQa^{-1}
        \end{array} \implies$ There is only one orbit $\text{Syl}_p(G)$.
        
        We know $\abs{\text{Syl}_p(G)} = \frac{\abs{G}}{\abs{G_Q}}$
        and $G_Q = N_G(Q)$. Then $n_p = \frac{\abs{G}}{\abs{G_Q}} \Div \abs{G}$.
        So $n_p \Div p^\alpha r \implies n_p \Div r$.
    \end{itemize}
  \end{proof}
\end{theorem}

\begin{prop}
  Let $\abs{G} = pq$ where $p, q$ are primes with $\begin{cases}
    p < q \\
    q \not\equiv 1 \pmod p.
  \end{cases}$
  Then $G \cong C_{pq}$.
  \begin{proof}
    $n_p = 1+kp \mid q \implies n_p = 1$ i.e.
    $H \in \text{Syl}_p(G) \implies H \lhd G$.

    $n_q = 1+kq \mid p \implies n_q = 1$ i.e.
    $K \in \text{Syl}_q(G) \implies K \lhd G$.

    Since $\gcd(p, q) = 1$, $H \cap K = 1$.
    Hence $G = H \times K \cong C_p \times C_q \cong C_{pq}$.
  \end{proof}
\end{prop}

\begin{example}
  Consider $\abs{G} = 255 = 3 \times 5 \times 17$.
  \begin{enumerate}
    \item 找兩個 normal subgroup (17, 5 or 3)
    \item quot 掉後發現剩下的是 abelian $\leadsto$ $[G, G]$ 在裡面
    \item $[G, G] = 1$
    \item 唱 f.g. xxx thm. 得到 $G \cong \Zb_3 \times \Zb_5 \times \Zb_{17}$.
    \item 中國剩飯定理 $G \cong C_{255}$.
  \end{enumerate}
\end{example}

\begin{exercise}
  If $\abs{G} = 7 \times 11 \times 19$, then $G$ is abelian.
\end{exercise}

\begin{example}
  No group $G$ of order $48 = 2^4 \times 3$ is simple.
  \begin{enumerate}
    \item $n_2 = 1 + 2k \Div 3 \leadsto n_2 = 1 \text{~or~} 3$.
    \item $n_2 = 1$ then OK.
    \item Assume $n_2 = 3$. Let $P \in \text{Syl}_2(G),
      X = \{\, \text{left cosets of $P$} \,\}$ ($\abs{X} = 3$).
    \item Consider $\arraycolsep=1pt \begin{array}{ccrcl}
        &G \times &X & \to & X \\
        (&a, &xP) & \mapsto & axP
      \end{array} \leadsto \varphi: G \to S_3$.
    \item 考慮 $\ker\varphi$.
  \end{enumerate}
\end{example}

\begin{exercise}
  No group $G$ of order $36$ is simple.
\end{exercise}

\begin{exercise}
  No group $G$ of order $30$ is simple.
\end{exercise}

\begin{exercise}
  Let $\abs{G} = 385$. Show that $\exists P \in \text{Syl}_7(G)$ s.t.
  $P \le Z_G$.
\end{exercise}

%1 TEX root=../main.tex
\subsection{Week 9}
\subsubsection{Classification}

To classify groups of small orders: 
\begin{itemize}
  \item $\abs{G} = 1$: $G = \{1\}$
  \item $\abs{G} = 2$: $G \cong C_2$
  \item $\abs{G} = 3$: $G \cong C_3$
  \item $\abs{G} = 4$: $G \cong \Zb_4 \text{~or~} \Zb_2 \times \Zb_2$
  \item $\abs{G} = 5$: $G \cong C_5$
  \item $\abs{G} = 6$: $n_3 = 1, n_2 = 1 \text{~or~} 3$. Let $H \in \Syl_3(G)$ and $H \lhd G$. Let $K \in \Syl_2(G)$. Also $H \cap K = \{1\}$ and $HK = G$ then $G \cong K \times_{\tau} H$
    \begin{itemize}
      \item If $\tau$ is trivial: $G \cong K \times H \cong C_2 \times C_3 \cong C_6$
      \item $\tau: b \mapsto \phi_2: \gen{a} \to \gen{a}$: $G \cong K \times_{\tau} H \cong \gen{a,b \mid a^3 = 1, b^2 = 1, bab^{-1} = a^2 = a^{-1}} \cong D_3$
    \end{itemize}
  \item $\abs{G} = 7$: $G \cong C_7$
  \item $\abs{G} = 8$:
    \begin{itemize}
      \item If abelian: $\Zb_8$ or $\Zb_4 \times \Zb_2$ or $\Zb_2 \times \Zb_2 \times \Zb_2$
      \item If non-abelian:
        \begin{itemize}
          \item $\not\exists a \in G$ with $\ord(a) = 8$
          \item Not each $a \in G$ with $a^2 = 1$, otherwise $G$ is abelian.
          \item $\exists a \in G$ with $\ord(a) = 4$: Let $H = \gen{a}$ and $H \lhd G$ since $[G:H] = 2$. Pick $b \in G \setminus H$ and $K = \gen{b}$
            \begin{itemize}
              \item $\ord(b) = 2$: $H \cap K = \{1\}$ and $HK = G$ then $G \cong K \times_{\tau} H$, $\tau: b \mapsto \phi: a \mapsto a^3$: $G \cong K \times_{\tau} H \cong \gen{a,b \mid a^4 = 1, b^2 = 1, bab^{-1} = a^3 = a^{-1}} \cong D_4$
              \item $\ord(b) = 4$: $H \cap K = \gen{a^2 = b^2}$. Then consider $bab^{-1} \in H \implies bab^{-1} = 1,a,a^2,a^3$
                \begin{enumerate}
                  \item $1,a$ obviously wrong.
                  \item $bab^{-1} = a^2$: $a = a^2aa^{-2} = b^2ab^{-2} = a^4 \implies a^3 = 1$ 矛盾
                  \item So $bab^{-1} = a^3 = a^{-1}$.
                \end{enumerate}
                $G \cong \gen{a,b \mid a^4 = 1, b^4 = 1, a^2 = b^2, bab^{-1} = a^3 = a^{-1}} \cong Q_8$
            \end{itemize}
        \end{itemize}
    \end{itemize}
  \item $\abs{G} = 9$: $G \cong \Zb_9 \text{~or~} \Zb_3 \times \Zb_3$
  \item $\abs{G} = 10$: $G \cong K \times H \cong C_2 \times C_5 \cong C_{10}$ or $G \cong D_5$
  \item $\abs{G} = 11$: $G \cong C_{11}$
  \item $\abs{G} = 12$: Claim: If $\abs{G} = 12$, then either $G$ has a normal Sylow $3$-subgroup or $G \cong A_4$.
    \begin{proof}
      By Sylow 3, $n_3 = 1+3k \mid 4 \implies n_3 = 1 \text{~or~} 4$.
      \begin{itemize}
        \item If $n_3 = 1$, then $G$ has a normal Sylow $3$-subgroup.
        \item Otherwise, let $P \in \Syl_3(G)$ and $X = \left\{ \text{left cosets of }P \right\}, \abs{X} = 4$.
          Consider $G \times X \to X$ defined by $(a,xP) \mapsto axP$ with $\phi: G \to S_4$.
          And $\ker \phi \le P$, $\abs{P} = 3$ and $P \centernot{\lhd} G$ (since $n_3=4$), so $\ker \phi = \{1\}$.

          And since $n_3 = 4$, there are $8$ elements of order $3$ which corresponds 
          to $8$ $3$-sycles in $A_4$, thus $\abs{\Image \phi \cap A_4} \ge 8$.
          But $\abs{\Image \phi \cap A_4} \Div \abs{A_4} = 12 \implies \Image \phi = A_4$
      \end{itemize}
    \end{proof}
    Now, for the case where $\exists H \in \Syl_3(G)$ and $H \lhd G$.
    Let $K \in \Syl_2(G)$, then $K \cap H = \{1\}$ and $KH = G \implies G \cong K \times_\tau H$
    for some $\tau: K \to \Aut(H) = \{\text{id}, \phi_2\}$
    \begin{itemize}
      \item $\tau$ is trivial: $\Zb_{12}$ or $\Zb_2 \times \Zb_6$.
      \item $\gen{b} = K \cong \Zb_4$: $\tau(b) = \phi_2 \implies G = \gen{a,b \mid a^3=1, b^4=1, bab^{-1} = a^{-1}} \not\cong D_6,A_4$
      \item $\gen{b} = K \cong \Zb_2 \times \Zb_2$: Let $K = \gen{b,c\mid b^2=1, c^2=1, bc=cb}$,
        then $\tau: b \mapsto \phi_2$ and $c \mapsto \text{id}$ (the other cases are equivalent to this one),
        $G = \gen{a,b,c \mid a^3 = 1, b^2 = 1, c^2 = 1, bc=cb,bab^{-1} = a^{-1}, cac^{-1} = a} \cong \gen{a,b\mid a^3=1,b^2=1, bab^{-1} = a^{-1}} \times \gen{c} \cong D_3 \times C_2 \cong D_6$
        \begin{fact}
          For odd $n$, $D_{2n} \cong D_n \times \quot{\Zb}{2\Zb}$.
          \begin{proof}
            \[D_{2n} = \gen{a,b\mid a^{2n}=1,b^2=1, bab^{-1} = a^{-1}}\]
            \[H = \gen{a^2,b\mid (a^2)^{n}=1,b^2=1, b(a^2)b^{-1} = a^{-2}} \cong D_n\]
            \[K = \gen{a^n} \cong C_2\]
            And $n$ is odd, so $H \cap K = \{1\}$ and $D_{2n} \cong D_n \times C_2$
          \end{proof}
        \end{fact}
    \end{itemize}
  \item $\abs{G} = 13$: $G \cong C_{13}$
  \item $\abs{G} = 14$: $G \cong C_{14} \text{~or~} D_7$
  \item $\abs{G} = 15$: $G \cong C_{15}$
\end{itemize}

\begin{exercise} \mbox{}
  Assume that $K$ is cyclic and $H$ is an arbitrary group.
  Let $\tau_1: K \to \Aut(H)$, $\tau_2: K \to \Aut(H)$ with $\tau_1(K) \sim \tau_2(K)$ (conjugate).
  If $\abs{K} = \infty$, then assume that $\tau_1$ and $\tau_2$ are injective. Show that $K \times_{\tau_1} H \cong K \times_{\tau_2} H$.
\end{exercise}

\begin{exercise}
  Classify $G$ if $\abs{G} = p^3$ with $p$ an odd prime and each nontrivial element of $G$ has order $p$.
\end{exercise}

\begin{exercise}
  Classify groups of order $30$.
\end{exercise}

\subsubsection{Free groups}
A free group generate by a non-empty set $X$ is that
there are no relations satisfied by any of elements in $X$.

\begin{definition}
  A free group on $X$ is a group $F$ with an inclusion map $i: X \to F$ satisfying the
  following universal property: For any group $G$ and any map $f: X \to G$,
  exists a unique group homo $\varphi : F \to G$ that the following diagram commutes.
  \[
    \begin{tikzcd}
    X \arrow[r]{i} \arrow[swap, dr]{f} & F \arrow{d}{\varphi} \\
    & G
    \end{tikzcd}
  \]
\end{definition}

\begin{theorem}
  $F$ exists and is unique up to isomorphism. (Denote it as $F(X) = F$).
\end{theorem}

\begin{proof}
  For $X$, we create a new disjoint set $X^{-1} = \{ x^{-1} : x \in X\}$
  and an element $1 \notin X \cup X^{-1}$.

  Define $F(X) = \{ 1 \} \cup \left\{ x_1^{\delta_1} x_2^{\delta_2} \cdots x_m^{\delta_m} :
   m \in \mathbb{N} , x_i \in X, \delta_i = \pm 1, x_{i+1}^{\delta_{i+1}} \neq
   \left( x_i^{\delta_i} \right)^{-1}\right\}$, and
  \[ x_1^{\delta_1} x_2^{\delta_2} \cdots x_m^{\delta_m} =
    y_1^{\epsilon_1} y_2^{\epsilon_2} \cdots y_m^{\epsilon_m}  \iff n = m \text{ and }
    \delta_i = \epsilon_i \text{ and } x_i = y_i , \forall i \]

  For each $y \in X \cup X^{-1}$, we define $\sigma_y : F(X) \to F(X)$ by
  \[
    \sigma_y (x_1^{\delta_1} x_2^{\delta_2} \cdots x_m^{\delta_m})
    =
    \begin{cases}
      y x_1^{\delta_1} x_2^{\delta_2} \cdots x_m^{\delta_m} & \text{if } x_1^{\delta_1} \neq y^{-1} \\
      \begin{cases}
        x_1^{\delta_1} x_2^{\delta_2} \cdots x_m^{\delta_m} & (m \geq 2) \\
        1 & (m = 1)
      \end{cases} & \text{if } x_1^{\delta_1} = y^{-1}
    \end{cases}
  \]

  Then $\sigma_y$ is a permutation of $F(X)$, since if
  $ \sigma_y(x_1^{\delta_1} x_2^{\delta_2} \cdots x_m^{\delta_m}) =
  \sigma_y(y_1^{\epsilon_1} y_2^{\epsilon_2} \cdots y_m^{\epsilon_m}) $.
  \begin{itemize}
    \item[m = n:] either $x_1^{\delta_1} = y_1^{\epsilon_1} = y^{-1}$ or not,
      then either $x_2^{\delta_1} x_3^{\delta_2} \cdots x_m^{\delta_m} =
  y_2^{\epsilon_1} y_3^{\epsilon_2} \cdots y_m^{\epsilon_m}$ or
  $y x_1^{\delta_1} x_2^{\delta_2} \cdots x_m^{\delta_m} =
  y y_1^{\epsilon_1} y_2^{\epsilon_2} \cdots y_m^{\epsilon_m}$. Both of them leads to
  $x_1^{\delta_1} x_2^{\delta_2} \cdots x_m^{\delta_m} =
  y_1^{\epsilon_1} y_2^{\epsilon_2} \cdots y_m^{\epsilon_m}$.
  \item[m = n+2:] Omimi
  \end{itemize}
  Also $\sigma_y$ is onto since omimi. And notice that $\sigma_{y^{-1}} \circ \sigma_y = id_{F(X)}$

  Define $A = \langle \sigma_x : x \in X \rangle \leq S_{F(X)}$. and define $\phi: F(X) \to A$ by
  $\phi(1) = id_{F(X)}$ and \\ $x_1^{\delta_1} \cdots x_m^{\delta_m} \mapsto \sigma_{x_1}^{\delta_1}
  \cdots \sigma_{x_m}^{\delta_m}$. The it is omimi that $\phi$ is a bijection. So we define
  $x::X \cdot y::X = \phi^{-1}( \phi(x) \circ \phi(y) )$.

  The $\phi$  in the universal property could be defined as
  $\phi(x_1^{\delta_1} x_2^{\delta_2} \cdots x_m^{\delta_m}) = f(x_1)^{\delta_1} \cdots f(x_m)^{\delta_m}$.
\end{proof}

\begin{prop}
  Let $G = \langle a_1, \cdots , a_n \rangle$ and $X = \{ x_1, \cdots, x_m \}$. Then
  $G \cong \quot{F(X)}{K}$ for some normal subgroup $K$. $K$ is called the subgroup of relations
  connecting the generators.

  Define $f = x_i :: X_i \to a_i :: G$. By universal property, $\exists \phi
  = x_i :: F(X) \mapsto a_i :: G$. Then
  $\quot{F(x)}{\ker \phi} \cong G$.
\end{prop}

\begin{definition}
  Let $X = \{x_1, x_2, \cdots, x_n\}$ and $R \subset F(X)$.
  Let $N(R)$ be the smallest normal subgroup of $F(X)$ containing $R$,
  Then $G = \quot{F(X)}{N(R)}$ is written as $\langle x_1, \cdots, x_n \mid \text{elements of } R \rangle$,
  which is called a presentation of $G$. If $\abs{R} < \infty$, then $G$ is said to be finitely
  presented.
\end{definition}

\begin{example}
  \[ D_n = \gen*{
    \begin{bmatrix}
      \cos \frac{2\pi}{n} & -\sin \frac{2\pi}{n} \\
      \sin \frac{2\pi}{n} &  \cos \frac{2\pi}{n}
    \end{bmatrix},
    \begin{bmatrix}
      1 & 0 \\
      0 & -1
    \end{bmatrix}
    }
  \]

  We find that $x^n , y^2 , xyxy \in \ker \phi$. Then $R = \{ x^n , y^2 , xyxy \} \subseteq \ker \phi
  \implies N(R) \leq \ker \phi$.  By factor theorem, $\exists \bar{\phi} :: \quot{F(X)}{N(R)} \to D_n$.
  But notice that
  \[ \abs*{\quot{F(x)}{N(R)}}  \leq 2n \]
  since $xyxy = 1 \implies xy = yx^{-1}$, so every element could be turn into
  $x^i y^j$. Hence $\bar{\phi}$ is an isomorphism.
\end{example}

\begin{prop}
  Let $X = \{x_1, x_2, \cdots, x_n\}$. Then $\quot{F(X)}{[F(X), F(X)]} \cong \mathbb{Z}^n$.
\end{prop}

\begin{proof}
  Define $f = x_i :: X \mapsto e_i :: \mathbb{Z}^n$. Then $\phi = x_i :: F(X) \mapsto e_i :: \mathbb{Z}^n$.
  By 1st isomorphism theorem $\quot{F(X)}{\ker \phi} \cong \mathbb{Z}^n$ which is abelian,
  so $[F(X), F(X)] \leq \ker \phi$.
  By factor theorem, 一個ㄛ圖.

  Claim that $\bar{\phi}$ is 1-1.
  \begin{proof}
    Since $\quot{F(X)}{[F(X), F(X)]}$ is abelian, $\forall a \in \quot{F(X)}{[F(X), F(X)]}$, we can write
    $a = \bar{x}_1^{n_1} \bar{x}_2^{n_2} \cdots \bar{x}_m^{n_m}$.
    If $\bar{\phi}(\bar{a}) = (m_1, \cdots, m_n) = 0$ in $\mathbb{Z}^n$, then $m_i = 0,\, \forall i
    \implies a = 1$
  \end{proof}
\end{proof}


%! TEX root=../main.tex
\section{Multilinear algebra}
\subsection{Week 11}
\subsubsection{Bilinear forms \& Groups preserving bilinear forms}

\begin{definition}
  Let $V$ be a vector space over a field $F$.
  \begin{itemize}
    \item A function $f: V \times V \to F$ is called a bilinear form if
      \[
        \begin{cases}
          f(rx_1 + x_2, y) &= rf(x_1, y) + f(x_2, y) \\
          f(x, ry_1 + y_2) &= rf(x, y_1) + f(x, y_2)
        \end{cases}
        \qquad \forall x_1, x_2, x, y_1, y_2, y \in V, r \in F
      \]
    \item $B_F(V, V) = \{\, \text{bilinear forms on $V$} \,\}$ can be regarded
      as a vector space over $F$.
  \end{itemize}
\end{definition}

\begin{theorem}
  Let $\dim V = n$ and $\beta = \{ v_1, \dots, v_n \}$ be a basis for $V$.
  Then $\exists$ an isomorphism $\psi_\beta: B_F(V, V) \to M_{n\times n}(F)$.
  \begin{proof}
    For $v, w \in V$, write $v = \sum_i a_iv_i, w = \sum_j b_j v_j$, i.e.
    $[v]_\beta = \begin{pmatrix}a_1\\ \vdots \\ a_n\end{pmatrix},
    [w]_\beta = \begin{pmatrix}b_1\\ \vdots \\ b_n\end{pmatrix}$.

    For $f \in B_F(V, V)$, $f(v, w) = \sum_i \sum_j a_ib_j f(v_i, v_j)
    = \begin{pmatrix}a_1 & \dots & a_n\end{pmatrix}
    \begin{pmatrix} \\ f(v_i, v_j) \\ \\ \end{pmatrix}
    \begin{pmatrix}b_1 \\ \vdots \\ b_n\end{pmatrix}$.

    Define $\psi_\beta(f) = A$ with $A_{ij} = f(v_i, v_j)$.
    \begin{itemize}
      \item $\psi_\beta$ is a linear transformation.
      \item $\psi_\beta$ is 1-1.
      \item $\psi_\beta$ is onto: $\forall A \in M_{n\times n}(F)$, we define
        $f(v, w) = [v]_\beta^t A [w]_\beta$.
        \qedhere
    \end{itemize}
  \end{proof}
\end{theorem}

\begin{definition}
  Let $f \in B_F(V, V)$
  \begin{itemize}
    \item $f$ is said to be symmetric if
      $f(v, w) = f(w, v) \quad \forall v, w \in V$.
    \item $f$ is said to be skew-symmetric if
      $f(v, w) = -f(w, v) \quad \forall v, w \in V$.
    \item $f$ is said to be alternating if $f(v, v) = 0 \quad \forall v \in V$.
  \end{itemize}
\end{definition}

\begin{remark} \mbox{}
  \begin{itemize}
    \item Alternating $\implies$ skew-symmetric.
    \item If $\Char F \ne 2$, skew-symmetric $\implies$ alternating.
    \item If $\Char F = 2$, symmetric $=$ skew-symmetric.
    \item $\forall f \in B_F(V, V)$ with $\Char F \ne 2$,
      \begin{align*}
        f_s(u, v) = \frac{1}{2}\left(f(u, v) + f(v, u)\right) \\
        f_a(u, v) = \frac{1}{2}\left(f(u, v) - f(v, u)\right)
      \end{align*}
      and $f(u, v) = f_s(u, v) + f_a(u, v)$.
  \end{itemize}
\end{remark}

So we only need to study ``symmetric'' \& ``alternating''.

\newpage

\begin{exercise} \mbox{}
  \begin{enumerate}
    \item If $A$ and $B$ are congruent $(B = Q^tAQ)$ in $M_{n\times n}(F)$,
      then they define the same bilinear form.
    \item $f$ is
      $\begin{cases}
        \text{symmetric} \\
        \text{skew-symmetric}
      \end{cases} \iff \psi_\beta(f) \text{~is~}
      \begin{cases}
        \text{symmetric} (A^t = A)\\
        \text{skew-symmetric} (A^t = -A)
      \end{cases}$
  \end{enumerate}
\end{exercise}

\begin{observation*}
  Let $f \in B_F(V, V)$ and $v_0 \in V$.
  \begin{align*}
    L_f(v_0) &= f(v_0, \cdot) \in V' = \Hom(V, F):
    \text{the dual space of $V$} \\
    R_f(v_0) &= f(\cdot, v_0) \in V'
  \end{align*}
\end{observation*}

The left radical of $f:
\Lrad(f) = N(L_f) = \{\, v\in V \mid f(v, w) = 0 \quad \forall w \in V \,\}$.

The right radical of $f:
\Rrad(f) = N(R_f) = \{\, w\in V \mid f(v, w) = 0 \quad \forall v \in V \,\}$.

\begin{exercise} \mbox{}
  \begin{enumerate}
    \item $\rank(\psi_\beta(f)) = \rank(R_f) = \rank(L_f)$.
    \item If $\dim V = n$, then TFAE ($\implies f:$ non degenerate)
      \begin{enumerate}[(a)]
        \item $\rank(f) = n$.
        \item $\forall v \in V, v \ne 0, \exists w \in V$ s.t. $f(v,w) \ne 0$.
        \item $\Lrad(f) = \{0\}$.
        \item $L_f: V \to V'$ is isom.
      \end{enumerate}
      (also, right)
  \end{enumerate}
\end{exercise}

\begin{theorem}[Principal Axis theorem]
  Let $\dim V = n$ and $\Char F \ne 2$.
  If $f \in B_F(V, V)$ is symmetric, then $\exists \beta$ s.t. $\psi_\beta(f)$
  is diagonal.

  \begin{proof}
    It is sufficient to find $\beta = \{v_1, \dots, v_n\}$ s.t.
    $f(v_i, v_j) = 0 \quad \forall i \ne j$.

    If $f = 0$, then done! Assume $f \ne 0$.
    By induction on $n$: If $n = 1$, done.
    Let $n > 1$.

    \underline{Claim 1}: $\exists v_1 \in V$ s.t. $f(v_1, v_1) \ne 0$.
    Assume that $f(v, v) = 0 \quad \forall v \in V$.
    %\[
      %f(v, w) = \frac{1}{4}f(v+w,v+w) - \frac{1}{4}f(v-w,v-w) = 0
    %\]
    {\color{red}
      \[ f(v, w) = \frac{1}{2} \big( f(v+w,v+w) - f(v, v) - f(w, w) \big) = 0. \quad \footnote{The argument in class requires $\Char F \geq 4$, omimi...} \] }
    So $f = 0$, which is a contradiction.

    Now let $v_1 \in V$ with $f(v_1, v_1) \ne 0$. Let $W = \gen{v_1}_F$ and
    $W^\perp = \{\, w \in V \mid f(v_1, w) = 0 \,\} \subseteq V$.

    \underline{Claim 2}: $V = W \oplus W^\perp$
    \begin{itemize}
      \item $V = W + W^\perp$: For all $v \in V$, let $a = f(v, v_1) / f(v_1, v_1)$, then $v = a v_1 + (v - a v_1)
        \triangleq w + w'$ where $w \in W$ and $f(w', v_1) = f(v - a v_1, v_1) = f(v, v_1) - a f(v_1, v_1)
        = 0$. So $w' \in W^\perp$ and thus $V = W + W^\perp$.
      \item $W \cap W^\perp = \{0\}$: obviously since if $a v_1 \in W$, $f(a v_1, v_1) = 0 \iff a = 0 \iff a v_1 = 0$.
    \end{itemize}

    Since $f \Big|_{W^\perp \times W^\perp}$ is a symmetric bilinear form
    on $W^\perp$ and $\dim W^\perp < \dim V$.
    By induction hypothesis, $\exists \{ v_2, \dots, v_n \}$ a basis for
    $W^\perp$ s.t. $f(v_i, v_j) = 0 \quad \forall i \ne j$. Then
    $\beta = \{v_1, \dots, v_n \}$.
  \end{proof}
\end{theorem}

\begin{theorem}[Sylvester's theorem]
  Let $f \in B_\Rb(V, V)$ be symmetric with $\dim V = n$. Then $\exists \beta$
  s.t. $\psi_\beta(f) = \begin{pmatrix}
    1 \\
    & \ddots \\
    & & 1 \\
    & & & -1 \\
    & & & & \ddots \\
    & & & & & -1 \\
    & & & & & & 0 \\
    & & & & & & & \ddots \\
    & & & & & & & & 0 \\
  \end{pmatrix}$.

  The triple (\# of 1, \# of -1, \# of 0) is well-defined.
  (called the signature of $f$)
  \begin{proof}


    Assume $V^+ = \gen{v_1, \dots, v_p}_F, V^- = \gen{v_{p+1}, \dots, v_{r}}_R,
    V^\perp = \gen{v_{r+1}, \dots, v_n}_F$. ($V = V^+ \oplus V^- \oplus V^\perp$)

    Claim: If $W$ is a subspace of $V$ s.t. $f$ is positive-definite on $W$,
    then $W, V^-, V^\perp$ are independent.

    Let $\gen{w_1, w_2, \cdots, w_s}$ be a basis of $W$. If
    \[ a_1 w_1 + a_2 w_2 + \cdots + a_s w_s = b_{p+1} v_{p+1} + \cdots + b_r v_r + 
      c_{r+1} v_{r+1} + \cdots + c_n v_n. \]
    Let $w \triangleq a_1 w_1 + \cdots + a_s w_s, v \triangleq b_{p+1} v_{p+1} + \cdots + b_r v_r + 
    c_{r+1} v_{r+1} + \cdots + c_n v_n$. Since $w = v$, $f(w, w) = f(v, v)$. 
    but $f(w, w) = \sum a_i^2 \geq 0$ and $f(v, v) = - \sum b_i^2 \leq 0$. Hence $a_i = 0, b_i = 0$.
    Since $v_{r+1}, \cdots, v_n$ is linear independent, $c_i = 0$. Therefor these vectors are linear 
    independent.

  \end{proof}
\end{theorem}

\begin{exercise}
  Let $f \in B_F(V, V)$ with $\Char F \ne 2$.
  If $f$ is skew-symmetric, then $\exists \beta$ s.t.
  \[
    \psi_\beta(f) = \begin{pmatrix}
      0 & 1 \\
      -1 & 0 \\
      & & 0 & 1 \\
      & & -1 & 0 \\
      & & & & \ddots \\
      & & & & & 0 & 1 \\
      & & & & & -1 & 0 \\
      & & & & & & & 0 \\
      & & & & & & & & \ddots \\
      & & & & & & & & & 0
    \end{pmatrix}
  \]
\end{exercise}

\begin{exercise}
  Study Hermitian form
\end{exercise}

${\sf T}: V \isoto V, f\in B_F(V, V)$.
${\sf T}$ preserves $f$ if $f({\sf T}(v), {\sf T}(w)) = f(v, w) \quad
\forall v,w \in V$.

In matrix form, let $\beta$ be a basis for $V$,
$M =[{\sf T}]_\beta, A = \psi_\beta(f)$, then $A = M^tAM$.

\begin{itemize}
  \item $f \in B_\Rb(V,V)$ symmetric, non-degenerate:
    $\exists \beta$ s.t. $\psi_\beta(f) = \begin{pmatrix}I_p \\ & -I_q\end{pmatrix}$.

    Then $\{\, {\sf T}: V \isoto V \text{~preserves~} f \,\} \leftrightarrow
    \left\{\,
      M \in \text{GL}_n(\Rb) \middle|
      M^t\begin{pmatrix}I_p \\ & -I_q\end{pmatrix}M = \begin{pmatrix}I_p \\ & -I_q\end{pmatrix}
      \,\right\} = \text{O}(p, q)$.
  \item $f \in B_\Rb(V,V)$ skew-symmetric, non-degenerate: $n = 2k$,
    $\exists \beta$ s.t. $\psi_\beta(f) = J$.

    Then $\{\, {\sf T}: V \isoto V \text{~preserves~} f \,\} \leftrightarrow
    \left\{\,
      M \in \text{GL}_n(\Rb) \middle|
      M^tJM = J
      \,\right\}$, where 
    \[ J = \begin{pmatrix}
        0 & I_k \\
        -I_k & 0
    \end{pmatrix} \]
\end{itemize}

\subsubsection{Tensor product}
From now on, $R$ is assumed to be commutative with $1$.

\begin{definition}
Let $M_1, \dots, M_n, L$ be $R$-modules.

A function $F: M_1 \times \dots \times M_n \to L$ is said to be $n$-multilinear
if $\forall i$,
\[
  f(x_1, \dots, rx_i + x_i', \dots, x_n) =
  rf(x_1, \dots, x_i, \dots, x_n) + f(x_1, \dots, x_i', \dots, x_n)
  \quad \forall r \in R, x_i, x_i' \in M_i
\]
If $n = 2$, $f$ is called a bilinear map.
\end{definition}

\begin{definition}
  Let $M, N$ be $R$-modules. A tensor product of $M$ and $N$ is an $R$-module
  $M \otimes_R N$ with a bilinear map $\rho: M\times N \to M\otimes_R N$
  satisfying the following universal property:

  for any $R$-momdule $W$ and any bilinear map $f: M\times N \to W$,
  $\exists!$ $R$-module homomorphism $\varphi: M \otimes_R N \to W$,
  \[
    \begin{tikzcd}
    M\times N \arrow{r}{\rho} \arrow[swap]{dr}{f}
      & M \otimes_R N \arrow{d}{\varphi} \\
    & W
    \end{tikzcd}
  \]
\end{definition}

\begin{theorem}[Main theorem]
  $M \otimes_R N$ exists and is unique up to isom.
  \begin{proof}
    Let $X = M\times N$.
    First we construct the free module
    $\displaystyle V_1 = \bigoplus_{(x, y) \in X} R \cdot (x, y)$. \\
    Notice that in $V_1$,
    \begin{itemize}
      \item $(x_1, y_1) + (x_2, y_2) \ne (x_1+x_2, y_1+y_2)$.
      \item $r(x, y) \ne (rx, ry)$.
      \item $r(r_1(x_1, y_1) + \dots + r_n(x_n, y_n)) =
        rr_1(x_1, y_1) + \dots + rr_n(x_n, y_n)$.
    \end{itemize}
    Let $V_0 = \gen*{
      \begin{gathered}
        (x_1+x_2, y) - (x_1, y) - (x_2, y), \\
        (x, y_1+y_2) - (x, y_1) - (x, y_2), \\
        r(x, y) - (rx, y), r(x, y) - (x, ry)
      \end{gathered}
      \middle| x_1, x_2, x \in M, y_1, y_2, y \in N, r \in R
    }_R$.

    Define $M \otimes_R N = \quot{V_1}{V_0}$ which is an $R$-module and
    $\arraycolsep=1pt
    \begin{array}{rcl}
      \rho: M \times N & \to & M \otimes_R N \\
      (x, y) & \mapsto & (x, y) + V_0 = x \otimes y
    \end{array}$
    which is $R$-bilinear. (check yourself)

    Universal property: $\forall (x, y) \in M \times N$,
    $\arraycolsep=1pt
    \begin{array}{rcl}
      R(x, y) & \to & W \\
      r(x, y) & \mapsto & rf(x, y)
    \end{array}$. So, by the universal property of $\oplus$,
    $\exists!$ $R$-module homo. $\varphi_1: V_1 \to W$:
    \[
      \begin{tikzcd}
      M\times N \arrow{r}{i} \arrow[swap]{dr}{f}
        & V_1 \arrow{d}{\varphi_1} \\
      & W
      \end{tikzcd}
    \]
    Claim: $V_0 \subseteq \ker \varphi_1$. (check yourself)
    Then by factor theorem,
    \[
      \begin{tikzcd}[cramped, column sep=tiny]
        \exists! \varphi: \quot{V_1}{V_0} \arrow{rr} & & W \\
        & M \times N \arrow{ul} \arrow{ur} &
      \end{tikzcd}
      \qedhere
    \]
  \end{proof}
\end{theorem}

\begin{example}
  $\Qb \otimes_\Zb \quot{\Zb}{n\Zb} = 0$.
\end{example}

\begin{example}
  $\Rb[x, y] \cong \Rb[x] \otimes_\Rb \Rb[y]$.
  \begin{proof}
    $\arraycolsep=1pt
    \begin{array}{rcl}
      \Rb[x] \times \Rb[y] & \to & \Rb[x, y] \\
      ( f(x), g(y) ) & \mapsto & f(x)g(y)
    \end{array}$
    is bilinear $\leadsto$
    $\arraycolsep=1pt
    \begin{array}{rcl}
      \exists! \varphi: \Rb[x] \otimes_\Rb \Rb[y] & \to & \Rb[x, y] \\
       f(x) \otimes g(y) & \mapsto & f(x)g(y)
    \end{array}$.

    Conversely, 
    $\arraycolsep=1pt
    \begin{array}{rcl}
      \Rb[x, y] & \to & \Rb[x] \otimes_\Rb \Rb[y] \\
      h(x, y) = \sum a_{ij} x^i y^j & \mapsto & \sum a_{ij} x_i \otimes y_j
    \end{array}$.
  \end{proof}
\end{example}

\begin{prop}
  If $M = \gen{x_1, \dots, x_n}_R$ and $N = \gen{y_1, \dots, y_m}_R$. Then
  \[ M \otimes_R N = \gen{x_i \otimes y_j \mid i = 1, \dots, n;
  j = 1, \dots, m}_R. \]
  In particular, if $R$ is a field $F$, then
  $\dim_F M\otimes_F N = (\dim_F M)(\dim_F N)$.
  \begin{proof}
    Note that $M \otimes_R N = \gen{x \otimes y \mid x \in M, y \in N}$.
    Let $x = \sum_i a_ix_i, y = \sum_j b_jy_j$. Then
    $x\otimes y = \sum_i \sum_j a_ib_j x_i \otimes y_j$.
  \end{proof}
\end{prop}

Some canonical isomorphisms:
\begin{itemize}
  \item $(M \otimes_R N) \otimes_R L \cong M \otimes_R (N \otimes_R L)$.
    \begin{proof}
      $\forall z \in L$, 
      $\arraycolsep=1pt
      \begin{array}{rcl}
        M \times N & \to & M \otimes_R (N \otimes_R L) \\
        (x, y) & \mapsto & x \otimes (y \otimes z)
      \end{array}$ is bilinear.
      $\exists!$ $R$-mod homo.
      $\varphi_z: M \otimes_R N \to M\otimes_R(N\otimes_R L)$.
      Similarly,
      $\arraycolsep=1pt
      \begin{array}{rcl}
        (M \otimes_R N) \times L & \to & M \otimes_R (N \otimes_R L) \\
        \left(\sum x_i\otimes y_i, z\right) & \mapsto &
        \sum x_i \otimes (y_i \otimes z)
      \end{array}$ is bilinear. (The right is due to $\varphi_z$ linear, and 
      the left is because $x \otimes (y \otimes (rz_1 + z_2)) = rx \otimes (y \otimes z_1) + 
      x \otimes (y \otimes z_2)$.)
      Hence exists unique $R$-mod homo.
      $\varphi: (M \otimes_R N) \otimes_R L \to M \otimes_R (N \otimes_R L)$.
      By the symmetric construction, we have $\varphi^{-1}$ and $\varphi^{-1} \circ 
      \varphi = \varphi \circ \varphi^{-1} = 1$, so the two are isomorphic.
    \end{proof}
  \item $(M \oplus M') \otimes_R N \cong (M \otimes_R N)\oplus(M'\otimes_R N)$.

    The mapping $\psi :: (M \oplus M') \times N \to (M \otimes_R N)\oplus(M'\otimes_R N)$
    by $\psi = ((x, x'), y) \mapsto (x\otimes y, x'\otimes y)$ is biliear, hence 
    exists R-mod homomorphism $\varphi :: (M \oplus M') \otimes_R N \to (M \otimes_R N)\oplus(M'\otimes_R N)$.

    On the other hand, The mapping $(x, y) :: M \times N \mapsto (x, 0) \otimes y :: (M \oplus M') \otimes_R N$ 
    is bilinear. So exists $\phi_1 :: M \otimes N \to (M \oplus M') \otimes_R N$, similarly there exists
    $\phi_2 :: M' \otimes N \to (M \oplus M') \otimes_R N$. Now by the universal property of direct sum, 
    there exists $\phi :: (M \otimes_R N)\oplus(M'\otimes_R N) \to (M \oplus M') \otimes_R N$.
    After a careful examine, we have
    \[ \varphi = (x, x') \otimes y \mapsto (x \otimes y, x' \otimes y),
      \phi = (x \otimes y, x' \otimes y) \mapsto (x, x') \otimes y \]
    Thus $\phi = \varphi^{-1}$ and hence the two are isomorphic.
\end{itemize}

\begin{exercise}\mbox{}
  \begin{enumerate}
    \item $R \otimes_R M \cong M$.
    \item $M \otimes_R N \cong N \otimes_R M$.
  \end{enumerate}
\end{exercise}

\begin{exercise}
  $\quot{R}{I} \otimes_R N \cong \quot{N}{IN}$ where
  $IN \defeq \left\{ \sum a_ix_i \mid a_i \in I, x_i \in N \right\}$.
\end{exercise}

\begin{exercise}
  Compute
  $
  \dim_\Qb (\Qb \otimes_\Zb \Qb), 
  \dim_\Rb (\Rb \otimes_\Rb \Cb), 
  \dim_\Rb (\Cb \otimes_\Rb \Cb), 
  \dim_\Cb (\Cb \otimes_\Rb \Cb)
  $
\end{exercise}

%! TEX root=../main.tex
\subsection{Week 12}
\subsubsection{Tensor product \RNum{2}}

By universal property, we get
$\{ R\text{-bilinear maps~} M \times N \to L \} \leftrightarrow
\Hom_R(M \otimes_R N, L)$.

Similarly,
\begin{gather*}
  \Hom\left(\bigoplus_{s\in \Lambda} M_s, L\right) \cong
  \prod_{s\in \Lambda} \Hom\left(M_s, L\right) \\
  \Hom\left(N, \prod_{s\in \Lambda} M_s\right) \cong
  \prod_{s\in \Lambda} \Hom\left(N, M_s\right) \\
\end{gather*}

\begin{fact}
  $f\in \Hom_R(M, M'), g \in \Hom_R(N, N') \leadsto
  f\otimes g \in \Hom_R(M\otimes N, M'\otimes N')$ by
  $(f \otimes g)(x \otimes y) = f(x) \otimes g(y)$.
  \begin{proof}
    Define
    $\arraycolsep=1pt
    \begin{array}{rcl}
      h: M \times N & \to & M' \otimes_R N' \\
      (x, y) & \mapsto & f(x) \otimes g(y)
    \end{array}$
  \end{proof}
\end{fact}

Restrition and extension of scalars.

Let $f: R \to S$ be a ring homomorphism and $R, S$ be commutative with $1$.
Then $S$ can be regarded as an $R$-module.
    $\arraycolsep=1pt
    \left(\begin{array}{rcl}
      R \times S & \to & S \\
      (r, x) & \mapsto & f(r)x
    \end{array}\right)$.

If $M$ is a $S$-module, then $M$ is also an $R$-module.
    $\arraycolsep=1pt
    \left(\begin{array}{rcl}
      R \times M & \to & M \\
      (r, a) & \mapsto & f(r)a
    \end{array}\right)$.

If $N$ is an $R$-module, then $S \otimes_R N$ an $S$-module.
    $\arraycolsep=1pt
    \left(\begin{array}{rrcl}
      S & \times (S \otimes_R N) & \to & S \otimes_R N \\
      (r, & x\otimes a) & \mapsto & rx \otimes a
    \end{array}\right)$.

\begin{example}[Important example]
  Let $V$ be a real vector space. The complexification of $V$ is
  $V^\Cb \defeq \Cb \otimes_\Rb V$ which is a $\Cb$-vector space.
\end{example}

\begin{exercise}
  Let $K \subseteq L$ be an inclusion of fields and let $E$ be a vector space
  over $K$. Show that $E^L \defeq L \otimes_K E$ satisfies the following
  universal property: For any vector space $U$ over $L$ and any
  $K$-linear map $f: E \to U$, $\exists!$ $L$-linear map $\varphi$:
  \[
    \begin{tikzcd}[cramped, column sep=tiny]
      \varphi: 1\otimes x :: E^L \arrow{rr} & & f(x) :: U \\
      & x :: E \arrow{ul} \arrow{ur}{f} &
    \end{tikzcd}
    \qedhere
  \]
\end{exercise}

\begin{exercise}
  $E \to E^L$ is a covariant functor from the category of vector spaces over
  $K$ to the category of vector spaces over $L$.
\end{exercise}

\begin{example}
  $\Zb^n \cong \Zb^m \leadsto
  \Qb \otimes_\Zb \Zb^n \cong \Qb \otimes_\Zb \Zb^m \leadsto n = m$.
\end{example}

\begin{example}
  $G \cong \quot{\Zb}{d_1\Zb} \oplus \dots \oplus \quot{\Zb}{d_l\Zb}
  \oplus \Zb^s, \Qb \otimes_\Zb G = \Qb^s$.
\end{example}

Let $M, N$ and $U$ be $R$-module. Then
\[
  \Hom_R\left(M \otimes_R N, U\right) \cong
  \Hom_R\left(N, \Hom_R(M, U)\right)
\]

\begin{proof} \mbox{}
  \begin{itemize}
    \item For $f \in \Hom_R\left(M \otimes_R N, U\right)$ and $a \in N$,
      define $f_a = x :: M \mapsto f(x \otimes a) :: U$.
      \begin{itemize}
        \item linear: easy.
        \item $\ob{f}: a \mapsto f_a$ is an $R$-mod homo.: easy.
        \item $\tau: f \mapsto \ob{f}$ is an $R$-mod homo.:
          $\tau(rf+g)(a)(x) = (rf+g)_a(x) = (rf+g)(x\otimes a)
          = rf(x\otimes a) + g(x\otimes a) = \dots
          = r\tau(f)(a)(x) + \tau(g)(a)(x)$
      \end{itemize}
    \item For $g \in \Hom_R\left(N, \Hom_R(M, U)\right)$,
      define $g' = (x, a) :: M \times N \mapsto g(a)(x) :: U$.
      \begin{itemize}
        \item $g'$ is $R$-bilinear: easy.
        \item $\exists! \tilde{g}: x\otimes a \mapsto g(a)(x)$.
        \item $\sigma: g\mapsto \tilde{g}$ is an $R$-mod homo.: easy.
      \end{itemize}
    \item $\sigma \tau = \text{id}, \tau \sigma = \text{id}$: easy...
      \qedhere
  \end{itemize}
\end{proof}

\begin{exercise}
  $\Hom_R(M, \cdot), M \otimes_R \cdot$ are covariant functors from the
  category of $R$-modules to itself.
  (is an adjoint pair)
\end{exercise}

\begin{fact}
  $\Hom_R(R, M) \cong M$. By $f \mapsto f(1)$.
\end{fact}

\begin{definition}
  An exact sequence $A \xrightarrow{f_1} B \xrightarrow{f_2} \cdots$ is
  a sequence satisfied $\text{im}\; f_k = \ker f_{k+1}$.
\end{definition}

\begin{itemize}
  \item $0 \to \Zb \xrightarrow{2} \Zb$.
  \item $\Qb \to \quot{\Qb}{\Zb} \to 0$.
\end{itemize}

Let $V, W$ be vector spaces over $F$. Then
$V^* \otimes_F W \cong \Hom_F(V, W)$.
\begin{proof}
  Let $\alpha = \{e_1, \dots, e_n\}$ and $\beta = \{f_1,\dots, f_m\}$ be
  bases for $V$ and $W$ respectively.
  Via $\alpha, \beta$, $\Hom_F(V, W) \cong
  \gen*{E_{ij} \middle|
  \begin{aligned}i &= 1,\dots,m\\j &= 1,\dots,n\end{aligned}}_F$.
  $V^* \otimes W \cong
  \gen*{e_j^* \otimes f_i \middle|
  \begin{aligned}i &= 1,\dots,m\\j &= 1,\dots,n\end{aligned}}_F$.
\end{proof}

\subsubsection{Tensor algebra}
\begin{definition} \mbox{}
  \begin{itemize}
    \item Let $R$ be a commutative ring with $1$.
      An $R$-algebra is a ring $A$ which is also an $R$-module s.t. the
      multiplication map $A \times A \to A$ is $R$-bilinear.
      ( $r(ab) = (ra)b = a(rb)$ )
    \item Let $A$ be an $R$-algebra. A grading of $A$ is a collection of
      $R$-submodules $\{ A_n \}_{n=0}^\infty$ ($n$-th homogeneous part) s.t.
      \[
        A = \bigoplus_{n=0}^\infty A_n \quad \text{and} \quad
        A_nA_m \subseteq A_{n+m} \quad \forall n,m
      \]
    \item A graded $R$-algebra is an $R$-algebra with a chosen grading.
    \item $\mathfrak{M}_R$ is the category of $R$-modules.
    \item $\mathfrak{Gr}_R$ is the category of graded $R$-algebras.
      ($f: A \to A'$ with $f(A_n) \subseteq A_n'$)
  \end{itemize}
\end{definition}

\begin{example}
  $A = R[x], A_n = \gen{x^n}_R$. If $I = \gen{x+1}_A$, $I$ is not graded.
  $I = \gen{x^2}_A$ is graded.
\end{example}

\begin{definition}
  \color{red}
  An ideal $I$ is graded in a graded ring $A$ if and only if
  $I = \raisebox{0.3ex}{$\bigoplus$} I \cap A_n$.
  \footnote{This is not mentioned in class}
\end{definition}

\begin{exercise}
  TFAE
  \begin{enumerate}[(1)]
    \item $I$ is graded.
    \item $\forall a \in I$ write $a = a_{k_1} + a_{k_2} + \dots + a_{k_m},
      a_{k_i} \in A_{k_i} \implies a_{k_i} \in I$.
      ($a_{k_i}$ is the homogenuous component of $a$)
    \item $\quot{A}{I}$ is a graded ring with
      $\left(\quot{A}{I}\right)_n = \quot{(A_n + I)}{I}
      \cong \quot{A_n}{I \cap A_n}$.
  \end{enumerate}
\end{exercise}

\begin{exercise} \mbox{}
  \begin{enumerate}[(1)]
    \item If $I$ is a f.g. graded ideal, then $I$ has a finite system of
      generators consisting of homogeneous elements alone.
    \item $I, J$ are graded $\implies I+J, IJ, I \cap J$ are graded.
  \end{enumerate}
\end{exercise}

\underline{Observation}:
Let $\{ M_i \}_{i=1}^\infty$ be a collection of $R$-modules.
\begin{itemize}
  \item $M_1 \otimes_R M_2$ exists.
  \item $(M_1 \otimes_R M_2) \otimes_R M_3 \cong
    M_1 \otimes_R (M_2 \otimes_R M_3) \implies
    M_1 \otimes_R M_2 \otimes_R M_3$ is well-defined.
    Universal property: for any $R$-module $L$ and a $3$-multilinear map
    $f: M_1 \times M_2 \times M_3 \to L$. (拆括號囉)
  \item By induction, $M_1 \otimes \dots \otimes M_n$ is well-defined and
    satisfies the universal property. ($n$-multilinear map)
\end{itemize}

Goal: For a given $R$-module $M$, we intend to construct an graded $R$-algebra
$T(M)$ containing $M$ that is ``universal'' w.r.t. $R$-algebras containing $M$.

That is, a tensor algebra is a pair $(T(M), i)$ where $T(M)$ is an $R$-algebra
and $i :: M \to T(M)$, such that for any $R$-algebra $A$ containing $M$,
which is to say that exist a $R$-module homomorphism $\varphi: M \to A$,
then exists an $R$-algebra homomorphism $\psi :: T(M) \to A$ such
that $\varphi = \psi \circ i$. \\[.5em]

\underline{Construction}:
\begin{itemize}
  \item $\forall k \in \Nb$, $T^k(M) \defeq
    \underbrace{M\otimes \dots \otimes M}_{k \text{~times}}$, each
    $x_1\otimes x_2\otimes \dots \otimes x_k \in T^k(M)$ is called a $k$-tensor.

    $T^0(M) \defeq R$ and
    \[
      T(M) \defeq \bigoplus_{k=0}^\infty T^k(M) = R \oplus T^1(M) \oplus \dots
    \]
  \item define multiplication on $T(M)$ by:
    \[
      \begin{tikzcd}[cramped, row sep=tiny]
        T^i(M) \times T^j(M) \arrow[r] & T^{i+j}(M) \\
        (x_1\otimes \dots \otimes x_i, y_1\otimes \dots \otimes y_j)
        \arrow[r, mapsto]
        & x_1\otimes \dots \otimes x_i\otimes y_1\otimes \dots\otimes y_j \\
      \end{tikzcd}
    \]
  \item Distribution law: easy.
\end{itemize}

Proving the universal property:
For any $R$-algebra $A$ containing $M$ and an $R$-module homo.
$\varphi: M \to A$.
$\forall k \ge 2$, we define
$f_k: M \times \dots \times M \to A$
\[
  \arraycolsep=1pt
  \begin{array}{rcl}
    f_k: & M \times \dots \times M & \to A \\
         & (x_1, \dots, x_k) & \mapsto
    \varphi(x_1) \dots \varphi(x_k)
  \end{array}
\]
$f_k$ is $k$-multilinear $\leadsto$
\[
  \arraycolsep=1pt
  \begin{array}{rcl}
    \exists! \tilde{f_k}: & M \otimes \dots \otimes M & \to A \\
         & x_1 \otimes \dots\otimes x_k & \mapsto
    \varphi(x_1) \dots \varphi(x_k)
  \end{array}
\]
By the universal property of $\bigOp$, exists a unique $R$-module homo.
$\tilde\varphi :: T(M) \to A$ which make the following diagram commutes.
\[
  \begin{tikzcd}[cramped, column sep=tiny]
    \tilde\varphi: T(M) \arrow{rr} & & A \\
      & T^k(M) \arrow{ul}{i} \arrow{ur}{f_k} &
  \end{tikzcd}
\]

$\tilde\varphi$ is an $R$-algebra homomorphism.

\begin{definition}
  $T(M)$ is called the tensor algebra of $M$.
\end{definition}

\begin{exercise}
  $T$ is a covariant functor from $\Mf_R$ to $\Grf_R$.
\end{exercise}

\begin{prop}
  Let $V$ be a vector space over $F$ with a basis $\beta = \{
  v_1, \dots, v_n \}$. Then
  \[
    \left\{
      v_{i_1} \otimes \dots \otimes v_{i_k} \,\middle|\,
      \forall j = 1, \dots, k,\; i_j = 1, \dots, n
    \right\}
  \]
  forms a basis for $T^k(V)$. $\dim_F T^k(V) = n^k$.
\end{prop}

$T(V)$ can be regarded as a non-commutative polynomial algebra over $F$.
\\[.5em]
$\odot$ Symmetrization ($\Char F = 0$)
\[
  \begin{tikzcd}[cramped, row sep=tiny]
    V \times \dots \times V \arrow[r] & T^n(V) \\
    (x_1, \dots, x_n) \arrow[r, mapsto]
    & \displaystyle \frac{1}{n!}\sum_{\tau \in S_n}
    x_{\tau(1)}\otimes \dots \otimes x_{\tau(n)}
  \end{tikzcd}
\]
is $n$-multilinear.

The symmetrizer operator $\sigma: T^n(V) \to T^n(V)$,
$\tilde{S}^n(V) \defeq \sigma(T^n(V)) \subseteq T^n(V)$.

\underline{Claim}:
$T^n(V) = \tilde{S}^n(V) \oplus C^n(V)$ where
\[ C^n(V) = C(V) \cap T^n(V) \quad
C(V) = \gen{v\otimes w - w\otimes v \mid v, w\in V} \]

%1 TEX root=../main.tex
\subsection{Week 13}
\subsubsection{Symmetric and Exterior algebra}
\paragraph{Symmetric algebra}
Define
\[
  \arraycolsep=1pt
  \begin{array}{rcl}
    S: & \Mf_R & \to \Grf_R \\
       & M & \mapsto \quot{T(M)}{C(M)}
  \end{array}
  \qquad S(M) \defeq \quot{T(M)}{C(M)}
\]
where $C(M)$ is the gradded two-sided ideal generated by
$u \Ot v - v\Ot u$ with $u, v \in M$.

\begin{itemize}
  \item $C^k(M) \defeq C(M) \cap T^k(M)$ is the submodule of $T^k(M)$ generated
    by all \[ x_1 \Ot \dots \Ot x_k - x_{\sigma(1)} \Ot \dots
    \Ot x_{\sigma(k)} \quad \forall x_i \in M, \sigma \in S_k. \]

    ``$\subseteq$'':
      $x_1 \Ot \dots \Ot x_s \Ot (u\Ot v - v\Ot u) \Ot y_1 \Ot \dots \Ot y_t
      \in C(M) \cap T^k(M)$ with $s + 2 + t = k$.

    ``$\supseteq$'': bubble sort
  \item $k \ge 2, S^k(M) = \quot{T^k(M)}{C^k(M)} =
    \gen{\ob{x}_1 \Ot \dots \Ot \ob{x}_k \mid x_i \in M}_R$ with
    $\ob{x}_1 \Ot \dots \Ot \ob{x}_k =
    \ob{x}_{\sigma(1)} \Ot \dots \Ot \ob{x}_{\sigma(k)} \quad
    \forall \sigma \in S_k$
\end{itemize}

Hence, $S(M) = \bigoplus_{k=0}^\infty S^k(M)$ is a graded commutative
$R$-algebra.

\begin{definition}
  $f: M \times \dots \times M \to L$ is a symmetric $k$-multilinear map if
  $f$ is $k$-multilinear and
  \[ f(x_1, \dots, x_k) = f(x_{\sigma(1)}, \dots, x_{\sigma(k)}) \quad
  \forall \sigma \in S_k \]
  \begin{itemize}
    \item $k \ge 2$, $S^k(M)$ is universal w.r.t. symmetric $k$-multilinear
      maps on $M$:
      By the universal property of $T^k(M)$, $\exists!$ $R$-module homo.
      $\tilde{f}: T^k(M) \to L$. Now $C^k(M) \subseteq \ker \tilde{f}
      \implies \exists!$ $R$-module homo. $\ob{f}: S^k(M) \to L$ by
      factor thm.
    \item $S(M)$ satisfies the universal property for maps to a commutative
      $R$-algebra:
      given a commmutative $R$-algebra $A$ and $f:M \to A$ $R$-module homo.,
      \[
        \begin{tikzcd}[cramped]
          M \arrow{r}{f} \arrow[hook]{d} & A \\
          T(M) \arrow{ur}{\exists! f'}
          \arrow{r} & \quot{T(M)}{C(M)} \arrow{u}
        \end{tikzcd}
      \]
    \item $S: \Mf_R \to \Grf_R$ is a covariant functor.
      \begin{itemize}
        \item $\varphi: M \to N$: $R$-module homo. $\leadsto
          T(\varphi): T(M) \to T(N) \to \quot{T(N)}{C(N)} = S(N)$
      \end{itemize}
  \end{itemize}
\end{definition}

\begin{exercise}
  Let $E$ be a vector space over $F$ with $\dim E = n$.
  \begin{enumerate}
    \item Show that $S(E) \cong F[x_1, \dots, x_n]$.
    \item Compute $\dim_F S^k(E)$.
  \end{enumerate}
\end{exercise}

\paragraph{Exterior algebra} ($\Char R \ne 2$)
\[
  \arraycolsep=1pt
  \begin{array}{rcl}
  \Lambda: & \Mf_R & \to \Grf_R \\
           & M & \mapsto \Lambda(M) = \quot{T(M)}{A(M)}
  \end{array}
\]
where $A(M)$ is the two sided graded generated by $v\Ot v \quad
\forall v \in M$.

\begin{itemize}
  \item $A^k(M) \defeq A(M) \cap T^k(M)$ is the submodule of $T^k(M)$
    generated by all $x_1 \Ot \dots \Ot x_k$ with $x_i = x_j$ for some $i\ne j$.

    (Note: $(x_1+x_2)\Ot(x_1+x_2) = x_1\Ot x_1 + x_1\Ot x_2 + x_2\Ot x_1 +
    x_2\Ot x_2 \leadsto x_1\Ot x_2 + x_2\Ot x_1 \in A(M)$)
  \item $\Lambda^k(M) \cong \quot{T^k(M)}{A^k(M)} = \gen{
    \ob{x_1\Ot \dots \Ot x_k} \mid x_i \in M
    }$ with $\ob{x_1\Ot \dots \Ot x_k} = \ob{0}$ if $x_i = x_j$ for some
    $i\ne j$. We use $x_1 \wedge \dots \wedge x_k \defeq
    \ob{x_1\Ot \dots \Ot x_k}$.

    Note: $x_1 \wedge x_2 = -x_2 \wedge x_1$.
\end{itemize}

\begin{definition}
  $f: M\times \dots \times M \to L$ is an alternating $k$-multilinear map if
  $f$ is $k$-multilinear and $f(x_1, \dots, x_k) = 0$ when $x_i = x_j$ for some
  $i \ne j$.
  \begin{itemize}
    \item $k \ge 2$, $\Lambda^k(M)$ is universal w.r.t. alternating
      $k$-multilinear maps on $M$:
      \[
        \begin{tikzcd}[cramped]
          M\times \dots \times M \arrow{r} \arrow{d} & L \\
          T^k(M) \arrow[swap]{ur}{\exists! f'}
          \arrow{r} & \Lambda^k(M) \arrow{u}
        \end{tikzcd}
      \]
    \item $\Lambda(M)$ satisfies the universal property for maps to an
      $R$-algebra $A$ with $a^2 0 \quad \forall a \in A$:
      given an $R$-algebra $A$ and $f:M \to A$ $R$-module homo.,
      \[
        \begin{tikzcd}[cramped]
          M \arrow{r}{f} \arrow[hook]{d} & A \\
          T(M) \arrow{ur}{\exists! f'}
          \arrow{r} & \Lambda(M) \arrow{u}
        \end{tikzcd}
      \]
    \item $\Lambda: \Mf_R \to \Grf_R$ is a covariant functor.
      \begin{itemize}
        \item $\varphi: M \to N$: $R$-module homo. $\leadsto
          T(\varphi): T(M) \to T(N) \to \quot{T(N)}{A(N)} = \Lambda(N)$
      \end{itemize}
  \end{itemize}
\end{definition}

\begin{exercise}
  Let $V$ be a vector space over $F$ with $\dim V = n$ and $\varphi: V\to V$
  be a linear transformation.
  \begin{enumerate}[(1)]
    \item Compute $\Lambda^k(V)$.
    \item Determine the map $\Lambda^n(\varphi): \Lambda^n(V)\to \Lambda^n(V)$.
  \end{enumerate}
\end{exercise}

\paragraph{Symmetrization and Skew-symmetrization}
\[
  \begin{tikzcd}[cramped, row sep=tiny]
    T^k(V) \arrow[r] & T^k(V) \\
    \Sym = \sigma: x_1\Ot \dots \Ot x_k \arrow[r, mapsto]
    & \displaystyle \frac{1}{k!}\sum_{\tau \in S_k}
    x_{\tau(1)}\Ot \dots \Ot x_{\tau(k)} \\
    \Alt = \sigma': x_1\Ot \dots \Ot x_k \arrow[r, mapsto]
    & \displaystyle \frac{1}{k!}\sum_{\tau \in S_k} \sgn(\tau)
    x_{\tau(1)}\Ot \dots \Ot x_{\tau(k)}
  \end{tikzcd}
\]
$\tilde{S}^k(V) = \sigma(T^k(V)) \quad \tilde{\Lambda}^k(V) = \sigma'(T^k(V))$

\begin{itemize}
  \item $\sigma^2 = \sigma$ easy $\leadsto T^k(V) = \Image \sigma \oplus
    \ker \sigma = \tilde{S}^k(V) \oplus \ker \sigma$.
  \item $\ker \sigma = C^k(V)$.
    $C^k(V) \subseteq \ker \sigma$ is obvious.
    Assume $\supsetneq$, i.e., $\exists t \in \ker \sigma$ s.t.
    $t \not\in C^k(V)$.
    Recall $q: T^k(V) \onto S^k(V)$, OMIMI.
\end{itemize}

\begin{exercise}
  $T^k(V) = \tilde{\Lambda}^k(V) \oplus A^k(V)$.
\end{exercise}

%1 TEX root=../main.tex
\section{Introduction to the linear representation theory of finite groups}
\subsection{Week 14}
\subsubsection{Generatlities on linear representations}
\paragraph{Notation}

\begin{itemize}
  \item $G$: finite group
  \item $V$: vector space of finite dim over $\Cb$
  \item $\text{GL}(V)$: the group of all linear isom. $V \to V$
\end{itemize}

\begin{definition}
  A group homo. $\rho: G \to \text{GL}(V)$ is called a linear representation
  of $G$.
  $\dim V$ is called the degree of $\rho$.
  ($V$ is a representation space)

  For a fixed basis $\beta = \{\, e_i \,\}$,
  \[
    \begin{tikzcd}
      G \arrow{r}{\rho} \arrow[swap]{dr}{R} & \text{GL}(V) \arrow[d, "\beta"', "\rotatebox{90}{\(\sim\)}"] \\
                                            & \text{GL}_n(\Cb)
    \end{tikzcd}
  \]
  ($R$ is a matrix representation)
\end{definition}

\begin{example}
  A representation of degree $1$ of $G$ is $\rho: G \to \text{GL}(\Cb)
  \cong \Cb^*$.

  $\ord(g)$ is finite $\leadsto$ $\rho(g)^m = 1$ for some $m \in \Nb$
  $\leadsto$ $\rho(g)$ is a root of unity, i.e. $\abs{\rho(g)} = 1$.

  Note: So, $\rho: G \to S^1$, $S^1$ is the unit circle.

  \begin{enumerate}
    \item $G = \quot{\Zb}{p\Zb}$,
      $\rho: \ob{1} :: G \mapsto s_p :: S^1$ with $s_p^p = 1$.
    \item $G = S_3, V = \Cb e_1 \oplus \Cb e_2 \oplus \Cb e_3$.

      A permutation representation is
      $\rho: \tau :: S_3 \mapsto (\rho(\tau): e_i \mapsto e_{\tau(i)})
      :: \text{GL}(V)$.
      
    \item $G = S_3, V = \bigoplus_{\sigma \in S_3} \Cb e_{\sigma}$.
      The regular representation is
      \[ \rho^{\text{reg}}: \tau :: G \mapsto
      (\rho^{\text{reg}}(\tau): e_{\sigma} \mapsto e_{\tau \sigma})
      :: \text{GL}(V). \]
  \end{enumerate}
\end{example}

For general $G$, with $V = \bigoplus_{g\in G} \Cb e_g$,
\[ \rho^{\text{reg}}: h :: G \mapsto
  (\rho^{\text{reg}}(h): e_{g} \mapsto e_{hg})
:: \text{GL}(V). \]

\begin{definition} \mbox{}
  \begin{itemize}
    \item $\rho: g :: G \mapsto \text{id} :: \text{GL}(V)$:
      trivial representation.
    \item $\rho: G \toone \text{GL}(V)$: faithful representation.
    \item $\rho, \rho'$ are said to be equivalent if $\exists$ a linear isom.
      ${\sf T}: V \isoto V'$ s.t.
  \end{itemize}
\end{definition}

\begin{remark}
  When we choose two bases $\beta, \beta'$ for $V$,
  \[
    \begin{tikzcd}
      G \arrow{r}{\rho} \arrow[swap]{dr}{R} & \text{GL}(V) \arrow[d, "\beta"', "\rotatebox{90}{\(\sim\)}"] \\
                                            & \text{GL}_n(\Cb)
    \end{tikzcd} \quad
    \begin{tikzcd}
      G \arrow{r}{\rho'} \arrow[swap]{dr}{R} & \text{GL}(V) \arrow[d, "\beta'"', "\rotatebox{90}{\(\sim\)}"] \\
                                            & \text{GL}_n(\Cb)
    \end{tikzcd}
  \]
  then $\rho, \rho'$ are equivalent.
\end{remark}

Let $T: e_i :: V \mapsto e_i' :: V$. For $g \in G, R(g) = \big(a_{ij}\big)$.

$T \circ \rho(g) = \rho'(g) \circ T$

\begin{definition}
  Let $\inpd{\cdot, \cdot}$ be a positive definite Hermitian form on $V$.

  Then $T: V \to V$ is called a unitary operator if
  $\inpd{T(x), T(y)} = \inpd{x, y} \quad \forall x, y \in V$.

  or $\forall \beta:$ orthonormal basis,
  $[T]_\beta^*[T]_\beta = [T]_\beta[T]_\beta^* = I_n$.
\end{definition}

\begin{theorem}
  $\forall \rho: G \to \text{GL}(V)$, $\exists$ a matrix representation
  $R: G \to U_n$.
  \begin{proof}
    We only need to $G$-invariant positive definite Hermitian form on $V$.
    ($\forall g \in G, \inpd{\rho(g)x, \rho(g)y} = \inpd{x, y} \quad
  \forall x, y \in V$)

  We start with an arbitrary positive definite Hermitian form
  $\inpd{\cdot, \cdot}'$ on $V$.

  Define a new form $\inpd{\cdot, \cdot}$ by
  \[
    \inpd{x, y} \defeq \frac{1}{\abs{G}} \sum_{g\in G}
    \inpd{\rho(g)(x), \rho(g)(y)}'
  \]
  which is a positive definite Hermitian form. (easy to check)
  \end{proof}
\end{theorem}

\begin{definition}
  Let $\rho: G \to \text{GL}(V)$, For $W \subset V$ (we use $\subset$ to denote
  subspace), if $\forall x \in W$, $\rho(g)(x) \in W, \forall g \in G$, then
  $W$ is said to be $G$-invariant and
  \[
    \arraycolsep=1pt
    \begin{array}{rcl}
      \rho^W: & G & \to \text{GL}(W) \\
              & g & \mapsto \rho(g) \big|_W
    \end{array}
  \]
  is called a subrepresentation of $\rho$.
\end{definition}

$W$ is $G$-invariant $\leadsto$ $\rho(g) \big|_W: W \isoto W$.

\begin{example}
  Let $\rho$ be the regular rep. of $S_3$.

  $W^\circ = \{\, \alpha_1e_1 + \dots + \alpha_6e_6 \mid \alpha_1 +\dots + \alpha_6 = 0 \,\}$
  is $G$-invariant.

  $W^1 = \gen{ e_1 + \dots + e_6}_\Cb$ is $G$-invariant.
\end{example}

\begin{theorem}
  Let $\rho: G \to \text{GL}(V)$ and $W \subset V$ be $G$-invariant.
  Then $\exists W^\circ \subset V$ is still $G$-invariant and
  $V = W \oplus W^\circ$.
  \begin{proof}
    We can pick an arbitrary $W'$ with $V = W \oplus W'$ and
    $\pi_1: V \to W$ is the projection to $W$. Then $W' = \ker \pi_1$.
    
    Now we need $\pi_1$ preserves the $G$ action ($G$-equivariant).
    Define
    \[
      \pi^\circ = \frac{1}{\abs{G}} \sum_{g\in G}
      \rho(g)^{-1} \circ \pi_1 \circ \rho(g) : V \to W
    \]
    \begin{itemize}
      \item well-defined: $\rho(g)(V) \subset V \leadsto
        \pi_1 \circ \rho(g)(V) \subset W \leadsto
        \rho(g)^{-1} \circ \pi_1 \circ \rho(g)(V) \subseteq W$.
      \item surjective: $\forall y \in W,
        \rho(g)^{-1}\circ \pi_1 \circ\rho(g)(y) = y$ since
        $\rho(g)(y) \in W$. Also, $(\pi^\circ)^2 = \pi^\circ$.
        So $V = \Image \pi^\circ \oplus \ker \pi^\circ$.
      \item $G$-equivariant: $\forall g' \in G$,
        \begin{align*}
          \pi^\circ \circ \rho(g')(x)
          &= \frac{1}{\abs{G}} \sum_{g\in G}
            \rho(g)^{-1}\circ\pi_1\circ\rho(g) (\rho(g')(x)) \\
          &= \rho(g') \frac{1}{\abs{G}} \sum_{gg'\in G}
            \rho(gg')^{-1}\circ\pi_1\circ\rho(gg')(x) \\
          &= \rho(g') \circ \pi^\circ(x)
        \end{align*}
      \item $W^\circ \defeq \ker \pi^\circ$ is $G$-invariant:
        $\forall x \in W^\circ$, $\pi^\circ(\rho(g)(x))
        = \rho(g)(\pi^\circ(x)) = \rho(g)(0) = 0$. So
        $\rho(g)(x) \in W^\circ$.
    \end{itemize}
  \end{proof}
\end{theorem}

\begin{remark}
  If $W \subset V$ is $G$-invariant, then $W^\perp$ is also $G$-invariant.
  (w.r.t. a $G$-invariant positive definite Hermitian form)
\end{remark}

\begin{definition}
  $\rho: G \to \text{GL}(V)$ is irreducible if $\rho$ has no proper notrivial
  subrepresentations.
\end{definition}

\begin{theorem}
  Each $\rho: G\to \text{GL}(V)$ is a direct sum of irreducible
  subrepresentations.
  \begin{proof}
    By induction on $\dim V$. For $\dim V = 1$, then $\rho$ is irr.

    For $\dim V > 1$, if $\rho$ is irr., then done.
    Otherwise, $\exists W, W^\circ$ are $G$-invariant s.t.
    $V = W \oplus W^\circ$ with $\dim W \ge 1, \dim W^\circ \ge 1$.
    By induction hypothesis, $\rho^W, \rho^{W^\circ}$ are direct sum
    of irr. subrep., and $\rho = \rho^W \oplus \rho^{W^\circ}$, done.
  \end{proof}
\end{theorem}

\begin{remark}
  Let $\rho: G \to \text{GL}(V)$ and $\rho': G \to \text{GL}(V')$.
  \begin{itemize}
    \item $\rho \oplus \rho': G \to \text{GL}(V\oplus V')$.
      矩陣是左上右下
    \item $\rho \otimes \rho': G \to \text{GL}(V\otimes V')$.
      矩陣是密密麻麻 ($\sum_{i,j} r_ip, r_{jq}' (e_i \otimes e_j')$)
  \end{itemize}
\end{remark}

\subsubsection{Character Theory \RNum{1}}

Main goal: To determine all equivalence classes of irreducible representations
of a finite group $G$.

\begin{definition}

\end{definition}

\begin{remark} \mbox{}
  \begin{enumerate}
    \item $\chi_\rho$ is independent of the choice of $\beta = \{ e_i \}$
      For another basis $\beta' = \{ e_i' \}$.
    \item $\rho \tikz[anchor=base, baseline]{ \node(rho-equiv) {$\cong$}; }
      \rho' \leadsto \chi_\rho = \chi_{\rho'}$.
      \begin{tikzpicture}[overlay]
        \node[inner sep=0pt,outer sep=0pt] (t) at ($(rho-equiv) + (2, -0.5)$) {
           \footnotesize equivalent};
         \path[->,shorten <= 3pt] (t.west) edge[bend left=20] 
           ($(rho-equiv.base) + (0, -0.05)$);
      \end{tikzpicture}
  \end{enumerate}
\end{remark}

\begin{definition}\mbox{}
  \begin{itemize}
    \item The degree of $\chi_\rho$ is defined to the degree of $\rho$
      ($ =\dim V$).
    \item $\chi_\rho$ is an irreducible character if $\rho$ is irr.
  \end{itemize}
\end{definition}

Basic facts:
\begin{enumerate}
  \item $\chi_\rho(1) = n$.
  \item $\chi_\rho$ is a class function, i.e., it is constant on each
    conjugacy class.
  \item $\chi_\rho(g^{-1}) = \ob{\chi_\rho(g)}$: Assume that the eigenvalues
    of $R(g)$ are $\lambda_1, \dots, \lambda_n$. Then the eigenvalues of
    $R(g^{-1})$ are $\lambda_1^{-1}, \dots, \lambda_n^{-1}$.
    \[
      0 = \det(\lambda I_n - A) =
      \det(\lambda I_n (A^{-1} - \lambda^{-1}I_n) A) =
      \det(\lambda I_n) \det(A^{-1} - \lambda^{-1} I_n) \det(A)
    \]
    So $\det(A^{-1} - \lambda^{-1} I_n) = 0$.
    Then $g^m = 1 \implies R(g)^m = I_n \implies \abs{\lambda_i} = 1
    \implies \lambda_i^{-1} = \ob{\lambda_i}$. Thus
    $\chi_\rho(g^{-1}) = \trace(R(g)^{-1}) =
    \ob{\lambda_1 + \dots + \lambda_n} = \ob{\chi_\rho(g)}$.
  \item $\chi_{\rho \oplus \rho'} = \chi_{\rho} + \chi_{\rho'}$.
  \item $\chi_{\rho \otimes \rho'} = \chi_{\rho} \chi_{\rho'}$.
\end{enumerate}


\begin{definition}
  $\Cc(G, \Cb)$ is the vector space of complex functions on $G$.

  $\chi_\rho \in \Cc(G) \subset \Cc(G, \Cb)$
  is the vector space of complex class functions of $G$.
\end{definition}


\begin{remark}
  Assume that $\{ C_1, \dots, C_k \}$ is the set of distinct conjugacy classes
  in $G$.
  Then $\{\, f_i(C_j) = \delta_{ij} \mid \forall i = 1, \dots, k \,\}$ forms
  a basis for $\Cc(G)$ over $\Cb$.
  \begin{itemize}
    \item $\forall f \in \Cc(G)$, let $f(C_i) = a_i$, then
      $f = \sum a_i f_i$.
    \item $\sum a_i f_i = 0$, pick $x_j \in C_j$, then
      $(\sum a_if_i)(x_j) = a_j = 0 \quad \forall j = 1, \dots k$.
  \end{itemize}
  So $\dim \Cc(G) = k$.
\end{remark}

\begin{definition}
  $\phi, \psi \in \Cc(G, \Cb)$, then
  \[ \inpd{\phi, \psi} \defeq \sum_{g\in G} \phi(g) \ob{\psi(g)} \]
  is a positive definite Hermitian form on $\Cc(G, \Cb)$.
\end{definition}

\begin{theorem}[Main theorem]
  The set of all irr. characters of $G$ forms an orthonormal basis for $\Cc(G)$
  over $\Cb$. So there are only $k$ irr. rep. up to equivalent.
\end{theorem}

\begin{lemma}[Schur's lemma]
  Let $\rho: G \to \text{GL}(V)$ and $\rho': G\to \text{GL}(V')$ be two irr.
  rep. of $G$.
  
  一個ㄛ圖

  Then
  \begin{enumerate}
    \item $\rho, \rho'$ are not equivalent $\implies$ ${\sf T} = 0$.
    \item $V = V', \rho = \rho' \implies {\sf T} = \lambda 1_V$ for some
      $\lambda \in \Cb$.
  \end{enumerate}

  \begin{proof}
    \begin{enumerate}
      \item Assume ${\sf T} \ne 0$. Since ${\sf T}$ is $G$-equivariant,
        $\ker {\sf T} \le V$ and $\Image {\sf T}\le V'$ are $G$-invariant.

        $\rho$ is irr $\leadsto$ $\ker {\sf T} = 0$ or $V$.

        $\rho'$ is irr $\leadsto$ $\Image {\sf T} = 0$ or $V$.

        ${\sf T}$ is an isom.

        $\rho, \rho'$ are equivalent.

      \item  Let $\lambda$ be an eigenvalue of ${\sf T}$, say
        ${\sf T}(v) = \lambda v$ with $v \ne 0$ in $V$.
        Put ${\sf T}' - {\sf T} - \lambda 1_V$.

        Also, ㄛ圖 since $\rho(g)$ is $\Cb$-linear.

        So ${\sf T}'$ is also $G$-equivariant. But $v \in \ker {\sf T}'$,
        i.e. ${\sf T}'$ is not 1-1. By 1., ${\sf T'} = 0$.
    \end{enumerate}
  \end{proof}
\end{lemma}

\begin{coro}
  $\rho, \rho'$ as above. Let ${\sf L}: V \to V'$ be a linear transformation.
  Define
  \[
    {\sf T} = \frac{1}{\abs{G}} \sum_{g\in G} \rho'(g)^{-1} {\sf L} \rho(g)
  \]
  is $G$-equivariant. Then
  \begin{enumerate}
    \item $\rho, \rho'$ are not equivalent $\implies {\sf T} = 0$.
    \item $V = V', \rho = \rho' \implies {\sf T} = \lambda 1_V,
      \lambda \frac{\trace({\sf L})}{\dim V}$.
  \end{enumerate}
\end{coro}

\begin{remark}
  Let $\rho \to_\beta R: G \to \text{GL}_n(\Cb)$ and $R(g) = (r_{ij}(g))$

  $\rho' \to_{\beta'} R': G \to \text{GL}_{n'}(\Cb)$ and $R'(g) = (r'_{ij}(g))$

  Let ${\sf L} .....> [{\sf L}]^{\beta'}_\beta =
  (x_{\mu\nu} \in M_{n' \times n}(\Cb)$

  Then ${\sf T} ...> [{\sf T}]^{\beta'}_\beta = (x^0_{tl})$ with
  \[
    x^0_{tl} = \frac{1}{\abs{G}}
    \sum_{\substack{g\in G \\ i=1,\dots, n \\ j=1,\dots, n'}}
    r'_{tj}(g^{-1})x_{ji}r_{il}(g)
  \]

  In case 1. of coro, $x^0_{tl} = 0 \quad \forall t, l$.


  In case 2. of coro, ${\sf T} = \lambda 1_V$, i.e.
  $x^0_{tl} = \lambda \delta_{tl}$.
  $\lambda = \frac{\trace({\sf L})}{n} = \frac{1}{n} \sum_{i=1}^n x_{ii}
  = \frac{1}{n} \sum_{i,j} \delta_{ji}x_{ji}$

  Hence,
  \[
    \frac{1}{\abs{G}} \sum_{g\in G}
    r_{tj}(g^{-1})r_{il}(g) = \frac{1}{n} \delta_{ji}\delta_{tl}
  \]
\end{remark}

\begin{prop} \mbox{}
  \begin{enumerate}
    \item If $\chi_\rho$ is irr., then $\inpd{\chi_\rho, \chi_\rho} = 1$.
    \item If two irr. rep. $\rho, \rho'$ are not equivalent, then
      $\inpd{\chi_\rho, \chi_{\rho'}} = 0$.
  \end{enumerate}

  \begin{proof}
    \begin{enumerate}
      \item
      \item 
    \end{enumerate}
  \end{proof}
\end{prop}

OMIMI above


\begin{remark}
  $\inpd{\chi_\rho,\chi_\rho} = 1 \implies \rho$ is irr.
  \begin{proof}
    We write $\rho = \rho_1^{\oplus m_1}\oplus\dots\oplus\rho^{\oplus m_l}$
    where $\rho_1, \dots, \rho_l$ are non-equivalent irr. rep.
    \[ \chi_\rho = \sum_{i=1}^l m_i \chi_{\rho_i} \]
    \[
      1 = \inpd{\chi_\rho, \chi_\rho} = \sum_{i=1}^l m_i^2
      \implies \exists m_i = 1 \text{~and~} m_j = 0 \text{~for~} j \ne i
    \]
    So $\rho \cong \rho_i$.
  \end{proof}
\end{remark}

%1 TEX root=../main.tex
\section{Introduction to the linear representation theory of finite groups}
\subsection{Week 15}
\subsubsection{Character Theory \RNum{2}}

\begin{prop}
  Let $\rho: G\to \text{GL}(V)$ and
  $\rho = \rho^{W_1} \oplus \dots \oplus \rho^{W_k}$ where
  $\rho_i = \rho^{W_i}$ is irr. $\forall i$.
  ($V \cong W_1 \oplus\dots\oplus W_k$)
  
  If $\tilde{\rho}: G\to \text{GL}(\tilde{W})$ is an irr. rep. then the number
  of $\rho_i$ isomorphic to $\tilde{rho}$ is equal to
  $\inpd{\chi_\rho, \chi_{\tilde{\rho}}}$.
  \begin{proof}
    We know $\chi_\rho = \chi_{\rho_1} + \dots + \chi_{\rho_k}$, so
    \[
      \inpd{\chi_\rho, \chi_{\tilde{\rho}}}
      = \sum_{i=1}^k = \inpd{\chi_{\rho_i}, \chi_{\tilde{\rho}}}
    \]
    Recall $\rho_i \cong \tilde{\rho} \implies
    \inpd{\chi_{\rho_i}, \chi_{\tilde{\rho}}} = 1$, otherwise 
    $\inpd{\chi_{\rho_i}, \chi_{\tilde{\rho}}} = 0$.
  \end{proof}
\end{prop}

\begin{remark} \mbox{}
  \begin{enumerate}
    \item The number of $W_i$ isomorphic to $\tilde{W}$ does not depend
      on the chosen decomposition. ($= \inpd{\chi_\rho, \chi_{\tilde{\rho}}}$)
    \item If $\chi_\rho = \chi_{\rho'}$, then $\rho \cong \rho'$:
      $\inpd{\chi_\rho, \chi_{\tilde{\rho}}} = \inpd{\chi_{\rho'}, \chi_{\tilde{\rho}}}$
      The type of irr. subrep of $\rho$ is the same as $\rho'$.

    \item If $\chi_1, \dots, \chi_l$ are distinct irr. characters of $G$, then
      since $x_1, \dots, x_l$ are orthonormal w.r.t. $\inpd{\cdot,\cdot}$ in
      $\Cc(G)$, $x_1, \dots, x_l$ are linearly indep. over $\Cb$ in $\Cc(G)$.

      But $\dim \Cc(G) = k =$ \# of conjugacy classes in $G$. So
      $l \le k$ i.e. we conclude that there are at most $k$ mutually
      non-equivalent irr. rep. of $G$, say $\rho_1, \dots, \rho_l, l \le k$.

      For any $\rho: G\to \text{GL}(V)$,
      $\rho \cong \rho_1^{\oplus m_1} \oplus\dots\oplus\rho_l^{\oplus m_l}$
      where $m_i = \inpd{\chi_\rho, \chi_{\rho_i}} \in \Zb^{\ge 0}$.
  \end{enumerate}
\end{remark}

\begin{theorem}[Orthogonality relations for $\chi$'s]
  The set of all irr. characters of $G$ forms an orthonormal {\bf basis}
  $\Cc(G)$ over $\Cb$. In particular, the number of irr. rep. of $G$ is equal
  to \# of conjugacy classes in $G$. (up to equivalence)

  \begin{proof}
    Let $\chi_i = \chi_{\rho_i}, i = 1, \dots, l$ be all irr.
    characters of $G$ and $\Dc = \gen{\chi_1, \dots, \chi_l}_\Cb
    \subseteq \Cc(G)$. Then $\Cc(G) = \Dc \oplus \Dc^\perp$.
    Claim: $\Dc^\perp = \{0\}$.

    Let $\varphi \in \Dc^\perp$, i.e. $\inpd{\varphi, \chi_i} = 0 \quad
    \forall i = 1,\dots, l$.
    \begin{itemize}
      \item For $\rho: G\to \text{GL}(V),
        \rho \cong \rho_1^{\oplus m_1} \oplus\dots\oplus\rho_l^{\oplus m_l}$
        with $m_i = \inpd{\chi_\rho, \chi_i}$ and
        $\chi_\rho = m_1\chi_1 + \dots + m_k\chi_l \leadsto
        \inpd{\varphi, \chi_\rho} = 0$.
      \item Define ${\sf T}_\rho \in \Hom_\Cb(V, V)$ by
        \[
          {\sf T}_\rho = \frac{1}{\abs{G}} \sum_{g\in G}
          \ob{\varphi(g)} \rho(g)
          \quad \leadsto \quad
          \trace({\sf T}) = \frac{1}{\abs{G}} \sum_{g\in G}
          \ob{\varphi(g)} \chi_\rho(g)
          = \ob{\inpd{\varphi, \chi_\rho}} = 0
        \]
        $\forall h \in G$, we need 
        ${\sf T} = \rho(h)^{-1}\circ{\sf T}\circ\rho(h)$, and
        \begin{align*}
          \rho(h)^{-1}\circ{\sf T}\circ\rho(h)
          &= \frac{1}{\abs{G}} \sum_{g\in G}
          \ob{\varphi(g)} \rho(h)^{-1}\circ\rho(g)\circ\rho(h) \\
          &= \frac{1}{\abs{G}} \sum_{i=1}^k
          \ob{\varphi(g_i)} \sum_{g\in C_i} \rho(h^{-1}gh) \\
          &= \frac{1}{\abs{G}} \sum_{i=1}^k
          \ob{\varphi(g_i)} \sum_{g'\in C_i} \rho(g') = {\sf T}_\rho
        \end{align*}
        where $\{C_1, \dots, C_k\}$ is the set of distinct conjugacy
        classes in $G$.
      \item For $\rho = \rho_i$, by Schur's lemma,
        ${\sf T}_{\rho_i} = \lambda_i 1_{W_i}$ where
        $\rho_i: G\to \text{GL}(W_i)$.
        But $\trace{{\sf T}_{\rho_i}} = 0 \implies \lambda_i = 0
        \implies {\sf T}_{\rho_i} = 0$.
      \item In general,
        $\rho \cong \rho_1^{\oplus m_1} \oplus\dots\oplus\rho_l^{\oplus m_l}$,
        so ${\sf T}_{\rho_i} = 0 \implies {\sf T}_\rho = 0$.

      \item In particular, $\rho = \rho^{\text{reg}}: G\to \text{GL}(V)$ with
        $V = \bigoplus_{g\in G} \Cb e_g$. Then
        ${\sf T}_\rho = 0 \implies {\sf T}_\rho(e_1) = 0$ and
        \[
          0 = {\sf T}_\rho(e_1) = \frac{1}{\abs{G}} \sum_{g\in G}
          \ob{\varphi(g)} \rho(g)(e_1) = \frac{1}{\abs{G}} \sum_{g\in G}
          \ob{\varphi(g)} e_g
        \]
        Since $\{ e_g \}$ is a basis, $\ob{\varphi(g)} = 0 \quad \forall g$.
        That is, $\varphi = 0$. \qedhere
    \end{itemize}
  \end{proof}
\end{theorem}

\begin{prop}
  Each irr. rep. $\rho_i: G\to \text{GL}(W_i)$ is contained in
  $\rho^\text{reg}$ with multiplicity equal to $\dim W_i = m_i$,
  $i = 1,\dots, k$.

  In particular, $\bigoplus_{g\in G} \Cb e_g \cong
  \underbrace{W_1\oplus\dots\oplus W_1}_{m_1 \text{times}} \oplus \dots \oplus
  \underbrace{W_1\oplus\dots\oplus W_k}_{m_k \text{times}}$.
  So $\abs{G} = m_1^2 + \dots + m_k^2$.

  \begin{proof}
    Let $\chi^\text{reg} \defeq \chi_{\rho^\text{reg}}$ and
    $\chi_i = \chi_{\rho_i}, i = 1, \dots, k$. Then
    \[
      \inpd{\chi^\text{reg}, \chi_i} = \frac{1}{\abs{G}} \sum_{g\in G}
      \chi^\text{reg}(g)\chi_i(g^{-1})
      = \frac{1}{\abs{G}} \abs{G} \chi_i(1) = m_i
    \]
  \end{proof}
\end{prop}

\begin{theorem}[Divisibility]
  $\forall i = 1, \dots, k, \quad \chi_i(1) = m_i \Div \abs{G}$.

  \begin{proof} \mbox{}
    \begin{itemize}
      \item For $\rho=\rho_i$, $\chi = \chi_i$,
        \[
          \sum_{g\in C_j} \rho(g) = \frac{\abs{C_j}\chi(f_0)}{m_i}
          \sI_{m_i} \text{~for any~} g_0 \in C_j
        \]
        Observe that $\forall h \in G$,
        \[
          \rho(h)^{-1}\circ\sT\circ\rho(h) = \sum_{g\in C_j} \rho(h^{-1}gh)
          \sum_{g'\in C_j} \rho(g') = \sT
        \]
        So $\sT$ is $G$-equivariant w.r.t. $\rho$.
        By Schur's lemma, $\sT = \lambda \sI_{m_i}$ for some $\lambda \in \Cb$.
        And $\trace(\sT) = \sum_{g\in C_j} \chi(g) = \abs{C_j}\chi(g_0)$
        for any $g_0 \in C_j \leadsto \sT =
        \frac{\abs{C_j}\chi(g_0)}{\chi_i(1)} \sI$ for $g_0 \in C_j$.
      \item $\lambda_\mu(C_j) = \frac{\abs{C_j}\chi_\mu(g_0)}{m_\mu}$ for
        $g_0 \in C_j$ is an algebraic integer $\forall \mu, j$:
        For $g\in C_l$, $a_{i,j,l} \defeq$ \# of $\{\, (g_i, g_j) \in C_i \times C_j \mid
        g_ig_j = g \,\}$ which is indep. of the choice of $g$.

        Claim: $\lambda_\mu(C_i)\lambda_\mu(C_j) = \sum_{l=1}^k a_{i,j,l} \lambda_\mu(C_j)
        \quad \forall i, j, \mu$. Then $\lambda_\mu(C_j)$ is aneigenvalue of
        $A$, i.e., $\lambda_\mu(C_j)$ satisfies $\det(\lambda I - A) = 0$.

        \begin{align*}
          \lambda_\mu(C_i)\lambda_\mu(C_j) I_{m_\mu}
          &= \left(\lambda_\mu(C_i) I_{m_\mu}\right)
             \left(\lambda_\mu(C_j) I_{m_\mu}\right)
          = \left(\sum_{g\in C_i} \rho(g)\right)
             \left(\sum_{g'\in C_j} \rho(g')\right) \\
          &= \sum_{\substack{g\in C_i\\ g'\in C_j}} \rho(gg')
          = \sum_{l=1}^k \sum_{\bar{g}\in C_l} a_{i,j,l}\rho(\bar{g}) \\
          &= \sum_{l=1}^k a_{i,j,l} \sum_{\bar{g}\in C_l}\rho(\bar{g}) \\
          &= \sum_{l=1}^k a_{i,j,l} \lambda_\mu(C_l) I_{m_\mu}
       \end{align*}
     \item $m_i = \chi_i(1) \Div \abs{G} \quad \forall i = 1, \dots, k$:
       \begin{align*}
         \frac{\abs{G}}{\chi_i(1)}
         &= \frac{\abs{G}}{\chi_i(1)} \inpd{\chi_i, \chi_i} \\
         &= \frac{\abs{G}}{\chi_i(1)} \frac{1}{\abs{G}} \sum_{g\in G}
         \chi_i(g) \chi_i(g^{-1}) \\
         &= \sum_{g\in G} \frac{\chi_i(g)}{\chi_i(1)} \chi_i(g^{-1}) \\
         &= \sum_{j=1}^k \sum_{g\in C_j} \frac{\chi_i(g)}{\chi_i(1)} \chi_i(g^{-1}) \\
         &= \sum_{j=1}^k \frac{\abs{C_j}\chi_i(g_j)}{\chi_i(1)} \chi_i(g_j^{-1})
       \end{align*}
    \end{itemize}
  \end{proof}
\end{theorem}

\begin{exercise} \mbox{}
  \begin{enumerate}
    \item Show that if $g\in G$ and $g\ne 1$, then
      $\sum_{i=1}^k m_i\chi_i(g) = 0$.
    \item Show that each character $\chi$ of $G$ with $\chi(g) = 0 \quad
      \forall g \ne 1$ is an integral multiple of $\chi^\text{reg}$.
  \end{enumerate}
\end{exercise}

\begin{exercise} \mbox{}
  \begin{enumerate}
    \item Let $\abs{G} < \infty$. then $G$ is abelian $\iff$ each irr. rep.
      of $G$ is of degree $1$.
    \item $\{ \text{the $\deg 1$ rep. of $G$} \} = \{
      \text{the irr. rep. of~} \quot{G}{[G,G]} \}$.
  \end{enumerate}
\end{exercise}


%1 TEX root=../main.tex
\section{Extensions of Groups}
\subsection{Week 16}
\subsubsection{Extensions of abelian groups}

\begin{definition}
  If a group $E$ contains a normal subgroup $N$ and $\quot{E}{N} \cong G$,
  then we call $E$ an extension of $N$ by $G$, denoted by
  $1 \to N \to E \to G \to 1$.
\end{definition}

\underline{Ques}: When $N$ and $G$ are given, how to obtain all extensions of
$N$ by $G$.

{\bf Now assume that $N$ is abelian.}

\begin{definition}
  $1 \to N \to E \xrightarrow{p} G \to 1$.
  $l: G \to E $ is a lifting if $p \circ l = \text{id}_G$ and $l(1) = 1$.
\end{definition}

\begin{remark}
  $G \cong \quot{E}{N} = \{\, xN \mid x \in E \,\}$,
  $p \circ l(\bar{x}) = \bar{x}$, $l(\bar{x})$ is a representative of $xN = \bar{x}$.
\end{remark}

\begin{prop} \mbox{}
  \begin{enumerate}
    \item $\forall \bar{x}\in G, \theta_{\bar{x}}: N \to N, a \mapsto l(\bar{x})al(\bar{x})^{-1}$.
      is independent of the choice of $l$.
    \item $\theta: G \to \Aut(N), \bar{x} \mapsto \theta_{\bar{x}}$ is a group homomorphism.
  \end{enumerate}
  \begin{proof} \mbox{}
    \begin{enumerate}
      \item Suppose $l': G \to E$ is another lifting. Then $l(\bar{x})N = l'(\bar{x})N$.
        So $l'(\bar{x}) = l(\bar{x})b$ for some $b \in N$.
        $\forall a \in N$, $l'(\bar{x})al'(\bar{x})^{-1} = l(\bar{x})bab^{-1}l(\bar{x})^{-1}
        = l(\bar{x})al(\bar{x})^{-1}$ since $N$ is abelian.
      \item $\theta_{\bar{x}\bar{y}}(a) = l(\bar{x}\bar{y})al(\bar{x}\bar{y})^{-1}$.
        \[
          \begin{cases}
            p \circ l(\bar{x}\bar{y}) = \bar{x}\bar{y} \\
            p \circ (l(\bar{x})l(\bar{y})) = \bar{x}\bar{y}
          \end{cases}
          \leadsto l(\bar{x}\bar{y}), l(\bar{x})l(\bar{y})
          \text{~are liftings of~} \bar{x}\bar{y}
          \qedhere
        \]
    \end{enumerate}
  \end{proof}
\end{prop}

\begin{definition}
  An extension $1\to N\to E\to G\to 1$ splits if $\exists$ a lifting
  $l: G\to E$ is a group homo.
\end{definition}

\begin{prop} TFAE
  \begin{enumerate}
    \item $1\to N\to E\to G\to 1$ splits.
    \item $\exists$ a subgroup $K \le E$ s.t. $K \cong G$ and
      $\begin{cases} K \cap N = \{1\} \\ NK = E\end{cases}
      \leadsto E \cong N \rtimes K (\cong N \rtimes G)$.
  \end{enumerate}
  \begin{proof}
    (1) $\Rightarrow$ (2): Let $K = \Image l$ which is a subgroup since $l$
    is a group homo.
    \begin{itemize}
      \item $l$ is an isomorphism from $G$ to $K$: If $l(\bar{x}) = l(\bar{y})$, then
        $p \circ l(\bar{x}) = p\circ l(\bar{y}) \leadsto \bar{x} = \bar{y}$.
        So $l$ is 1-1.
      \item $E = NK$: $\forall x \in E, \bar{x} = p(x) \leadsto
        y = l(\bar{x})$ and $p(x) = p(y) \leadsto \exists a \in N$ s.t.
        $x = ay$.
      \item $K \cap N = \{1\}$:
        $a = l(\bar{x}) \in K \cap N \leadsto 1 = p(a) = p(l(\bar{x})) = \bar{x}
        \leadsto a = l(1) = 1$.
    \end{itemize}

    (2) $\Rightarrow$ (1):
    \begin{itemize}
      \item $p \big|_K: K \to G$ is an isom.:
        onto: $p(K) = p(NK) = p(E) = G$, 1-1: $\ker(p\big|_K) = N \cap K = \{1\}$.
      \item $l = \left(p\big|_K\right)^{-1}$ is a group homo.

        \underline{Observation}: Let $l: G\to E$ be a lifting.
        Then $E = \bigcup_{\bar{x}\in G} Nl(\bar{x}), \forall x, y \in E$,
        write $x = al(\bar{x}), y = bl(\bar{y}), a, b\in N, \bar{x},\bar{y}\in G$.
        \[
          xy = (al(\bar{x})bl(\bar{y})) = al(\bar{x})bl(\bar{x})^{-1}l(\bar{x})l(\bar{y})
          = a \theta_{\bar{x}}(b)l(\bar{x})l(\bar{y})
        \]
        Notice that $l(\bar{x})l(\bar{y})$ and $l(\bar{x}\bar{y})$ are liftings,
        so we can write $l(\bar{x})l(\bar{y}) = f(\bar{x}, \bar{y})l(\bar{x}\bar{y})$
        for some $f(\bar{x}, \bar{y})\in N$.
        \qedhere
    \end{itemize}
  \end{proof}
\end{prop}

\begin{exercise}
  $B^2(G, N) \le Z^2(G, N)$.
\end{exercise}
\begin{exercise}
  Show that there are inequivalent  extensions of $N$ by $G$ with isomorphic
  middle groups.
  (Hint: $N=\quot{\Zb}{p\Zb}$ with $p$ is odd, $E=\quot{\Zb}{p^2\Zb}$,
  $a:: N \mapsto x^p::E$ and please give another morphism $N\to E$ by yourself.)
\end{exercise}

\begin{definition}
  Given $1\to N\to E\xrightarrow{p} G\to 1$ and $l: G\to E$, a factor set is
  a function $f: G\times G \to N$ s.t.
  $\forall \bar{x}, \bar{y}\in G, l(\bar{x})l(\bar{y})
  = f(\bar{x}, \bar{y})l(\bar{x}\bar{y})$.
\end{definition}

\begin{prop}
  Let $1\to N \to E\xrightarrow{p} G\to 1$ and $l: G\to E$.
  If $f$ is a factor set, then
  \begin{enumerate}[(1)]
    \item $f(x, 1) = 1 = f(1, y) \quad \forall x, y\in G$.
    \item (cocycle identity) $\forall x,y,z\in G,
      f(x,y)f(xy, z)
      = \theta_{x}(f(y, z))f(x, yz)$.

      (i.e. $f(x, y) + f(xy,z)
      = xf(y,z) + f(x, yz)$)
  \end{enumerate}
  \begin{proof} \mbox{}
    \begin{enumerate}[(1)]
      \item Trivial since $l(x)l(1) = l(1 \cdot x)$.
      \item By associativity. $(l(x)l(y))l(z) = l(x)(l(y)l(z))$. \\
        $(l(x)l(y))l(z) = f(x, y) l(xy) l(z) = f(x, y) f(xy, z) l(xyz)$, and \\
        $l(x)(l(y)l(z)) = l(x) f(y, z) l(yz) = l(x) f(y, z) l^{-1}(x) l(x) l(yz)
        = \theta_x(f(y, z)) f(x, yz) l(xyz)$.
        Thus $f(x, y) f(xy, z) = \theta_x(f(y, z)) f(x, yz)$.
        \qedhere
    \end{enumerate}
  \end{proof}
  \label{prop:factor-set}
\end{prop}

\begin{theorem}
  Let $\sigma: G \to \Aut(N), x\mapsto \sigma_x$  be a group homo. and
  $f: G\times G\to N$ satisfies (1),(2) in Prop. \ref{prop:factor-set}.
  Then $\exists 1\to N\to E\to G\to 1$ and $l: G\to E$ s.t. $\theta = \sigma$
  and $f$ is the corresponding factor set.

  \begin{proof}
    \begin{itemize}
      \item Define $E = N \times G$ equipped with the operation
        \[
          (a, x)(b, y) = (a \sigma_{x}(b)f(x,y),
          xy)
        \]
        \begin{itemize}
          \item associativity:
            \begin{align*}
              \big( (a, x) (b, y) \big) (c, z) &= (a \sigma_x(b) f(x, y), xy) (c, z) \\
              &= (a \sigma_x(b) f(x, y) \sigma_{xy}(c) f(xy, z), xyz) \\
              &= (a \sigma_x(b) \sigma_{xy}(c) f(x, y) f(xy, z), xyz) \quad (\because N \text{ abelian})
            \end{align*}
            and
            \begin{align*}
              (a, x) \big( (b, y) (c, z) \big) &= (a, x) (b \sigma_y(c) f(y, z)) \\ 
              &= \big( a \sigma_x \big( b \sigma_y(c) f(y, z) \big) f(x, yz), xyz \big) \\
              &= \big( a \sigma_x ( b ) \sigma_{xy}(c) \sigma_x(f(y, z)) f(x, yz), xyz \big) \\
              &= \big( a \sigma_x ( b ) \sigma_{xy}(c) f(x, y) f(xy, z), xyz \big) \\
            \end{align*}
          \item indentity: $(1, 1)$.
          \item inverse: $(a, x)^{-1} =
            (\sigma_{x^{-1}}(a^{-1}f(x,x^{-1})^{-1}), x^{-1})$.
        \end{itemize}
      \item $p: E\to G, (a, x) \mapsto x$ is a group homo by def.
      \item $i: N\to E, a \mapsto (a, 1)$ is a group homo.
        $(a, 1)(b, 1) = (a\sigma_1(b)f(1, 1), 1) = (ab, 1)$.
      \item $\ker p = \Image i$.
      \item $\Fix l: G\to E, a\in N, x\in G$, say $l(x) = (b, x)$.
        \begin{align*}
          l(x)(a, 1)l(x)^{-1}
          &= (b, x) (a, 1) (b, x)^{-1} = (b \sigma_x(a) , x)
          \big(\sigma_{x^{-1}}(a^{-1}f(x,x^{-1})^{-1}), x^{-1} \big) \\
          &= (b \sigma_{x}(a) \cdot (\sigma_x \circ \sigma_{x^{-1}}) \big( b^{-1} f(x, x^{-1})^{-1} \big)
          \cdot f(x, x^{-1}), 1) \\
          &= (\sigma_x(a), 1)
        \end{align*}
        So $\theta_{x} = \sigma_{x}$.
      \item Let $l: G\to E, x\mapsto (1, x)$.
        Check $l(x)l(y)l(xy)^{-1} = (f(x,y), 1)$.
        Then $f$ is the corresponding factor set.
        \qedhere
    \end{itemize}
  \end{proof}
\end{theorem}

\begin{prop}
  Let $1\to N\to E \xrightarrow{p}G \to 1$ with two liftings
  $l_1: G \to E,\ l_2: G\to E$ with $f_1: G\times G\to N,\ f_2: G\times G\to N$
  respectively.

  Then $\exists h: G\to N$ with $h(1) = 1$ and $\forall x,y\in G,
  f_2(x, y)f_1(x,y)^{-1} = \theta_{x}(h(y))h(xy)^{-1}h(x)$. 
  ($f_2(x,y) - f_1(x,y) = xh(y) - h(xy) + h(x)$)

  \begin{proof}
    For $x\in G$, $\exists h(x) \in N$ s.t.
    $l_2(x) = h(x)l_1(x)$.
    Since $l_1(1) = l_2(1) = 1$, $h(1) = 1$.

    Now, $l_2(x)l_2(y) = f_2(x, y) l_2(x, y) = f_2(x, y) h(xy) l_1(x, y)$. and
    \begin{align*}
      l_2(x) l_2(y) &= h(x) l_1(x) h(y) l_1(y) = h(x) l_1(x) h(y) l_1^{-1}(x) l_1(x) l_1(y) \\
      &= h(x) \theta_x(h(y)) l_1(x) l_1(y) = f_1(x, y) h(x) \theta_x(h(y)) l_1(x, y)
    \end{align*}
    So $f_2(x, y) f_1(x, y)^{-1} = \theta_x(h(y)) h(xy)^{-1} h(x)$.
  \end{proof}
\end{prop}

\begin{remark}
  A map which has the form $\tilde{h}: G\times G \to N, (x,y) \mapsto
  xh(y) - h(xy) + h(x)$ is called a coboundary map.
\end{remark}

\begin{definition}
  $Z^2(G, N) =$ the abelian group of all factor sets.

  $B^2(G, N) =$ the abelian group of all coboundary maps.

  $H^2(G, N) = \quot{Z^2(G, N)}{B^2(G, N)}$
\end{definition}

\begin{definition}
  Two extensions $\begin{cases}
    1\to N\to E\to G\to 1 \\
    1\to N\to E'\to G\to 1
  \end{cases}$
  are equivalent if exists an isomorphism $\varphi: E \xrightarrow{\sim} E'$ which let the
  following diagram comutes.
  \[
    \begin{tikzcd}
      1 \arrow[r]
      & N \arrow[d, "1_N"] \arrow[r]
      & E \arrow[d, "\rotatebox{90}{\(\sim\)}", "\varphi"'] \arrow[r]
      & G \arrow[d, "1_G"] \arrow[r]
      & 1 \\
      1 \arrow[r] & N \arrow[r] & E \arrow[r] & G \arrow[r] & 1
    \end{tikzcd}
  \]
\end{definition}

\begin{theorem}
  Two extensions $\begin{cases}
    1\to N\to E\to G\to 1 \\
    1\to N\to E'\to G\to 1
  \end{cases}$ are equivalent $\iff$ \\
  Exists mappings $l: G\to E, l': G\to E'$ with two factor sets $f, f'$ respectively satisfies
  $f - f' \in B^2(G, N)$.

  \begin{proof}
    ``$\Rightarrow$'': Choose $l: G\to E$ which has a corresponding factor set $f: G\times G \to N$.
    Now define $l': G\to E'$ by $l' = \varphi \circ l$. Since 
    $p' \circ l' = p' \circ \varphi \circ l = p \circ l = 1$, $l'$ is a lifting.
    Let   $f': G\times G \to N$ be its factor set.
    
    Since $1_N = 1_N \circ \varphi$, $\varphi \big|_N = 1_N$. And
    \begin{align*}
      & l(x) l(y) = f(x, y) l(x, y) \\
      \Rightarrow \ & \varphi(l(x) l(y)) = \varphi(f(x, y) l(x, y)) \\
      \Rightarrow \ & l'(x) l'(y) = \varphi(f(x, y)) l'(x, y) \\
      \Rightarrow \ & f'(x, y) = \varphi(f(x, y))
    \end{align*}
    But $f(x, y) \in N$, $\varphi(f(x, y)) = \varphi\big|_N(f(x, y)) = f(x, y)$. So $f(x, y) = f'(x, y)$,
    hence $f - f' = 0 \in B^2(G, N)$.
    \begin{exercise} \mbox{}
      \begin{enumerate}[(1)]
        \item Show that $f' - f \in B^2(G, N)$.
        \item ``$\Leftarrow$'': Show all details of the following steps:
          \begin{itemize}
            \item $\begin{cases}
                1\to N\to E \to G \to 1 \\
                1\to N\to E(N, G, f, \theta) \to G\to 1
              \end{cases}$ are equivalent.
            \item Similarly $\begin{cases}
                1\to N\to E' \to G\to 1 \\
                1\to N\to E(N, G, f', \theta')\to G\to 1
              \end{cases}$ are equivalent.
            \item $f'-f \leadsto h:G\to N$,
          \end{itemize}
          \qedhere
      \end{enumerate}
    \end{exercise}
  \end{proof}
\end{theorem}


%! TEX root=../main.tex
\section{Fields}

\subsection{Algebraic extensions}

\begin{definition} \hfill
  \begin{itemize}
    \item $L / K$ is called an field extension if $L$ is a field and $K$ is a subfield of $L$.
    \item $L / K$ is called an algebraic extension if $\forall \alpha \in L, \exists f(x) \in K[x]$
      such that $f(\alpha) = 0$.
    \item $K(\alpha_1, \alpha_2, \dots, \alpha_n) \triangleq \big\{ P(\alpha_1, \dots, \alpha_n)
      / Q(\alpha_1, \dots, \alpha_n) : P, Q \in K[x_1, x_2, \dots, x_n] \text{ and } Q \neq 0 \big\}$
  \end{itemize}
\end{definition}

\begin{theorem}[Eisenstein criterion] \mbox{} \\
  Let $f(x) = a_n x^n + \dots + a_1 x + a_0 \in \Zb[x]$ with $\gcd(a_0, a_1, \cdots, a_n) = 1$.
  Assume that there exists a prime $p$ s.t. $p \nmid a_n$ but $p \mid a_i$ for other $i \neq n$,
  and $p^2 \nmid a_0$, then $f$ is irreducible.

  \begin{proof}
    Consider $\bar{f}(x)$, by assumption, $\bar{f}(x) = \bar{a}_n x^n$. So if $f(x) = g(x) h(x)$
    with $\deg g, \deg h \geq 1$, let $g(x) = b_r x^r + \dots + b_0, h(x) = c_{n-r} x^{n-r} + \dots + c_0$,
    then $\bar{g}(x) = \bar{b}_r x^r, \bar{h}(x) = \bar{c}_{n-r} x^{n-r}$ for some
    $r$. But then we would find out that $\bar{b}_0 = \bar{c}_0 = 0$, and thus $p^2 \mid a_0$,
    which is a contradiction, hence $f$ is irreducible.
  \end{proof}
\end{theorem}

\begin{prop}
  Given $L/K$ and $\alpha \in L$, if $\alpha$ is algebraic over $K$, then
  there exists a unique monic irreducible polynomial $m_{\alpha, K}(x) \in K[x]$
  of minimal degree s.t. $m_{\alpha, K}(\alpha) = 0$ and for any other
  $f(x) \in K[x]$ with $f(\alpha) = 0$, we have $m_{\alpha, K} \mid f$.
  We call $m_{\alpha, K}$ the {\bf minimal polynomial} of $\alpha$ over $K$.

  \begin{proof}
    %Consider the evaluation map on $\alpha$:
    %\[ \deffunc{\text{ev}_\alpha}{K[x]}{K[\alpha]}{f(x)}{f(\alpha)} \]
    %The map is

    Let $I$ be the set of all polynomials such that $f(\alpha) = 0$, since $\alpha$ algebraic,
    $I \neq \varnothing$, so pick a monic polynomial $g(x)$ of minimal degree in $I$.
    For any other $f(x) \in I$, write $f(x) = g(x) q(x) + r(x)$ with $\deg r < \deg g$.
    If $r(x) \neq 0$, then $r(\alpha) = f(\alpha) - q(\alpha) g(\alpha)$.
    But then $r(\alpha) = f(\alpha) - q(\alpha) g(\alpha) = 0$ with $\deg r < \deg g$,
    which contradicts the minimality of $g$, thus $r = 0$, and hence $g \mid f$.

    Finally, if $g(x) = h_1(x) h_2(x)$ with $\deg h_1, \deg h_2 < \deg g$,
    then one of them, say $h_1(\alpha) = 0$ again contradicts the minimality of $g$,
    hence $g$ is irreducible.
  \end{proof}
\end{prop}

\begin{prop}
  Let $L/K$ be an extension and $\alpha \in L$, the following are equivalent:
  \begin{enumerate}[(\arabic*)]
    \item $\alpha$ is algebraic over $K$.
    \item $K[\alpha] = K(\alpha)$.
    \item $[K(\alpha): K] < \infty$.
  \end{enumerate}

  \begin{proof}
    (1) $\Rightarrow$ (2): ``$\subset$'' trivial. \\
    ``$\supset$'': For all $\beta \in K(\alpha), \beta = g(\alpha) / h(\alpha)$ with $h(\alpha) \neq 0$.
    So $m_{\alpha, K} \nmid h$. Since $m_{\alpha, K}$ is irreducible, $\gcd(m_{\alpha, K}, h) = 1$,
    hence there exists $a(x), b(x) \in K[x]$ such that $1 = a(x) h(x) + b(x) m_{\alpha, K}(x)$
    Subsitute $\alpha$ and we get $1/h(\alpha) = a(\alpha)$, hence $\beta = g(\alpha) a(\alpha) \in K[\alpha]$.

    (2) $\Rightarrow$ (1): Since $1 / \alpha \in K[\alpha]$, thus $1 / \alpha = f(\alpha)$ for some
    polynomial $f$, hence if $g(x) = xf(x) - 1, g(\alpha) = 0$ which implies $\alpha$ is algebraic.

    (1) $\Rightarrow$ (3): Assume that $\deg m_{\alpha, K} = n$, it is easy to see that
    $K[\alpha] = \gen{ 1, \alpha, \dots, \alpha^{n-1} }_K$. Since (1) $\implies$ (2),
    we have $[K(\alpha): K] = [K[\alpha], K] = n$.

    (3) $\Rightarrow$ (1): Since $[K(\alpha): K] = n$, consider $1, \alpha, \alpha^2, \dots, \alpha^n$.
    Some of these $n+1$ elements may be coincident, but nevertheless these elements are linearly dependent.
    Hence there exists $a_0, \dots, a_n$ not all zero in $K$ s.t.
    $a_0 + a_1 \alpha + \dots + a_n \alpha^n = 0 \implies \alpha$ is algebraic.
  \end{proof}
\end{prop}

\begin{prop}
  Given $M/L$ and $L/K$, $[M: K] = [M: L] [L: K]$.

  \begin{proof}
    If $[M:L] = m < \infty$ and $[L:K] = n < \infty$, then $L \cong K^{\oplus n}, M \cong L^{\oplus m}$.
    So $M \cong \left( K^{\oplus n} \right)^{\oplus m} \cong K^{\oplus mn}$, thus $[M: K] = mn$.

    Now if $[M: K] = l < \infty$, then there exists a basis $\{ z_1, z_2, \dots, z_l \}$
    which is a basis for $M$ over $K$. Then $M = K z_1 + \dots + K z_l \subset L z_1 +
    \dots + L z_l \subset M \implies M = L z_1 + \dots + L z_l$. Hence $[M: L] < \infty$.
    Also, since $L$ is a $K$-linear subspace of $M$, $[L: K] \leq l \implies [L: K] < \infty$.
    Thus if $[M: L] = \infty$ or $[L: K] = \infty$, then $[M: K] = \infty$.
  \end{proof}
\end{prop}

\begin{prop} \label{prop:alg-elements-form-a-field}
  Given $L/K$, define $L^{\text{alg}} \triangleq \{ \alpha \in L \mid \alpha \text{ is algebraic over } K \}$,
  then $L^{\text{alg}}$ is a subfield of $L$.

  \begin{proof}
    Notice that if $\alpha, \beta \in L^{\text{alg}}$, then $\beta$ is algebraic over $K$
    implies that $\beta$ is algebraic over $K(\alpha)$. Thus
    \[ [K(\alpha, \beta): K] = [K(\alpha)(\beta): K(\alpha)] [K(\alpha): K]
      < \infty \]

    Also, since $K(\alpha + \beta), K(\alpha - \beta), K(\alpha \beta), K(\alpha / \beta)$ are
    all contained in $K(\alpha, \beta)$, they are all algebraic over $K$, thus
    these elements are all algebraic, and hence $L^{\text{alg}}$ is a subfield.
  \end{proof}
\end{prop}

\begin{prop}
  $[L: K] < \infty$ if and only if $L = K(\alpha_1, \alpha_2, \dots, \alpha_n)$ with each $\alpha_i$
  algebraic over $K$. In this case, $L / K$ is algebraic (but the other side may not hold).

  \begin{proof}
    ``$\Rightarrow$'': Let $[L: K] = n$, so there is a basis $\{ \alpha_1, \alpha_2, \dots, \alpha_n \}$
    for $L$ over $K$. It is easy to see that $L = K(\alpha_1, \dots, \alpha_n)$.
    Also $[K(\alpha_i): K] \leq [L: K] < \infty$, thus $\alpha_i$ is algebraic.

    ``$\Leftarrow$'': Since $\alpha_i$ is algebraic over $K$, $\alpha_i$ is algebraic over $K(\alpha_1, \dots, \alpha_{i-1})$.
    Thus
    \[ [L: K] = [K(\alpha_1, \dots, \alpha_n): K(\alpha_1, \dots, \alpha_{n-1})]
      [K(\alpha_1, \dots, \alpha_{n-1}): K(\alpha_1, \dots, \alpha_{n-2})] \dots [K(\alpha_1): K] < \infty \]
    Moreover, $\forall \alpha \in L, [K(\alpha): K] \leq [L: K] < \infty$, so $\alpha$ is algebraic over $K$.
  \end{proof}
\end{prop}

\begin{coro}
  Given $L/K$, and $S$ a subset of $L$, if $\forall \alpha \in S$, $\alpha$ is
  algebraic over $K$, then $K(S) / K$ is algebraic.

  \begin{proof}
    If $\beta \in K(S)$, by definition we know that there exists
    $\alpha_1, \dots, \alpha_n$ such that $\beta \in K(\alpha_1, \dots, \alpha_n)$.
    Thus $\beta$ is algebraic over $K$.
  \end{proof}
\end{coro}

\begin{prop} \label{prop:alg-tower-implies-alg}
  If $M/L$ and $L/K$ are algebraic, then $M/K$ is algebraic.

  \begin{proof}
    For all $\alpha \in M$, since $\alpha$ is algebraic over $L$,
    there exists $a_{n-1}, \dots, a_0$ so that $\alpha^n + a_{n-1} \alpha^{n-1} + \dots + a_0 = 0$,
    that is, $\alpha$ is algebraic over $K(a_0, \dots, a_{n-1})$.

    So $[K(a_0, \dots, a_{n-1}, \alpha): K] = [K(a_0, \dots, a_{n-1})(\alpha): K(a_0, \dots, a_{n-1})]
    [K(a_0, \dots, a_{n-1}): K] < \infty$, thus $\alpha$ is algebraic over $K$.
  \end{proof}
\end{prop}

\begin{definition}
  Given $L/L_1$ and $L/L_2$, $L_1 L_2$ is defined as the smallest subfield of $L$
  containing both $L_1$ and $L_2$.
\end{definition}

\begin{prop}
  Let $[L_1: K] = m$ and $[L_2: K] = n$.
  \begin{enumerate}[(\arabic*)]
    \item $[L_1 L_2: K] \leq mn$.
    \item If $\gcd(m, n) = 1$, then $[L_1 L_2: K] = mn$.
  \end{enumerate}

  \begin{proof}
    (1): Assume $L_1 = K(\alpha_1, \dots, \alpha_m), L_2 = K(\beta_1, \dots, \beta_n)$.
    We could find that $L_1 L_2 = K(\alpha_1, \dots, \alpha_m, \beta_1, \dots, \beta_n)$.
    Notice that $[K(\beta_1, \dots, \beta_m)(\alpha_i): K(\beta_1, \dots, \beta_m)] \leq [K(\alpha_i): K]$,
    and thus $[L_1 L_2: K] = [K(\alpha_1, \dots, \alpha_m, \beta_1, \dots, \beta_n)
    : K(\beta_1, \dots, \beta_n)] [K(\beta_1, \dots, \beta_m): K] \leq [K(\alpha_i, \dots, \alpha_n): K]
    [K(\beta_1, \dots, \beta_n): K] = [L_1: K][L_2: K]$.

    (2): Notice that $[L_i: K] \mid [L_1 L_2: K]$, so $mn \mid [L_1 L_2: K]$. By
    (1), $[L_1 L_2: K] \leq nm$, hence $[L_1 L_2: K] = nm$.
  \end{proof}
\end{prop}

\begin{definition}
  Let $R$ be a commutative ring with $1$, and $I$ be an ideal of $R$, then
  \begin{itemize}
    \item $I$ is called a {\bf maximal ideal}\index{Ideal!maximal ideal} if for any ideal $J$ satisfying
      $I \subseteq J$ we have $J = I \text{ or } J = R$.
    \item $I$ is called a {\bf prime ideal}\index{Ideal!prime ideal}
      if $I \neq R$ and $ab \in I \implies a \in I \text{ or } b \in I$.
  \end{itemize}
\end{definition}

\begin{prop} \label{prop:max-prime-to-field-int-domain}
  Suppose $R$ is a ring and $I \subsetneq R$ is an ideal, then
  \begin{enumerate}
    \item $I$ is maximal $\iff$ $R / I$ is a field.
    \item $I$ is a prime ideal $\iff$ $R / I$ is an integral domain.
  \end{enumerate}

  \begin{proof} \hfill
    \begin{enumerate}
      \item ``$\Rightarrow$'': For any $\bar{r} \in R/I$ with $\bar{r} \neq 0$, then $r \not\in I$.
        Consider $\gen{ r } + I$ which contains $I$ and is not equal to $I$ because $r \not\in I$.
        Since $I$ is maximal, $\gen{ r } + I = R$, and thus $\exists x \in R, y \in I$ such that
        $xr + y = 1$, so $\bar{x} \bar{r} = \bar{1}$. Hence every non-zero element has multiply inverse
        and $R / I$ is a field.

      ``$\Leftarrow$'': If $J$ is an ideal such that $I \subsetneq J$, pick $x \in J \setminus I$,
      then $\bar{x} \neq 0$, so $\exists r \in J$ such that $\bar{x} \bar{r} = 1$. Then
      $xr + I = 1 + I \implies \exists y \in I \text{ s.t. } xr + y = 1$. So $1 \in J$, and
      because $J$ is an ideal, $J = R$.

      \item By the fact that $(ab \in I \implies a \in I \text{ or } b \in I) \iff
        (\bar{a}\bar{b} = 0 \implies \bar{a} = 0 \text{ or } \bar{b} = 0)$ the proof is complete.
        \qedhere
    \end{enumerate}
  \end{proof}
\end{prop}

\begin{prop} \label{prop:irr-to-max-ideal}
  If $f(x) \in K[x]$ is irreducible, where $K$ is a field, then $\gen{ f(x) }$ is maximal ideal.
  \begin{proof}
    We know that $K[x]$ is a principle ideal domain, so if $\gen{ f(x) } \subseteq J$, then
    $J$ is generated by a element, say $g(x)$. Since $f(x) \in J$, we could write $f(x) = g(x) h(x)$.
    By the fact that $f(x)$ is irreducible, either $g(x)$ is an unit then $J = R$, or $h(x)$ is
    an unit then $J = \gen{ f(x) }$.
  \end{proof}
\end{prop}

\begin{example}
  $f(x) = x^2 + 1$ has roots $\alpha = \pm \sqrt{-1}$, so $\Rb(\sqrt{-1}) \cong \Rb[x] / \gen{ x^2 + 1 }$.
\end{example}

\begin{theorem} \label{thm:field-ext-1}
  Let $f(x) \in K[x]$ be monic, irreducible and of degree $n$. Then there exists
  $L / K$ and $\alpha \in L$ s.t. $f(\alpha) = 0, L = K(\alpha)$ and $[L: K] = n$.
  \begin{proof}
    Since $f(x)$ is irreducible, by prop. \ref{prop:irr-to-max-ideal}
    $\gen{ f(x) }$ is a maximal ideal. Then by prop.
    \ref{prop:max-prime-to-field-int-domain} $L = K[x] / \gen{ f(x) }$ is a field,
    and $K$ is a subfield of $L$ by the inclusion map $\alpha \mapsto \bar\alpha$.
    The map is 1-1 since $\bar{1} \neq 0$ and a field homomorphism is either a
    1-1 map or a zero(全洪)map.

    Notice that $L \cong K[\bar{x}]$, where $\bar{x}$ is the coset $x + \gen{ f(x) }$.
    Now let $\alpha = \bar{x}$, and it is easy to see that $f(\alpha) = f(x) + \gen{ f(x) } = 0$.
    Also $L \cong K[\bar{x}] \cong K(\alpha)$. Finally, $m_{\alpha, K} \mid f$ and by the fact that
    $f$ is monic and irreducible, $m_{\alpha, K} = f$ and thus $[L: K] = \deg m_{\alpha, K} = \deg f = n$.
  \end{proof}
\end{theorem}

\begin{theorem}
  Let $f(x) \in K[x]$ be of degree $n > 0$. Then there exists $L/K$ s.t. $f$
  splits over $L$, that is,
  \[ f(x) = \lambda (x - \alpha_1) (x - \alpha_2) \dotsm (x - \alpha_n) \text{ with }
    \alpha_1, \alpha_2, \dots, \alpha_n \in L,\, \lambda \in K \]
  In fact, $L$ can be chosen to be the smallest field over which $f$ splits and in this case $[L : K] \leq n!$.\\
  $L$ is called a \emph{splitting field} \index{splitting field} for $f$ over $K$.
\end{theorem}

\begin{proof}
  By induction on $n$, $n = 1$ is trivial, simply pick $L = K$.

  For $n > 1$, let $p(x)$ be an monic irreducible factor of $f(x)$.
  By theorem \ref{thm:field-ext-1}, there exists an extension $K(\alpha_1)$ s.t. $p(\alpha_1) = 0$.
  By division algorithm, $f(x) = (x - \alpha_1) f_1(x)$ where $f_1(x) \in K(\alpha_1)[x]$
  and $\deg f_1 = n - 1$. Using the induction hypothesis, we know that there exists $L$,
  which is an extension of $K(\alpha_1)$, s.t. $f_1$
  splits over $L$. Hence $\exists \alpha_2, \alpha_3, \dots, \alpha_n \in L$ s.t.
  $ f_1(x) = \lambda (x - \alpha_2) \dots (x - \alpha_n)$,
  thus $f(x) = \lambda (x - \alpha_1) (x - \alpha_2) \dots (x - \alpha_n)$. Compare the
  coefficient of $x^n$ we know that $\lambda \in K$.

  More over, observe that $K(\alpha_1, \dots, \alpha_n)$ is the smallest field containing $K$ and
  $\{ \alpha_1, \dots, \alpha_n\}$. So if we choose $L = K(\alpha_1, \alpha_2, \dots, \alpha_n)$,
  then
  \[ [L: K] = [K(\alpha_1, \alpha_2, \dots, \alpha_n): K(\alpha_1, \alpha_2, \dots, \alpha_{n-1})]
    \dotsm
    [K(\alpha_1): K] \leq n! \]
  Since $[K(\alpha_1, \alpha_2, \dots, \alpha_k): K(\alpha_1, \alpha_2, \dots, \alpha_{k-1})]
  = [K(\alpha_1, \alpha_2, \dots \alpha_{k-1})(\alpha_k): K(\alpha_1, \alpha_2, \dots, \alpha_{k-1})]$
  and $\alpha_k$ is a root of $p(x) \in K(\alpha_1, \alpha_2, \dots, \alpha_{k-1})[x]$
  where $f(x) = (x - \alpha_1)(x - \alpha_2) \dotsm (x - \alpha_{k-1}) p(x)$.
\end{proof}

\begin{example}
  Find a splitting field $L$ for $x^8 - 2$ over $\Qb$ and determine $[L : \Qb]$.
\end{example}

\begin{remark}
  $\quot{\Qb[x]}{\gen{x^8 - 2}} = \Qb(\bar{x}) \cong \Qb(\sqrt[8]{2})
  \cong \Qb(\sqrt[8]{2} \zeta)$
\end{remark}

\begin{prop} \label{prop:after-homo-still-irr}
  Let $K, L$ be two fields and $\tau: K \to L$ be a nontrivial homomorphism.
  We define $\bar\tau : K[x] \to \tau(K)[x] \subseteq L[x]$ by
  \[ a_n x^n + \dots + a_0 \mapsto \bar\tau(f) \triangleq \tau(a_n)x^n + \dots + \tau(a_0) \]
  which is an isomorphism. Also, $f$ is irreducible implies $\bar\tau(f)$ is irreducible in $\tau(K)[x]$.
\end{prop}

\begin{lemma} \label{lemma:extension-exists-condition}
  Let $K(\alpha) / K$ be algebraic and $\tau: K \to L$ be a nontrivial homo,
  then there exists an extension $\sigma$ of $\tau$ from $K(\alpha)$ to $L$ if
  and only if $\exists \beta \in L$ s.t. $\bar\tau(m_{\alpha, K})(\beta) = 0$.

  In this case $m_{\beta, \tau(K)} = \bar\tau(m_{\alpha, K})$.

\begin{proof}
  ``$\Rightarrow$'': Let $\beta = \sigma(\alpha)$ and $m_{\alpha, K} = x^n + a_{n-1} x^{n-1} + \dots + a_0$.
  Then $\bar\tau(m_{\alpha, K})(\beta) = \beta^n + \tau(a_{n-1})\beta^{n-1} + \dots + \tau(a_0)
  = \tau(\alpha^n + a_{n-1} \alpha^{n-1} + \dots + a_0) = 0$

  ``$\Leftarrow$'': Observe that $m_{\beta, \tau(K)} = \bar\tau(m_{\alpha, K})$ since
  $\bar\tau(m_{\alpha, K})(\beta) = 0$ and $\bar\tau(m_{\alpha, K})$ is monic and irreducible
  by prop \ref{prop:after-homo-still-irr}. $\sigma$ is then given by the following diagram.
  \[
    \begin{tikzcd}
      & K[x] \arrow[r, "\sim", "\bar\tau"'] \arrow[d, two heads]
      & \tau(K)[x] \arrow[d, two heads] \\
      K(\alpha) \arrow[r, Leftrightarrow, "\cong"]
      & \raisebox{.1em}{$K[x]$} \Big/ \raisebox{-.1em}{$\gen{ m_{\alpha, K} }$}
      \arrow[r, "\sim", "\sigma"']
      & \raisebox{.1em}{$\tau(K)[x]$} \Big/ \raisebox{-.1em}{$\gen{ m_{\beta, \tau(K)} }$}
      \arrow[r, Leftrightarrow, "\cong"]
      & \tau(K)(\beta) \subseteq L
    \end{tikzcd}
  \]
\end{proof}
\end{lemma}

\begin{coro} \label{coro:num-of-extensions}
  Let $K(\alpha)/K$ be an algebraic extension and $\tau: K \hookrightarrow L$.
  If $\bar\tau(m_{\alpha, K})$ has $r$ distinct roots in $L$, then there are exactly $r$ extensions of $\tau$.
\end{coro}

\begin{theorem} \label{thm:two-splitting-field-are-isom}
  Let $\tau: K \to K'$ be an isomorphism of fields.
  If $L$ is a splitting field for $f$ over $K$ and $L'$ is a splitting field for $\bar\tau(f)$
  over $K'$, then $L \cong L'$

\begin{proof} \label{coro:extension-exists-splitting-field}
  By induction on $n = \deg f$. When $n = 1$, $L = K, L' = K'$, so $L \cong L'$.

  Now if $n > 1$, assume $f(\alpha) = 0$ for $\alpha \in L$. Then
  $\bar\tau(m_{\alpha, K}) \mid \bar\tau(f)$ and by the fact that $L'$ is a
  splitting field for $\bar\tau(f)$, $\exists \beta \in L'$ s.t. $\bar\tau(m_{\alpha, K})(\beta) = 0$.
  By lemma \ref{lemma:extension-exists-condition}, $\exists \tau_{\circ}:
  K(\alpha) \xrightarrow\sim K'(\beta)$ with $\tau_\circ \big|_K = \tau$.

  Now, write $f = (x - \alpha) f_\circ$, then $\bar\tau(f) = \bar\tau_\circ(f) = (x - \tau_\circ(\alpha))
  \bar\tau_\circ(f_\circ) = (x - \beta) \bar\tau_\circ(f_\circ)$. Then $L$ and $L'$ is a splitting
  field for $f_{\circ}$ over $K(\alpha)$ and $\bar\tau_\circ(f_\circ)$ over $K(\beta)$ respectively.
  By induction hypothesis, $L \cong L'$.
\end{proof}
\end{theorem}

\begin{coro}
  Let $\tau: K \to K'$ be an isomorphism of fields, and
  $L$ is a splitting field of $f$ over $K$, $L'$ is a splitting field of $\bar\tau(f)$ over $K'$.
  Then $\tau$ could be extend to $\sigma: L \to L'$ such that $\sigma\big|_K = \tau$.
\end{coro}

\begin{example}
  $L = \Qb(\sqrt{2}, \sqrt{3})$.
\end{example}

%! TEX root=../main.tex
\subsection{Week 2}

\begin{definition}
  A polynomial $f(x) \in K[x]$ is said to be \emph{separable}\index{seperable}
  if its irreducible factors have no multiple roots in a splitting field $L$.
\end{definition}

\begin{definition}
  If $f(x) = a_n x^n + \dots + a_1 x + a_0$, then define $f'(x) \triangleq n a_n x^{n-1} + \dots + 2a_2 x + a_1$.
\end{definition}

\begin{theorem} \label{thm:multiple-root-condition}
  Let $f(x) \in K[x]$ be monic, irreducible of positive degree, then all the roots of $f(x)$
  in a splitting field are simple if and only if $\gcd(f(x), f'(x)) = 1$.

  \begin{proof}
    ``$\Rightarrow$'': We can write $f(x) = (x - \alpha_1) (x - \alpha_2) \dotsm (x - \alpha_n)$ where
    $\alpha_i$ are distinct roots of $f$. Then $f'(x) = \sum_{i = 1}^n f(x) / (x - \alpha_i)$
    and we have $(x - \alpha_i) \nmid f(x)$ for all $i$.

    ``$\Leftarrow$'': Assume $f(x) = (x - \alpha)^k g(x)$ with $k \geq 2$.
    Then $f'(x) = k (x - \alpha)^{k-1} g(x) + (x - \alpha)^k g'(x)$ which implies $(x - \alpha) \mid f(x)$.
    So $(x - \alpha) \mid \gcd(f(x), f'(x))$ and thus $\gcd(f(x), f'(x)) \neq 1$.
  \end{proof}
\end{theorem}

\begin{remark}
  The following are equivalent:
  \begin{enumerate}
    \item $\alpha$ is a multiple root of $f(x)$.
    \item $\alpha$ is a common root of $f(x)$ and $f'(x)$.
    \item $m_{\alpha, K} \mid f(x)$ and $m_{\alpha, K} \mid f'(x)$.
  \end{enumerate}
\end{remark}

\begin{theorem} \label{thm:size-of-finite-field}
  There is a finite field $K$ with $\abs{K} = q \iff q = p^n$ for some prime $p$ and $n \in \Nb$.
  In this situation, $K$ is unique up to isomorphism, denote by $\Fb_{p^n}$.

  \begin{proof}
    ``$\Rightarrow$": Let $p = \Char K$ and $[K : \Zb/ p\Zb] = n$, then $\abs{K} = p^n$.

    ``$\Leftarrow$": Let $K$ be a splitting field for $f(x) = x^{p^n} - x$ over $\Fb_p$.
    We claim that the set of all roots of $f(x)$ forms a field. Since if $\alpha, \beta$ are
    two roots of $f$, obviously $\alpha \beta, \alpha \beta^{-1}$ are also roots,
    and by $(\alpha \pm \beta)^{p^n} = \alpha^{p^n} \pm \beta^{p^n} = \alpha \pm \beta$ because
    $\Char K = p$.
    $\alpha \pm \beta$ are also roots, hence the roots form a field. By definition, $K$
    is the smallest field containing $\Fb_p$ and roots of $f(x)$, so
    $K$ is exactly the set of roots of $f(x)$.

    Also, $f'(x) = -1$ has no root, so $f(x)$ has no multiple root which implies $\abs{K} = p^n$.

    Moreover, if $K'$ is another finite field with $\abs{K'} = p^n$, then for all $\alpha \in K'$,
    $\alpha^{p^n} = \alpha$, so $\alpha$ is a root of $f(x)$, which implies that $K'$ is
    a splitting field for $f(x)$ over $\Fb_p$. By theorem~\ref{thm:two-splitting-field-are-isom},
    $K \cong K'$.
  \end{proof}
\end{theorem}

\begin{theorem} \label{thm:aut-of-finite-field}
  Let $n \in \Nb$ and $\Fb_q$ be a finite field. Then there exists
  a unique extension $\Fb_{q^n} / \Fb_q$ s.t. $[\Fb_{q^n} : \Fb_q] = n$, and
  $\Aut \Big( \Fb_{q^n} / \Fb_q \Big) = \gen{\sigma_q}$ with
  $\sigma_q = \alpha :: \Fb_{q^n} \mapsto \alpha^q :: \Fb_{q^n}$.
  $\sigma_q$ is called the \emph{\Index{Frobenius homomorphism}}.

  \begin{proof}
    By theorem~\ref{thm:size-of-finite-field},
    $q = p^r$ for some prime $p$ and $r \in \Nb$, so $q^n = p^{nr}$ which is a power of
    a prime. Again by theorem~\ref{thm:size-of-finite-field},
    $\Fb_{q^n}$ is the splitting field for $x^{p^{nr}} - x$ over $\Fb_p$.
    Since $x^q - x \mid x^{q^n} - x$, $\Fb_q \subseteq \Fb_{q^n}$ and thus $[\Fb_{q^n}: \Fb_{q}] = n$.

    Then we proof that $\sigma_q$ is indeed in $\Aut \Big( \quot{\Fb_{q^n}}{\Fb_q} \Big)$.
    We check that
    \[
      \begin{aligned}
        \sigma_q(\alpha+\beta) &= (\alpha+\beta)^q = \alpha^q + \beta^q = \sigma_q(\alpha) + \sigma_q(\beta) \\
        \sigma_q(\alpha\beta) &= (\alpha\beta)^q = \alpha^q \beta^q = \sigma_q(\alpha) \sigma_q(\beta)
      \end{aligned}
    \]
    Now $\sigma_q$ is nontrivial since $\sigma_q$ send $1$ to $1$, so $\sigma_q$ is 1-1 and hence an
    isomorphism since $\Fb_q$ is finite. Also, for all $\alpha \in \Fb_q$, $\sigma_q(\alpha)
    = \alpha^{q} = \alpha$, hence $\sigma_q$ fixes $\Fb_q$.

    Finally we prove that the order of $\sigma_q$ is $n$. Assume not, so $\ord(\sigma_q) = m < n$.
    Then $\sigma_q^m = \Id \implies x^{q^m} - x=0$ for each $x \in \Fb_{q^n}$.
    But $x^{q^m} - x = 0$ has at most $q^m < q^n$ roots, which leads to a contradiction.
  \end{proof}
\end{theorem}

\begin{remark}
  By theorem~\ref{thm:finite-subgroup-of-field-is-cyclic}, the multiplication group
  of $\Fb_{q^n}$ is cyclic, so $\Fb_{q^n}^\times = \gen{ \alpha }
  \subseteq \Fb_q(\alpha) \setminus \{0\} \subseteq \Fb_{q^n} \setminus \{0\}$,
  hence $\Fb_{q^n} = \Fb_q(\alpha)$.
\end{remark}

\begin{lemma}
  Every irreducible polynomial $f(x)$ in $\Fb_{p^n}[x]$ is separable.

  \begin{proof}
    Without lost of generality, assume $f(x)$ is monic.

    Since $\sigma_p$ is an isomorphism, $\Fb_{p^n} = \Fb_{p^n}^p = \{ \alpha^p \mid \alpha \in \Fb_{p^n}\}$.
    Now assume $f(x)$ has a multiple root $\alpha$, then $m_{\alpha, \Fb_p} = f(x)$ since $f$
    is irreducible. By theorem~\ref{thm:multiple-root-condition} we also have
    $f(x) = m_{\alpha, \Fb_p} \mid f'(x)$, but $\deg f'(x) < \deg f(x)$ so we must have $f'(x) \equiv 0$.

    Write $f(x) = a_n x^n + \ldots + a_1 x + a_0$, then $f'(x) \equiv 0$ implies $k a_k = 0_{\Fb_p}$ for each $k$,
    which means that if $a_k \neq 0 \implies p \mid k$. So
    \[ f(x) = a_{mp} x^{mp} + a_{(m-1)p} x^{(m-1)p} + \dots + a_p x^p + a_0 =
    (a_{mp} x^m + \ldots + a_p x + a_0)^p. \]
    But this implies $f(x)$ is reducible, which is a contradiction.
  \end{proof}
\end{lemma}

\begin{theorem}
  $x^{p^n} - x$ equals the product of all monic irreducible polynomials in
  $\Fb_p[x]$ of degree $d$ where $d$ runs through all divisors of $n$. i.e.

  \begin{proof}
    By lemma, each irreducible polynomial is separable, and if $f(x), g(x) \in \text{RHS}$,
    and $f(\alpha) = g(\alpha) = 0$, then $f = m_{\alpha, \Fb_p} = g$. Thus RHS is separable.
    LHS is separable since $f' = 1$, so we could prove the equality by checking that
    they have same roots.

    $\text{LHS} \mid \text{RHS}$: $\forall \alpha \in \Fb_{p^n}$,
    $[\Fb_p(\alpha): \Fb_p] \mid [\Fb_{p^n}: \Fb_p] = n$, thus $\deg m_{\alpha, \Fb_p} \mid n$
    and hence $m_{\alpha, \Fb_p}$ appears in RHS.

    $\text{RHS} \mid \text{LHS}$: Assume $\deg m_{\alpha, \Fb_p} = d \mid n$, then
    $[\Fb_p(\alpha): \Fb_p] = d$, so $\alpha^{p^d} = \alpha$, and hence
    $\alpha = \alpha^{p^d} = \alpha^{p^{2d}} = \dots = \alpha^{p^n}$.
  \end{proof}
\end{theorem}

\begin{definition}
  M\"{o}bius $\mu$-function:
  Let $d = p_1^{k_1} p_2^{k_2} \dotsm p_n^{k_n}$, then
  \[
    \mu(d) = \begin{cases}
      1 & \text{if $n$ is even and all $k_i = 1$} \\
      -1 & \text{if $n$ is odd and all $k_i = 1$} \\
      0 & \text{otherwise}
    \end{cases}
  \]
\end{definition}

\begin{theorem}[M\"{o}bius inversion formula]
  If $f(n) = \sum\limits_{d \mid n} g(d)$, then
  $g(n) = \sum\limits_{d \mid n} \mu(d) f\left(\frac{n}{d}\right)$.
\end{theorem}

%! TEX root=../main.tex
\subsection{Algebra closure}

\begin{definition} \hfill
  \begin{itemize}
    \item $L$ is called an {\bf algebraic closure}\index{algebraic closure} of $K$ if $L/K$ is algebraic and
      each polynomial $f(x) \in K[x]$ splits over $L$.
    \item $L$ is said to be {\bf algebraically closed}\index{algebraically closed} if for each $f(x) \in L[x]$,
      $f(x)$ has a root in $L$.
  \end{itemize}
\end{definition}

\begin{prop}
  Given $L/K$, if $L$ is algebraically closed, then $L_a \triangleq
  \Set{\alpha \in L \given \alpha \text{ is algebraic over } K}$ is an
  algebraic closure of $K$.

  \begin{proof}
    By prop~\ref{prop:alg-elements-form-a-field}, $L_a$ is a field, and
    by definition, $L/K$ is algebraic.

    Now we proof that for any $K \subseteq L$ and $f(x) \in K[x]$, $f(x)$
    splits over $K$.  Using induction, $\deg f = 1$ is trivial.
    If $\deg f > 1$, then since $f(x) \in K[x] \subseteq L[x]$,
    $f$ has a root, say $\alpha$. so we could write $f(x) = (x - \alpha) g(x)$.
    Then $g(x) \in K(\alpha)[x] \subseteq L[x]$. by induction, $g(x)$ splits
    and hence $f(x)$ splits.

    So for any $f(x) \in K[x]$, $f$ splits. Write $f(x) = (x - \alpha_1) \dots (x - \alpha_n)$,
    then each $\alpha_i$ is algebraic over $K$, $\alpha_i \in L_a$ and
    hence there product $f(x)$ splits in $L_a[x]$.
  \end{proof}
\end{prop}

\begin{coro}
  If $K$ is algebraically closed, then $K$ is an algebraic closure of $K$ itself.
\end{coro}

\begin{prop}
  If $L$ is an algebraic closure of $K$, then $L$ is algebraically closed.

  \begin{proof}
    For $f(x) \in L[x]$, let $\alpha$ be a root of $f(x)$. Since $L(\alpha)/L$ and
    $L/K$ is algebraic, by prop~\ref{prop:alg-tower-implies-alg}, $L(\alpha)/K$ is algebraic.
    So $\alpha$ must be in $L$, hence $f(x)$ has a root in $L$.
  \end{proof}
\end{prop}

\begin{prop}
  The following are equivalent.
  \begin{enumerate}
    \item $K$ has no nontrivial algebraic extension.
    \item For all irreducible polynomial in $K[x]$ has degree $1$.
    \item Every polynomial of  positive degree in $K[x]$ has at least one root in $K$.
    \item Every polynomial of  positive degree in $K[x]$ splits over $K$.
  \end{enumerate}
\end{prop}

In below we would use the Zorn's lemma heavily.
\begin{lemma}[Zorn's lemma]
  Suppose a partially order set $P$ has the property that every chain (i.e., a total order subset)
  has an upper bound in $P$, then the set $P$ contains at least one maximal element.
\end{lemma}

\begin{lemma} \label{lemma:max-ideal-exists}
  In a commutative ring $R$ with $1$, any proper ideal $I \subsetneq R$ is contained in a maximal ideal.

  \begin{proof}
    Consider $S = \Set{J \subsetneq R \given I \subseteq J} \neq \varnothing$ since $I \in S$.
    Define a partial order on $S$ by $J_1 \preceq J_2 \iff J_1 \subseteq J_2$.

    Given a chain $\Set{J_i \given i \in \Lambda}$, let $J = \bigcup_{i \in \Lambda} J_i$. $J$ is an
    ideal, since if $x, y \in J$, then $x \in J_1, y \in J_2$.
    Let $\tilde{J} = \max(J_1, J_2)$, then $x, y \in \tilde{J}$
    which implies $x + y \in \tilde{J}$, and it is easy to check that for any $x \in R, y \in J$, $xy \in J$.

    Also, $J$ is proper since $1 \not\in J$, or else $1 \in J_i$ and thus $J_i = R$ which leads to
    an contradiction.

    By Zorn's lemma, $\exists$ a maximal element in $S$, and thus it is a maximal ideal which contains $I$.
  \end{proof}
\end{lemma}

\begin{theorem}
  If $K$ is a field, then $\exists$ an algebraic closure $L$ of $K$.

  \begin{proof}
    Let $S = \Set{x_f \given f(x) \in K[x] \text{ with } \deg f \geq 1}$ be the set of variables indexed by non-constant
    polynomial in $K[x]$. Consider the polynomial ring $K[S]$ and $I = \langle f(x_f) : f \in K[x] \text{ with } \deg f \geq 1 \rangle$,
    which is an ideal in $K[S]$.

    We claim that $I \neq K[S]$. If not then $1 \in I \implies 1 = \sum_{i = 1}^n g_i f_i(x_{f_i})$.
    Write $x_i \triangleq x_{f_i}$ for $i = 1, 2, \cdots, n$. Also, by definition $g_i$ only involves a finite number of
    variable in $S$, so we could set $g_i \in K[x_1, x_2, \cdots, x_m]$ with $m \geq n$. That is, $1 = \sum_{i = 1}^n g_i(x_1, x_2,
    \cdots, x_m) f_i(x_i)$. Let $\Sigma$ be a splitting field for $f(x) = f_1(x) f_2(x) \cdots f_n(x)$ and define $\alpha_i \in \Sigma$
    which satisfies $f_i(\alpha_i) = 0$ and $a_i = 0$ for $n+1 \leq i \leq m$. Then
    $1 = \sum_{i = 1}^n g(\alpha_1, \alpha_2, \cdots, \alpha_m) f_i(\alpha_i) = 0$ which leads to an contradiction.

    By lemma~\ref{lemma:max-ideal-exists}, $\exists$ a maximal ideal $M$ s.t. $I \subseteq M$.

    Consider $K \hookrightarrow F_1 \triangleq K[S] / M$, and then for all $f \in K[x]$, $f(\bar{x}_f) = \bar{0}$ in $F_1$.
    By induction, $\exists F_1 \subseteq F_2 \subseteq F_3 \subseteq \cdots$
    which satisfies $f(x) \in F_n[x]$ has a root in $F_{n+1}$
    Let $F = \bigcup_{i = 1}^\infty F_i$ which is algebraically closed since if $f(x) \in F[x]$ then $f(x) \in F_m[x]$
    for some $m$ and thus $f(x)$ has a root in $F_{m+1} \subseteq F$.

    Finally $L \triangleq \{ \alpha \in F \mid \alpha \text{ is algebraic over } K \}$ is an algebraic closure of $K$.
  \end{proof}

  \begin{lemma} \label{lemma:homo-extend-to-alg-closed-extension}
    If $L_1 / K$ is algebraic and $\tau: K \to L_2$ is a non-zero homomorphism with $L_2$ being algebraically closed,
    then $\tau$ could be extend to $\sigma: L_1 \to L_2$.

    \begin{proof}
      Consider $S = \{ (M, \theta) \mid K \subset M \subset L_1,\ \theta: M \to L_2 \text{ with } \theta\big|_K = \tau\}$,
      which is not an empty set since $(K, \tau) \in S$.

      Define a partial order on $S$ by $(M_1, \theta_1) \preceq (M_2, \theta_2)
      \iff M_1 \subseteq M_2 \,\land\, \theta_2 \big|_{M_1} = \theta_1$.
      Given any chain $\{(M_i, \theta_i) : i \in \Lambda \}$, let $N = \bigcup_{i = 1}^\infty M_i$ and
      $\theta = \alpha :: N \mapsto \theta_i(\alpha)$ if $\alpha \in M_i$. It could
      be check easily that this map is well defined, and $(N, \theta)$ is
      a least upper bound in $S$ for this chain.  By Zorn's lemma, $\exists$ a max element $(M, \sigma)$ in $S$.

      Now, if $M \neq L_1$, then pick $\alpha \in L_1 \setminus M$. Since $L_1/K$ is algebraic,
      the minimal polynomial $m_{\alpha, K}$ exists. Since $L_2$ algebraically closed, $\bar\sigma(m_{\alpha, K})$
      has a root in $L_2$, and thus by lemma~\ref{lemma:extension-exists-condition},
      $\sigma$ could be extend to $\sigma': M(\alpha) \to L_2$ which contradict the maximality of $(M, \sigma)$.
      Thus $M = L_1$.
    \end{proof}
  \end{lemma}

  \begin{theorem}
    Any two algebraic closures $L_1, L_2$ of $K$ are isomorphic.
    \begin{proof}
      Consider the inclusion map $\text{id}_K :: K \hookrightarrow L_1$.
      By Lemma~\ref{lemma:homo-extend-to-alg-closed-extension},
      $\text{id}_K$ could be extend to $\sigma :: L_2 \to L_1$ such that $\sigma\big|_K = \text{id}_K$.
      Since $\sigma \neq 0$, $\sigma(L_2) \cong L_2$.
      Also, $L_2$ is algebraically closed implies $\sigma(L_2)$ is algebraically closed.
      So for any $\alpha \in L_1$, $\alpha$ is algebraic over $K$ and thus over $\sigma(L_2)$,
      which implies $\alpha \in \sigma(L_2)$, so $\sigma$ is onto, hence $\sigma$ is an
      isomorphism between $L_1$ and $L_2$.
    \end{proof}
  \end{theorem}

  \begin{example}
    Let $p$ be a prime.
    \begin{itemize}
      \item Any finite field $L$ with $\Char L = p$, $L \cong \Fb_{p^n}$ for
        some $n \in \Nb$.
      \item $\Gal\left(\Fb_{p^n} / \Fb_p\right) = \gen{\sigma_p}$ with
        $p = \alpha :: \Fb_{p^n} \mapsto \alpha^p :: \Fb_{p^n}$.
      \item A subfield $L$ of $\Fb_{p^n}$ is isomorphic to $\Fb_{p^m}$ with
        $m \mid n$ since $[\Fb_{p^n} : \Fb_{p^m}] = d \leadsto p^{md} = p^n$.
      \item $\bigcup_{n=1}^\infty \Fb_{p^n}$ is a field, and it is the
        algebraic closure of $\Fb_p$.
    \end{itemize}
  \end{example}
\end{theorem}

\subsection{Separable extension}

\begin{definition} \hfill
  \begin{itemize}
    \item $\alpha$ is separable over $K$ if $m_{\alpha, K}$ is separable over $K$.
    \item $L/K$ is called a separable extension if $\forall \alpha \in L$, $\alpha$ is separable over $K$.
  \end{itemize}
\end{definition}

\begin{example}
  Let $\Char K = p$ and $K^p \subsetneq K$. Pick $b \in K \setminus K^p$ and consider $L$ to be
  the splitting field of $x^p - b$ over $K$, say $\alpha \in L$ with $\alpha^p = b$.
  Notice that $x^p - b = x^p - a^p = (x - a)^p$, and $x^p - b$ is irreducible in $K$, or else
  if $x^p - b = g(x) h(x)$ in $K[x]$, then write $g(x) = (x - \alpha)^k, h(x) = (x - \alpha)^{n-k}$,
  but then expand $g(x)$ and we would get $\alpha^k \in K$, since $\alpha^p \in K$ and
  $\gcd(k, p) = 1$ implies $\alpha \in K$ which leads to an contradiction.

  By above we know that $x^p - b$ is inseparable.
\end{example}

\begin{definition}
  $K$ is said to be \emph{\Index{perfect}} if either $\Char K = 0$ or ``$\Char K = p$ and $K = K^p$''.
\end{definition}

\begin{example}
  If $\Char K = p$ and $\quot{K}{\Fb_p}$ is algebraic, then $K$ is perfect.

  \begin{proof}
    Consider \deffunc{\sigma_p}{K}{K}{\alpha}{\alpha^p}, which is a monomorphism which fixes $\Fb_p$.
    Since $\quot{K}{\Fb_p}$ is algebraic, by the exercise problem, $\sigma_p$ is an automorphism, so $K = K^p$.
  \end{proof}
\end{example}

\begin{fact}
  $K$ is perfect if an only if for any irreducible polynomial $f(x) \in K[x]$, $f$ is separable.

  Also, we can find that any irreducible polynomial $f(x) \in K[x]$ is not separable over $K$
  if and only if $\Char K = p > 0$ and $f(x) = g(x^p)$ for some $g(x) \in K[x]$, where
  $g(x)$ is irreducible and not all coefficient of $g$ is in $K^p$.

  Finally, if $\Char K = 0$, then $K$ is separable.
\end{fact}

\begin{prop} \label{prop:separable-and-split-have-most-embeddings}
  Give $\quot{K(\alpha)}{K}$ with degree $m_{\alpha, K} = d$ and $\tau :: K \to L \neq 0$.
  If $\alpha$ is separable over $K$ and $\bar\tau(m_{\alpha, K})$ splits over $L$, then
  there are exactly $d$ monomorphisms $\sigma :: K(\alpha) \to L$ with $\sigma\big|_K = \tau$.

  Otherwise, if $\alpha$ is not separable or $\bar\tau(m_{\alpha, K})$ doesn't split over $L$,
  then there are $r < d$ such monomorphisms.

  \begin{proof}
    Observe that $m_{\alpha, K}$ is separable over $K$ if an only if $\bar\tau(m_{\alpha, K})$ is separable over $\tau(K)$.
    Extend $K$ to $\Sigma$, $\tau(K)$ to $\Sigma'$, where $\Sigma, \Sigma'$ are the splitting
    field of $m_{\alpha, K}$ and $\bar\tau(m_{\alpha, K})$, respectively.
    Since $K \cong \tau(K)$, by theorem~\ref{thm:two-splitting-field-are-isom}, $\Sigma \cong \Sigma'$.
    Let $\tau'$ be the isomorphism which is an extension of $\tau$.

    If $m_{\alpha, K} = (x - \alpha_1) (x - \alpha_2) \cdots (x - \alpha_d)$, then
    $\bar\tau(m_{\alpha, K}) = (x - \tau'(\alpha_1)) (x - \tau'(\alpha_2)) \cdots (x - \tau'(\alpha_n))$.
    where $\tau' :: \Sigma \isoto \Sigma'$ and $\alpha_i \neq \alpha_j \iff \tau'(\alpha_i) \neq \tau'(\alpha_j)$.
    Thus if $\alpha$ is separable, $\bar\tau(m_{\alpha, K})$ has $d$ distinct roots in $L$.
    By corollary~\ref{coro:num-of-extensions}, there are exactly $d$ monomorphisms $\sigma$ with $\sigma\big|_K = \tau$.

    Otherwise, there are $r$ roots in $L$ where $r < d$, and thus there are $r < d$ such monomorphisms.
  \end{proof}
\end{prop}

\begin{prop}\label{prop:separable-field-split-have-most-embeddings}
  Let $[K': K] = d$ and $\tau:: K \to L \neq 0$. Then $\quot{K'}{K}$ is separable and $\forall \alpha \in K'$,
  $\bar\tau(m_{\alpha, K})$ splits over $L$, if and only if there are exactly
  $d$ monomorphisms $\sigma::K' \to L$ with $\sigma\big|_k = \tau$.
  Otherwise $\exists r < d$ of such monomorphisms.

  \begin{proof}
    By induction on $d$, if $d = 1$ we could simply let $\sigma = \tau$.

    Now for $d > 1$, let $\alpha \in K' \setminus K$.
    By prop $1$, $\exists$ exactly $[K(\alpha): K]$ monomorphisms $\tau_1: K(\alpha) \to L$.

    Now, for any $\beta \in \quot{K'}{K(\alpha)}$,
    $m_{\beta, K(\alpha)} \mid m_{\beta, K}$ and thus $m_{\beta, K(\alpha)}$ is separable
    and $\bar\tau(m_{\beta, K(\alpha)})$ splits in $L$ since $\bar\tau(m_{\beta, K})$ splits.
    These implies that $\quot{K'}{K(\alpha)}$ is separable and $\forall \beta \in K(\alpha),
    m_{\beta, K(\alpha)}$ splits in $L$. Thus, $K(\alpha)$ satisfies the hypothesis,
    and by induction, there are exactly $[K': K(\alpha)]$ monomorphisms $\sigma :: K' \to L$
    such that $\sigma\big|_{K(\alpha)} = \tau_1$, thus there are $[K': K(\alpha)][K(\alpha): k]
    = [K': K]$ such monomorphisms.

    Otherwise, we could choose $\alpha \in K'$ such that $\bar\tau(m_{\alpha, K})$ has fewer
    then $[K(\alpha): K]$ roots in L, then there are $r' < [K(\alpha): K]$ monomorphism $\tau_1 :: K(\alpha)
    \to L$. By induction, each $\tau_1$ has $r''$ extensions $\sigma :: K' \to L$ and $r'' \leq [K': K(\alpha)]$
    Hence the number of monomorphism equals $r' r'' < [K': K]$.
  \end{proof}
\end{prop}

\begin{lemma} \label{lemma:element-in-alg-ext-splits}
  If $\quot{K(\alpha_1, \alpha_2, \dots, \alpha_n)}{K}$ is algebraic and $L$ is a splitting field of
  $f(x) = \prod_{i=1}^n m_{\alpha_i, K}$
  over $K$, then for all $\beta \in K(\alpha_1, \alpha_2, \dots, \alpha_n)$,
  $m_{\beta, K}$ also splits over $L$.

  \begin{proof}
    Let $L = K(R)$ with $R$ being the set of all roots of $f(x)$. Pick any root $\gamma$ of $m_{\beta, K}$.
    Observe the following diagram:
  \[
    \begin{tikzcd}
      K(R) \arrow[rr, "\sim", "\text{(2) } \sigma"'] & & K(R, \gamma) \\
      K(\beta) \arrow[u, hookrightarrow] \arrow[rr, "\sim", "\text{(1) } \tau"'] & & K(\gamma) \arrow[u, hookrightarrow] \\
      & K \arrow[ur, hookrightarrow] \arrow[ul, hookrightarrow] &
    \end{tikzcd}
  \]
  Where (1) holds because these field are both isomorphic to $\quot{K[x]}{\langle m_{\beta, K} \rangle}$. \\
  (2) holds because $\tau$ obviously fixes $K$, and hence $K(R)$ is a splitting field of $f$
  and $K(R, \gamma)$ is a splitting field of $\bar\tau(f)$. By theorem~\ref{thm:two-splitting-field-are-isom},
  $K(R)$ and $K(R, \gamma)$ is then isomorphic.

  Thus we have $[K(R): K] = [K(R): K(\beta)][K(\beta): K] = [K(R, \gamma): K(\gamma)][K(\gamma): K]
  = [K(R, \gamma): K]$, and $[K(R): K] = [K(R, \gamma): K] = [K(\gamma, R): K(R)][K(R): K]$
  which implies $[K(\gamma, R): K(R)] = 1$, hence $\gamma \in R$.

  \end{proof}
\end{lemma}

\begin{theorem} \label{thm:separable-elements-form-separable-field}
  Given $\quot{K(\alpha_1, \alpha_2, \dots, \alpha_n)}{K}$, if $\alpha_i$ is
  separable over $K_{i-1} \triangleq K(\alpha_1, \dots, \alpha_{i-1})$, then
  $\quot{K(\alpha_1, \alpha_2, \dots, \alpha_n)}{K}$ is separable.

  \begin{proof}
    Let $L$ be a splitting field of $f(x) = \prod m_{\alpha_i, K}$.

    We claim that there are $[K_j: K]$ monomorphisms $\tau_j:: K_j \to L$ with $\tau_j \big|_K = \text{id}_K$.
    Use induction on $j$, if $j = 0$, then there are only $1$ such monomorphism, namely itself $\text{id}_K$.

    For $j > 0$, observe that $m_{\alpha_j, K_{i-1}} \mid m_{\alpha_j, K}$, and since $\bar\tau_{j-1}
    (m_{\alpha_j, K}) = m_{\alpha_j, K}$ splits over $L$, $m_{\alpha_j, K_{i-1}}$ also splits over $L$.
    By hypothesis, $\alpha_j$ is separable over $K_{j-1}$, so by prop~\ref{prop:separable-and-split-have-most-embeddings},
    there are $[K_j: K_{j-1}]$ such monomorphisms $\tau_j:: K_j \to L$ with $\tau_j \big|_{K-1} = \tau_{j-1}$.
    By induction, there are $[K_{j-1}: K]$ monomorphisms $\tau_{j-1}:: K_{j-1} \to L$
    with $\tau_j \big|_K = \text{id}_K$. Compose these monomorphisms, we know that there
    exists exactly $[K_j: K_{j-1}][K_{j-1}: K] = [K_j: K]$ monomorphisms $\tau_j:: K_j \to L$ such that
    $\tau_j \big|_K = \text{id}_K$.

    So there are exactly $[K_n: K]$ monomorphisms $\tau :: K(\alpha_1, \dots, \alpha_n) \to L$
    with $\tau\big|_K = \text{id}_K$.
    By prop~\ref{prop:separable-field-split-have-most-embeddings}, $K(\alpha_1, \dots, \alpha_n)$ is separable.

  \end{proof}
\end{theorem}

\begin{theorem}
  $\quot{L}{K}$ is separable if and only if $\quot{L}{M},\, \quot{M}{K}$ are separable.

  \begin{proof}
    ``$\Rightarrow$'': If $L/K$ is separable, then $M/K$ is obviously separable. For any $\beta \in L$,
    $m_{\beta, M} \mid m_{\beta, K}$ so $m_{\beta, M}$ is separable which implies $L/M$ is separable.

    ``$\Leftarrow$'': For any $\alpha \in L$, write $m_{\alpha, M} = x^n + a_{n-1}x^{n-1} + \dots + a_1 x + a_0$,
    then $m_{\alpha, M}$ is separable implies that $\alpha$ is separable over $K(a_0, \dots, a_{n-1})$.
    By theorem~\ref{thm:separable-elements-form-separable-field},
    $K(a_0, a_1, \dots, a_{n-1}, \alpha) / K$ is separable, hence each $\alpha$ is separable over $K$,
    thus $L/K$ is separable.
  \end{proof}
\end{theorem}

%! TEX root=../main.tex
\subsection{Normal extension (week 4)}

\begin{definition}
  $\quot{L}{K}$ is called a {\bf normal extension}\index{Field extension!normal extension}
  if $\forall \alpha \in L$, $m_{\alpha, K}$ splits over $L$.
\end{definition}

\begin{theorem} \label{thm:splitting-field-iff-finite-normal}
  $L$ is a splitting field of some polynomial $f(x)$ over $K$ if
  and only if $L/K$ is finite and normal.

  \begin{proof}
    ``$\Rightarrow$'': Let $\alpha_1, \alpha_2, \dots, \alpha_n$ be the roots of $f$,
    so $L = K(\alpha_1, \alpha_2, \dots, \alpha_n)$, and $L$ is also a splitting field
    of $\prod m_{\alpha_i, K}$ since $m_{\alpha_i, K} \mid f$. By lemma~\ref{lemma:element-in-alg-ext-splits},
    for any $\beta$ in $L$, $m_{\beta, K}$ splits, thus $L/K$ is normal and also finite obviously.

    ``$\Leftarrow$'': Since $L/K$ is a finite extension, we could write
    $L = K(\alpha_1, \alpha_2, \dots, \alpha_n)$. Let $f = \prod m_{\alpha_i, K}$, then
    since $L/K$ normal, each $m_{\alpha_i, K}$ splits. It is also easy to see that $L$
    is the smallest field where $f$ splits, thus $L$ is a splitting field of $f$.
  \end{proof}
\end{theorem}

\begin{remark}
  If $L/K$ is normal, then for any $M$ with $K \subset M \subset L$, we have $L/M$ is normal,
  this is because $\forall \alpha, \, m_{\alpha, M} \mid m_{\alpha, K}$, and thus
  $m_{\alpha, M}$ splits since $m_{\alpha, K}$ splits.

  But $M/K$ need not to be normal. For example, Let $K = \Qb$, $L$ be the splitting field of $x^3 - 2$,
  by theorem~\ref{thm:splitting-field-iff-finite-normal} $L/K$ is normal.
  Then $L = \Qb(\sqrt[3]{2}, \omega)$ where $\omega \triangleq \mathrm{e}^{2 \pi \mathrm{i} / 3}$.
  Let $M = \Qb(\sqrt[3]{2})$ then $m_{\sqrt[3]{2}, K}$ doesn't split in $M$, so $M/K$ is not normal.
\end{remark}

\begin{prop} \label{prop:TFAE-of-normal-extension}
  Let $\quot{L}{K}$ be a finite, normal extension and $L \supset M \supset K$, then the following
  are equivalent.

  \begin{enumerate}[(\alph*)]
    \item $\quot{M}{K}$ is normal.
    \item $\forall \sigma \in \Aut(\quot{L}{K})$, $\sigma(M) \subset M$.
    \item $\forall \sigma \in \Aut(\quot{L}{K})$, $\sigma(M) = M$.
  \end{enumerate}

  \begin{proof}
    (a) $\Rightarrow$ (b): $\forall \alpha \in M$, $m_{\alpha, K}(\sigma(\alpha))
    = \sigma(m_{\alpha, K}(\alpha)) = 0$. So $\sigma(\alpha)$ is a root of $m_{\alpha, K}$.
    Since $\quot{M}{K}$ normal, $m_{\alpha, K}$ splits in $M$ and thus each root
    of $m_{\alpha, K}$ is in $M$, hence $\forall m, \, \sigma(m) \in M \implies \sigma(M) \subset M$.

    (b) $\Rightarrow$ (c): Since $L/K$ is algebraic and $\sigma$ is 1-1,
    by a homework problem, $\sigma$ onto.

    (c) $\Rightarrow$ (a): For any $\alpha \in M$, let $\beta \in L$ be a root of $m_{\alpha, K}$.
    By theorem~\ref{thm:splitting-field-iff-finite-normal}, we could assume $L$
    is a splitting field of $f$ over $K$. Consider the following diagram,
    \[
      \begin{tikzcd}
        L \arrow[rr, "\sim", "\sigma"'] & & L \\
        K(\alpha) \arrow[u, hookrightarrow] \arrow[rr, "\sim", "\tau"'] & & K(\beta) \arrow[u, hookrightarrow] \\
        & K \arrow[ur, hookrightarrow] \arrow[ul, hookrightarrow] &
      \end{tikzcd}
    \]
    Where isomorphism $\tau$ with $\tau(\alpha) = \beta$ exists
    since $\alpha, \beta$ share the same minimal polynomial,
    and $\sigma$ with $\sigma\big|_K = \tau$ exists by theorem~\ref{thm:two-splitting-field-are-isom}.
    Since $\sigma \in \Aut(\quot{L}{K})$, $\beta = \sigma(\alpha) \in M$, thus $M/K$ normal.
  \end{proof}
\end{prop}

\begin{definition}
  Let $L/K$ is called a \emph{Galois}\index{Extension!Galois extension} extension
  if $L/K$ is finite, normal and separable.
  That is, $L$ is a splitting field of some separable polynomial over $K$.
\end{definition}

\begin{theorem}
  If $L/K$ is Galois, then $\abs{\Aut(L/K)} = [L:K]$. Otherwise, $\abs{\Aut(L/K)} < [L:K]$.

  \begin{proof}
    Since $L/K$ is normal, for any $\alpha$, $m_{\alpha, K}$ splits over $L$.
    Since $L/K$ is separable, $m_{\alpha, K}$ has no multiple roots. So there are exactly $[L:K]$
    extensions $\sigma:: L \to L$ of $\Id_K$.
  \end{proof}
\end{theorem}

\begin{definition}
  Given a field $L$, define the {\bf fixed field}\index{fixed field} of $G$ by
  $L^G \triangleq \Set{ \alpha \in L \given \sigma(\alpha) = \alpha, \, \forall \sigma \in G }$.
\end{definition}

\begin{theorem} \label{thm:extension-of-fix-field-is-galois}
  If $G$ is a subgroup of $\Aut(L)$ with $\abs{G} < \infty$, then $\abs{G} = [L: L^G]$,
  $G = \Aut(L / L^G)$ and $L / L^G$ is Galois.

  \begin{proof}
    First we prove that $[L: L^G] \leq \abs{G}$ by contradiction.
    Assume $\abs{G} < [L: ^G]$. \\
    Let $G = \{\sigma_1, \sigma_2, \dots, \sigma_n\}$ and $\alpha_1, \alpha_2, \dots, \alpha_{n+1} \in L$
    with $\Set{\alpha_i}$ are linearly independent over $L^G$.

    Consider the equations
    \[
      \left\{
        \arraycolsep=2pt
        \begin{array}{ccc}
          \sigma_1(\alpha_1) x_1 + \dots + \sigma_1(\alpha_{n+1}) x_{n+1} &=& 0 \\
          \sigma_2(\alpha_1) x_1 + \dots + \sigma_2(\alpha_{n+1}) x_{n+1} &=& 0 \\
          \vdots & & \vdots \\
          \sigma_n(\alpha_1) x_1 + \dots + \sigma_n(\alpha_{n+1}) x_{n+1} &=& 0 \\
        \end{array}
      \right.
    \]
    Since the number of variables is more than the number of equations, there is a
    non-trivial solution. Choose one solution $(a_1, \dots, a_{n+1})$ having the
    least amount of nonzero element. By reordering, we could assume
    the solution is $(a_1, a_2, \dots, a_m, 0, 0, \dots, 0)$ and it is no harm to assume $\sigma_1 = 1_G$.
    If $m = 1$, then $\sigma_1(\alpha_1) a_1 = \alpha_1 a_1 = 0 \implies a_1 = 0$,
    which is a contradiction.

    So assume that $m > 1$, we have
    \[
      \left\{
        \arraycolsep=2pt
        \begin{array}{ccc}
          \sigma_1(\alpha_1) a_1 + \dots + \sigma_1(\alpha_{m}) a_{m} &=& 0 \\
          \sigma_2(\alpha_1) a_1 + \dots + \sigma_2(\alpha_{m}) a_{m} &=& 0 \\
          \vdots & & \vdots \\
          \sigma_n(\alpha_1) a_1 + \dots + \sigma_n(\alpha_{m}) a_{m} &=& 0 \\
        \end{array}
      \right.
    \]
    By multipling $a_m^{-1}$, we could assume $a_m = 1$. The equation about $\sigma_1$ gives
    $\alpha_1 a_1 + \dots + \alpha_m a_m = 0$, since $\alpha_i$ is linearly independent,
    one of $\Set{a_i}$, say $a_k$ is not in $L^G$, and thus there exists $t$
    such that $\sigma_t(a_k) \neq a_k$. Apply $\sigma_t$ to each equation, we have
    \[ \sigma_t\sigma_i(\alpha_1) \sigma_t(a_1) + \dots + \sigma_t \sigma_i(\alpha_m) \sigma_t(a_m) = 0,\quad
      \forall 1 \leq i \leq n \]

    But since $\{\sigma_t \sigma_1, \dots, \sigma_t \sigma_n\} = \{\sigma_1, \dots, \sigma_n\}$,
    $(\sigma_t(a_1), \sigma_t(a_2), \dots, \sigma_t(a_m), 0, \dots, 0)$ is a solution
    and thus $(a_1 - \sigma_t(a_1), \dots, a_m - \sigma_t(a_m), 0, \dots)$ is also a solution of the equations.
    Since $\sigma_t(a_k) \neq a_k$, the solution is not trivial, and because $a_m = 1$, $a_m - \sigma_t(a_m) = 0$.
    Hence this solution has $m-1$ nonzero element, which contradicts the minimality of the original solution.
    Thus $[L: L^G] \leq \Aut(L/L^G)$.

    Finally, $\abs{\Aut(L/L^G)} \leq [L: L^G]$ by theorem~\ref{thm:separable-elements-form-separable-field},
    thus $\abs{G} \leq \abs{\Aut(L/L^G)} \leq [L: L^G] \leq \abs{G}$, hence they are all equal.
  \end{proof}
\end{theorem}

\begin{definition}
  Let $f(x) \in K[x]$ and $L$ be a splitting field of $f(x)$ over $K$. We use
  $\Gal(L/K)$ to denote $\Aut(L/K)$ and call it the
  {\bf Galois group}\index{Galois group} of $f(x)$.
\end{definition}

\begin{prop}
  Let $f(x) \in \Qb[x]$ be irreducible polynomial of degree $p$ where $p$ is a prime.
  If $f(x)$ has exactly $p-2$ roots and $2$ complex roots, then the Galois group of $f(x)$ is $S_p$.

  \begin{proof}
    Let $L$ be a splitting field of $f$ over $\Qb$ and $R = \{ \alpha_1, \alpha_2, \dots, \alpha_p \}$ be
    the set of all roots of $f(x)$. Since $f(x)$ is irreducible, $f(x) / a_p = m_{\alpha_i, \Qb}, \, \forall i$.
    By lemma~\ref{lemma:extension-exists-condition}, for any $\sigma \in \Gal(L/\Qb)$, $\sigma$ sends
    $\alpha_i$ to another root $\alpha_j$. Also, $\{\alpha_i\}$ generates $L$ so
    $G \triangleq \Gal(L/\Qb) \leq S_p$.

    Now, we define an equivalence relation on $R$ such that $\alpha_i \sim \alpha_j \iff
    \cycle{\alpha_i, \alpha_j} \in G$, that is, $\exists \sigma \in G$ such that $\sigma(\alpha_i) = \alpha_j,
    \sigma(\alpha_j) = \alpha_i$ and $\sigma(\alpha_t) = \alpha_t, \, \forall t \neq i, j$.

    We claim that each equivalence class has the same size. Let $[\alpha_i], [\alpha_j]$ be
    two equivalence classes. Since $\alpha_i, \alpha_j$ share the same minimal polynomial,
    by lemma~\ref{lemma:extension-exists-condition}, $\exists \sigma,\, \sigma(\alpha_i) = \alpha_j$,
    and $\sigma$ sends $[\alpha_i]$ to $[\alpha_j]$, since if $\alpha_k \in [\alpha_i]$,
    $\cycle{\alpha_i, \alpha_k} \in G$ and thus $\sigma \cycle{\alpha_i, \alpha_k} \sigma^{-1}
    = \cycle{\alpha_j, \sigma(\alpha_k)} \in G$. Since $\sigma$ is 1-1,
    $\abs[\big ]{[\alpha_i]} \leq \abs[\big ]{[\alpha_j]}$,
    and by symmetry we have $\abs[\big ]{[\alpha_i]} = \abs[\big ]{[\alpha_j]}$.

    But then if $[\alpha_i] = n$, $p = \abs{R} = \sum \abs[\big ]{[\alpha_j]} = kn$,
    so either there are $p$ equivalence classes with size of $1$, which is impossible since
    the two complex root are equivalent by conjugation, or there are is one equivalence class,
    which means that every 2 cycle is in $G$, and thus $G = S_p$.
  \end{proof}
\end{prop}

\subsection{Fundamental theorem of Galois theory}
\begin{theorem}[Main theorem]
  Let $L/K$ be a Galois extension, where $L$ be a splitting field of a separable polynomial $f$,
  and let $G = \Gal(L/K)$. Then:

  \begin{enumerate}[(\arabic*)]
    \item There is a 1-1 correspondence from the set of intermediate field to the set of subgroup:
      \[
        \begin{array}{ccc}
          \{ M : K \subseteq M \subseteq L \} & \xleftrightarrow{\quad\quad} & \{ H : H \leq G \} \\
          M & \xmapsto{\quad\quad} & \Gal(L/M) \\
          L^H & \reflectbox{\ensuremath{\xmapsto{\quad\quad}}} & H
        \end{array}
      \]

      \begin{proof}
        We check these two mappings are the inverse of each other.

        By theorem~\ref{thm:extension-of-fix-field-is-galois}, $\Gal(L/L^H) = H$.

        Now we have $M \subseteq L^{\Gal(L/M)}$. Since $L/M$ is galois, $[L: M] = \abs{\Gal(L/M)}$.
        By theorem~\ref{thm:extension-of-fix-field-is-galois} again,
        $\abs{\Gal(L/M)} = [L: L^{\Gal(L/M)}]$, thus $[L: M] = [L: L^{\Gal(L/M)}] \implies M = L^{\Gal(L/M)}$.
      \end{proof}

    \item If $M_1 = L^{H_1}, M_2 = L^{H_2}$, then $M_1 \subseteq M_2 \iff H_2 \leq H_1$.
      \begin{proof}
        Obvious.
      \end{proof}
    \item If $M = L^H$, then $M/K$ is normal if and only if $H \lhd G$.
    \begin{proof}
      For any $\sigma \in G$,
      \begin{align*}
        \tau \in \Gal(L/\sigma(M)) &\iff \tau(\sigma(x)) = \sigma(x), \, \forall x \in M \\
        &\iff \sigma^{-1} \tau \sigma(x) = x, \, \forall x \in M \\
        &\iff \sigma^{-1} \tau \sigma \in \Gal(L/M) \\
        &\iff \tau \in \sigma \Gal(L/M) \sigma^{-1}
      \end{align*}
      By prop~\ref{prop:TFAE-of-normal-extension}, $M/K$ is normal if and only if for all $\sigma \in G$,
      $\sigma(M) = M \iff \Gal(L/M) = \Gal(L/\sigma(M))$.
      By the discussion above, $\Gal(L/\sigma(M)) = \sigma \Gal(L/M) \sigma^{-1} = \sigma H \sigma^{-1}$.
      Hence $M/K$ is normal $\iff H = \sigma H \sigma^{-1},\, \forall \sigma \in G \iff H \lhd G$.
    \end{proof}

    \item If $H \lhd G$, then $G / H \cong \Gal(M/K)$.
      \begin{proof}
        Since $H \lhd G$, by (3) we know that $M/K$ is Galois. Define $\varphi = \sigma :: \Gal(L/K)
        \mapsto \sigma\big|_M :: \Gal(M/K)$. The mapping is well defined
        since $\sigma(M) = M$ (by prop~\ref{prop:TFAE-of-normal-extension}).
        Also, this map is onto since by corollary~\ref{coro:extension-exists-splitting-field},
        each $\tau \in \Gal(M/K)$ could be extended to $\sigma \in \Gal(L/K)$ because
        $\bar\tau(f) = f$. Finally, notice that $\ker \varphi = H$, thus by the
        first isomorphism theorem, $G/H \cong \Gal(M/K)$.
      \end{proof}

    \item If $M_1 = L^{H_1}, M_2 = L^{H_2}$, then $M_1 \cap M_2 = L^{\gen{H_1, H_2}}$ and
      $M_1 M_2 = L^{H_1 \cap H_2}$.
  \end{enumerate}
\end{theorem}

\begin{theorem}
  Let $L/K$ be Galois, and $N/K$ be any extension, then $LN / N$ is galois and
  $\Gal(\quot{LN}{N}) \cong \Gal(\quot{L}{L \cap N})$ by the isomorphism
  $\varphi : \sigma \mapsto \sigma\big|_L$.

  \begin{proof}
    Let $L$ be a splitting field of the separable polynomial $f(x)$ over $K$,
    say $L = K(\alpha_1, \dots, \alpha_n)$. Then $LN = N(\alpha_1, \dots, \alpha_n)$,
    which can be regarded as a splitting field of $f(x)$ over $N$.
    Thus by theorem~\ref{thm:splitting-field-iff-finite-normal}, $LN/N$ is Galois.

    Now we check that $\varphi$ is well defined, notice that $f(\sigma(\alpha_i))
    = \sigma(f(\alpha_i)) = 0$ since $\sigma$ fixes $K$, and thus $f$ sends $\alpha_i$
    to some $\alpha_j$. Also, $\{ \alpha_i \}$ generate $L$ over $K$, thus $\sigma\big|_L(L) = L$.

    If $\sigma\big|_L = \Id_L$, then $\sigma(\alpha_i) = \alpha_i, \, \forall i$.
    Since $\{\alpha_i\}$ generate $LN$ over $N$, $\sigma = \Id_{LN}$. Thus $\varphi$ is 1-1.

    Finally, let $H = \Image \varphi$, we claim that $L^H = L \cap N$, since
      \begin{align*}
        \alpha \in L^H &\iff \alpha \in L \text{ and } \forall \sigma \in \Gal(LN/N),\, \sigma\big|_L(\alpha) = \alpha \\
        &\iff \alpha \in L \text{ and } \forall \sigma \in \Gal(LN/N),\, \sigma(\alpha) = \alpha \\
        &\iff \alpha \in L \text{ and } \alpha \in (LN)^{\Gal(LN/N)} \\
        &\iff \alpha \in L \text{ and } \alpha \in N \iff \alpha \in L \cap N
      \end{align*}
  \end{proof}
\end{theorem}

\begin{remark}
  If $L/K$ is Galois and $N/K$ is finite, then $[LN: K] = [L: K][N: K] / [L \cap N: K]$.
  \begin{proof}
    \[ [LN: K]/[N:K] = [LN: N] = \Gal(LN/N) = \Gal(L / L \cap N) = [L: L \cap N] = [L: K]/[L \cap N: K] \]
    and the proof is completed.
  \end{proof}
\end{remark}

%! TEX root=../main.tex
\subsection{Abelian extension}

\begin{definition}
  $L/K$ is called an abelian extension if $L/K$ is Galois and $\Gal(L/K)$ is abelian.
\end{definition}

\begin{example}
  For an extension $\Fb_{q^n} / \Fb_q$ of a finite field, $\Fb_{q^n}$ is a splitting field of $x^{q^n}-x$
  over $\Fb_p$, so $\Fb_{q^n} / \Fb_q$ is Galois by theorem~\ref{thm:splitting-field-iff-finite-normal}.
  By theorem~\ref{thm:aut-of-finite-field}, we know that $\Gal(F_{q^n} / F_q) = \langle \sigma_q \rangle$
  is a cyclic group.
\end{example}

\begin{definition} \hfill
  \begin{itemize}
    \item The cyclotomic field $\Qb(\zeta_n)$ is the splitting field of $x^n - 1$ over $\Qb$.
    \item $\zeta$ is called an nth root of unity if $\zeta^n = 1$. $\mathcal{U} = \langle \zeta \rangle$
      is the multiplicative group of nth roots of unity.
    \item $\zeta_n$ is called a primitive nth root of unity if $\zeta^n = 1$ but $\zeta^m \ne 1, \, \forall 0 < m < n$.
    \item The nth cyclotomic polynomial is defined as
      \[ \Phi_n \triangleq \prod_{\gcd(k, n) = 1} (x - \zeta_n^k) \implies \deg \Phi_n = \varphi(n) \]
  \end{itemize}
\end{definition}

\begin{prop} \hfill
  \begin{itemize}
    \item $x^n - 1 = \prod_{d \mid n} \Phi_d$.
      \begin{proof}
        First, Both sides have no multiple root. Then since $\alpha^n = 1 \iff \ord_{\times}(\alpha) \mid n$,
        we know that two sides has equal roots.
      \end{proof}
    \item $\Phi_n \in \Zb[x]$.
      \begin{proof}
        By induction on $n$. $n = 1$ is trivial.
        Assume that the statement is true for all $k < n$, then since
        \[ x^n - 1 = \Phi_n \prod_{\substack{d \mid n, d < n}} \Phi_d \triangleq \Phi_n \Phi_{< n} \]
        But notice that $\Phi_{<n}$ is monic, so by the long division algorithm, it is easy to
        see that $\Phi_n = (x^n - 1) / \Phi_{<n}$ has all coefficient in $\Zb$.
      \end{proof}
    \item $\Phi_n$ is irreducible.
      \begin{proof}
        Suppose $\Phi_n = f(x) g(x)$ with $f$ irreducible, and both $f, g$ are monic.
        By Gauss lemma, we could assume $f(x), g(x) \in \Zb[x]$.
        Let $\zeta_n$ be a primimtive nth root of unity  which satisfied $f(\zeta_n) = 0$
        and $p$ be a prime with $p \nmid n$.

        Assume that $g(\zeta_n^p) = 0$, $m_{\zeta_n, \Qb} = f \implies f \mid g(x^p)$,
        say $g(x^p) = f(x) h(x)$.
        By the long division algorithm, we know that $h(x) \in \Zb[x]$, since $f(x) \in \Zb[x]$
        and monic.

        In $\Zb / p\Zb[x]$, we have $\bar{g}(x)^p = \bar{g}(x^p) = \bar{f}(x) \bar{h}(x)$,
        which implies $\bar{g}, \bar{f}$ has common root, thus $\bar\Phi_n = \bar{f}\bar{g}$ and
        hence $x^n - \bar{1}$ has a multiple root.
        But $(x^n - \bar{1})' = nx^{n-1} \neq 0$, and $0$ is not a root of $x^n - \bar{1}$,
        which leads to an contradiction.

        So we conclude that $f(\zeta_n^p) = 0$ for any $p \mid n$, which could be extend
        and show that $f(\zeta_n^k) = 0$ for any $\gcd(k, n) = 1$, thus $f = \Phi_n$.
      \end{proof}
    \item $\Qb(\zeta_n) / \Qb$ is Galois with $[\Qb(\zeta_n): \Qb] = \deg \Phi_n = \varphi(n)$.
    \item $\Gal(\Qb(\zeta_n) / \Qb) \cong \Fb_n^\times$.

      \begin{proof}
        Let $\sigma_k = (\zeta_n \mapsto \zeta_n^k) \in \Gal(\Qb(\zeta_n) / \Qb)$.
        The isomorphism is given by $\sigma_k \mapsto \bar{k}$.
        Clearly, it is a homomorphism since $\sigma_k \sigma_h = (\zeta_n \mapsto \zeta_n^{kh}) = \sigma_{kh}$.
        Also $\sigma_k = 1 \iff \bar{k} = 1$. Finally, $\abs{\Gal(\Qb(\zeta_n) / \Qb)} = \abs{\Fb_n^\times} = \varphi(n)$,
        so the map is onto.
      \end{proof}
    \item Suppose $n = p_1^{n_1} p_2^{n_2} \cdots p_k^{n_k}$ with $p_1, \dots, p_k$ be distinct primes.
      Define $L_i \triangleq \Qb(\zeta_{p_i^{n_i}})$. Obviously, $L_i \subseteq \Qb(\zeta_n)$ hence
      $L_1 L_2 \dotsm L_k \subseteq \Qb(\zeta_n)$, but $\zeta_n = \zeta_{p_1^{n_1}} \zeta_{p_2^{n_2}}
      \dotsm \zeta_{p_k^{n_k}}$, so $L_1 L_2 \dotsm L_k \supseteq \Qb(\zeta_n)$. Thus we have
      $L_1 L_2 \dotsm L_k = \Qb(\zeta_n)$.
  \end{itemize}
\end{prop}

\begin{example}
  Let $n = p$ be a prime.
  \begin{itemize}
    \item $\Gal(\Qb(\zeta_p) / \Qb) = \Fb_p^\times = \Zb / (p-1)\Zb$.
    \item For $H \leq \Gal(\Qb(\zeta_p) / \Qb)$, we shall find $\Qb(\zeta_p)^H$.
      Let $\alpha = \sum_{\tau \in H} \tau(\zeta_p)$, then it is easy to
      see that $\alpha \in \Qb(\zeta_p)^H$. Also, since $[\Qb(\zeta_p): \Qb] = p-1$,
      $\zeta_p, \zeta_p^2, \dots, \zeta_p^{p-1}$ is linear independent,
      so if some $\sigma \in G$ satisfy $\sigma(\alpha) = \alpha$, then since
      both $\sigma(\alpha), \alpha$ are a sum of linear independent elements,
      $\sigma$ must send $\zeta_p$ to an element $\tau(\zeta_p)$ for some $\tau \in H$,
      then $\sigma = \tau \implies \sigma \in H$. Thus $\Qb(\zeta_p)^H = \Qb(\alpha)$.

  \end{itemize}
\end{example}
\begin{example}
\end{example}

\begin{lemma}
  If $L_1 / K$, $L_2 / K$ are Galois, then $L_1 \cap L_2 / K$, $L_1 L_2 / K$ are Galois and
  \[ \Gal( L_1 L_2 / K) \cong \big\{ (\sigma, \tau) \bigm| \sigma\big|_{L_1 \cap L_2} = \tau\big|_{L_1 \cap L_2} \big\}
    \leq \Gal(L_1 / K) \times \Gal(L_2 / K) \]
  In particular, if $L_1 \cap L_2 = K$, then $\Gal(L_1 L_2 / K) \cong \Gal(L_1 / K) \times \Gal(L_2 / K)$.

  \begin{proof}
    We know that $L_1 \cap L_2 / K$ is finite and separable. Also, for each $\alpha \in L_1 \cap L_2$,
    $m_{\alpha, K}$ splits in both $L_1, L_2$ since they are normal, thus $m_{\alpha, K}$ splits
    in $L_1 \cap L_2$, hence $L_1 \cap L_2 / K$ is galois.

    Similary, $L_1 L_2$ is finite and separable. Let $L_1$ be the splitting field of $f_1$,
    and $L_2$ be the splitting field of $f_2$, then $L_1 L_2$ is the splitting field
    of the square-free part of $f_1 f_2$, hence $L_1 L_2 / K$ normal.

    Define $\varphi = \sigma :: \Gal(L_1 L_2 / K) \mapsto \big( \sigma \big|_{L_1}, \sigma \big|_{L_2} \big)
    :: \Gal(L_1 / K) \times \Gal(L_2 / K)$. Since $L_1, L_2$ are normal,
    by proposition~\ref{prop:TFAE-of-normal-extension}, $\sigma \big|_{L_i}(L_i) = L_i$ so they are well-defined.
    Also, it is clear that the map is 1-1.

    Now we count the number of the pair $(\sigma \big|_{L_1}, \sigma \big|_{L_2})$,
    There are $[L_1: K]$ of $\tau = \sigma \big|_{L_1}$, and fixing one, each $\sigma \big|_{L_2}$
    is an extension of $\tau \big|_{L_1 \cap L_2}$, so there are $[L_1 L_2 : L_1]$ of such.
    On the other hand we have $\abs{\Gal(L_1 L_2 / K)} = [L_1 L_2 : K] = [L_1 L_2 : L_1]
    [L_1 : K] = [L_1 : L_1 \cap L_2] [L_1 : K]$, thus $\Gal(L_1 L_2 / K)$ and
    $\Set{(\sigma \big|_{L_1}, \sigma \big|_{L_2})}$ has the same size, and hence the
    map is onto.
  \end{proof}
\end{lemma}

Back to our problem, $[L_1 L_2 \cdots L_k: \Qb] = [\Qb(\zeta_n): \Qb] = \varphi(n) = \varphi(p_1^{n_1})
\cdots \varphi(p_k^{n_k}) = [L_1 : \Qb] [L_2 : \Qb] \cdots [L_k : \Qb]$, thus
\[ \Gal\big( \Qb(\zeta_n) / \Qb \big) \cong
  \Gal\big( \Qb(\zeta_{p_1^{n_1}}) / \Qb \big) \times
  \Gal\big( \Qb(\zeta_{p_1^{n_2}}) / \Qb \big) \times \cdots \times
  \Gal\big( \Qb(\zeta_{p_1^{n_k}}) / \Qb \big) \]

\begin{theorem}
  Let $G$ be a finite abelian group. Then exists a subfield $L$ of a cyclotomic field satisfied $\Gal(L / \Qb) \cong G$.

  \begin{proof}
    By the FTFGAG,
    \[ G \cong \Zb / n_1 \Zb \times \Zb / n_2 \Zb \times \dots \times \Zb / n_k \Zb \]
    By dirichlet theorem, there are infinitely many prime $p$ such that $n \mid p - 1$.
    Let $p_i$ be a prime such that $n_i \mid p_i - 1$ and $p_i$ are all distinct.
    Then $G$ is a subgroup of $\Zb / (p_1 - 1) \Zb \times \dots \times \Zb / (p_k - 1) \Zb \cong \Gal(\Qb(\zeta_n) / \Qb)$
    where $n = p_1 p_2 \dotsm p_k$.
  \end{proof}
\end{theorem}

In this section, we assume that $\Char K \nmid n$ and $\eta$ is a primitive $n$th root of unity.

\begin{definition} \hfill
  \begin{itemize}
    \item $L/K$ is called a kummer extension of exponent $n$ if $\zeta \in K$ and $L$ is a splitting field
      of $(x^n - a_1) (x^n - a_2) \dots (x^n - a_k)$ over $K$.
    \item Let $\abs{G} < \infty$, the exponent $e(G)$ of $G$ is the least positive integer $m$
      satisfied $g^m = 1$ for any $g \in G$.
  \end{itemize}
\end{definition}

\begin{theorem}
  Let $L$ be a splitting field of $x^n - a$ over $K$, then $\Gal(L / K(\zeta))$ is cyclic of
  degree dividing $n$. More over $x^n - a$ is irreducible over $K(\zeta)$ $\iff$ $[L: K(\zeta)] = n$.

  \begin{proof}
    If $\alpha$ is a root of $x^n - a = 0$, then $\alpha, \alpha \zeta, \dots, \alpha \zeta^{n-1}$
    are roots of $x^n - a$, so $L = K(\alpha, \zeta) = K(\zeta)(\alpha)$.

    Consider $\varphi = (\alpha \mapsto \alpha \zeta^k) :: \Gal(L / K(\zeta)) \mapsto \bar{k} :: \Zb / n\Zb$.
    Then it is easy to see that $\varphi$ is a homomorphism. Also, if $\varphi(\sigma) = 0$,
    $\sigma = (\alpha \mapsto \alpha) = \mathrm{id}$. Thus $\varphi$ 1-1 and
    $\Gal(L / K(\zeta)) \hookrightarrow \Zb/n\Zb$.
  \end{proof}
\end{theorem}

\begin{definition}
  $L/K$ is called a cyclic extension if $L/K$ is Galois and $\Gal(L/K)$ is cyclic.
\end{definition}

\begin{theorem} \label{thm:kummer-base-theorem}
  If $L/K$ is a cyclic extension of degree $n$ and $\zeta \in K$, then $L$ is a splitting field of
  some irreducible polynomial $x^n - a$ over $K$.

  \begin{proof}
    Recall a result proved in HW problem: Distinct automorphisms of $L$ are linearly independent over $L$.
    
    Let $\Gal(L/K) = \langle \sigma \rangle$ with $\ord(\sigma) = n$. Then
    \[ \text{id}_L + \zeta \sigma + \zeta^2 \sigma^2 + \dots + \zeta^{n-1} \sigma^{n-1} \neq 0
      \implies \exists c \in L, \text{ s.t. } \alpha = c + \zeta \sigma(c) + \zeta^2 \sigma^2(c)
      + \dots + \zeta^{n-1} \sigma^{n-1}(c) \neq 0 \]

    Observe that $\sigma(\alpha) = \zeta^{-1} \alpha$, so $\alpha \not\in K$. Also $\sigma(\alpha^n)
    = \sigma(\alpha)^n = \zeta^{-n}\alpha^n = \alpha^n$, so $\alpha^n$ is fixed by all automorphisms
    in $\Gal(L/K)$, thus $\alpha_n \in K$, and hence $K(\alpha)$ is a splitting field of $x^n - a$ over $K$.

    Now $\sigma(\alpha) = \zeta^{-1}\alpha \in K(\alpha)$, so $\sigma\big|_{K(\alpha)} \in \Gal(K(\alpha) / K)$.
    Also $\sigma^k(\alpha) = \eta^{-k} \alpha \implies \ord(\sigma) = n$.
    Thus
    \[ n = [L: K] \geq [K(\alpha): K] = \Gal(K(\alpha) / K) \geq n \implies L = K(\alpha) \]
  \end{proof}
\end{theorem}

\begin{theorem}
  If $L/K$ is Galois which satisfied $\Gal(L/K)$ is abelian of exponent $n$ and $\zeta_n \in K$,
  then $L/K$ is a Kummer extension.

  \begin{proof}
    By induction on $[L: K]$. If $[L: K] = 1$ then $L = K$ and is trivial.

    Assume $[L: K] > 1$, then by FTFGAG, $G \triangleq \Gal(L/K) \cong \quot{\Zb}{d_1 \Zb}
    \times \quot{\Zb}{d_2 \Zb} \times \dots \times \quot{\Zb}{d_s \Zb}$ with $d_i \mid d_{i+1}$.
    If $s = 1$ then the theorem degenerates to theorem~\ref{thm:kummer-base-theorem}.

    So assume $s > 1$. Let $H = \Zb / d_1 \Zb \times \dots \times \Zb / d_{s-1} \Zb,
    N = \Zb / d_s \Zb$ be the corresponding subgroup in $\Gal(L/K)$.
    Set $M = L^N$, we have $[M: K] \lneq [L: K]$. Since any subgroup
    of abelian group is normal, we have $\Gal(M/K) \cong \Gal(L/K) / \Gal(L/M) = G / N = H$.

    Denote $m = d_{s-1}, n = d_{s}$, we have $m \mid n$. Then $\zeta_n \in K \implies \zeta_m = \zeta_n^{n/m} \in K$,
    thus we could pass down the induction, and assume $M$ is a kummer extension which is a splitting
    field of $g = (x^m - b_1) (x^m - b_2) \dotsm (x^m - b_{k-1})$ over $K$ with each $b_i \in K$.
    Let $\alpha_1, \dots, \alpha_{k-1}$ be all the roots of $g$, then $\alpha_i$
    is also a root of $(x^n - b_1^{n/m})$. Thus if we define $a_i \triangleq b_i^{n/m}$, then
    $M$ is also the splitting field of $(x^n - a_1) (x^n - a_2) \dotsm (x^n - a_{k-1})$ over $K$
    since $\zeta_n \in K$.

    Now, if $N = \langle \sigma \rangle$, then $G \cong H \times N = \{\sigma^k \tau : 0 \leq k < n, \tau \in H\}$.
    Since automorphisms are linear independent, exists $c \in L$ satisfied
    \[ 0 \neq \alpha = \sum_{\tau \in H} \tau(c) + \zeta \sum_{\tau \in H} \sigma \tau(c)
    + \dots + \zeta^{n-1} \sum_{\tau \in H} \sigma^{n-1} \tau(c) \]
  Then $\sigma(\alpha) = \zeta^{-1} \alpha$, so $\alpha \not\in M$. Also $\sigma(\alpha^n) = \alpha^n$
  and $\tau(\alpha^n) = \tau(\alpha)^n = \alpha^n$, so $a_k \alpha^n \in K$.
  Thus $M(\alpha)$ is a splitting field of $(x^n - a_k)$ over $M$.

  Finally, $n = [L: M] \geq [M(\alpha): M] = \abs{\Gal(M(\alpha) / M)} \geq n$,
  thus $L = M(\alpha)$, and hence $L$ is a splitting field of
  $(x^n - a_1) (x^n - a_2) \dotsm (x^n - a_k)$.
  \end{proof}
\end{theorem}

\begin{theorem}
  If $L/K$ is a kummer extension of exponent $n$, then $\Gal(L/K)$ is abelian of exponent dividing $n$.

  \begin{proof}
    Let $L$ be the splitting field of $(x^n - a_1)(x^n - a_2) \dotsm (x^n - a_k)$ with $\alpha_i = \sqrt[n]{a_i}$.
    If $\sigma \in \Gal(L/k)$, then $\sigma$ sends each $\alpha_i$ to some $\zeta^{k_{\sigma, i}} \alpha_i$.
    So $\sigma^n = \alpha_i \mapsto \zeta^{k_{\sigma, i} n} \alpha_i = \alpha_i \mapsto \alpha_i = \mathrm{id}$
    and $\sigma \circ \tau = \alpha_i \mapsto \zeta^{k_{\sigma, i} k_{\tau, i}} \alpha_i = \tau \circ \sigma$.
    by the fact that  $\Set{\alpha_i}$ is the generating set of $L$. Hence $\Gal(L/K)$ is abelian and of
    exponent dividing $n$.
  \end{proof}
\end{theorem}

\subsubsection{Cubic equations}

\begin{lemma}
  Let $\Char K \neq 2$ and $f(x) \in K[x]$ with $\deg f = n$. Let $L = K(\alpha_1, \dots, \alpha_n)$ be
  a splittig field of $L$ over $K$.

  Define $\delta = \prod_{1 \leq i < j \leq n} (\alpha_i - \alpha_j)$, then $L^{\Gal(L/K) \cap A_n} = K(\delta)$
  Here $\Gal(L/K) \hookrightarrow S_n$.

  \begin{proof}
    Notice that any transposition maps $\delta$ to $-\delta$, so $\forall \sigma \in \Gal(L/K) \cap A_n$,
    $\sigma(\delta) = \delta$, thus $K(\delta) \subseteq L^{\Gal(L/K) \cap A_n}$.

    Now, $\big| \quot{\Gal(L/K)}{\Gal(L/K) \cap A_n} \big|$ is either $1$ or $2$.
    If it is $1$, then $\Gal(L/K) \leq A_n$ ,thus $\delta \not\in K$ and is trivial.
    So assume it is $2$, then $\delta$ is not fixed by all permutation, thus $\delta \in K$.
    We have $2 = \big| \Gal \big( L^{\Gal(L/K) \cap A_n}/K \big) \big| = [L^{\Gal(L/K) \cap A_n}: K] = 2
    = [K(\delta): K]$, thus $K(\delta) = L^{\Gal(L/K) \cap A_n}$.
  \end{proof}
\end{lemma}

\begin{prop}
  Let $f(x) = x^3 + px + q$ be irreducible in $K[x]$ and $L$ be a splitting field,
  \begin{itemize}
    \item If $\Gal(L/K) \cong S_3$ then $\sqrt{D} \not\in K$.
    \item If $\Gal(L/K) \cong A_3$ then $\sqrt{D} \in K$.
  \end{itemize}
\end{prop}

\begin{definition}
  $H \leq S_n$ is said to be transitive if for any $i, j$, exists $\sigma \in H$ such that $\sigma(i) = j$.
\end{definition}

\begin{fact}
  Let $f(x)$ be a separable polynomial with degree $n$, then
  \[ f(x) \text{ is irreducible} \iff \text{ The Galois group of } f \text{ is transitive in } S_n \]
\end{fact}

%! TEX root=../main.tex
\subsection{Solution by radicals}

\begin{definition} \mbox{}
  \begin{enumerate}
    \item Given $L/K$ and $\alpha \in L$, $\alpha$ is called a radical over $K$
      if $\alpha^n \in K$ for some $n\in \Nb$.
    \item $L/K$ is called an extension by radicals if there exist
      $L = L_n \supset L_{n-1} \supset \dots \supset L_1 \supset L_0 = K$
      s.t. $\forall i = 1,\dots, n, \quad L_i = L_{i-1}(\alpha_i)$ with
      $\alpha_i$ a radical over $L_{i-1}$.
    \item $f(x) \in K[x]$ is solvable by radicals if there exists $L/K$,
      an extension by radicals, s.t. $f$ splits over $L$.
  \end{enumerate}
\end{definition}

\begin{definition}
  (Recall) Let $G$ be a finite group. $G$ is solvable if
  $\exists\: \{1\} = G_m \lhd G_{m-1} \lhd \dots \lhd G_0 = G$ s.t.
  $\quot{G_{i-1}}{G_i}$ is cyclic $\forall i$.
\end{definition}


\begin{lemma} \label{lemma:galois-radical-ext}
  Given a Galois extension $L/K$ and $M = L(\alpha)$ is an extension by
  a radical, where $\alpha^n = a \in L$. Assume that $\Char K \nmid n$. Then
  $\exists\: N$ s.t. $N/M$ is an extension by radicals and $N/K$ is Galois
  and $N$ contains $\zeta_n$.
  \begin{proof}
    We know that $M(\zeta_n) = L(\zeta_n, \alpha)$ is a splitting field of
    $x^n - a$ over $L$. If we set
    \[ f(x) = \prod_{\sigma \in \Gal(L/K)} (x^n - \sigma(a)), \]
    then the coefficients of $f(x)$ are elementary symmetric polynomials in
    $\Set{\sigma(a) \given \sigma \in \Gal(L/K) }$, which are fixed by
    $\Gal(L/K)$, so $f(x) \in K[x]$.

    Let $L$ be a splitting field of $g(x)$ over $K$. (since $L/K$ is Galois)
    Choose $N$ as a splitting field of $f(x)g(x)$ over $K$.
    By def., $N/K$ is Galois. Let $L = K(\beta_1,\dots,\beta_s)$ where
    $\beta_1, \dots, \beta_s$ are the roots of $g(x)$, then
    \[
      N = K\big(\beta_1,\dots,\beta_s, \zeta_n,
        \alpha_\sigma : \sigma \in \Gal(L/K)
      \big),
      \qquad \alpha_\sigma^n = \sigma(a) \in L
    \]
    So $N = M(\zeta_n, \alpha_\sigma : \sigma \in \Gal(L/K) \setminus \{ \Id \})$
    $\implies$ $N/M$ is an extension by radicals.
  \end{proof}
\end{lemma}

\begin{lemma} \label{lemma:radical-ext-chain-galois}
  Let $L = L_m \supset L_{m-1} \supset \dots \supset L_0 = K$ s.t.
  $L_i = L_{i-1}(\alpha_i)$ with $\alpha^{n_i} = a_i \in L_{i-1}$.
  If $\Char K \nmid n_1n_2\cdots n_m$, then there exists $N/L$ s.t.
  $N/K$ is a Galois extension by radicals and $\zeta_{n_i} \in N, \,
  \forall i = 1, \dots, m$.

  \begin{proof}
    By induction on $m$. For $m = 1$, $L_1 \supset L_0 = K$ and
    $L_1 = L_0(\alpha_1) = K(\alpha_1)$ where $\alpha_1^{n_1} \in K$ for some
    $n_1\in \Nb$. Set $N = L(\zeta_{n_1}) = K(\zeta_{n_1}, \alpha_1)$, done.

    For $m > 1$, by induction hypothesis, $\exists\: N'/L_{m-1}$ s.t.
    $N'/K$ is Galois extension by radicals and $N'$ contains
    $\zeta_{n_i},\, \forall i = 1, \dots, m-1$.
    By lemma \ref{lemma:galois-radical-ext}, $\exists\: N/N'(\alpha_m)$ is
    an extension by radicals s.t. $N/K$ is Galois and $N$ contains $\zeta_{n_m}$.
  \end{proof}
\end{lemma}

\begin{prop} \label{prop:quot-solvable}
  Let $H \lhd G$. Then $G$ is solvable $\iff$ $H, \quot{G}{H}$ are solvable.

  \begin{proof}
    ``$\Leftarrow$'': Let $q: G \to \quot{G}{H}$ be the quotient map,
    $Q = \quot{G}{H}$. The solvable series is given by
    \[
      G = q^{-1}(Q) = q^{-1}(Q_0) \rhd q^{-1}(Q_1) \rhd \dots \rhd q^{-1}(Q_n)
      = H = H_0 \rhd H_1 \rhd \dots \rhd H_m = \{ 1 \}
    \]
    ``$\Rightarrow$'': \\
    \underline{Claim:} Define $G^{(i)} = [G^{(i-1)}, G^{(i-1)}], \quad
    i \in \Nb; G^{(0)} = G$.
    Then $G$ is solvable $\iff$ $G^{(n)} = \{1 \}$ for some $n$.
    \begin{proof}
      ``$\Leftarrow$'': O.K.

      ``$\Rightarrow$'': Given $G = G_0 \rhd G_1 \rhd \dots \rhd G_m = \{1\}$
      with $\quot{G_{i-1}}{G_i}$ abelian.
      We have $G^{(1)} \le G_1 \leadsto G^{(2)} \le [G_1, G_1] \le G_2 \leadsto
      \dots \leadsto G^{(n)} \le G_n = \{ 1 \} \leadsto G^{(n)} = \{1\}$.
    \end{proof}
    By the claim above:
    \begin{itemize}
      \item $H^{(n)} \le G^{(n)} = \{1\} \leadsto H^{(n)} = \{1\} \implies H$ 
        is solvable.
      \item $q\big([G, G]\big) = [q(G), q(G)] = [\quot{G}{H}, \quot{G}{H}] =
        \left(\quot{G}{H}\right)^{(1)} \leadsto \dots \leadsto
        q(G^{(n)}) = \left(\quot{G}{H}\right)^{(n)} \implies \quot{G}{H}$
        is solvable.
    \end{itemize}
  \end{proof}
\end{prop}

\begin{theorem}[Main Theorem] \label{thm:solvable-iff-solvable}
  Under some proper assumption on $\Char K$, a separable polynomial $f(x) \in K[x]$
  is solvable by radicals $\iff$ the Galois group of $f$ is solvable.

  \begin{enumerate}[label={\bf Part \Alph*:}]
    \item Let $L = L_m \supset \dots \supset L_0 = K$ s.t. $L_i = L_{i-1}(\alpha_i)$
      with $\alpha^{n_i} = a_i \in L_{i-1}$ and $\Char K \nmid n_1\cdots n_m$.
      If a separable poly. $f(x) \in K[x]$ splits over $L$, then the Galois group
      of $f$ over $K$ is solvable.

  \begin{proof}
    By lemma \ref{lemma:radical-ext-chain-galois}, we can first extend
    the extension tower and thus assume that $L/K$ is
    Galois with each $\zeta_{n_i}$ in $L$. Then each $L/L_i$ is Galois.
    If we set $n = \lcm(n_1, \dots, n_m)$,
    $L$ also contains $\zeta = \zeta_n = \zeta_{n_1}^{r_1} \cdots \zeta_{n_m}^{r_m}$.

    Consider $L = L(\zeta) \supset L_{m-1}(\zeta) \supset \dots \supset
    L_0(\zeta) = K(\zeta)$ (Note that $K(\zeta) \supset K$ and $L/K$ is Galois)
    and let $G_i = \Gal(L/L_i(\zeta))$ for each $i = 0, \dots, m$.

    Define $L'_i \triangleq L_i(\zeta)$ for all $i$. We can find that
    \begin{itemize}
      \item $G_m = \{1\}, G_0 = \Gal(L/K(\zeta))$.
      \item Since $\zeta_n \in L_{i-1}$, $L_i / L_{i-1}$ is normal,
        so
        \[ \quot{G_{i-1}}{G_i} = \quot{\Gal(L/L'_{i-1})}{\Gal(L/L'_i)}
        \cong \Gal(L'_{i-1} / L'_i) =  \Gal(L'_i(\alpha_i) / L'_i) \] is cyclic.
    \end{itemize}
    So $G_0$ is solvable.
    Moreoevr, $K(\zeta)$ is a splitting field of $x^n - 1$ over $K$ and
    $\Gal(K(\zeta)/K) \le \left(\quot{\Zb}{n\Zb}\right)^\times$, which is
    abelian, so it is solvable. Also, $\Gal(K(\zeta)/K) \cong \quot{\Gal(L/K)}{G_0}$
    is solvable. $\implies \Gal(L/K)$ is solvable.
    Let $N$ be a splitting field of $f$ over $K$ $\leadsto L \supset N \leadsto
    \Gal(N/K) \cong \quot{\Gal(L/K)}{\Gal(L/N)}$.

    By prop \ref{prop:quot-solvable}, $\Gal(N/K)$ is solvable.
  \end{proof}

  \item Let $f \in K[x]$ be separable and $L$ be a splitting field of $f$ over $K$.
  Assume $\Char K \nmid \abs{\Gal(L/K)}$. If $\Gal(L/K)$ is solvable, then
  $f$ is solvable by radicals.

  \begin{proof}
    Let $n = \abs{\Gal(L/K)}$ and $\zeta = \zeta_n$.
    Let $N$ be a splitting field of $f$ over $K(\zeta)$, i.e. $N = LK(\zeta)$.
    $\implies \Gal(N/K(\zeta)) \cong \Gal(L/L\cap K(\zeta)) \le \Gal(L/K)$.

    So $\Gal(N/K(\zeta))$ is solvable, say $\Gal(N/K(\zeta)) = G_0 \rhd G_1
    \rhd \dots \rhd G_m = 1$, $\quot{G_{i-1}}{G_i}$ is cyclic.
    
    If we set $N_j = N^{G_j}$, then
    $N = N_m \supset N_{m-1} \supset \dots \supset N_0 = K(\zeta)$ and
    $G_j = \Gal(N/N_j)$, $\quot{G_{i-1}}{G_i} \cong \Gal(N_i/N_{i-1})$ is
    cyclic $\implies N_i = N_{i-1}(\alpha_i), \alpha_i^{n_i} \in N_{i-1}$.
    (kummer extension)
    \begin{mdframed}
      Note that $n_i = [L_i : L_{i-1}] = \abs{G_{i-1}} / \abs{G_i}$ dividing
      $\abs{G_0}$ and $\abs{G_0} \mid n$, so $\zeta_n$ generates $\zeta_{n_i}$
      and $\Char K \nmid n_i$.
    \end{mdframed}
    $\implies N/K(\zeta)$ is an extension by radicals $\leadsto$
    $N/K$ is an extension by radicals.
  \end{proof}
\end{enumerate}
\end{theorem}

\begin{remark}
  In Part A of theorem \ref{thm:solvable-iff-solvable},
  $\Gal(K(\zeta)/K) \le \left(\quot{\Zb}{n\Zb}\right)^\times$ may be proper
  subgroup. We can check the if $[K(\zeta) : K] \overset{?}{=} 4 = \varphi(5)$.
\end{remark}

\subsection{Ruffini-Abel theorem}

\begin{theorem}[Main theorem]
  Assume $\Char F = 0$.
  The general equation of the $n$-th degree is not solvable by radicals if
  $n \ge 5$. In fact, let $f(x) = x^n - t_1x^{n-1} + t_2 x^{n-2} - \dots +
  (-1)^nt_n \in \underbrace{F(t_1, \dots, t_n)}_{=K}[x]$ with
  $t_1, \dots, t_n$ variables and $L$ be a splitting field of $f$ over $K$.
  Then $\Gal(L/K) \cong S_n$. $S_n$ is not solvable for $n \ge 5$.
\end{theorem}

\begin{lemma} \label{lemma:symm-poly-symm-group}
  Let $L = F(x_1, \dots, x_n)$ and $s_1, \dots, s_n$ be the elementary
  symmetric polynomials in $x_1, \dots, x_n$.
  \[
    s_k = \sum_{1 \le j_1 < \dots < j_k \le n} \prod_{i=1}^k x_{j_i}
  \]
  If $K = F(s_1, \dots, s_n) \subset L$, then $L/K$ is Galois and
  $\Gal(L/K) \cong S_n$.

  write $f(x) = (x - x_1) \cdots (x - x_n) = x^n - s_1x^{n-1} + s_2x^{n-2} -
  \dots + (-1)^ns_n \in K[x]$.
  Clearly, $L$ is a splitting field of $f$ over $K \leadsto L/K$ is Galois
  and $\Gal(L/K) \toone S_n$.

  Now, for $\sigma \in S_n$, $\sigma$ can be regarded as an element in
  $\Gal(L/K)$:
  \[
    \arraycolsep=1pt
    \begin{array}{rcl}
      \sigma: & L & \to L \\
              & x_i & \mapsto x_{\sigma(i)}
    \end{array}
  \]
  Since $\Set{ \sigma(x_1), \dots, \sigma(x_n) } = \Set{ x_1, \dots, x_n }
  \leadsto \sigma(s_i) = s_i \quad \forall i \leadsto \sigma\big|_K = \Id_K
  \leadsto \sigma \in \Gal(L/K)$.
\end{lemma}

\begin{coro}
  $L^{S_n} = K = F(s_1, \dots, s_n)$. \\
  $L^{S_n} = \Set{ f(x_1, \dots, x_n) \in L \given f(x_{\sigma(1)}, \dots,
  x_{\sigma(n)}) = f(x_1, \dots, x_n) \quad \forall \sigma \in S_n }$ is
  all symmetric poly.
\end{coro}

\begin{coro}
  For any finite group $G$, by Cayley thm, $G \toone S_n$ for some $n$.
  so $\Gal(L/L^G) \cong G$.
\end{coro}

Now we prove the Main theorem:
\begin{proof}
  Let $L = K(z_1, \dots, z_n)$. Since $t_1, \dots, t_n$ are the symmetric
  poly. w.r.t. $z_1, \dots, z_n$, $L = F(z_1, \dots, z_n)$.

  Let $F(s_1, \dots, s_n)$ and $F(x_1, \dots, x_n)$ be given as in lemma
  \ref{lemma:symm-poly-symm-group}.

  since $t_1, \dots, t_n$ are variables,
  $\exists\: \tau: F[t_1, \dots, t_n] \onto F[s_1, \dots, s_n]$ with $\tau:
  t_i \mapsto s_i$.
  Also, Since $x_1, \dots, x_n$ are variables, $\exists\: \sigma:
  F[x_1, \dots, x_n] \onto F[z_1, \dots, z_n]$ with $\sigma: x_i \mapsto z_i$.

  now, $\sigma \circ \tau(t_i) = \sigma(s_i)
  = \sigma\left(\sum x_{j_1}\cdots x_{j_i}\right)
  = \left(\sum z_{j_1}\cdots z_{j_i}\right) = t_i \implies \sigma\circ\tau =
  \Id \implies \tau$ is 1-1 and thus an isom.
  So there exists an extension $\tau': F(t_1, \dots, t_n) \isoto F(s_1, \dots, s_n)$.
  Note $\bar{\tau}': f(x) \mapsto g(x) = x^n - s_1x^{n-1} + \cdots + (-1)^ns_n$.

  Let $F(z_1, \dots, z_n)$ be a splitting field of $f$ over $F(t_1, \dots, t_n)$
  and $F(x_1, \dots, x_n)$ be a splitting field of $g$ over $F(s_1, \dots, s_n)$
  where $g = \bar{\tau}'(f)$.
  There exists $\sigma': F(z_1, \dots, z_n) \isoto F(x_1, \dots, x_n)$ with
  $\sigma'\big|_{F(t_1, \dots, t_n)} = \tau'$.
  So $\Gal(L/K) \cong S_n$ by lemma \ref{lemma:symm-poly-symm-group}.
\end{proof}

\begin{remark} \mbox{}
  \begin{itemize}
    \item Since $S_n$ is transitive, $f$ is irr.
    \item $\Char F = 0 \leadsto f$ is separable.
  \end{itemize}
\end{remark}

%! TEX root=../main.tex
\subsection{Calculation of Galois groups}
Let $f(x)$ be separable in $K[x]$ and $L = K(\alpha_1, \dots, \alpha_n)$ be a splitting field of
$f$ over $K$. The goal is to find $\Gal(L/K)$ which is in $S_n$.

Define $\theta \triangleq y_1 \alpha_1 + \dots + y_n \alpha_n$. For each $\sigma \in S_n$,
define $\sigma_y(\theta) \triangleq y_{\sigma(1)} \alpha_1 + \dots + y_{\sigma(n)} \alpha_n$ and
$\sigma_{\alpha}(\theta) = y_1 \alpha_{\sigma(1)} + \dots + y_n \alpha_{\sigma(n)}$.
It is easy to see that $\sigma_y^{-1} = \sigma_\alpha$. 

In $L(x, y_1, \dots, y_n)$, we consider $F(x, \bm{y})
= \prod\limits_{\sigma \in S_n} (x - \sigma_y(\theta))
= \prod\limits_{\sigma^{-1} \in S_n} (x - \sigma_{\alpha}(\theta))
= \prod\limits_{\sigma \in S_n} (x - \sigma_\alpha(\theta))$.
Since each coefficient of $F$ is a symmetric polynomial of $\alpha_1, \dots, \alpha_n$,
by the fundamental theorem of symmetric polynomials, these symmetric polynomials are
polynomials of the elementary symmetric polynomials. Thus $F(x, y) \in K[x, y_1, \dots, y_n]$.

Decompose $F$ into irreducible factors in $K[x, y_1, \dots, y_n]$, say $F = F_1 F_2 \dotsm F_r$.
Notice that for any $\sigma \in S_n$, $F = \sigma_y F = \sigma_y F_1 \cdot \sigma_y F_2 \dotsm \sigma_y F_r$.
And each $F_i$ is map to some $F_j$, thus $\sigma$ induces a permutation of $F_1, F_2, \dots, F_r$.

For convenience, assume $(x - \theta) \mid F_1$. We have the following lemma:

\begin{lemma}
  \[ Q \triangleq \Set{ \sigma : \sigma_y F_1 = F_1 } = \Set{ \sigma : \sigma_y(x - \theta) \mid F_1} \]
  \begin{proof}
    ``$\subseteq$'': Since $x - \theta \mid F_1$, so $\sigma_y (x - \theta) \mid \sigma_y F_1 = F_1$.

    ``$\supseteq$'': $\sigma_y(x - \theta) = x - \sigma_y(\theta) \mid \sigma_y(F_1)$, so $\sigma_y(F_1)$
    and $F_1$ has a common factor. Since $F$ is separable, $\sigma_y(F_1) = F_1$.
  \end{proof}
\end{lemma}

\begin{prop}
  $\Gal(L/K) = Q$.

  \begin{proof}
    ``$\subseteq$'': For each $\sigma \in \Gal(L/K) \hookrightarrow S_n$, extend $\sigma$ to
    \[
      \arraycolsep=1pt
      \begin{array}{rccc}
        \tilde\sigma: & L(y_1, \dots, y_n) & \to & L(y_1, \dots, y_n) \\
        & \alpha \in L & \mapsto & \sigma(\alpha) \\
        & y_i & \mapsto & y_i
      \end{array}
    \]
    The automorphism fixes $K(y_1, \dots, y_n)$, so
    $\tilde\sigma(\theta) = \sigma_\alpha(\theta)$ and $\theta$ share the
    same minimal polynomial over $K(y_1, \dots, y_n)$.
    By Gauss's lemma, $F_1$ is irreducible in $K[y_1, \dots, y_n][x] \implies F_1$
    is irreducible in $K(y_1, \dots, y_n)[x]$, thus
    $F_1 = m_{\theta, K(y_1, \dots, y_n)}
    = m_{\sigma_\alpha(\theta), K(y_1, \dots, y_n)}$, which implies
    $(x - \sigma_\alpha(\theta)) \mid F_1$.
    So $\sigma_y^{-1} F_1 = F_1 \implies \sigma^{-1} \in Q \implies \sigma \in Q$.

    ``$\supseteq$'': For any $\sigma \in Q$, $F_1 = m_{\theta, K(y_1, \dots, y_n)}
    = m_{\sigma^{-1}_\alpha(\theta), K(y_1, \dots, y_n)}$, so there exists
    $\tau \in \Aut(L(\bm{y}) / K(\bm{y}))$ satisfying
    $\tau(\theta) = \sigma_\alpha^{-1}(\theta) = \sigma_y(\theta)$.
    Since $L/K$ normal, $\tau(L) = L$ and thus $\tau\big|_L \in \Gal(L/K)$ with
    $\tau\big|_L(\alpha_i) = \alpha_{\sigma^{-1}(i)}$, which implies that
    $\sigma^{-1} \in \Gal(L/K) \implies \sigma \in \Gal(L/K)$.
  \end{proof}
\end{prop}

\begin{theorem}
  Let $f(x)$ be monic, separable, in $\Zb[x]$. Assume $p \nmid D = \prod_{i < j} (\alpha_i - \alpha_j)^2$,
  then the Galois group of $\bar{f}(x)$ in $\Fb_p[x]$ is a subgroup of the Galois group of $f(x)$.

  \begin{proof}
    Since $f$ is separable, $D \neq 0$. The discriminant could be calculate
    by $D = (-1)^{n(n+1)/2} R(f, f')$ which only depends on the coefficients,
    so $\bar{D} \neq 0$ in $\Fb_p$ since $p \nmid D$. Thus $f$ separable.

    As above, let $F = F_1 F_2 \dotsm F_r$ in $\Zb[x, \bm{y}]$.
    Assume $f(x) = x^n + a_{n-1} x^{n-1} + \dots + a_0$, then $\bar{f}(x) =
    x^n + \bar{a}_{n-1}x^{n-1} + \dots + \bar{a}_0$.  Let $\alpha_1, \dots, \alpha_n$
    and $\beta_1, \dots, \beta_n$ be their roots, respectively.
    Define $\theta_p \triangleq y_1 \beta_1 + \dots + y_n \beta_n$.
    Since the coefficients of $F$ are symmetric polynomials of
    $\alpha_1, \dots, \alpha_n$, which only depends on the coefficients of $f$,
    and so is $F_p(x, y) = \prod_{\sigma \in S_n}(x - \sigma_y(\theta_p))$,
    we know that $F_p(x, y) = \bar{F}(x, y)$.

    Now $\bar{F} = \bar{F}_1 \bar{F}_2 \dotsm \bar{F}_r
    = (G_{1, 1} \dotsm G_{1, q_1})(G_{2, 1} \dotsm G_{2, q_2}) \dotsm (G_{r, 1}
    \dotsm G_{r, q_r})$

    The Galois group of $\bar{f}$ is
    \[ \Set{ \sigma \in S_n : \sigma_y G_{1, j} = G_{1, j}, \, \forall j }
      \subseteq \Set{ \sigma \in S_n : \sigma_y \bar{F}_1 = \bar{F}_1 } =
    \Set{\sigma \in S_n : \sigma_y F_1 = F_1} \]
    Where the equality holds because $\sigma_y \bar{F}_1 = \bar{F}_1 \iff
    (x - \sigma_y(\theta_p)) \mid \bar{F}_1 \iff
    (x - \sigma_y(\theta)) \mid F_1 \iff \sigma_y F_1 = F_1$. Thus the galois group
    of $\bar{f}$ is a subgroup of $f$.
  \end{proof}
\end{theorem}

\begin{fact} \hfill
  \begin{itemize}
    \item Every finite extension of $\Fb_p$ is cyclic, so the Galois group of
      $\bar{f}(x)$ in $\Fb_p[x]$ is cyclic.
    \item If $\bar{f}$ is irreducible, then the Galois group of $\bar{f}$ is
      transitive on its roots, thus the only possibililty is a cycle of length
      $n = \deg \bar{f}$ in $S_n$.
    \item If $\bar{f} = \bar{f}_1 \dotsm \bar{f}_r$, with each $\bar{f}_i$ irreducible.
      Let the Galois group be $\gen{\sigma}$, then $\sigma$ should be
      transitive on the roots of each $\bar{f}_i$. The only possibility of $\sigma$
      is a permutation composited by cycles of length
      $\deg \bar{f}_1, \dots, \deg \bar{f}_r$.
      That is, $\sigma = \cycle{\alpha_{1,1}, \dots, \alpha_{1, m_1}} \dotsm
      \cycle{\alpha_{r,1}, \dots, \alpha_{r, m_r}}$ where $m_i \triangleq
      \deg \bar{f}_i$.
  \end{itemize}
\end{fact}

%! TEX root=../main.tex
\subsection{Transcendental extensions}

\begin{definition}
  Let $L/K$ be an extension and $S \subset L$.
  \begin{itemize}
    \item $S$ is algebraically dependent over $K$ if for some $n \in \Nb$,
      exists $f(x_1, \dots, x_n) \in K[x_1, \dots, x_n]$ satisfied
      $f(a_1, \dots, a_n) = 0$ for some distinct $a_1, \dots, a_n \in S$.
    \item $S$ is algebraically independent over $K$ if $S$ is not algebraically dependent.
    \item $S$ is called a transcendence base for $L/K$ if $S$ is algebraically independent
      and $L/K(S)$ is algebraic.
  \end{itemize}
\end{definition}

\begin{theorem}
  Any two transcendence bases for $L/K$ have the same cardinality.

  \begin{proof}
    Pick any transcendence base $S = \{s_1, \dots, s_n\}$ for $L/K$.
    Let $T$ be another transcendence base for $L/K$.
    First we deal with the case which $S$ is finite.

    We claim that $\exists t_1 \in T$ such that
    $t_1$ is algebraically independent over $K(s_2, \dots, s_n)$.
    \begin{proof}
      If not, then all elements of $T$ is algebraically dependent over $K(s_2, \dots, s_n)$.
      This implies $K(s_2, \dots, s_n)(T) / K(s_2, \dots, s_n)$ is algebraic.
      And $L/K(T)$ is algebraic implies $L/K(T)(s_2,\dots,s_n)$ is algebraic.
      Then $L/K(s_2, \dots, s_n)$ is algebraic, which is a contradiction ($s_1$ is not).
    \end{proof}
    By the claim, $\Set{t_1, s_2, \dots, s_n}$ is algebraic indepedent.
    Also, there exists $f \ne 0$ in $K[x_1, \dots, x_{n+1}]$ such that
    $f(t_1, s_1, \dots, s_n) = 0$ since $t_1$ is algebraic over $K(s_1, \dots, s_n)$.
    Since $\Set{s_1, \dots, s_n}$ and $\Set{t_1, s_2, \dots, s_n}$ are both
    algebraically indepedent, $t_1, s_1$ must occur in $f$ $\implies$
    $s_1$ is algebraic over $K(t_1, s_2, \dots, s_n)$.
    Then $K(t_1, s_1, \dots, s_n) / K(t_1, s_2, \dots, s_n)$ is algebraic.
    Since $L/K(t_1, s_1, \dots, s_n)$ is algebraic, $L/K(t_1, s_2, \dots, s_n)$
    is algebraic.

    Repeating this process, we get find $t_1, \dots, t_n \in T$ s.t.
    $L/K(t_1, \dots, t_n)$ is algebraic. But $T$ is a transcendence base,
    so we must have $T = \Set{t_1, \dots, t_n}$.

    Now assume $S$ is infinite.
    For another transcendence base $T$, we have $\abs{T} = \infty$.
    For $s \in S$, $s$ is algebraic over $K(T)$, and
    in fact is over $K(T_s)$ such that $T_s$ is finite,
    since algebraic relation involves.
    Let $m_{s, K(T)} \in K(T_s)[x]$ for some finite set $T_s \subset T$.
    We claim that $\bigcup\limits_{s\in S} T_s = T$.
    \begin{proof}
      Let $U = \bigcup\limits_{s\in S} T_s$.
      Clearly $U \subseteq T$. And by def, $K(U)(S)/K(U)$ is algebraic.
      Also, $L/K(U)(S)$ is algebraic. So $L/K(U)$ is algebraic
      $\implies T = U$ since $T$ is a transcendence base.
    \end{proof}
    By well ordering principle, we can well-order $S$ and denote its
    $1$st element by $s_1$.
    Let
    \[
      \begin{cases}
        T_{s_1}' = T_{s_1} \\
        T_{s}' = T_s \setminus \bigcup_{l < s} T_l
      \end{cases}
      \quad \implies \quad
      \Set{ T_s' }_{s\in S} \text{~are mutually disjoint}
    \]
    For all $T_s'$, choose a fixed ordering of the elements in $T_s'$, says
    $t_{s,1}, \dots, t_{s, {k_s}}$. Define an injection
    $\varphi: \bigcup\limits_{s\in S} T_s' \to S\times \Nb$ with
    $\varphi: t_{s,i} \mapsto (s, i)$.
    So $\abs{T} = \abs*{\bigcup\limits_{s\in S} T_s} \le \abs{S\times \Nb}
    = \abs{S}\abs{\Nb} = \abs{S}$ since $\abs{S} = \infty$.
  \end{proof}
\end{theorem}

\begin{definition}
  Let $S$ be a transcendence base of $L/K$, then we use
  $\trdeg_K L$ to denote $\abs{S}$.
\end{definition}

\begin{remark}
  If $S_1, S_2$ are two transcendence base for $L/K$, then it is {\bf not
  necessarily true} that $K(S_1) = K(S_2)$.
\end{remark}

\begin{definition}
  $L/K$ is called purely transcendental if exists a transcendental base $S$
  such that $L = K(S)$.
\end{definition}

\begin{theorem}[L\"{u}roth's theorem]
  If $L$ is purely transcendental of degree $1$ over $K$, then any proper
  intermediate field $E$ is also purely transcendental of degree $1$.
\end{theorem}

\begin{lemma} \label{lemma:rational-trans-alg}
  Let $L = K(t)$ with $t$ being transcendental over $K$ and $u = f(t) / g(t) \in L \setminus K$
  with $\gcd(f(t), g(t)) = 1$.
  Assume $n = \max(\deg f, \deg g)$, then $L/K(u)$ is algebraic and $[L: K(u)] = n$.
  \begin{proof}
    Write
    \[ f(t) = a_n t^n + \dots + a_1 t + a_0, \quad g(t) = b_n t^n + \dots + b_1 t + b_0 \]
    (note that either $a_n \ne 0$ or $b_n \ne 0$)
    Let $F(x) = f(x) - ug(x) =-(a_n - ub_n) x^n + \dots + (a_1 - ub_1)x + (a_0 - ub_0)$.
    Since $a_n - ub_n \ne 0$, $F(x) \ne 0$ and $\deg F(x) > 0$.
    By def. of $u$, we have $F(t) = 0 \implies t$ is algebraic over $K(u)$ and
    $[K(t):K(u)] \le n$.
    Now we prove that $F(x)$ is irreducible over $K(u)$.
    By Gauss lemma, it suffices to show that $F(x)$ is irreducible in
    $K[u][x] = K[u, x]$.
    Assume that $F(x) = p(u, x) q(u, x)$ with $\deg_u p = 1$ and $q \in K[x]$.
    Since $F(x) = f(x) - ug(x)$, we have
    $q \mid f, q\mid g \implies q \mid \gcd(f, g) = 1 \implies q \in K$.
    So $[K(t):K(u)] = n$.
  \end{proof}
\end{lemma}

Now we prove the L\"{u}roth's theorem:
\begin{proof}
  For $v \in E \setminus K$, by lemma \ref{lemma:rational-trans-alg},
  $t$ is algebraic over $K(v) \leadsto t$ is algebraic over $E$.

  Let $m(x) = m_{t,E}$, then there exists $\beta(t) \in K(t)$ s.t.
  $\beta(t)m(x) = a_n(t)x^n + \dots + a_1(t)x + a_0(t)$ is primitive
  in $K[t][x] = K[t, x]$. Let $F(t, x) = \beta(t)m(x)$.

  Since $t$ is not algebraic over $K$, there exists some $u = \frac{a_i(t)}{a_n(t)} \not\in K$.
  Write $u = \frac{f(t)}{g(t)}$ with $\gcd(f, g) = 1$.
  (Note that $u \in E$)

  By lemma \ref{lemma:rational-trans-alg}, $[K(t): K(u)] = r \ge n$.
  Now we show that $r \le n$, then $r = n \implies E = K(u)$.

  Let $l = f(t)g(x) - g(t)f(x)$, which is skew-symmetric in $t$ and $x$.
  Notice that $g(t)^{-1} l \in E[x]$ and has $t$ as a zero.
  So $m(x) \mid g(t)^{-1} l$ in $E[x]$ $\implies \beta(t)m(x) \mid \beta(t)g(t)^{-1} l$.
  Since $\beta(t)g(t)^{-1} \in K[t]$, $F(t, x) \mid l$ in $K(t)[x]$.
  Since $F(t, x)$ is primitive in $K[t][x]$, $F(t, x) \mid l$ in $K[t][x]$.

  Say $l = Fq$ for some $q(t,x) \in K[t][x]$.
  Note that $\deg_t l \le r, \deg_t F \ge r \leadsto \deg_t l =\deg_t F = r,
  \deg_t q = 0$. So $q \in K[x] \leadsto q$ is primitive in $K[t][x]$.
  By Gauss lemma, $F, q$ are primitive, then $l$ is also primitive in $K[t][x]$.
  Since $l$ is skew-symmetric in $t$ and $x$, $l$ is also primitive in $K[x][t]$.
  But $q\in K[x]$ and $q \mid l$, we have $q \in K$.
  Hence $n = \deg_x F = \deg_x l = \deg_t l  = \deg_t F \ge r$.
\end{proof}

\subsection{Hilbert theorem 90 and Normal basis}

Let $L = K(\alpha)$ with $f = m_{\alpha, K} = x^n + a_{n-1} x^{n-1} + \dots + a_0$ being separable.
We have known that exists exactly $n$ monomorphisms $\sigma_i :: L \to \overline{K}$ fixing $K$,
and $\Set{\sigma_1(\alpha), \dots, \sigma_n(\alpha)}$ consists of all roots of $f$.
So
\begin{multline*}
  x^n + a_{n-1} x^{n-1} + \dots + a_0 = (x - \sigma_1(\alpha)) \dotsm (x - \sigma_n(\alpha)) \\
  \implies -a_{n-1} = \sigma_1(\alpha) + \dots + \sigma_n(\alpha) \text{ and }
  (-1)^n a_0 = \sigma_1(\alpha) \dotsm \sigma_n(\alpha)
\end{multline*}

Consider the $K$-linear transformation:
\[ \deffunc{T_\alpha}{K(\alpha)}{K(\alpha)}{v}{\alpha v} \]
Then
\[ [T_\alpha]_\gamma = \begin{pmatrix}
    0 & 0 & \cdots & 0 & -a_0 \\
    1 & 0 & \cdots & 0 & -a_1 \\
    0 & 1 & \cdots & 0 & -a_2 \\
    \vdots & \vdots & \ddots & \vdots & \vdots \\
    0 & 0 & \cdots & 1 & -a_{n-1}
  \end{pmatrix}, \quad \text{ where } \gamma = \Set{1, \alpha, \alpha^2, \dots, \alpha^{n-1}} \]
And $\trace(T_\alpha) = -a_{n-1}, \det(T_\alpha) = (-1)^n a_0$.

\begin{definition}
  Let $L/K$ be a Galois extension with $G = \Gal(L/K)$.
  for all $\alpha \in L$, define
  \begin{alignat*}{4}
    N_{L/K}(\alpha) &= \prod_{\sigma \in G} \sigma(\alpha) \quad\quad && N_{L/K} :: L^\times \to K^\times
    \text{ is multiplicative} \\
    \trace_{L/K}(\alpha) &= \sum_{\sigma \in G} \sigma(\alpha) \quad\quad && \trace_{L/K} :: L \to K
    \text{ is additive}
  \end{alignat*}
\end{definition}

\begin{theorem}[Hilbert theorem 90] \label{thm:Hilbert-theorem-90}
  Let $L/K$ is cyclic and $G = \gen{\sigma}$ with $\ord(\sigma) = n$, then
  \begin{enumerate}
    \item $\alpha \in L^\times$ and $N_{L/K}(\alpha) = 1$ $\iff$ $\exists \beta \in L^\times, \alpha = \beta / \sigma(\beta)$.
    \item $\alpha \in L^{\ }$ and $\trace_{L/K}(\alpha) = 0$ $\iff$ $\exists \beta \in L, \alpha = \beta - \sigma(\beta)$.
  \end{enumerate}

  \begin{proof} \hfill
    \begin{enumerate}
      \item ``$\Leftarrow$'': $N_{L/K}(\alpha) = \prod_{k=0}^{n-1} \sigma^k(\beta / \sigma(\beta)) = 1$.

        ``$\Rightarrow$'':  Since automorphisms are linearly independent, exists $c \in L$
        such that
        \[ 0 \neq \beta = \Id(c) + \alpha \sigma(c) + \alpha \sigma(\alpha) \sigma^2(c)
          + \dots + \alpha \sigma(\alpha) \sigma^2(\alpha) \dotsm \sigma^{n-2}(\alpha) \sigma^{n-1}(c) \]
        Since $\alpha\sigma(\alpha \sigma(\alpha) \sigma^2(\alpha) \dotsm \sigma^{n-2}(\alpha)) = N_{L/K}(\alpha) = 1$,
        it is easy to check that $\alpha \sigma(\beta) = \beta$.

      \item ``$\Leftarrow$'': $\trace_{L/K}(\alpha) = \trace_{L/K}(\beta - \sigma(\beta))
        = \sum \left(\sigma^k(\beta) -\sigma^{k+1}(\beta)\right) = 0$.

        ``$\Rightarrow$'':
        Choose $c$ such that $\beta_1 = c + \sigma(c) + \dots + \sigma^{n-1}(c) \neq 0$,
        so $\sigma(\beta_1) = \beta_1$.  Let
        \[ \beta_2 = \alpha \sigma(c) + (\alpha + \sigma(\alpha)) \sigma^2(c) + \dots +
          \big( \alpha + \sigma(\alpha) + \dots + \sigma^{n-2}(\alpha) \big) \sigma^{n-1}(c) \]
        Then
        \[ \beta_2 - \sigma(\beta_2) = \alpha \sigma(c) + \alpha \sigma^2(c) + \dots + \alpha \sigma^{n-1}(c)
          + \alpha c = \alpha \beta_1. \]
        So let $\beta \triangleq \beta_2 / \beta_1$, we obtain $\beta_2/\beta_1 - \sigma(\beta_2/\beta_1) =
        (\beta_2 - \sigma(\beta_2)) / \beta_1 = \alpha$.
        \qedhere
    \end{enumerate}
  \end{proof}
\end{theorem}

\begin{coro}
  Let $\Char K = p$ and $[L: K] = p$, then $L/K$ is Galois and cyclic $\iff$ $L = K(\alpha)$ where
  $\alpha$ is a root of $x^p - x - a$.

  \begin{proof}
    ``$\Rightarrow$'': Let $\Gal(L/K) = \gen{\sigma}$ with $\ord(\sigma) = p$. Then
    $\trace_{L/K}(1) = p = 0$. By theorem~\ref{thm:Hilbert-theorem-90},
    exists $\alpha$ satisfied $1 = \sigma(\alpha) - \alpha$.
    So $\alpha \not\in K$. Then we have $1 < [K(\alpha): K] \bigm| [L: K] = p$,
    so $[K(\alpha): K] = p \implies K(\alpha) = L$.

    Notice that $\sigma^k(\alpha) = \alpha + k$.
    Since $\sigma^k(\alpha)$ iterates through all roots of $m_{\alpha, K}$ and $\sigma^k(\alpha) = \alpha + k$,
    $\alpha, \alpha+1, \dots, \alpha +p-1$ are all the roots of $m_{\alpha, K}$.
    We claim that $m_{\alpha, K} = x^p - x - a$ where $a \triangleq \alpha^p - \alpha$.
    Since $\sigma(a) = \sigma(\alpha)^p - \alpha = \alpha^p + p - \alpha = a$,
    $a$ is fixed by all automorphisms, so $a \in K$. Moreover, $m_{\alpha, K}(\alpha+k)
    = \alpha^p + k^p - \alpha - k - a = 0$, thus the proof is completed.

    ``$\Leftarrow$'': Similarly, we know that all roots of $x^p - x - a$
    are $\alpha, \alpha+1, \dots, \alpha+{p-1}$. Define $\sigma(\alpha) = \alpha+1$,
    then $\sigma^i(\alpha) = \alpha+i$, and thus $\ord(\sigma) = p$. Hence $\Gal(L/K) = \gen{\sigma}$.
  \end{proof}
\end{coro}

\begin{coro}
  If $x^2 + d y^2 = 1$ where $-d$ is not a square, then $L \triangleq \Qb(\sqrt{-d})$
  is a splitting field of $x^2 + d$ over $\Qb$,
  so $N_{L/\Qb}(a + b \sqrt{-d}) = a^2 + d b^2$. Since $[L: \Qb] = 2$,
  the galois group is obviously cyclic and in fact is $\gen{\sigma}$,
  where $\sigma = (a + b \sqrt{-d}) \mapsto (a - b \sqrt{-d})$.
  By theorem~\ref{thm:Hilbert-theorem-90},
  \[ x^2 + d y^2 = 1 \iff \exists a + b \sqrt{-d} \quad\text{s.t.}\quad x + y \sqrt{-d} = \frac{a+b\sqrt{-d}}{a-b\sqrt{-d}}
    = \frac{(a^2 - db^2) + 2ab\sqrt{-d}}{a^2 + db^2} \]
\end{coro}

\begin{definition}
  Let $L/K$ be Galois and $\Gal(L/K) = \Set{ \text{Id} = \sigma_1, \dots, \sigma_n }$.
  A basis for $L/K$ of the form $\Set{ \sigma_1(\alpha), \sigma_2(\alpha), \dots, \sigma_n(\alpha)}$
  with $\alpha \in L$ is called  a normal basis for $L/K$.
\end{definition}

\begin{lemma} \label{lemma:basis-iff-det-neq-0}
  $\alpha_1, \dots, \alpha_n \in L$ form a basis for $L/K$ if and only if
  \[
    \begin{vmatrix}
      \sigma_1(\alpha_1) & \sigma_1(\alpha_2) & \cdots & \sigma_1(\alpha_n) \\
      \sigma_2(\alpha_1) & \sigma_2(\alpha_2) & \cdots & \sigma_2(\alpha_n) \\
      \vdots & \vdots & \ddots & \vdots \\
      \sigma_n(\alpha_1) & \sigma_n(\alpha_2) & \cdots & \sigma_n(\alpha_n)
    \end{vmatrix} \neq 0
  \]

  \begin{proof}
    ``$\Rightarrow$'': If not, then the determinant is $0$. Then
    \[ \left\{
        \arraycolsep=2pt
        \begin{array}{ccc}
          \sigma_1(\alpha_1) x_1 + \dots + \sigma_n(\alpha_1) x_n &=& 0 \\
          \sigma_1(\alpha_2) x_1 + \dots + \sigma_n(\alpha_2) x_n &=& 0 \\
          \vdots & & \vdots \\
          \sigma_1(\alpha_n) x_1 + \dots + \sigma_n(\alpha_n) x_n &=& 0
        \end{array}
      \right.
    \]
    has a non-zero solution $\bm{c} = (c_1, \dots, c_n) \in L^n$. (i.e., $\sum c_j \sigma_j(\alpha_i) = 0$
    for each $i$.) So $\big( \sum_{j} c_j \sigma_j \big)(\alpha_i) = 0$ for each $\alpha_i$,
    but $\alpha_i$ is a basis, so $\sum_{j} c_j \sigma_j = 0$, then these automorphisms
    are linear dependent, which leads to a contradiction.

    ``$\Rightarrow$'': If not, then exists $\bm{0} \neq \bm{c} = (c_1, \dots, c_n)$
    satisfied $\sum c_i \alpha_i = 0$. Then $\sum_i c_i \sigma_j(\alpha_i) = 0$
    for each $j$. Thus the determinant is $0$ which leads to a contradiction.
  \end{proof}
\end{lemma}

\begin{lemma} \label{lemma:automorphisms-are-alg-indep}
  Let $\abs{K} = \infty$. Then $\sigma_1, \dots, \sigma_n$ are algebraically independent over $L$.

  \begin{proof}
    Let $f(x_1, \dots, x_n) \in L[x_1, \dots, x_n]$ such that $f(\sigma_1, \dots, \sigma_n) = 0$.
    Let $\Set{\alpha_1, \dots, \alpha_n}$ be a basis for $L/K$. Then
    \[ 0 = f(\sigma_1, \dots, \sigma_n)\left( \sum_{i = 1}^n r_i \alpha_i \right)
      = f \left(r_1 \sigma_1 \left( \sum_{i = 1}^n \alpha_i \right), \dots,
        r_n \sigma_n \left( \sum_{i = 1}^n \alpha_i \right) \right) \]
    So let
    \[ g(x_1, \dots, x_n) \triangleq f \left(\sum_i \sigma_1(\alpha_i) x_1,
      \dots, \sum_i \sigma_n(\alpha_i) x_n \right) \]
    and write $g(x_1, \dots, x_n) = \sum_j g_j(x_1, \dots, x_n) \alpha_j$.
    Then $g_j(r_1, \dots, r_n) = 0, \, \forall \bm{r} \in K^n$. The only
    polynomial which has infinite zeros is the zero polynomial, thus $g_j = 0$ for each $j$.

    Now, by lemma~\ref{lemma:basis-iff-det-neq-0}, $\det([\sigma_i(\alpha_j)]) \neq 0$.
    So it is possible to solve $\bm{x} = (x_i)$ satisfied
    $\bm{y} = (y_j) = \big( \sum_i \sigma_j(\alpha_i) x_i \big)$.
    Thus $g = 0 \implies f = 0$.
  \end{proof}
\end{lemma}

\begin{theorem}
  Any Galois extension $L/K$ has a normal basis.

  \begin{proof}
    Case 1: $L/K$ is cyclic (so all finite field is included). \\
    Let $\Gal(L/K) = \gen{\sigma}$ with $\ord(\sigma) = n$. $\sigma$ could be view as
    a linear transformation of $L$ over $K$. Thus $\sigma$ gives $L$ a
    $K[x]$-module structure by $(f(x), \alpha) \mapsto f(\sigma)(\alpha)$.

    Since $K[x]$ is a PID. By the structure theorem, we could write
    \[ L \cong \quot{K[x]}{\gen{d_1(x)}} \oplus \dots \oplus \quot{K[x]}{\gen{d_s(x)}} \quad
      \text{with } d_i \mid d_{i+1} \]
    Since $\Id, \sigma, \dots, \sigma^{n-1}$ are linearly independent over $K$,
    $m_{\sigma, K}$ should have degree at least $n$, thus it is clear that $x^n - 1$
    is the minimal polynomial of $\sigma$, thus $d_s(x) = x^n - 1$. But
    the characteristic polynomial of $\sigma$ has degree at most $n$, thus
    $d_1(x) \dotsm d_s(x) = x^n - 1$. So $L \cong K[x] / \gen{x^n - 1}$.
    Let $\alpha \in L$ such that $\Ann(\alpha) = \gen{x^n - 1}$, then $L = K[x]\alpha$.
    Hence $L = \gen{\alpha, \sigma(\alpha), \dots, \sigma^{n-1}(\alpha)}$.

    Case 2: $\abs{K} = \infty$. Let $\Gal(L/K) = \Set{\sigma_1, \dots, \sigma_n}$.
    Define $y_{i, j} = x_k$ so that $\sigma_i \sigma_j = \sigma_{k}$.
    Consider
    \[
      f(x_1, \dots, x_n) = \begin{vmatrix}
        y_{1, 1} & y_{1, 2} & \cdots & y_{1, n} \\
        \vdots & \vdots & \ddots & \vdots \\
        y_{n, 1} & y_{n, 2} & \cdots & y_{n, n} \\
      \end{vmatrix}
    \]
    This determinant is a non-zero polynomial in $x_1, x_2, \dots, x_n$.
    Since if we fix $\sigma_1$, for each $\sigma_i$, exists unique $j$ so that $\sigma_i \sigma_j = \sigma_1$.
    So the determinant has a $x_1^n$ term and is not zero.
    Then $f(\sigma_1, \dots, \sigma_n) \neq 0$ by lemma~\ref{lemma:automorphisms-are-alg-indep}.
    Thus there exists $\alpha \in L$ s.t. $\det\big([\sigma_{i}\sigma_{j}(\alpha)] \big)
    = f(\sigma_1, \dots, \sigma_n)(\alpha) \neq 0$.
    So by lemma~\ref{lemma:basis-iff-det-neq-0}, $\Set{ \sigma_i(\alpha) }$ is a basis.
  \end{proof}
\end{theorem}

%! TEX root=../main.tex
\section{Commutative Algebra}

\subsection{ED, PID and UFD}

We shall consider $R$ to be a integral domain below.
\begin{definition}
  A function $N: R \to \Nb$ with $N(0) = 0$ is called a norm on $R$.
\end{definition}

\begin{definition}
  $R$ is called a Euclidean domain if exists a norm $N$ on $R$
  satisfy
  \[ \forall a, b \in R, \ \exists q, r \in R \text{ s.t. } a = qb + r \text{ with } r = 0 \text{ or } N(r) < N(b) \]
\end{definition}

\begin{example} \hfill
  \begin{itemize}
    \item $\Zb$ is a ED with $N(n) = \abs{n}$.
    \item $K[x]$ is a ED with $N(f) = \deg f, \, \forall f \in K[x]$.
  \end{itemize}
\end{example}

\begin{definition}
  $A_d$ is defined to be the ring of integers in the quadratic field $\Qb(\sqrt{d})$
  with $d \neq 1$ and $d$ is square-free. That is,
  \[ A_d \triangleq \Set{ \alpha \in \Qb(\sqrt{d}) \mid \alpha \text{ is integral over } \Zb} \]
\end{definition}

\begin{theorem} \hfill
  \begin{itemize}
    \item If $d \equiv 1 \pmod{4}$, then
      \[ A_d = \big\{ a + b \frac{1 + \sqrt{d}}{2} : a, b \in \Zb \big\} \]
    \item Else, $d \equiv 2, 3 \pmod{4}$, then
      \[ A_d = \big\{ a + b \sqrt{d} : a, b \in \Zb \big\} \]
  \end{itemize}
\end{theorem}

\begin{theorem}
  $A_d$ is a ED if $d = 2, 3, 5, -1, -2, -3, -7, -11$. Hence $A_d$ is also PID and UFD.
\end{theorem}

\begin{example}
  $A_{-5}$ is not a ED.

  \begin{proof}
    Consider $6 = 2 \cdot 3 = (1 + \sqrt{-5})(1 - \sqrt{-5})$.
    Notice that $1 + \sqrt{-5}$ is irreducible, since if $1 + \sqrt{-5} = \alpha \beta$,
    then $6 = N(1 + \sqrt{-5}) = N(\alpha) N(\beta)$. But there is
    $a^2 + 5b^2 = 2 \text{ or } 3$ has no integer solution.
    Also $1 + \sqrt{-5} \nmid 2, 3$. Since if $(1 + \sqrt{-5}) \alpha = 2$,
    then $N(1 + \sqrt{-5}) N(\alpha) = N(2)$, but $N(1 + \sqrt{-5}) = 6$.
  \end{proof}
\end{example}

\subsubsection{$A_{-1}$ and $A_{-3}$}
First, $\alpha$ is a unit $\iff$ $N(\alpha) = 1$.
so we have:
\begin{itemize}
  \item $A_{-1}$: $\pm 1, \pm \mathrm{i}$.
  \item $A_{-3}$: $\pm 1, \pm \omega, \pm \omega^2$.
\end{itemize}

If $\alpha$ is a prime in $A_{-1}$ or $A_{-3}$, then $N(\alpha) = p \text{ or } p^2$ for some prime integer $p$.

Let $N(\alpha)  = \alpha \bar\alpha = p_1 \dotsm p_n$ in $\Zb$

\begin{definition}
  If $p$ is add and $a \not\equiv 0 \pmod{p}$, then
  \begin{itemize}
    \item If $x^2 \equiv a \pmod{p}$ is solvable, then define $\left( \frac{a}{p} \right) = 1$.
    \item Else $x^2 \equiv a \pmod{p}$ is not solvable and define $\left( \frac{a}{p} \right) = -1$.
  \end{itemize}
\end{definition}

\begin{prop} \hfill
  \begin{itemize}
    \item $a \equiv b \pmod{p} \implies \left( \frac{a}{p} \right) = \left( \frac{b}{p} \right)$.
  \end{itemize}
\end{prop}

\subsection{Primary decomposition}
\begin{definition} \hfill
  \begin{itemize}
    \item The radical of an ideal $I$ is defined by $\sqrt{I} = \Set{ a \in R \mid a^n \in I \text{ for some } n \in \Nb}$.
    \item $I$ is radical if $\sqrt{I} = I$.
  \end{itemize}
\end{definition}

\begin{definition}
  The {\bf nilradical}\index{nilradical} is defined as $\sqrt{\langle 0 \rangle} \triangleq
  \Set{ a \in R \mid a^n = 0 \text{ for some } n \in \Nb}$.
  Elements in it are called nilpotent.
\end{definition}

\begin{prop}
  $\sqrt{ \langle 0 \rangle } = \bigcap_{P \in \Spec R} P$, where $\Spec R$ is the set of prime
  ideals in $R$.

  \begin{proof}
    ``$\subset$'': Notice that $a^n = 0 \in P$ for any prime ideal $P$. By the definition of
    prime ideal, either $a \in P$ or $a^{n-1} \in P$. No matter which, eventually we would get
    $a \in P$.

    ``$\supset$'':
    Let $\mathcal{S} \triangleq \Set{ I : \text{ ideal in } R \mid a^n \not\in I, \, \forall n \in \Nb}$.
    By the routine argument of Zorn's lemma, exists maximal element $Q$ in $\mathcal{S}$.
    We claim that $\mathcal{S}$ is a prime ideal.

    For each $x, y \not\in Q$, we have $Q + Rx \supsetneq Q$ and $Q + Ry \supsetneq Q$.
    By the maximality of $Q$, these two ideals are not in $\mathcal{S}$.
    So exists $n, m$ such that $a^n \in Q + Rx,\, a^m \in Q + Ry$ which implies
    $a^{n+m} \in Q + Rxy$, so $Q + Rxy \not\in \mathcal{S}$, thus $xy \not\in Q$,
    hence $Q$ is prime.
  \end{proof}
\end{prop}

\begin{coro} \label{coro:equation-of-sqrt-ideal}
  \[ \sqrt{I} = \bigcap_{\substack{P \supset I \\ P \in \Spec R}} P \]

  \begin{proof}
    Notice that $\Spec \quot{R}{I} = \Set{P \in \Spec R \mid R \subset I}$.
    By the proposition above,
    \[ \sqrt{\langle \bar0 \rangle} = \bigcap_{\bar{P} \in \Spec \quot{R}{I}} \bar{P}
      \quad \implies \quad \sqrt{I} = \bigcap_{\substack{P \supset I \\ P \in \Spec R}} P \]
    \end{proof}
\end{coro}

\begin{definition}
  An ideal $q$ of $R$ is called primary if $q \neq R$ and ``$xy \in q$ and $x \not\in q$''
  implies $y^n \in q$ for some $n \in \Nb$.
\end{definition}

\begin{prop} \hfill
  \begin{itemize}
    \item $\text{prime} \implies \text{primary}$.
    \item $\sqrt\text{primary} \implies \text{prime}$. Also, if $q$ is primary, then $p = \sqrt{q}$
      is the smallest prime ideal containing $q$, we say $q$ is $p$-primary.
  \end{itemize}

  \begin{proof}
    The first one is obvious.

    If $q$ is primary and $\sqrt{q} = p$. For any $xy \in p$ and $x \not\in p$,
    there exists $n$ so that $x^n y^n \in q$, and for this $n$, $x^n \not\in q$.
    Thus $(y^n)^m \in q$ for some $m$, hence $y \in p$. We conclude that $p$ is a prime ideal.

    Finally, by corollary~\ref{coro:equation-of-sqrt-ideal},
    \[ p = \sqrt{q} = \bigcap_{\substack{P \supset q \\ P \in \Spec R}} P \subset P, \quad \forall P \text{ prime }, \]
    thus $p$ is indeed the smallest.
  \end{proof}
\end{prop}

\begin{example}
  The primary ideals in $\Zb$ are $\langle 0 \rangle$ and $\langle p^m \rangle$
  where $p$ is a prime.

  \begin{proof}
    If $q = \langle a \rangle$ is primary, then $\sqrt{q} = \langle p \rangle$ is
    prime, and $p^n \in \langle a \rangle$. So $ab = p^n$ which implies $a = p^m$
    for some $m$.
  \end{proof}
\end{example}

\begin{definition}
  An ideal $I$ is said to be {\bf irreducible} \index{Ideal!irreducible}
  if $I = q_1 \cap q_2 \implies I = q_1 \lor I = q_2$.
\end{definition}

\begin{definition}
  Define $(I: x) = \Set{ a \in R \mid ax \in I}$.
\end{definition}

\begin{theorem} \label{thm:noeth-irr-ideal-is-primary}
  In a Noetherian ring $R$, every irreducible ideal $I$ is primary.

  \begin{proof}
    Let $xy \in I$ and $x \not\in I$. Consider $(I : y) \subseteq (I: y^2) \subseteq \dotsm$.
    Since $R$ is Noetherian, exists $n$ such that $(I: y^n) = (I: y^m)$ for any $m \geq n$.

    We claim that $I = (I + Ry^n) \cap (I + Rx)$.
    \begin{itemize}
      \item ``$\subset$'': Obvious.
      \item ``$\supset$'': For any $b \in (I + ry^n) \cap (I + Rx)$,
        write $b = a_1 + r_1 y^{n} = a_2 + r_2 x$. Then
        $r_1 y^{n+1} = a_2 y - a_1 y + r_2 x y \in I$ since $a_1, a_2, xy \in I$.
        So $r_1 \in (I: y^{n+1}) = (I: y_n) \implies r_1 y^n \in I$.
        Thus $b = a_1 + r_1 y^n \in I$.
    \end{itemize}

    Now by the fact that $I$ is irreducible and $I \neq I + Rx$ since $x \not\in I$,
    thus $I = I + Ry^n \implies y^n \in I$.
  \end{proof}
\end{theorem}

\begin{theorem} \label{thm:noeth-ideal-is-finite-intersection}
  In a Noetherian ring $R$, every ideal is a finite intersection of irreducible ideals.

  \begin{proof}
    If not, let $\mathcal{I} \triangleq \Set{I: \text{ ideal in } R \mid I \text{ is not a finite intersection
        of irreducible ideals }}$ and $\mathcal{I}$ is not an empty set.
    Since $R$ is Noetherian, the set has a maximal element $I_0$. Then $I_0$ is not
    irreducible (or else it is an intersection of itself, which is irreducible).
    Write $I_0 = I_1 \cap I_2$, with $I_1, I_2 \neq I_0$. Then $I_1, I_2 \not\in \mathcal{I}$,
    so these two ideals could be written as a finite intersection of irreducible ideals,
    implying that $I_0$ could also be written as a finite intersection of irreducible ideals,
    which is an contradiction.
  \end{proof}
\end{theorem}

\begin{prop} \label{prop:primary-divide-by-element}
  Let $q$ be a $p$-primary ideal and $x \in R$.
  \begin{enumerate}
    \item If $x \in q$, then $(q: x) = R$.
      \begin{proof}
        In this case $1 \in (q: x)$, thus $(q: x) = R$.
      \end{proof}
    \item If $x \not\in q$, then $(q: x)$ is $p$-primary.
      \begin{proof}
        For any $y \in (q: x)$, $xy \in q$ but $x \not\in q$, thus $y^n \in q \implies y \in p$.
        Hence
        \[ q \subset (q: x) \subset p \implies p = \sqrt{q} \subset \sqrt{(q: x)} \subset \sqrt{p} = p \]
        and thus $(q: x)$ is $p$-primary.

        For any $y, z$ with $yz \in (q: x)$ but $y \not\in (q: x)$, which is equivalent
        to $xyz \in q$ but $xy \not\in q$. Since $q$ primary, $z^n \in q \subset (q: x)$.
      \end{proof}
    \item If $x \not\in p$, then $(q: x) = q$.
      \begin{proof}
        \[
          \left\{ \begin{array}{l}
            y \in (q: x) \\
            x \not\in p \\
          \end{array} \right. \implies
          \left\{ \begin{array}{l}
            xy \in (q: x) \\
            x^n \not\in q, \ \forall n \in \Nb \\
          \end{array} \right. \implies y \in q
        \]
      \end{proof}
  \end{enumerate}
\end{prop}

\begin{prop}  \label{prop:intersection-of-primary-is-primary}
  If each $q_i$ are $p$-primary, then $q \triangleq \cap_{i = 1}^n q_i$ is $p$-primary.

  \begin{proof}
    We check that $\sqrt{q} = \bigcap_{i = 1}^n \sqrt{q_i} = \bigcap_{i = 1}^n p = p$.

    Also, if $xy \in q$ with $x \not\in q$, then $x \not\in q_k$ for some $k$.
    But $xy \in q_k$, thus $y^n \in q_k$. Since $\sqrt{q} = q_k$, $(y^n)^{m'} = y^m \in p \subset q$,
    thus $q$ is $p$-primary.
  \end{proof}
\end{prop}

\begin{definition}
  A {\bf primary decomposition} of $I = q_1 \cap \dots \cap q_n$ is {\bf minimal} if $\sqrt{q_1}, \dots, \sqrt{q_n}$
  are distinct and $q_i \not\supseteq \bigcap_{j \neq i} q_j$.
\end{definition}

A minimal primary decomposition of an ideal always exists in Noetherian ring since by
theorem~\ref{thm:noeth-ideal-is-finite-intersection}, the ideal could be written
as a finite intersection of irreducible ideals, and then by theorem~\ref{thm:noeth-irr-ideal-is-primary},
these ideals are primary. Now If $\sqrt{q_i} = \sqrt{q_j}$ happen in these ideal,
we could remove these two ideals and add $q' = \sqrt{q_i} \cap \sqrt{q_j}$.
By proposition~\ref{prop:intersection-of-primary-is-primary}, $q'$ is also primary.
And if $q_i \subseteq \bigcap_{j \neq i} q_j$, we could simply remove $q_i$.

\medskip

\begin{theorem}[Uniqueness of primary decomposition]
  Let $I = \cap_{i = 1}^n q_i$ be a minimal decomposition of $I$.
  If $p_i = \sqrt{q_i}, \, \forall i$, then we have
  \[ \Set{p_i} = \Big\{ \sqrt{(I: x)} \Bigm| x \in R \land \sqrt{(I: x)} \in \Spec R \Big\} \]
  which is independent of the decomposition.

  \begin{proof}
    ``$\supset$'': Let $x \in R \setminus I$, then $(I: x) = \big( \bigcap_{i=1}^n q_i : x \big)
    = \bigcap_{i = 1}^n (q_i: x)$. By proposition~\ref{prop:primary-divide-by-element},
    we have $\sqrt{(I: x)} = \bigcap \sqrt{(q_i: x)} = \bigcap_{x \not\in q_i} p_i$.

    Now, we have the following observation. ``If $p \in \Spec R$ with $p = \bigcap_{i=1}^n J_i$,
    then $p = J_j$ for some $j$.'' If not, then $J_i \not\subset p$ for all $i$,
    so we could pick $x_i \in J_i \setminus p$.
    But then $x_1 x_2 \dotsm x_n \in \cap J_i \in p$ since $J_i$ are ideals,
    which leads to a contradiction since $p$ is prime.

    So if $\sqrt{(I: x)}$ is a prime, then it is equal to some $p_i$.

    ``$\subset$'': By assumption, $q_i \not\subseteq \bigcap_{j \neq i} q_j$ for each $i$,
    thus we could pick $x \in \bigcap_{j \neq i} q_j \setminus q_i$,
    then $\sqrt{(I: x)} = \bigcap_j \sqrt{(q_j: x)} = \sqrt{(q_i: x)} = p_i$.
  \end{proof}
\end{theorem}

\begin{definition}
  If $\Set{p_i}$ is the unique prime ideals from the minimal primary decomposition of $I$.
  \begin{itemize}
    \item $\Set{p_i}$ is said to be associated with $I$ or to belong to $I$.
    \item The minimal elements in $\Set{p_i}$ are called isolated primes.
    \item The other are called embedded primes.
  \end{itemize}
\end{definition}

\begin{example}
  Let $R = k[x, y]$ and $I = \langle x^2, xy \rangle$. If $P_1 = \langle x \rangle,
  P_2 = \langle x, y \rangle$, then $I = P_1 \cap P_2^2$.
  $P_1$ is isolated, while $P_2$ is embedded.
\end{example}

%! TEX root=../main.tex
\subsection{The equivalence of algebra and geometry}

\begin{definition}
  Let $k$ be an algebraically closed field.
  The category of affine algebraic sets $\mathbb{g}$, which its objects and morphisms are defined as following.

  An affine algebraic set is the common zero set in $k^n$ of $\{ F_i \}_{i \in \Lambda} \subset k[x_1, \dots, x_n]$.
  We denote it by $V = v(\Set{F_i}_{i \in \Lambda}) \subset k^n$. (In fact, $V = v(\langle F_i : i \in \Lambda \rangle)
  = I = \langle F_1, \dots, F_n \rangle$.)

  These objects define closed sets in a topology, called the Zariski topology. We denote the
  topological space $k^n$ by $\mathbb{A}^n_k$.

  The morphisms is defined by
  \[ \deffunc{}{k^n}{k^m}{\alpha}{(F_1(\alpha), \dots, F_m(\alpha))} \text{ where } F_i \in k[x_1, \dots, x_n] \]
  
  Given two affine algebraic sets $V \subset k^n$ and $W \subset k^m$, a map $F: V \to W$ is a morphism
  if it is the restriction of a polynomial map from $k^n$ to $k^m$.
  ($V \cong W$ if $F: V \to W$ and $G : W \to V$ satisfy $F \circ G = \Id$ and $G \circ F = \Id$.)
\end{definition}



\begin{definition}
  The category of finitely generated reduced $k$-algebra $\mathcal{A}$.
  A finitely generated $k$-algebra $R$ is reduced if $R$ has no nilpotent elements.

  \[
    \begin{array}{ccc}
      \Set{ \text{affine algebraic sets in } \mathbb{A}^n_k } & \leftrightarrow & \Set{ \text{ radical ideals in } k[x_1, \dots, x_n]}
    \end{array}
  \]
\end{definition}

\begin{lemma}
  Given $T/S/R$. If $T/S$ is a module finite and $T/R$ is a ring finite, then $S/R$ is a ring finite.
\end{lemma}

\begin{lemma}
  If $S = k(z_1, \dots, z_p), \, p > 0$ with each $z_i$ variable, then $S/k$ is not ring finite.
\end{lemma}

\begin{lemma}
  If $A/k$ is an extension of fields and ring finite, then $A/k$ is algebraic.
\end{lemma}

\begin{theorem}[Hilbert Nullstellensatz]
  \[ I \subsetneq k[x_1, \dots, x_n] \implies v(I) \neq \varnothing \]
\end{theorem}

\begin{theorem}
  $\mathcal{I}(v(I)) = \sqrt{I}$
\end{theorem}

\begin{definition}
  Let $V \in \mathcal{G}$, the coordinate ring of $V$ is $k[V] \triangleq k[x_1, \dots, x_n] / \mathcal{I}(V)$
\end{definition}

%! TEX root=../main.tex
\subsection{Gr\"{o}bner basis}

\subsubsection{Division algorithm in $K[X_1,\dots,X_n]$}

\begin{example}
  $I = \langle xy-1,y^2-1\rangle \subseteq K[x,y]$, $f_1 = xy-1$ and $f_2 = y^2-1$ $G=\{f_1,f_2\}$. Does $f = x^2y+xy^2+y^2 \in I?$
  \begin{itemize}
      \item Choose a lexicographic monomial ordering: $x > y$
      \item The multidegree $\partial(f) = (2,1)$, $\partial(f_1) = (1,1)$, $\partial(f_2) = (0,2)$
      \item The leading term $LT(f) = x^2y$, $LT(f_1) = xy$, $LT(f_2) = y^2$
      \item $LT(f) = xLT(f_1) \Rightarrow f = xf_1+xy^2+y^2+x \Rightarrow f = \underset{h_1}{(x+y)}f_1+\underset{h_2}{(1)}f_2+\underset{\bar{f}^G}{(x+y+1)}$ or $f = \underset{h_1}{x}f_1+\underset{h_2}{(x+1)}f_2+\underset{\bar{f}^G}{(2x+1)}$.
  \end{itemize}
  Note:  Divisor $h_1$, $h_2$ and remainder $\bar{f}^G$ are not unique!! 
\end{example}

\begin{definition}
  Fix a monomial ordering and let I be an ideal of $K[X_1,\dots,X_n]$. The ideal of leading terms in I is defined to be $LT(I) = \langle LT(f)|f\in I \rangle$.
\end{definition}

\begin{remark}
  Let $I = \langle f_1,\dots,f_n \rangle$. In general, $\langle LT(f_1),\dots,LT(f_n) \rangle \subsetneq LT(I)$.
\end{remark}

\begin{example}
  Let $f_1=xy^2+y$, $f_2=xy^2$. And, $xf_1+yf_2=xy \in \langle f_1,f_2 \rangle$ but $xy \notin \langle xy^2, x^2y \rangle$.
\end{example}

\begin{definition}
  $G = \Set{g_1,\dots,g_m}$ is called a Gr\"{o}bner basis of I if $I = \langle g_1,\dots,g_m \rangle$ and $LT(I) = \langle LT(g_1),\dots,LT(g_m) \rangle$
\end{definition}

\begin{prop} \label{prop:Leading-term-ideal-equal-so-does-ideal}
  $LT(I) = \langle LT(g_1),\dots,LT(g_m) \rangle$, and $\langle g_1,\dots,g_m \rangle \subseteq I$ $ \Rightarrow I = \langle g_1,\dots,g_m \rangle$
  \begin{proof}
    $\forall f \in I$, do division process. Then $f = \overset{m}{\underset{i = 1}{\sum}} h_ig_i+r$, either $r=0$ or \uline{$\bigstar = \text{no term of r is divisible}$ by any of $LT(g_1),\dots,LT(g_m)$}. Assume $r \neq 0$, then $r = f - \overset{m}{\underset{i = 1}{\sum}} h_i g_i \in I \Rightarrow LT(r) \in LT(I) = \langle LT(g_1),\dots, LT(g_m) \rangle$, a contradiction. Hence, r = 0 (i.e. $f\in \langle g_1,\dots,g_m \rangle$).
  \end{proof}
\end{prop}

\begin{theorem} \label{thm:Grobner_existense}
  Each ideal I has a Gr\"{o}bner basis.
  \begin{proof}
    By Hilbert basis thm, $LT(I) = \langle f_1,\dots,f_m \rangle$ for some $f_i$'s. Write $f_i = \overset{m_i}{\underset{j = 1}{\sum}}h_{ij}LT(g_{ij})$ $h_{ij} \in K[X_1,\dots,X_m]$, $g_{ij} \in I$. Then $LT(I) = \langle LT(g_{ij}) \rangle$ $i=1,\dots,m$ $j=1,\dots,m_i$. By prop~\ref{prop:Leading-term-ideal-equal-so-does-ideal}, This is Gr\"{o}bner basis.
  \end{proof}
\end{theorem}

\begin{theorem} \label{thm:Grobner_property}
  Let $G = \Set{g_1,\dots,g_m}$ be a Gr\"{o}ner basis of I, then
  \begin{itemize}
    \item $\forall f \in K[X_1,\dots,X_n]$, $f = f_I + r$ where $f_I,r$ are unique.
      \begin{proof}
        By division algorithm, $f = f_I +\underset{\bigstar}{r} = f'_I+\underset{\bigstar}{r'}$, then $\underset{\bigstar}{r-r'} = f_I-f'_I$. But if $r-r' \neq 0$, then $LT(r-r') \in LT(I) = \langle LT(g_1,\dots,g_2) \rangle$, a contradiction. Hence, $r-r' = 0\Rightarrow f_I = f'_I$.
      \end{proof}
    \item $f \in I \iff r=0$.
      \begin{proof}
        Suppose $f \in I$, then $f = f_I + \underset{\bigstar}{r}$, and if $r\neq 0$ $\underset{\bigstar}{r} = f - f_I\in I$, a contradiction. Hence, r = 0. Conversly, if $r = 0$, $f = f_I \in I$. 
      \end{proof}
  \end{itemize}
\end{theorem}


\subsubsection{Buchberger's algorithm}

\begin{definition}
  Let $f,g \in K[X_1,\dots,X_n]$ and $M$ be the monic least common multiple of LT(f) and LT(g). $S(f,g) = \frac{M}{LT(f)}f- \frac{M}{LT(g)}g$ is called an S-polynomial of f, g.
\end{definition}

Let $I = \langle g_1, \dots, g_m \rangle$ and $G = \Set{ g_1, \dots, g_m }$.
A Gr\"{o}ner basis of $I$ can be constructed by the following algorithm:
\begin{enumerate}
  \item Initially let $G_0 \gets G$.
  \item Repeatly construct $G_{i+1} \gets G_i \cup \big( \Set{S(f, g) \mod G_i \mid f, g \in G_i} \setminus \Set{0} \big)$,
    until once $G_{i+1} = G_i$, then $G_i$ is a Gr\"{o}ner basis.
\end{enumerate}

\begin{lemma} \label{lemma:sum-of-equal-degree-f-is-less}
  Let $f_1, \dots, f_m \in K[x_1, \dots, x_n]$ with $a_1, \dots, a_m \in K$ satisfy
  $\partial(f_1) = \partial(f_2) = \dots = \partial(f_m) = \alpha$ and $h =\sum_{i = 1}^m a_i f_i $ with $\partial(h) < \alpha$.
  Then $h = \sum_{i = 2}^m b_i S(f_{i-1}, f_i)$ for some $b_i \in K$.
  \begin{proof}
    Write $f_i = c_if'_i$ with $c_i \in K$ and $f'_i$ being monic of multidegree = $\alpha$. Note: $S(f_i, f_j) = f'_1 - f'_2$, since all multidegree are equal. Then, 
    \begin{equation}
      \begin{split}
        h &= \overset{m}{\underset{i = 1}{\sum}} \big( a_ic_if'_i \big) \\
        &= a_1c_1(f'_1-f'_2) + (a_1c_1+a_2c_2)(f'_2-f'_3) + \dots+ (a_1c_1 + \dots + a_{m-1}c_{m-1})(f'_{m-1}-f'_m) \\
        &+ (a_1c_1+\dots+a_mc_m)f'_m \\
        &= \overset{m}{\underset{i = 2}{\sum}}b_iS(f_{i-1},f_i) + b_{m+1}f'_m\text{ with }b_i = \sum_{1}^{i-1}a_1c_1.
      \end{split}
    \end{equation}
      Also, $b_{m+1} = 0$ , since $\Big\{\partial(h)$, $\partial(\overset{m}{\underset{i = 2}{\sum}}b_iS(f_{i-1},f_i) ) \Big\} < \alpha$ and $\partial(f'_m) = \alpha$ (By direct comparison multidegree). Then, we have $h = \sum_{i = 2}^m b_i S(f_{i-1}, f_i)$.
  \end{proof}
\end{lemma}

\begin{theorem}[Buchberger's criterion]
  Assume $I = \langle g_1, \dots, g_m \rangle$, then
  $G = \Set{g_1, \dots, g_m}$ is a Gr\"{o}bner basis of $I$ $\iff$ $S(g_i, g_j) \equiv 0 \pmod{G}$ for each $i, j$.
  \begin{proof}
    \begin{description} [leftmargin=0cm,labelindent=0cm]
      \item
      \item[$\cdot$] Suppose G is a Gr\"{o}bner basis of I. $S(g_i, g_j) \in I \Rightarrow S(g_i, g_j) \equiv 0$(mod G) by thm~\ref{thm:Grobner_property}.
      \item[$\cdot$] Converely, suppose $S(g_i, g_j) \equiv 0$ (mod G) $\forall i, j$. For $f \in I$, $f \underset{not\ division}{=} \overset{m}{\underset{i = 1}{\sum}}h_ig_i$ for some $h_i \in K[X_1,\dots,X_2]$. Define $\alpha = max\{\partial(h_1g_1),\dots,\partial(h_mg_m)\}$. We have $\partial(f) \leq \alpha$ and we can select an expression $f = \overset{m}{\underset{i = 1}{\sum}}h_ig_i$ for f s.t $\alpha$ is minimal.
      \item[] \uline{Claim}:   $\partial(f) = \alpha$
      \item (pf) Rewrite,
          \begin{equation}
            \begin{split}f &= \overset{m}{\underset{i = 1}{\sum}}h_ig_i \\
            &= \underset{\partial(h_ig_i) = \alpha}{\sum}h_ig_i +  \underset{\partial(h_ig_i) < \alpha}{\sum}h_ig_i \text{ \ \ Note:}  \underset{\partial(h_ig_i) = \alpha}{\sum}h_ig_i \neq 0 \text{ , since } \alpha \text{ is minimal.}\\
            &= \underset{\partial(h_ig_i) = \alpha}{\sum}LT(h_i)g_i + \underset{\partial(h_ig_i) = \alpha}{\sum}(h_i-LT(h_i)g_i) + \underset{\partial(h_ig_i) < \alpha}{\sum}{h_ig_i}
            \end{split}
          \end{equation}
          Let $LT(h_i) = a_ih_i^0$ with $h_i^0$ being a monic monomial. Comparing the multidegree on both side, $\partial\left(\underset{\partial(h_ig_i) = \alpha}{\sum}a_ih_i^0g_i\right) < \alpha$ By lemma~\ref{lemma:sum-of-equal-degree-f-is-less}, $\underset{\partial(h_ig_i) = \alpha}{\sum}\left(a_ih_i^0g_i\right) = c_{12}S(h_{i_1}^0g_{i_1},h_{i_2}^0g_{i_2}) + \dots$(finite) where $\partial(h_{i_1}g_{i_1}) = \partial(h_{i_2}g_{i_2}) = \dots = \alpha $. By def, if we set $\beta_{st} = M_{st} =$ the monic lcm of $LT(g_{i_s}), LT(g_{i_t})$, then
          \begin{equation}
            \begin{split}
              S(h_{i_s}^0g_{i_s}, h_{i_t}^0g_{i_t}) &= \frac{X^\alpha}{LT(h_{i_s}^0g_{i_s})}h_{i_s}^0g_{i_s} - \frac{X^\alpha}{LT(h_{i_t}g_{i_t})}h_{i_t}^0g_{i_t} \\
              &= X^{\alpha-\beta_{st}} \left( \frac{X^{\beta_{st}}}{\bcancel{h_{i_s}^0}LT(g_{i_s})}\bcancel{h_{i_s}^0}g_{is} - \frac{X^{\beta_{st}}}{\bcancel{h_{i_t}^0}LT(g_{i_t})}\bcancel{h_{i_t}^0}g_{i_t}  \right) \\
              &= X^{\alpha-\beta_{st}} S\left(g_{i_s},g_{i_t}\right) \\
              &= X^{\alpha-\beta_{st}}\overset{m}{\underset{j = 1}{\sum}}{l_jg_j} \text{ (by division)}
            \end{split}
          \end{equation}
      \item Then, $\partial(l_jg_j) < \beta_{st} \Rightarrow \partial(f) \geq \alpha$. Therefore, $\partial(f) = \alpha \Rightarrow LT(f) = \underset{\partial(h_ig_i) = \alpha}{\sum}LT(h_i)LT(g_i) \Rightarrow LT(f) \in \langle LT(g_1,\dots, LT(g_m \rangle$.
 
    \end{description}
  \end{proof}
\end{theorem}

\begin{theorem}
  The Buchberger's algorithm will terminate
  \begin{proof}
    $.$
    \begin{itemize}
      \item $\langle LT(G_i) \rangle \subsetneq \langle LT(G_{i+1}) \rangle$ if $G_i \neq G_{i+1}$
        \[
          G_i \neq G_{i+1} \Rightarrow \exists f, g \in G_i \text{ s.t. } S(f,g) \cancel{\equiv} 0 \text{(mod G) } \Rightarrow LT(S(s,g)) \notin \langle LT(G_i) \rangle
        \]
      \item $\langle LT(G_0) \rangle \subsetneq \langle LT(G_1) \rangle \subsetneq \dots$ (Noetherian ACC condition).
    \end{itemize}
  \end{proof}
\end{theorem}

\subsection{Applications of Gr\"{o}bner basis}
\begin{definition}
  Let $I \subseteq k[x_1, \dots, x_n]$ and $x_1 > x_2 > \dotsm > x_n$.
  $I_i \triangleq I \cap K[x_{i+1}, \dots, x_n]$ is called the $i$-th elimination ideal of $I$.
\end{definition}

\begin{theorem}[Elimination theorem]
  Let $G = \Set{g_1, \dots, g_m}$ be a Gr\"obner basis of $I \neq 0$ with ordering
  $x_1 > \dotsm > x_n$. Then $G_i \triangleq G \cap K[x_{i+1}, \dots, x_n]$
  is a Gr\"obner basis of $I_i$ (i.e., $\langle \LT(G_i) \rangle = \LT(I_i)$).
\end{theorem}

\begin{example}
  Find $V = \Vc(x+y-z, x^2+y^2-z^3, x^3+y^3-z^5)$. \\
  We compute a Gr\"obner basis of $I = \langle f_1, \dots, f_3 \rangle$ with respect to the ordering $x > y > z$.
  The Gr\"obner basis is $\Set{x+y-z, 2y^2-2yz-z^3+z^2, 2z^5-3z^4+z^3}$.
\end{example}

\begin{example}
  \[
    \begin{tikzcd}[column sep=0.8cm,row sep=0ex]
      f: &[-0.7cm] \mathbb{A}^1 \arrow[r] & \mathbb{A}^3 \\
      & t \arrow[r, mapsto] & (t^4, t^3, t^2)
    \end{tikzcd}
  \]

  We compute a Gr\"obner basis of $I = \langle t^4 - x, t^3 - y, t^2 - z \rangle$ with respect to
  $t > x > y > z$. The Gr\"obner basis is $\Set{-t^2+z, ty-z^2, tz-y, x-z^2, y^2-z^3}$.
\end{example}

\begin{example}
  \[
    \begin{tikzcd}[column sep=0.8cm,row sep=0ex]
      f: &[-0.7cm] V = \Vc(x^3 - x^2z - y^z) \arrow[r] & \mathbb{A}^3 \\
      & (x, y, z) \arrow[r, mapsto] & (x^2 z - y^2 z, 2xyz, -z^3)
    \end{tikzcd}
  \]

  The ideal is $\langle x^3 - x^2 z - y^2 z, u - x^2 z + y^2 z, v - 2 x y z, w + z^3 \rangle$
  has a Gr\"obner basis $\langle \dots, u^2 + v^2 - w^2 \rangle$.
\end{example}

\begin{theorem}
  Let $I, J$ be two ideals of $k[x_1, \dots, x_n]$, then $I \cap J = (t \tilde{I} + (1 - t) \tilde{J}) \cap K[x_1, \dots, x_n]$.
\end{theorem}

\begin{example}
  $I = \langle y^2 , x - yz \rangle, \, J = \langle x, z \rangle$. We shall find $I \cap J$. \\
  $tI + (1-t)J = \langle tx - tyz, ty^2, (1-t)x, (1-t)z \rangle$ has a Gr\"obner basis
  $\Set{f_1, f_2, f_3, f_4, xy, x - yz}$, so $I \cap J = \langle xy, x-yz \rangle$.
\end{example}

\begin{theorem}
  Let $L = \langle f_1, \dots, f_s \rangle \subsetneq k[x_1, \dots, x_n]$, then
  $f \in \sqrt{I} \iff \langle f_1, \dots, f_s, 1 - tf \rangle = k[x_1, \dots, x_n, t]$.
\end{theorem}

\begin{example}
  Let $I = \langle xy^2 + 2y^2, x^4 - 2x^2 + 1 \rangle$, and we are to determine $f = y - x^2 + 1$
  is in $\sqrt{I}$ or not.
\end{example}

\begin{prop}
  An affine algebraic set $V$ in $\mathbb{A}^n_k$ has a unique minimal decomposition.
  $V = V_1 \cap V_2 \cap \dotsm \cap V_m$ with $V_i$ irreducible and $V_i \not\subset V_j$.
\end{prop}

\begin{theorem}[Decomposition]
  Assume $\sqrt{I} = I$ and $I \subset J$, then $\Vc(I : J) = \overline{\Vc(I) \setminus  \Vc(J)}$.
  and $\Vc(I) = \Vc(J) \cup \Vc(I: J)$.
\end{theorem}

\begin{example}
  Let $I = \langle xz - y^2, x^3 - yz \rangle$ and $V = \Vc(I)$. \\

  Notice that $\langle xz - y^2 , x^3 - yz \rangle \subseteq \langle x, y \rangle = J$,
  so $(I: J) = (I: \langle x \rangle) \cap (I: \gen{y})$.

  First we calculate $(I: x)$. Notice that we know how to calculate $I \cap \gen{x}$ now.
  After a calculation, $I \cap \gen{x} = \Set{x^2z - xy^2, x^4 - xyz, x^3y - xz^2}$,
  so $(I: x) = \gen{f_1/x, f_2/x, f_3/x} = I + \gen{x^2y - z^3}$.
  Simarly one could find that $(I: y) = (I: x)$, thus $(I: J) = (I: x)$.

  Hence $V = \Vc(x, y) \cap \Vc(xz-y^2, x^3-yz, x^2y - z^2)$.
\end{example}

\begin{prop}
  In general, if $W \subseteq \mathbb{A}_k^n$ is an affine algebraic set defined by $x_i = f_i(t_1, \dots, t_m)$,
  then $W$ is irreducible.
\end{prop}

\begin{prop}
  If $f : V \to W$ with $\overline{f(V)} = \Vc(\ker f^*)$, where $f^* : k[W] \to k[V]$.
\end{prop}


\printindex
%%%%%%%%%%%%%%%%%%%%%%%%%%%%%%%%%%%%%%%%%%%%%
% \bibliographystyle{plain}
% \bibliography{journal.bib}
% \begin{thebibliography}{99}
% \bibitem[1]{ex}\url{http://www.example.com/}
% \end{thebibliography}
\end{document}
