%! TEX root=main.tex
\section{Group theory}

\subsection{Basic}

\begin{definition}
  A non-empty set $G$ with a binary function $f: G \times G \to G,
  (a, b) \mapsto ab$
  is a {\it group} if it satisfies
  \begin{enumerate}
    \item $(ab)c = a(bc)$.
    \item $\exists 1 \in G$ s.t. $1a = a1 = a, \forall a \in G$.
    \item $\exists a^{-1} \in G$ s.t. $aa^{-1} = a^{-1}a = 1$.
  \end{enumerate}
\end{definition}

CONCON

\begin{definition}
  Let $G$ be a group. Then $G$ is said to be {\it abelian} if
  $\forall a, b \in G, ab = ba$.
\end{definition}

\begin{exercise}
  Let $G$ be a semigroup. Then TFAE (the following are equivalent)
  \begin{enumerate}
    \item $G$ is a group.
    \item For all $a, b \in G$ and the equations $bx=a, yb=a$, each of them
      has a solution in $G$.
    \item $\exists e \in G$ s.t. $ae=a \; \forall a \in G$ and if we fix
      such $e$, then $\forall b \in G \; \exists b' \in G$ s.t. $bb' = e$.
  \end{enumerate}
\end{exercise}

\begin{exercise}
  Let $G$ be a group. Show that
  \begin{enumerate}
    \item $\forall a \in G, a^2 = 1$, then $G$ is abelian.
    \item $G$ is abelian $\iff \forall a, b \in G, (ab)^n = a^n b^n$ for three
      consecutive integer $n$.
  \end{enumerate}
\end{exercise}

\begin{definition}
  Let $G$ be a group and $H \subseteq G, H \ne \phi$.
  Then $H$ is said to be a subgroup of $G$, denoted by $H \le G$, if
  \begin{enumerate}
    \item $\forall a, b \in H, ab \in H$.
    \item $1 \in H$.
    \item $\forall a \in H, a^{-1} \in H$.
  \end{enumerate}
  \underline{useful criterion}:
  $H \le G \iff \forall a, b \in H, ab^{-1} \in H$.
  \begin{proof} \mbox{}
    \begin{description}[style=nextline]
      \item[$\Rightarrow$] $b \in H \implies b^{-1} \in H$, and $a \in H$, so
        $a b^{-1} \in H$.
      \item[$\Leftarrow$]
        \begin{enumerate}
          \item $H \ne \phi \implies \exists a \in H \implies
            aa^{-1} = 1 \in H$.
          \item $1, a \in H \implies 1 a^{-1} = a^{-1} \in H$.
          \item $a, b^{-1} \in H \implies a (b^{-1})^{-1} = ab \in H$.
            \qedhere
        \end{enumerate}
    \end{description}
  \end{proof}
\end{definition}

\begin{example}
  $(\Zb, +, 0) \le (\Qb, +, 0) \le (\Rb, +, 0) \le (\Cb, +, 0)$ ;
  $(\Qbx, \times, 1) \le (\Rbx, \times, 1) \le (\Cbx, \times, 1)$
\end{example}

\begin{example} \mbox{}
  \begin{itemize}
    \item Special linear group $\text{SL}(n, \Fb) = \{\, A \in
      \text{GL}(n, \Fb) \mid \det A = 1 \,\}$
    \item Orthogonal group $\text{O}(n) = \{\, A \in \text{GL}(n, \Rb)
      \mid \tr{A} A = I_n \,\}$
    \item Unitary group $\text{U}(n) = \{\, A \in \text{GL}(n, \Cb) \mid
      A^* A = I_n \,\}$
    \item Special orthogonal group $\text{SO}(n) = \text{SL}(n, \Rb) \cap
      \text{O}(n)$
    \item Special unitary group $\text{SU}(n) = \text{SL}(n, \Cb) \cap
      \text{U}(n)$
  \end{itemize}
\end{example}

\begin{definition}
  Let $f: G_1 \to G_2$. f is called an {\it isomorphism} if
  \begin{enumerate}
    \item $f$ is 1-1 and onto.
    \item $\forall a, b \in G_1, f(ab) = f(a)f(b)$. ({\it homomorphism})
  \end{enumerate}
  , denoted by $G_1 \cong G_2$.
\end{definition}

\begin{remark} (practice)
  \begin{enumerate}
    \item $f(1) = 1$.
    \item $f(a^{-1}) = f(a)^{-1}$.
    \item If $f$ is an isomorphism, then $\exists f^{-1}$ is also a
      homomorphism.
  \end{enumerate}
\end{remark}

\begin{example} \mbox{}
  \begin{itemize}
    \item $\text{U}(1) = \{\, z \in \Cbx \mid \bar{z}z = 1 \,\},
      z = \cos \theta + \sin \theta i$
    \item $\text{SO}(2) = \left\{ \begin{pmatrix}
        \cos \theta & - \sin\theta \\
        \sin \theta & \cos\theta
      \end{pmatrix} : \theta \in \Rb \right\}$
  \end{itemize}
  notice that $\text{U}(1) \cong \text{SO}(2)$.
  $S^1 = \{\, (a, b) \in \Rb^2 \mid a^2 + b^2 = 1 \,\}$, 可被賦予群的結構.
\end{example}

\begin{example}
  Let $A \in \text{SU}(2) \implies A = \begin{pmatrix}
    \alpha & \beta \\
    -\bar{\beta} & \bar{\alpha} \end{pmatrix},
    \alpha\bar{\alpha} + \beta\bar{\beta} = 1, \alpha, \beta \in \Cb$.
\end{example}

\underline{Quaternion}(四元數): $\Hb = \{\,
  a + bi + cj + dk \mid a, b, c, d \in \Rb \,\}$ with $i^2 = j^2 = k^2 = -1,
  ij = k, jk = i, ki = j (\implies ij = -ji)$.

Let $x = a + bi + cj + dk, \bar{x} = a - bi - cj - dk$, then
$N(x) = x\bar{x} = a^2 + b^2 + c^2 + d^2$, For $x \ne 0, N(x) \ne 0,
x^{-1} = \frac{1}{N(x)} \bar{x}$

Now, for $x = a + bi + cj + dk = (a + bi) + (c + di) j$.
So $\text{SU}(2) \cong \{\, x \in \Hbx \mid N(x) = 1 \,\}$.
$S^3 = \{\, (a, b, c, d) \in \Rb^4 \mid a^2 + b^2 + c^2 + d^2 = 1 \,\}$,
可被賦予群的結構.

$\bigstar$ The only spheres with continuous group law are $S^1, S^3$.

\begin{exercise}
  Find a way to regard $M_{n\times n}(\Hb)$ as a subset of
  $M_{2n \times 2n}(\Cb)$, which preserves addition and multiplication,
  and then there is a way to characterize $\text{GL}(n, \Hb)$.
\end{exercise}

\begin{definition}[symplectic group]
  $\text{Sp}(n, \Fb) = \{\, A \in \text{GL}(2n, \Fb) \mid \tr{A} JA = J \,\}$
  where $J = \begin{pmatrix} O & I_n \\ -I_n & O \end{pmatrix}$.
  ($\tr{A} JA = J$ preserving non-degenerate skew-symmetric forms)

  $\text{Sp}(n) = \{\, A \in \text{GL}(n, \Hb) \mid A^* A = I_n \,\}$.
\end{definition}

\begin{exercise}
  Show $\text{Sp}(n) \cong \text{U}(2n) \cap \text{Sp}(n, \Cb)$.
\end{exercise}

\underline{Ques}: Find the smallest subgroup of $\text{SU}(2)$ containing
$\begin{pmatrix}i & 0 \\ 0 & -i\end{pmatrix}$.

\subsection{Permutation groups and Dihedral groups}
\begin{definition}
  A permutation of a set $B$ is a 1-1 and onto function from $B$ to $B$.

  Let $S_B \defeq \text{the set of permutations of $B$}$. Then
  $(S_B, \cdot, {\rm Id}_B)$ forms a group.

  If $B = \{a_1, \dots, a_n\}$, then $S_B \cong S_{\{1,\dots,n\}}$ and write
  $S_n = S_{\{1,\dots,n\}}$, called the symmetric group of degree $n$.
\end{definition}

\begin{theorem}[Cayley theorem]
  Any group is isomorphic to a subgroup of some permutation group.
  
  (Hint): Let $G$ be a group. Set $B = G$. Consider $a \in G$ as
  $\sigma_a: G \to G, x \mapsto ax$.
  Then $\sigma_a \in S_G \implies G \le S_G$.
\end{theorem}

\begin{fact}
  $S_n$ is a finite group of order $n!$, i.e. $\abs{S_n} = n!$.
  \begin{proof}
    EASY =O
  \end{proof}
\end{fact}

\underline{Cyclic notation}: $\sigma \in S_5$, say $\sigma = \begin{pmatrix}
  1 & 2 & 3 & 4 & 5 \\
  4 & 3 & 5 & 1 & 2
\end{pmatrix}$.
Write $\sigma = \cycle{1, 4}\cycle{2, 3, 5}$.

$\Rightarrow$ Any permutation can be written as a product of disjoint cycles.

\begin{example}
  In $S_7$, $\sigma_1 = \cycle{1,2,3}\cycle{4,5,6}\cycle{7},
  \sigma_2 = \cycle{1,3,5,6}\cycle{2,4,7}$.

  Then $\sigma_1\sigma_2 = \cycle{2,5,4,7,3,6},
  \sigma_1^{-1} = \cycle{1,3,2}\cycle{4,6,5}$.
\end{example}

\begin{definition}
  A 2 cycle is called a {\it transposition}.
\end{definition}

\begin{example}
  $\cycle{1,2,3} = \cycle{1,3}\cycle{1,2},
  \cycle{1,2,3,4,5} = \cycle{1,5}\cycle{1,4}\cycle{1,3}\cycle{1,2}$.

  Any permutation is a product of 2 cycles.
\end{example}

\underline{Useful formula}: $\sigma \in S_n$,
$\sigma \cycle{j_1, \dots, j_m} \sigma^{-1} =
\cycle{\sigma(j_1),\dots,\sigma(j_m)}$.

\begin{example}
  Let $\sigma = \cycle{1,2,3}\cycle{4,5,6,7}$,
  $\sigma \cycle{2,3,4} \sigma^{-1} = \cycle{3,1,5}$.
\end{example}

\begin{proof}
  Note that both sides are functions. For $i \in \{1,\dots,n\}$,
  \begin{enumerate}[\underline{Case \arabic*}:]
    \item $\exists k$ s.t.  $\sigma(j_k) = i$, CONCON
    \item Otherwise, CONCON
  \end{enumerate}
\end{proof}

\begin{fact}
  $S_n = \langle \cycle{1,2},\dots,\cycle{1,n} \rangle$.
  \begin{proof}
    $\cycle{1,i}^{-1} = \cycle{1,i}$ and
    $\cycle{i,j} = \cycle{1,i}\cycle{1,j}\cycle{1,i}^{-1}$.
  \end{proof}
\end{fact}

\begin{definition}
  Let $G$ be a group and $S \subset G$. The subgroup generated by $S$ defined
  to be the smallest subgroup of $G$ which contains $S$, denoted by
  $\langle S \rangle$.
\end{definition}

\begin{exercise} \mbox{}
  \begin{enumerate}
    \item $S_n = \langle \cycle{1,2},\cycle{2,3},\dots,\cycle{n-1,n} \rangle$.
    \item $S_n = \langle \cycle{1,2},\cycle{1,2,\dots,n} \rangle$.
  \end{enumerate}
\end{exercise}

\begin{definition}
  $A_n = \{ \text{even permutations of $S_n$} \} \le S_n,
  \abs{A_n} = \frac{n!}{2}$.
\end{definition}

\begin{exercise} \mbox{}
  \begin{enumerate}
    \item $A_n = \langle\cycle{1,2,3},\cycle{1,2,4},\dots,\cycle{1,2,n}\rangle, n \ge 3$.
    \item $A_n = \langle\cycle{1,2,3},\cycle{2,3,4},\dots,\cycle{n-2,n-1,n}\rangle, n \ge 3$.
  \end{enumerate}
\end{exercise}

\begin{remark}
  $\langle S \rangle = \bigcap\limits_{S \subseteq H \le G} H =
  \{ a_1a_2\dots a_k \mid k \in \Nb, a_i \in S \cup S^{-1}\} \cup \{1\}$
\end{remark}

The orthogonal transformations on $\Rb^2$: $\text{O}(2)$.

Let $A = \begin{pmatrix}a_1 & a_2 \\ b_1 & b_2\end{pmatrix} \in \text{O}(2)$.

略... (這邊討論旋轉和反射的矩陣)

\begin{enumerate}[\underline{Case \arabic*}:]
  \item $A = \begin{pmatrix}
      \cos\alpha & -\sin\alpha \\
      \sin\alpha & \cos\alpha
    \end{pmatrix}$ is counterclockwise roration w.r.t. $\alpha$.
  \item $A = \begin{pmatrix}
      \cos\alpha & \sin\alpha \\
      \sin\alpha & -\cos\alpha
    \end{pmatrix}$ is the reflection.
    $A^2 = I_2 \implies$ eigenvalues are $\pm 1$.

    Easy to show that $\text{L}_A(v) = v - 2 \inpd{v, v_2} v_2$.
\end{enumerate}

$\text{O}(2) = \{\text{rotations}\} \cup \{\text{reflections}\}$.

\begin{definition}
  The dihedral group $D_n$ is the group of symmetries of a regular $n$-gon.

  In general, $D_n = \langle {\rm T, R} \mid
  {\rm T}^n = 1, {\rm R}^2 = 1, {\rm TR} = {\rm RT}^{-1} \rangle \le O(2)
  \le S_n, \abs{D_n} = 2n$.
\end{definition}

\begin{definition}
  Let $\rm T$ be a linear transformation from $\Rb^n \to \Rb^n$.
  \begin{itemize}
    \item $\rm T$ is called a rotation if $\exists$ a $\rm T$-invariant
      subspace $W \subseteq \Rb^n$ with $\dim W = 2$ s.t.
      $\begin{cases}
        {\rm T} \big|_W \text{~is a rotation} \\
        {\rm T} \big|_{W^\bot} = {\rm id}_{W^\bot}
      \end{cases}$
    \item $\rm T$ is called a reflection if $\exists$ a $\rm T$-invariant
      subspace $W \subseteq \Rb^n$ with $\dim W = 2$ s.t.
      $\begin{cases}
        {\rm T} \big|_W = -{\rm id}_W \\
        {\rm T} \big|_{W^\bot} = {\rm id}_{W^\bot}
      \end{cases}$
  \end{itemize}
\end{definition}

\underline{Main result}: the group of orthogonal transformations
$= \langle \text{rotations}, \text{reflections} \rangle$.

\underline{Prop}: For ${\rm T}: \Rb^n \to \Rb^n$, $\exists$ a $\rm T$-invariant
subspace $W \subseteq \Rb^n$ with $1 \le \dim W \le 2$.
\begin{proof}
  Let $A=[{\rm T}]_\alpha \in M_{n\times n}(\Rb) \subseteq M_{n\times n}(\Cb)$.
  Consider $\widetilde{{\rm L}_A}: \Cb^n \to \Cb^n, v \mapsto Av$.

  Then $\exists$ an eigenvalue $\lambda \in \Cb$ and an eigenvector
  $v \in \Cb^n$ for $\widetilde{{\rm L}_A}$.
  Let $\lambda = \lambda_1 + \lambda_2 i, v = v_1 + v_2 i$. By definition,
  we have
  \[
    Av = \widetilde{{\rm L}_A}(v) = \lambda v =
    (\lambda_1 + \lambda_2 i)(v_1 + v_2 i)
    \implies \begin{cases}
      Av_1 = \lambda_1 v_1 - \lambda_2 v_2 \\
      Av_1 = \lambda_2 v_1 + \lambda_1 v_2
    \end{cases},
  \]
  so $W = \langle v_1, v_2 \rangle$.
\end{proof}

\begin{exercise} \mbox{}
  \begin{enumerate}
    \item If $\rm T$ is orthogonal, then $W^\bot$ is also $\rm T$-invariant.
    \item Use induction on $n$ to show the main result.
  \end{enumerate}
\end{exercise}

For $n = 3, A \in \text{O}(3)$, we have $A \sim \begin{pmatrix}
  \cos\alpha & -\sin\alpha & \\
  \sin\alpha & \cos\alpha & \\
   & & \pm 1
\end{pmatrix}$.
