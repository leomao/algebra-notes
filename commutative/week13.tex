%! TEX root=../main.tex
\subsection{Artinian rings and DVR (week 13)}
\subsubsection{Artinian rings}

\begin{definition}
  $R$ is called an Artinian ring if one of the followings holds:
  \begin{itemize}
    \item any non-empty set of ideals has a minimal element.
    \item any descending chain of ideals is stationary (DCC).
  \end{itemize}
\end{definition}


\underline{Goal:}
\begin{enumerate}
  \item $R = \cong R_1 \times \dots \times R_l$ where $R_i$ is an Artinain
    local rings.
  \item Artinian $\iff$ Noetherian $+ \dim = 0$.
\end{enumerate}

\begin{prop} \leavevmode \label{prop:radical-power-relation} 
  \begin{itemize}
    \item  $\sqrt{\mf_i^{n_i} + \mf_j^{n_j}} = \sqrt{\sqrt{\mf_i^{n_i}} + 
      \sqrt{\mf_j^{n_j}}}$
      \begin{proof} $ $\\ 
        "$\subseteq$" $\forall a \in \text{LHS}$, that is, $a^n  = b + c$ with 
        $b \in m_i^{n_i} \subseteq \sqrt{m_i^{n_i}}$ and 
        $c \in m_j^{n_j} \subseteq{m_j^{n_j}}$ then $a \in \text{RHS}$. \\
        "$\supseteq$" $\forall a \in \text{RHS}$, that is, $a^n = b + c$ with 
        $b^k \in m_i^{n_i}$ and $c^t \in m_j^{n_j}$. Then 
        $(a^n)^{k+t} = (b + c)^{k + t} = b^{k+t} + \dots + C_{t}^{k}b^kc^t
        + \dots + c^{k+t}$. Every term either in $m_i^{n_i}$ or $m_j^{m_j}$,
         then $(a^n)^{k+t} = c + d$ with $c \in m_i^{n_i}\ d \in m_j^{n_j}$
         $\Rightarrow a \in \text{LHS}$
      \end{proof}
    \item If $m$ is prime, $\sqrt{m^{n}} = m$
      \begin{proof} $ $\\
        $"\subseteq"$ $a \in \text{LHS} \Rightarrow a^k \in m^{n}$ and $m$ is prime.
        $\Rightarrow a \in m$. \\
        $"\subseteq"$ $a \in \text{RHS} a \in m \Rightarrow a^n \in \text{LHS}$.
      \end{proof}
     \item If $m$, $m_i, i = 1, \cdots, n$ are prime and $m \supseteq m_1 \cap
       \cdots \cap m_n$, then $m \supseteq m_i$ for some i.
       \begin{proof} $ $\\
         Suppose not, then we pick $a_i \in m_i \ m$. $b = a_1\cdots a_n \in m_i 
         \forall i$. $\leadsto b \in m_1 \cap \cdots \cap m_n \subseteq m$. But,
         m is prime, exist $a_i \in m$, a contradiction. 
       \end{proof}
  \end{itemize}
\end{prop}


\begin{prop} \label{artin-ring-basic-property}
  Let $R$ be an Artinian ring
  \begin{enumerate}[(1)]
    \item $I \subseteq R \leadsto \quot{R}{I}$ is also Artinian.
    \item If $R$ is an integral domain, then $R$ is a field.
      \begin{proof}
        $\forall \underset{\ne 0}{a} \in R, \gen{a} \supseteq \gen{a^2} \supseteq \dotsm$
        is a descending chain of ideals $\implies \gen{a^l} = \gen{a^{l+1}} = \dotsm$
        for some $l \in \Nb \implies a^l = ba^{l+1} \implies a^l(1 - ab) = 0
        \implies ab = 1$ since $a^l \ne 0$.
      \end{proof}
    \item $\Spec R = \Max R$. ($\implies \dim R = 0$)
      \begin{proof}
        $\forall p \in \Spec R, \quot{R}{p}$ is an integral domain
        $\leadsto \quot{R}{p}$ is a field $\leadsto p \in \Max R$.
      \end{proof}
    \item $\abs{\Max R} < \infty$.
      \begin{proof}
        Consider the set $\Set*{ \bigcap\limits_{\text{finite}} \mf \given \mf \in \Max R}
        \ne \varnothing$. So there exists a minimal element in this set ($R$ is Artinian),
        say $\mf_1 \cap \dots \cap \mf_k$.  Now, for $\mf \in \Max R$, we have
        $\mf\cap \mf_1\cap \dots \cap\mf_k = \mf_1\cap \dots \cap\mf_k$ since the
        latter is minimal $\implies \mf \supseteq \mf_1\cap \dots\cap\mf_k
        \leadsto \mf \supseteq \mf_i$ for some $i$, by Prop~\ref{prop:radical-power-relation}.
        $\leadsto m = m_i$, since $m_i$ is max. So $\Max R = \set{\mf_1, \dots, \mf_k}$.
      \end{proof}
    \item $\exists n_1, \dots, n_k \in \Nb$ s.t.
      $\gen{0} = \mf_1^{n_1}\mf_2^{n_2} \dotsm \mf_k^{n_k}
      = \mf_1^{n_1}\cap \mf_2^{n_2} \cap \dots \cap \mf_k^{n_k}$.
      \begin{proof} $ $\\
        \begin{itemize}
          \item $ \mf_1^{n_1}\mf_2^{n_2} \dotsm \mf_k^{n_k} 
            = \mf_1^{n_1}\cap \mf_2^{n_2} \cap \dots \cap \mf_k^{n_k}$. \\
            Recall $I_i, I_j$ are coprime for $i \ne j \leadsto
            \prod\limits_{i=1}^n I_i = \bigcap\limits_{i=1}^n I_i$.
            And, by Prop~\ref{prop:radical-power-relation}
             \[
              \sqrt{\mf_i^{n_i} + \mf_j^{n_j}}
              = \sqrt{\sqrt{\mf_i^{n_i}} + \sqrt{\mf_j^{n_j}}}
              = \sqrt{\mf_i + \mf_j} = \sqrt{R} = R
              \leadsto \mf_i^{n_i} + \mf_j^{n_j} = R.
            \]
          \item $\gen{0} = \mf_1^{n_1}\mf_2^{n_2} \dotsm \mf_k^{n_k} $ for suitable $\{n_i\}$
            that $\mf_i^{n_i} = \mf_i^{n_i+1}$\\
             Let $\Sc = \Set{J \subseteq R \given J \mf_1^{n_1}\mf_2^{n_2} \dotsm \mf_k^{n_k} \neq 0}$.
             If $\gen{0} \neq \mf_1^{n_1}\mf_2^{n_2} \dotsm \mf_k^{n_k} $, then $\mf_i \in S$.
             $\leadsto \Sc \neq \varnothing$. Since $R$ is Artinian, there exists
             a minimal element $J_0 \in \Sc$.
             By definition of S, $\exists x \in J_0$, $x\mf_1^{n_1}\mf_2^{n_2} \dotsm \mf_k^{n_k} 
             \neq 0 \leadsto \gen{x} \in S$ and $\gen{x} \subseteq J_0 \Rightarrow \gen{x} = J_0$.\\
             Also, $x \mf_1^{n_1+1}\mf_2^{n_2+1} \dotsm \mf_k^{n_k+1} =
             x \mf_1^{n_1}\mf_2^{n_2} \dotsm \mf_k^{n_k} \ne 0 \leadsto 
             I = x \mf_1\dots \mf_k\in S$ and $I \subseteq J_0 = xR \leadsto I = xR$.
            \[
              (\mf_1\mf_2 \dotsm \mf_k) x R = xR \leadsto
              (\mf_1 \cap \mf_2 \cap \dotsm \cap \mf_k) x R = xR \leadsto
              (\Jac R) xR = xR
            \]
            By Nakayama's lemma, $xR = 0 \implies x = 0$, which is a contradiction.
        \end{itemize}
      \end{proof}
    \item The nilradical $\nf_R$ of $R$ is nilpotent.
      \begin{proof}
        By (3), $\nf_R = \Jac R$. Let $n = \max \set{n_1, \dots, n_k}$ in (5),
        then $\nf_R^n = (\mf_1\mf_2 \dotsm\mf_k)^n = 0$.
      \end{proof}
  \end{enumerate}
\end{prop}

\underline{Goal 1:} $R \cong R_1 \times R_k$ where $R_i$ is Artinian local ring.
\begin{proof}
  By Chinese Remainder theorem,
  \[
    R \cong \quot{R}{\gen{0}} = \quot{R}{\mf_1^{n_1}\mf_2^{n_2} \dotsm \mf_k^{n_k}}
    \cong 
    \quot{R}{\mf_1^{n_1}} \times \quot{R}{\mf_2^{n_2}} \times \dots \times \quot{R}{\mf_k^{n_k}}
  \]
  Let $R_i = \quot{R}{\mf_i^{n_i}}$, then $\bar{\mf} \in \Max R_i$ if
  $\mf \in \Max R$ and $\mf \supset \mf_i^{n_i} \leadsto \mf = \mf_i$.
  So $\Max R_i = \Set{\bar{\mf_i}} \implies R_i$ is a local ring.
\end{proof}

\begin{lemma} \label{vector-space-dim-acc-dcc-relation}
  Let $V$ be a $K$-vector space, TFAE
  \begin{enumerate}[(1)]
    \item $\dim_k V < \infty$
    \item $V$ has DCC on subspaces.
    \item $V$ has ACC on subspaces.
  \end{enumerate}

  \begin{proof} $ $ \\
    \underline{Fact} : If $V_1 \subseteq V_2$ is finite dim vector space over K, then $V_1 = V_2$
    iff $\dim_k V_1 = \dim_k V_2$. Otherwise, $\dim_k V_1 \lneq \dim_k V_2$\\
    (1) $\Leftrightarrow$ (3) \\
    $"\Rightarrow"$ Suppose exists a chain in vector space $V$ with strictly increasing and infinite
    length,
    $$
      V_1 \subset V_2 \subset \cdots \subseteq V \Rightarrow \dim_k V_1 \lneq \dim_k V_2 \lneq \cdots 
      \leq \dim_k V
    $$
    Then, $\dim_k V$ must be infinite. \\
    $"\Leftarrow"$ If $\dim_k V$ is ininite, let S = $\{b_1, b_2, \dots \}$ be basis of V.
    $$
      \gen{b_1}_K \subset \gen{b_1 , b_2}_K \subset \cdots
    $$
    is a infinite ascending chain.\\
    Similarily, (1) $\Leftrightarrow$ (2).
  \end{proof}
\end{lemma}

\underline{Observation:} If $R$ is Northerian and $\dim R = 0$, then
$\gen{0} = \bigcap_{i=1}^k q_i$ (primary decomposition) and
$\sqrt{q_i} = \mf_i \in \Spec R = \Max R$. Also,  $\exists n_i $ 
$\mf_1^{n_1}\mf_2^{n_2} \dotsm \mf_k^{n_k} = \gen{0}$


Since $\mf_i$ is finitely generated, $\exists n_i$ s.t.
$\mf_i^{n_i} \subseteq q_i$. Hence 
$$
\mf_1^{n_1}\mf_2^{n_2} \dotsm \mf_k^{n_k} = 
\mf_1^{n_1} \cap \mf_2^{n_2} \cap \dotsm \cap \mf_k^{n_k} \subseteq
q_1 \cap q_2 \cap \dotsm \cap q_k = \gen{0} \\
$$
$\implies \mf_1^{n_1}\mf_2^{n_2} \dotsm \mf_k^{n_k} = \gen{0}$ \\

\underline{Goal 2:} In a ring $R$, let $\mf_1, \dots, \mf_n$ be,
not necessarily different, maximal ideals in $R$ s.t.
$\mf_1\mf_2 \dotsm \mf_n = 0$. Then $R$ is Artinian $\iff$ $R$ is Noetherian.

\begin{proof}
  We have a chain of ideals in $R$:
  $\mf_0 = R \supset \mf_1 \supset \mf_1\mf_2 \supset \dots \supset
  \mf_1\mf_2\dotsm\mf_n = 0$.

  Let $M_i = \quot{\mf_1\mf_2\dotsm\mf_{i-1}}{\mf_1\mf_2\dotsm\mf_i}$
  as $R$-module.
  Notice that $\mf_i M_i = 0$, we can treat $M_i$ as $\quot{R}{\mf_i}$-module.
  But $\quot{R}{\mf_i}$ is a field, so $M_i$ can be regarded as a vector
  space. Hence, by lemma~\ref{vector-space-dim-acc-dcc-relation}
  \[ 
  \text{ $M_i$ is Artinian $\iff$ $M_i$ is Noetherian. } 
  \]
  By definition, 
  $$
  0 \to \mf_1\mf_2\dotsm\mf_i \to \mf_1\mf_2\dotsm\mf_{i-1}
  \to M_i \to 0
  $$
  By Ex1,
  $$
    \begin{aligned}
      \mf_0 = R \text{ Artinian} & \iff \mf_1 , M_1 \text{ Artinian} \\
      & \iff \mf_1 \mf_2 , M_1, M_2 \text{ Artinian} \\
      & \hspace{13pt} \vdots \\
      & \iff \underset{\text{is } 0}{\mf_1 \mf_2 \cdots \mf_n}, M_1, 
      \dots, M_n \text{ Artinian} \\
      & \iff \underset{\text{is } 0}{\mf_1 \mf_2 \cdots \mf_n}, M_1, 
      \dots, M_n \text{ Neotherian} \\
      & \hspace{13pt} \vdots \\
      & \iff \mf_1 \mf_2 , M_1, M_2 \text{ Neotherian} \\
      & \iff \mf_1 , M_1 \text{ Neotherian} \iff \mf_0 = R \text{ Neotherian}
    \end{aligned}
  $$
  Note: \underline{Goal 2} is accomplished by recongnizing that,
  \begin{itemize}
    \item $R$ is Artinian $\implies \mf_1^{n_1}\cdots \mf_k^{n_k} = \gen{0}$
      by prop~\ref{artin-ring-basic-property} (4).
  \item $R$ is Neotherian $+ \dim 0 \implies \mf_1^{n_1} \cdots \mf_k^{n_k} = \gen{0}$ by \underline{Observation}.
  \end{itemize}
\end{proof}


\subsubsection{DVR (Discrete Valuation Ring)}

\begin{definition} \mbox{}
  \begin{enumerate}[(1)]
    \item Let $K$ be a field. A discrete valuation of $K$ is
      $\nu: K^\times \onto \Zb$ ($\nu(0) = \infty$) s.t.
      \begin{itemize}
        \item $\nu(xy) = \nu(x) + \nu(y)$.
        \item $\nu(x \pm y) = \min\set{\nu(x), \nu(y)}$.
      \end{itemize}
    \item The valuation ring of $\nu$ is $R = \Set{x \in K \given \nu(x) \ge 0}$,
      called a DVR.
      \begin{itemize}
        \item Fact \\
          $\nu(1) = 0$ : $\nu(1) = \nu(1) + \nu(1) \implies \nu(1) = 0$ \\
          $\nu(x) = -\nu(x^{-1})$ : $0 = \nu(xx^{-1}) = \nu(x) + \nu(x^{-1})$
        \item $\mf = \Set{x \in R \given \nu(x) > 0}$ is the unique maximal ideal
          in $R$ since $\nu(x) = 0 \iff x$ is a unit.
          \begin{proof} $ $\\
            $"\Rightarrow"$
              $\nu(x) = 0 \leadsto \nu(x^{-1}) = 0 \leadsto x^{-1} \in R$ \\
            $"\Leftarrow"$
              $\nu(x^{-1}),\nu(x) \ge 0.$ And, $\nu(x) = -\nu(x) \le 0 \leadsto 
              \nu(x) = 0$
          \end{proof}
        \item Let $t \in R$ with $\nu(t) = 1$, then $\mf = \gen{t}$.\\
          $\forall x \in \mf, \nu(x) = k > 0. \leadsto \nu(x(t^k)^-1) = 
          \nu(x) - k\nu(t) = 0 \leadsto x = t^ku$, $u$ is unit in R.
        \item Let $I \subseteq \mf$ and define
          $m = \min\Set{l \in \Nb \given x = t^l u \quad \forall x \in I}$.
          Then $I = \gen{t^m}$.\\
      \end{itemize}
  \end{enumerate}
\end{definition}
\iffalse
\begin{prop} \label{local-neotherian-ring-set-gen-max-gen-qoetient-max}
  Let R be a local Noetherian ring with maximal ideal $\mf$. Then $t_1, \cdots t_n
  \in \mf$ generate $\mf$ if and only if thier images generate $\mf / \mf^2$ as an
  $R/\mf$ vector space.
  \begin{proof} \hfill
    The $"\Rightarrow"$ is simple. \\
    $"\Leftarrow"$ \\
    Let $N = \gen{t_1, \cdots , t_n} \subseteq \mf$ and images of $t_1, \cdots, t_n
    \in \mf / \mf^2$ generate $\mf / \mf^2$ as $R/\mf$-vector space, then we have
    $$
      N+\mf^2 = \mf + \mf^2 \implies (N + \mf^2) / N = (\mf + \mf^2) / N
      \implies \mf(\mf/N) = \mf / N
    $$
    where we have used $N/N = 0$ and $\mf + \mf^2 = \mf$. By Nakayama's lemma, $M =
     \mf / N$ is zero module, so $\mf = N$ and $t_1, \cdots, t_n$ generate $\mf$.
  \end{proof}
\end{prop}
\fi

\begin{prop} \label{dvr-equiv-condition}
  $R$ is a DVR $\iff$ $R$ is 1-dimensional normal, Noetherian local domain.
  \begin{proof} \hfill \\
    ``$\Rightarrow$'':
    \begin{tikzpicture}[baseline,
      implies/.style={double,double equal sign distance,-implies} 
      ]
      \node(DVR) at (0, 0) {DVR};
      \node(PID) at (1.5, 0) {PID};
      \node(UFD) at (3, 0.3) {UFD};
      \node(normal) at (5, 0.3) {normal};
      \node(Noeth) at (3.5, -0.3) {Noetherian};
      \draw (DVR.east) edge[implies] (PID.west);
      \draw (PID.east) edge[implies] (UFD.west);
      \draw (UFD.east) edge[implies] (normal.west);
      \draw (PID.east) edge[implies] (Noeth.west);
    \end{tikzpicture} \\
    $\forall P \neq 0 \in \Spec R$, $P = \gen{t^k} = m^k$ for some $k \in \Nb$
    $\leadsto P = \sqrt{P} = \sqrt{m^k} = m \leadsto P = m \leadsto
    \gen{0} \subset m \leadsto \dim R = 1$. \\
    $"\Leftarrow"$:
    \begin{itemize}
      \item $\mf \ne \mf^2$ : \\
        If $\mf = \mf^2$ and $\mf = \Jac R$, then $m = \gen{0}$ by Nakayama's lemma,
        which is a contradiction to $\dim R = 1$.
      \item Let $t \in \mf - \mf^2$ and $\mf = \gen{t}$ \\
        Consider $M = \mf / \gen{t}$ and assume $M \neq 0$ \\
        \underline{Fact}: $I = \Ann(\bar{x})\ \bar{x} \in M \implies I \in \Spec R$
        Since $ab \in I,\ a,b \notin I$, then $ab\bar{x} = 0$, 
        and $b\bar{x} \neq 0$. \\
        Suppose I is maximal, $\Ann(b\bar{x}) \supseteq \Ann(\bar{x})
        \leadsto \Ann(b\bar{x}) = \Ann(\bar{x})$ Then, $a \in \Ann(b\bar{x})
        = \Ann(\bar{x})$, which is a contradiction. \\
        By the Fact, $\exists \bar{x} \neq 0 \in M$ s.t. $\Ann(\bar{x}) = \mf \leadsto
        x\mf = \gen{t} = tR \leadsto \frac{x}{t}\mf \subseteq R$. \\
        (1) If $\frac{x}{t} = R \leadsto \frac{xy}{t} = 1$ for some $y \in \mf
        \leadsto t = xy \in \mf$, which is a contratiction. \\
        (2) If $\frac{x}{t} \subset \mf$, let $\mf = \gen{y_1,\cdots, y_n}_R$ Write
        $\frac{x}{t} y_i = \sum\limits_{j = 1}^{l} a_{ij}y_j \forall i = 1, \cdots,
        l$ By using determinant trick, we have $\frac{x}{t}$ is integral over $R$,
        but $R$ is normal $\leadsto \frac{x}{t} \in R \leadsto x \in \gen{t} \leadsto 
        \bar{x} = \bar{0}$, which is a contradiction. \\
        Therefore, $\mf = \gen{t}$.
      \item By Ex3, $\bigcap\limits_{n = 0}^{\infty} m^n = 0$. Thus, $\forall x \in R$,
        $\existsuniq k$ s.t. $x \in m^k$ and $x \notin m^k+1$. $\leadsto x = t^ku$,
        $u$ is units.
      \item Define $\nu(x) = k$ and $\forall \frac{x}{y} \in \Frac{R} 
        \nu(\frac{x}{y}) = \nu(x) - \nu{y}$. \\
        (1) $\frac{x}{y} = \frac{x'}{y'} \leadsto xy' = x'y \leadsto \nu(xy') = 
        \nu(x'y)$. \\
        (2) $\nu(\frac{a}{b} \frac{c}{d}) = \nu(ac) - \nu(bd) = \big[\nu(a) - 
        \nu(b)\big] - \big[\nu(c) - \nu(d)\big] = \nu(\frac{a}{b}) - \nu(\frac{c}{d})$ \\
        (3) $\nu(\frac{a}{b} + \frac{c}{d})$, $\nu(a) = v_a$, $\nu(b) = v_b$,
        $\nu(c) = v_c$,$\nu(d) = v_d$. $\nu{\frac{a}{b} + \frac{c}{d}} = 
        \min\big\{\nu(\frac{a}{b}), \nu(\frac{c}{d})\big\} = \min\big\{\nu(\frac{ad}{bd}), 
        \nu(\frac{bc}{bd})\big\} = \nu(\frac{ad + bd}{bd})$.
    \end{itemize}
    Therefore, R is DVR.
  \end{proof}
\end{prop}

\subsubsection{Dedekind domains}
\begin{definition}
  A Dedekind domain is a Noetherian normal domain of $\dim 1$.
\end{definition}

\begin{definition}
  Let $R$ be an integral domain and $K = \Frac(R)$.
  A nonzero $R$-submodule $I$ of $K$ is called a fractional ideal of $R$ if
  $\exists 0 \ne a \in R$ s.t. $aI \subset R$.
\end{definition}

\begin{example}
  If $I = \gen{f_1, \dots, f_n}_R$ with $f_i = \frac{a_i}{b_i} \in K$, then
  $a = b_1b_2 \dotsm b_n$ and $aI \subset R \implies I$ is fractional.

  In general, if $R$ is a Noetherian, then every fractional ideal $I$ of $R$
  is finitely generated.
\end{example}

\begin{definition}
  A fractional ideal $I$ of $R$ is invertible if $\exists J:$ a fractional ideal
  of $R$ s.t. $IJ = R$.
\end{definition}

\begin{prop}\mbox{}
  \begin{enumerate}
    \item If $I$ is invertible, then $J = I^{-1}$ is unique and equals
      $J = (R:I) \triangleq \Set{a \in K \given aI \subset R}$.
      \begin{proof}
        $ J \subseteq (R:I) \subseteq (R:I)R \subseteq (R:I)IJ \subseteq RJ = J
        \leadsto J = (R:I) $
      \end{proof}
    \item If $I$ is invertible, then $I$ is a finitely generated $R$-module.
      \begin{proof}
        $I(R:I) = R \leadsto 1 = \sum\limits_{i = 0}^{k} x_iy_i$, $x_i \in I$ and
        $y_i \in (R:I)$. Then, $\forall x \in I$, $x = \sum\limits_{i = 0}^{k} 
        \underbrace{(xy_i)}_{\in R}x_i \leadsto I = \gen{x_0, \dots, x_k}_R$.
      \end{proof}
    \item Let $R$ be a local domain but not a field, $K = \Frac(R)$.
      Then $R$ is a DVR $\iff$ every nonzero fractional ideal $I$ of $R$ is
      invertible.
      \begin{proof} $ $\\
        $"\Rightarrow"$:
        Let I be fractional ideal of R, then $\exists a \in R \leadsto aI 
        \subseteq R$. And, $\mf = \gen{t}$, since R is not a field $t \ne 0$.
        $\leadsto a = t^k u$ where $u$ is a unit in $R$.
        If $aI = R$, then $J = \gen{a}_R$. $\leadsto IJ = R$. If not,
        $aI \subsetneq R$ $\leadsto aI \subseteq \mf \leadsto aI = \gen{t^l}
        \leadsto I = \gen{t^{l - k}}_R$. Let $J = \gen{t^{k - l}}_R$, then $IJ
         = R$.

        $"\Leftarrow"$: \mbox{}
        \begin{itemize}
          \item R is Neotherian:
            $\forall I \subsetneq R$ and $I$ is invertible $\leadsto I$ is f.g.
            $R$-module.
          \item $\mf \neq \mf^2$, $\mf$ is the unique maximal in $R$:
            $\mf = \mf^2$ and $\mf = \Jac R \leadsto \mf = 0 \leadsto R$ is a field,
             which is a contradiction.
          \item $\mf = \gen{t}_R$:
            Pick $t \in \mf - \mf^2$ and let $\mf\mf^{-1} = R \leadsto t\mf^{-1}
            \subseteq R$. If $t\mf \subseteq \m$, then $t\mf\mf^{-1} = tR \subseteq
            \mf^{2}$, a contradiction. Therefore, $\mf \subset t\mf \leadsto t\mf 
            = R \leadsto t\mf\mf^{-1} = tR = \mf \leadsto \mf = \gen{t}_R$. 
          \item Using the same construction $\nu$ in prop~\ref{dvr-equiv-condition}
            we have the $R$ is DVR.
            \qedhere
        \end{itemize}
      \end{proof}
  \end{enumerate}
\end{prop}

\begin{theorem}
  Let $R$ be an integral domain and $K= \Frac(R)$. TFAE
  \begin{enumerate}[(a)]
    \item $R$ is a Dedekind domain.
    \item $R$ is Noetherian and $R_P$ is a DVR for all $P \in \Spec R$.
    \item Every nonzero fractional ideal of $R$ is invertible.
    \item Every nonzero proper ideal of $R$ can be written (uniquely) as a
      product of powers of prime ideals.
  \end{enumerate}

  \begin{proof} \mbox{}
    \begin{description}
      \item[\rm (a)$\Leftrightarrow$(b):] $ $
        \begin{itemize}
          \item $R$ is normal $\iff$ $R_P$ is normal for all $P \in \Spec R$.
          \item $\dim R_P = 1 \quad \forall P \in \Spec R \iff h(P) = 1 \quad
            \forall 0 \ne P \in \Spec R \iff \dim R = 1$.
        \end{itemize}
      \item[\rm (b)$\Leftrightarrow$(c):] $ $
        \begin{itemize}
          \item (b) and (c) $\implies R$ is Noetherian
          \item $I_P(R:I)_P = I_P(R_P:I_P)$ (Hint: $I$ is f.g.)
          \item $R_P$ is DVR $\forall P \in \Spec R \iff$ Every nonzero fractional
            ideal of $R$ is invertable.
            $$
            \forall P \in \Spec R\ R_P = I_P(R_P:I_P) = I_P(R:I)_P
             = I(R:I)_P \iff R = I(R:I)
            $$
        \end{itemize}
      \item[\rm (a)(b)(c)$\Rightarrow$(d):] $ $ \\
        Existence:
        \begin{itemize} 
          \item $I = q_1 \cap \cdots \cap q_n$ and $\sqrt{q_i} = P_i \in R $ by
             primary decomposition thm.
          \item $q_1 \cap \cdots \cap q_n = q_1 \cdots q_n$ \\
            Since $\dim R = 1$, $P_i \in Max R$. And, $R$ is Noetherian,
            $\exists n_i \in \Nb$ s.t. $m_i^{n_i} \subseteq q_i$. Then,
            $m_i^{n_i} + m_j^{n_j} = R \forall i \neq
            j \leadsto q_i + q_j  = R \leadsto q_1 \cap \cdots \cap q_n = q_1 \cdots
             q_n$
           \item $I = m_i^{r_i} \cdots m_n^{r_n}$ \\
             Since $R_{m_i}$ is DVR $\leadsto (q_i)_{m_i} = (m_i^{r_i})_{m_i} \leadsto
              q_i = m_i^{r_i}$ by prime is 1-1 correspondence in localization. 
              Therefore,  $I = m_i^{r_i} \cdots m_n^{r_n}$. 
        \end{itemize}
        Uniqueness: 
        \begin{itemize}
          \item $P_1\cdots P_k = Q_1\cdots Q_r$ $P_i, Q_i$ is prime. Then, $P_1\cdots
             P_k \subseteq Q_1 \leadsto P_i \subseteq Q_1$, say $i = 1$ by 
             prop~\ref{prop:radical-power-relation}. \\
             Since $Q_1$ is invertible, then $P_2\cdots P_k = Q_2\cdots Q_r$. By 
             induction by hypothesis, we have the uniqueness result.
        \end{itemize}
      \item[\rm (d)$\Rightarrow$(c):] \mbox{}
        \begin{itemize}
          \item Every invertible prime is maximal: \\
            If not, let $p + aR = P_1 \cdots P_k$ and $p + a^2R = Q_1 \cdots Q_r$. 
            $\leadsto p \subseteq P_i$ and $Q_j$\\
            \underline{Claim} $(p + aR)^2 = (p + a^2R)$: \\
            In $R/p$, $\gen{\bar{a}} = \big(P_1/p\big) \cdots \big(P_k/p\big)$ and 
            $\gen{\bar{a}^2}= \big(Q_1/p\big)\big(Q_r/p\big)$. And, $\gen{\bar{a}}
             = \big(P_1/p\big)^2 \cdots \big(P_k/p\big)^2 = \big(Q_1/p\big) \cdots
             \big(Q_r/p\big)$
          \item $P = P^2 + aP$
             $$
             P \subseteq P + a^2R = (P + aR)^2 \subseteq P^2 + aR
             $$
             Then,
             $$
             \forall x \in P, \quad \underset{\in P}{x} = \underset{\in P^2}{y}
             + a\underset{\in R}{z} \leadsto z \in P
             $$
             Therefore, $P\subseteq P^2 + aP \leadsto P = P^2 + aP$. 
        By invertibility of $P$, we have $R = P + aR$, which is a contradiction.
        \item Every nonzero prime is invertible: \\
          Let $a \neq 0 \ in P$, and $P \supseteq \gen{a}$ is invertible. $\gen{a}
            = P_1 \cdots P_n$ and $P_i$ is invertible. And, $P \subseteq P_i 
            \leadsto P = P_i$ since $P_i$ is maximal.
        \item $\forall$ ideal $I \neq 0 \subseteq R$, $I = P_1 \cdots P_m \leadsto 
          I$ is invertible.
        \item If $I$ is fractional ideal of $R$, say $aI \subseteq R \leadsto 
          \exists J$ ideal in $R$ $\leadsto aIJ = R \leadsto I(aJ) = R$. $I$ is
            invertible.
        \end{itemize}
  \end{description}
  \end{proof}
\end{theorem}
