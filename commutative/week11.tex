%! TEX root=../main.tex
\subsection{Gr\"{o}bner basis (week 11)}

\subsubsection{Division algorithm in $K[X_1,\dots,X_n]$}

\begin{example}
  $I = \gen{ xy-1,y^2-1 } \subseteq K[x,y]$, $f_1 = xy-1$ an
  d $f_2 = y^2-1$ $G=\Set{f_1,f_2}$. Does $f = x^2y+xy^2+y^2 \in I$?
  \begin{itemize}
      \item Choose a lexicographic monomial ordering: $x > y$
      \item The multidegree $\partial(f) = (2,1)$, $\partial(f_1) = (1,1)$,
        $\partial(f_2) = (0,2)$
      \item The leading term $\LT(f) = x^2y$, $\LT(f_1) = xy$, $\LT(f_2) = y^2$
      \item $\LT(f) = x\LT(f_1) \Rightarrow f = xf_1+xy^2+y^2+x \Rightarrow f = \underset{h_1}{(x+y)}f_1+\underset{h_2}{(1)}f_2+\underset{\bar{f}^G}{(x+y+1)}$ or $f = \underset{h_1}{x}f_1+\underset{h_2}{(x+1)}f_2+\underset{\bar{f}^G}{(2x+1)}$.
  \end{itemize}
  Note:  Divisor $h_1$, $h_2$ and remainder $\bar{f}^G$ are not unique!! 
\end{example}

\begin{definition}
  Fix a monomial ordering and let $I$ be an ideal of $K[X_1,\dots,X_n]$.
  The ideal of leading terms in $I$ is defined to be
  $\LT(I) = \Gen{ \LT(f) \given f\in I }$.
\end{definition}

\begin{remark}
  Let $I = \gen{ f_1,\dots,f_n }$. In general, $\gen{\LT(f_1),\dots,\LT(f_n)}
  \subsetneq \LT(I)$.
\end{remark}

\begin{example}
  Let $f_1=xy^2+y$, $f_2=x^2y$. And, $xf_1-yf_2=xy \in \gen{ f_1,f_2 }$ but
  $xy \notin \gen{ xy^2, x^2y }$.
\end{example}

\begin{definition}
  $G = \Set{g_1,\dots,g_m}$ is called a Gr\"{o}bner basis of $I$ if
  $I = \gen{ g_1,\dots,g_m }$ and $\LT(I) = \gen{ \LT(g_1),\dots,\LT(g_m) }$.
\end{definition}

\begin{prop} \label{prop:leading-term-ideal-equal-so-does-ideal}
  Let $g_1, \dots, g_m \in I$, then
  $\LT(I) = \gen{\LT(g_1),\dots,\LT(g_m)} \implies I = \gen{g_1,\dots,g_m}$.
  \begin{proof}
    $\forall f \in I$, do the division process. Then
    $f = \sum\limits_{i=1}^m h_ig_i+r$, either $r=0$ or
    \uline{$\bigstar =$ no term of $r$ is divisible by any of
    $\LT(g_1),\dots,\LT(g_m)$}. Assume $r \neq 0$, then
    $r = f - \sum\limits_{i=1}^m h_i g_i \in I \Rightarrow \LT(r) \in \LT(I) =
    \gen{ \LT(g_1),\dots, \LT(g_m) }$, which is a contradiction.
    Hence, $r = 0$ (i.e. $f\in \gen{ g_1,\dots,g_m }$).
  \end{proof}
\end{prop}

\begin{theorem} \label{thm:Grobner-existense}
  Each ideal $I$ has a Gr\"{o}bner basis.
  \begin{proof}
    By Hilbert basis thm, $\LT(I) = \gen{ f_1,\dots,f_m }$ for some $f_i$'s.
    Write $f_i = \sum\limits_{j=1}^{m_i} h_{ij}\LT(g_{ij})$ with
    $h_{ij} \in K[X_1,\dots,X_n]$, $g_{ij} \in I$. Then
    $\LT(I) = \Gen{ \LT(g_{ij}) \given i=1,\dots,m,\, j=1,\dots,m_i}$.
    By prop~\ref{prop:leading-term-ideal-equal-so-does-ideal}, 
    This is Gr\"{o}bner basis.
  \end{proof}
\end{theorem}

\begin{theorem} \label{thm:Grobner-property}
  Let $G = \Set{g_1,\dots,g_m}$ be a Gr\"{o}bner basis of $I$, then
  \begin{itemize}
    \item $\forall f \in K[X_1,\dots,X_n]$, $f = f_I + r$ where
      $f_I \in I, r = \bigstar$ are unique.
      \begin{proof}
        By division algorithm, $f = f_I +\underset{\bigstar}{r} =
        f_I'+\underset{\bigstar}{r'}$, then $\underset{\bigstar}{r-r'} = f_I-f_I'$.
        But if $r-r' \neq 0$, then $\LT(r-r') \in \LT(I) =
        \gen{ \LT(g_1),\dots,\LT(g_m) }$, which is a contradiction.
        Hence, $r-r' = 0\Rightarrow f_I = f_I'$.
      \end{proof}
    \item $f \in I \iff r=0$.
      \begin{proof}
        Suppose $f \in I$, then $f = f_I + \underset{\bigstar}{r}$, and if $r\neq 0$,
        $\underset{\bigstar}{r} = f - f_I\in I$, which is a contradiction.
        Hence, $r = 0$. Conversly, if $r = 0$, $f = f_I \in I$. 
      \end{proof}
  \end{itemize}
\end{theorem}


\subsubsection{Buchberger's algorithm}

\begin{definition}
  Let $f,g \in K[x_1,\dots,x_n]$ and $M$ be the monic least common multiple of
  $\LT(f)$ and $\LT(g)$. $S(f,g) = \frac{M}{\LT(f)}f - \frac{M}{\LT(g)}g$ is
  called an S-polynomial of $f, g$.
\end{definition}

Let $I = \gen{ g_1, \dots, g_m }$ and $G = \Set{ g_1, \dots, g_m }$.
A Gr\"{o}bner basis of $I$ can be constructed by the following algorithm:
\begin{enumerate}
  \item Initially let $G_0 \gets G$.
  \item Repeatly construct $G_{i+1} \gets G_i \cup \big(
    \Set{S(f, g) \mod G_i \given f, g \in G_i} \setminus \Set{0} \big)$,
    until once $G_{i+1} = G_i$, then $G_i$ is a Gr\"{o}bner basis of $I$.
\end{enumerate}

\begin{lemma} \label{lemma:sum-of-equal-degree-f-is-less}
  Let $f_1, \dots, f_m \in K[x_1, \dots, x_n]$ with $a_1, \dots, a_m \in K$
  satisfying $\partial(f_1) = \partial(f_2) = \dots = \partial(f_m) = \alpha$
  and $h =\sum_{i = 1}^m a_i f_i $ with $\partial(h) < \alpha$.
  Then $h = \sum_{i = 2}^m b_i S(f_{i-1}, f_i)$ for some $b_i \in K$.
  \begin{proof}
    Write $f_i = c_if'_i$ with $c_i \in K$ and $f'_i$ being monic of multidegree
    $\alpha$. Note: $S(f_i, f_j) = f'_i - f'_j$ since all multidegree are
    equal. Then, 
    \[
      \begin{split}
        h &= \sum_{i=1}^m \big( a_ic_if'_i \big) \\
        &= a_1c_1(f'_1-f'_2) + (a_1c_1+a_2c_2)(f'_2-f'_3) + \dots +
        (a_1c_1 + \dots + a_{m-1}c_{m-1})(f'_{m-1}-f'_m) \\
        &+ (a_1c_1+\dots+a_mc_m)f'_m \\
        &= \sum\limits_{i=2}^m b_iS(f_{i-1},f_i) + b_{m+1}f'_m\text{ with }
        b_i = \sum_{j=1}^{i-1}a_jc_j.
      \end{split}
    \]
    Also, in this equality, $f'_m$ is the only term that has multidegree $\alpha$
    (other terms have multidegree less than $\alpha$). So $b_{m+1} = 0$ must hold.
    Then, we have $h = \sum_{i = 2}^m b_i S(f_{i-1}, f_i)$.
  \end{proof}
\end{lemma}

\begin{theorem}[Buchberger's criterion]
  Assume $I = \gen{ g_1, \dots, g_m }$, then
  $G = \Set{g_1, \dots, g_m}$ is a Gr\"{o}bner basis of $I$ $\iff$
  $S(g_i, g_j) \equiv 0 \pmod{G}$ for each $i, j$.
  \begin{proof} \mbox{}
    \begin{itemize}
      \item Suppose $G$ is a Gr\"{o}bner basis of $I$.
        $S(g_i, g_j) \in I \Rightarrow S(g_i, g_j) = 0$
        by thm~\ref{thm:Grobner-property}.
      \item Converely, suppose $S(g_i, g_j) \equiv 0 \pmod G \forall i, j$.
        For $f \in I$, $f \underset{not\ division}{=}
        \sum\limits_{i=1}^m h_ig_i$ for some $h_i \in K[x_1,\dots,x_n]$.
        Define $\alpha = \max\{\partial(h_1g_1),\dots,\partial(h_mg_m)\}$.
        We have $\partial(f) \leq \alpha$ and we can select an expression
        $f = \sum\limits_{i=1}^m h_ig_i$ for f s.t $\alpha$ is minimal.
      \item \uline{Claim}: $\partial(f) = \alpha$.
      \item (pf) If not, we rewrite $f$
        \[
          \begin{split}
            f &= \sum_{i = 1}^m h_ig_i \\
            &= \sum_{\partial(h_ig_i) = \alpha} h_ig_i
              + \sum_{\partial(h_ig_i) < \alpha} h_ig_i
              \qquad (\text{the first term $\neq 0$ since $\alpha$ is minimal.}) \\
            &= \sum_{\partial(h_ig_i) = \alpha} \LT(h_i)g_i
              + \sum_{\partial(h_ig_i) = \alpha} (h_i-\LT(h_i)g_i)
              + \sum_{\partial(h_ig_i) < \alpha} h_ig_i
          \end{split}
        \]
        Let $\LT(h_i) = a_ih_i^0$ with $h_i^0$ being a monic monomial.
        Comparing the multidegree on both side,
        $\partial\left(\sum\limits_{\partial(h_ig_i) = \alpha} a_ih_i^0g_i\right)
        < \alpha$ By lemma~\ref{lemma:sum-of-equal-degree-f-is-less},
        $\underset{\partial(h_ig_i) = \alpha}{\sum}\left(a_ih_i^0g_i\right) =
        c_{12}S(h_{i_1}^0g_{i_1},h_{i_2}^0g_{i_2}) + \dots$(finite) where
        $\partial(h_{i_1}g_{i_1}) = \partial(h_{i_2}g_{i_2}) = \dots = \alpha$.
        By def, if we set $M_{st} = X^\beta_{st} =$ the monic LCM of
        $\LT(g_{i_s}), \LT(g_{i_t})$, then
        \[
          \begin{split}
            S(h_{i_s}^0g_{i_s}, h_{i_t}^0g_{i_t}) &= \frac{X^\alpha}{\LT(h_{i_s}^0g_{i_s})}h_{i_s}^0g_{i_s} - \frac{X^\alpha}{\LT(h_{i_t}g_{i_t})}h_{i_t}^0g_{i_t} \\
            &= X^{\alpha-\beta_{st}} \left( \frac{X^{\beta_{st}}}{\bcancel{h_{i_s}^0}\LT(g_{i_s})}\bcancel{h_{i_s}^0}g_{is} - \frac{X^{\beta_{st}}}{\bcancel{h_{i_t}^0}\LT(g_{i_t})}\bcancel{h_{i_t}^0}g_{i_t}  \right) \\
            &= X^{\alpha-\beta_{st}} S\left(g_{i_s},g_{i_t}\right) \\
            &= X^{\alpha-\beta_{st}}\overset{m}{\underset{j = 1}{\sum}}{l_jg_j} \text{ (by division)}
          \end{split}
        \]
      \item Then, $\partial(l_jg_j) < \beta_{st} \implies$ we found a expression
        with multidegree less than $\alpha$, which is a contradiction.
        Therefore, $\partial(f) = \alpha \implies \LT(f) =
        \sum\limits_{\partial(h_ig_i) = \alpha} \LT(h_i)\LT(g_i)$
        $\implies \LT(f) \in \gen{ \LT(g_1),\dots, \LT(g_m) }$.
    \end{itemize}
  \end{proof}
\end{theorem}

\begin{theorem}
  The Buchberger's algorithm will terminate
  \begin{proof}
    $.$
    \begin{itemize}
      \item $\gen{ \LT(G_i) } \subsetneq \gen{ \LT(G_{i+1}) }$ if $G_i \neq G_{i+1}$
        \[
          G_i \neq G_{i+1} \implies \exists f, g \in G_i \text{ s.t. } S(f,g)
          \not\equiv 0 \pmod G \implies \LT(S(s,g)) \notin \gen{\LT(G_i)}
        \]
      \item $\gen{ \LT(G_0) } \subsetneq \gen{ \LT(G_1) } \subsetneq \dotsm$
        is not possible since $K[x_1, \dots, x_n]$ is a Noetherian ring.
        (Noetherian ACC condition).
    \end{itemize}
  \end{proof}
\end{theorem}

\subsection{Applications of Gr\"{o}bner basis}
\begin{definition}
  Let $I \subseteq K[x_1, \dots, x_n]$ and $x_1 > x_2 > \dotsm > x_n$.
  $I_i \triangleq I \cap K[x_{i+1}, \dots, x_n]$ is called the $i$-th elimination
  ideal of $I$.
\end{definition}

\begin{theorem}[Elimination theorem]
  Let $G = \Set{g_1, \dots, g_m}$ be a Gr\"obner basis of $I \neq 0$ with ordering
  $x_1 > \dotsm > x_n$. Then $G_i \triangleq G \cap K[x_{i+1}, \dots, x_n]$
  is a Gr\"obner basis of $I_i$ (i.e., $\gen{ \LT(G_i) } = \LT(I_i)$).
  \begin{proof}
    % TODO
  \end{proof}
\end{theorem}

\begin{example}
  Find $V = \Vc(x+y-z, x^2+y^2-z^3, x^3+y^3-z^5)$. \\
  We compute a Gr\"obner basis of $I = \gen{ f_1, \dots, f_3 }$ with respect
  to the ordering $x > y > z$.
  The Gr\"obner basis is $\Set{x+y-z, 2y^2-2yz-z^3+z^2, 2z^5-3z^4+z^3}$.
\end{example}

\begin{example}
  \[
    \begin{tikzcd}[column sep=0.8cm,row sep=0ex]
      f: &[-0.7cm] \Ab^1 \arrow[r] & \Ab^3 \\
      & t \arrow[r, mapsto] & (t^4, t^3, t^2)
    \end{tikzcd}
  \]

  We compute a Gr\"obner basis of $I = \gen{ t^4 - x, t^3 - y, t^2 - z }$ with
  respect to $t > x > y > z$. The Gr\"obner basis is
  $\Set{-t^2+z, ty-z^2, tz-y, x-z^2, y^2-z^3}$.
\end{example}

\begin{example}
  \[
    \begin{tikzcd}[column sep=0.8cm,row sep=0ex]
      f: &[-0.7cm] V = \Vc(x^3 - x^2z - y^z) \arrow[r] & \Ab^3 \\
      & (x, y, z) \arrow[r, mapsto] & (x^2 z - y^2 z, 2xyz, -z^3)
    \end{tikzcd}
  \]

  The ideal is $\gen{ x^3 - x^2 z - y^2 z, u - x^2 z + y^2 z, v - 2 x y z, w + z^3 }$
    has a Gr\"obner basis $\gen{ \dots, u^2 + v^2 - w^2 }$.
\end{example}

\begin{theorem}
  Let $I, J$ be two ideals of $K[x_1, \dots, x_n]$, then $I \cap J =
  (t \tilde{I} + (1 - t) \tilde{J}) \cap K[x_1, \dots, x_n]$.
  \begin{proof}
    % TODO
  \end{proof}
\end{theorem}

\begin{example}
  $I = \gen{ y^2 , x - yz }, \, J = \gen{ x, z }$. We shall find $I \cap J$. \\
  $tI + (1-t)J = \gen{ tx - tyz, ty^2, (1-t)x, (1-t)z }$ has a Gr\"obner basis
  $\Set{f_1, f_2, f_3, f_4, xy, x - yz}$, so $I \cap J = \gen{ xy, x-yz }$.
\end{example}

\begin{theorem}
  Let $L = \gen{ f_1, \dots, f_s } \subsetneq K[x_1, \dots, x_n]$, then
  $f \in \sqrt{I} \iff \gen{ f_1, \dots, f_s, 1 - tf } = K[x_1, \dots, x_n, t]$.
  \begin{proof}
    % TODO
  \end{proof}
\end{theorem}

\begin{example}
  Let $I = \gen{ xy^2 + 2y^2, x^4 - 2x^2 + 1 }$, and we are to determine
  $f = y - x^2 + 1$ is in $\sqrt{I}$ or not.
\end{example}

\begin{prop}
  An affine algebraic set $V$ in $\Ab^n_k$ has a unique minimal decomposition.
  $V = V_1 \cap V_2 \cap \dotsm \cap V_m$ with $V_i$ irreducible and
  $V_i \not\subset V_j$.
  \begin{proof}
    % TODO
  \end{proof}
\end{prop}

\begin{theorem}[Decomposition]
  Assume $\sqrt{I} = I$ and $I \subset J$, then $\Vc(I : J) = \overline{\Vc(I) \setminus \Vc(J)}$.
  and $\Vc(I) = \Vc(J) \cup \Vc(I: J)$.
  \begin{proof}
    % TODO
  \end{proof}
\end{theorem}

\begin{example}
  Let $I = \gen{ xz - y^2, x^3 - yz }$ and $V = \Vc(I)$. \\

  Notice that $\gen{ xz - y^2 , x^3 - yz } \subseteq \gen{ x, y } = J$,
  so $(I: J) = (I: \gen{ x }) \cap (I: \gen{y})$.

  First we calculate $(I: x)$. Notice that we know how to calculate $I \cap \gen{x}$ now.
  After a calculation, $I \cap \gen{x} = \Set{x^2z - xy^2, x^4 - xyz, x^3y - xz^2}$,
  so $(I: x) = \gen{f_1/x, f_2/x, f_3/x} = I + \gen{x^2y - z^3}$.
  Simarly one could find that $(I: y) = (I: x)$, thus $(I: J) = (I: x)$.

  Hence $V = \Vc(x, y) \cap \Vc(xz-y^2, x^3-yz, x^2y - z^2)$.
\end{example}

\begin{remark}
  In general, if $W \subseteq \Ab_k^n$ is an affine algebraic set defined by
  $x_i = f_i(t_1, \dots, t_m)$, then $W$ is irreducible.
  \begin{proof}
    % TODO
  \end{proof}
\end{remark}

\begin{prop}
  Let $f : V \to W$, then $\overline{f(V)} = \Vc(\ker f^*)$ where $f^* : k[W] \to k[V]$.
  \begin{proof}
    % TODO 這邊好像出了問題
  \end{proof}
\end{prop}
