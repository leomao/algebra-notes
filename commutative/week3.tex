%! TEX root=../main.tex
\subsection{Gr\"{o}bner basis}

\subsubsection{Buchberger's algorithm}

Let $I = \langle g_1, \dots, g_m \rangle$ and $G = \Set{ g_1, \dots, g_m }$.
A Gr\"{o}ner basis of $I$ can be constructed by the following algorithm:
\begin{enumerate}
  \item Initially let $G_0 \gets G$.
  \item Repeatly construct $G_{i+1} \gets G_i \cup \big( \Set{S(f, g) \mod G_i \mid f, g \in G_i} \setminus \Set{0} \big)$,
    until once $G_{i+1} = G_i$, then $G_i$ is a Gr\"{o}ner basis.
\end{enumerate}

\begin{lemma}
  Let $f_1, \dots, f_m \in K[x_1, \dots, x_n]$ with $a_1, \dots, a_m \in K$ satisfy
  $\partial(f_1) = \partial(f_2) = \dots = \partial(f_m) = \alpha$ and $\partial\big( \sum_{i = 1}^m a_i f_i \big) < \alpha$.
  Then $h = \sum_{i = 2}^m b_i S(f_{i-1}, f_i)$ for some $b_i \in K$.
\end{lemma}

\begin{theorem}[Buchberger's criterion]
  Assume $I = \langle g_1, \dots, g_m \rangle$, then
  $G = \Set{g_1, \dots, g_m}$ is a Gr\"{o}bner basis of $I$ $\iff$ $S(g_i, g_j) \equiv 0 \pmod{G}$ for each $i, j$.
\end{theorem}
