%! TEX root=../main.tex
\subsection{Local Rings (week 12)}

From now on, $R$ is a commutative ring with $1$.

\begin{definition}
  $R$ is called a local ring if it has a unique maximal ideal.
\end{definition}

\begin{prop} \mbox{}
  \begin{enumerate}[(1)]
    \item $R$ is a local ring.
    \item The set of non-units forms an ideal.
    \item $\exists M \in \Max R$ s.t. $1 + m$ is a unit $\forall m \in M$.
  \end{enumerate}

  \begin{proof} \mbox{}
    \begin{description}
      \item[$(1)\Rightarrow(2)$:]
        Let $M$ be the unique maximal ideal of $R$. Then $M$ couldn't contain
        any unit. For each non-unit $x$,
        $\gen{x} \neq R$ and is contained in a maximal ideal by
        lemma~\ref{lemma:max-ideal-exists}, thus $x \in M$.
        Hence $M = \Set{ \text{non-units} }$.
      \item[$(2) \Rightarrow (3)$:]
        This ideal must be a maximal ideal $M$ since it can't be extended.
        Now, $1 \notin M \leadsto 1 + m \notin M$. So $1 + m$ is a unit.
      \item[$(3)\Rightarrow(1)$:]
        If there exists another maximal ideal $N$, then $M + N = R$.
        Say $m \in  M, n \in N$ s.t. $m + n = 1$, then $n = 1 - m$ is a unit
        $\implies N = R$, which is a contradiction.
        \qedhere
    \end{description}
  \end{proof}
\end{prop}

\begin{example}
  $k[[x]]$ is a local ring with the unique maximal ideal $\gen{x}$.
  \begin{proof}
    For each $f = \sum a_n x^n \in k[[x]]$, one could see that $f$ is
    an unit if and only if $a_n \neq 0$, and the leftovers form an
    ideal $\gen{x}$.
  \end{proof}
\end{example}

\begin{example}
  Let $P \in \Spec R$. If $S = R \setminus P$, then $S$ is a multiplicatively
  closed set with $1 \in S$ and $R_P \triangleq R_S$ is a local ring.

  \begin{proof}
    $S$ is a multiplicatively closed set simply follow from the definition of prime ideal.
    One could then easily see that $P_P \triangleq \Set*{\frac{x}{s} \mid x \in P, s \in S}$
    contains all non-unit, thus $R_P$ is local.
  \end{proof}
\end{example}


\begin{prop}
  The following sets are correspondent ($k$ is algebraically closed):
  \begin{enumerate}[(1)]
    \item $\Ab_k^n$
    \item $\Max k[x_1, \dots, x_n]$
    \item $\Hom_k(k[x_1, \dots, x_n], k)$
  \end{enumerate}
  \begin{proof}
    (1) $\Rightarrow$ (2): For any $(a_1, \dots, a_n) \in \Ab_k^n$,
    $k[x_1, \dots, x_n] / \gen{x_1 - a_1, \dots, x_n - a_n} \cong k$ is a field,
    hence $\gen{x_1 - a_1, \dots, x_n - a_n}$ is a maximal ideal.

    (2) $\Rightarrow$ (1): Let $M \in \Max k[x_1, \dots, x_n]$,
    by theorem~\ref{thm:weak-hilbert-null}, $\Vc(M) \neq \varnothing$,
    so exists $(a_1, \dots, a_n) \in \Vc(M)$.
    Now $M = \sqrt{M} = \Ic(\Vc(M)) \subseteq \Ic((a_1, \dots, a_n)) = \gen{\dots, x_i - a_i, \dots}$
    which is maximal,
    We conclude that $(a_1, \dots, a_n)$ is the only element in $\Vc(M)$
    and $M = \gen{\dots, x_i - a_i, \dots}$.

    (1) $\Rightarrow$ (3): For each $(a_1, \dots, a_n)$, define $\varphi \in \Hom_k(\dotsm)$ by
    evaluation:
    \[
      \begin{tikzcd}[column sep=0.8cm,row sep=0ex]
        \varphi: &[-0.7cm] k[x_1, \dots, x_n] \arrow[r] & k \\
        & x_i \arrow[r, mapsto] & a_i
      \end{tikzcd}
    \]

    (3) $\Rightarrow$ (1): Similarly, for each $\varphi \in \Hom_k(\dotsm)$,
    recover $(a_1, \dots, a_n)$ by $(\varphi(x_1), \dots, \varphi(x_n))$.
  \end{proof}
\end{prop}

\begin{remark}
  Inspired by the correspondence,
\end{remark}

\begin{definition}
  A property of an $R$-module $M$ is said to be a local property if
  \[
    M \text{ has this property } \iff M_P
    \text{ (as an $R_P$-module) has this property } \forall P \in \Spec R
  \]
\end{definition}

\begin{prop} \label{prop:module-is-zero-iff-zero-in-localization}
  TFAE
  \begin{enumerate}[(1)]
    \item $M = 0$
    \item $M_P = 0 \quad \forall P \in \Spec R$
    \item $M_Q = 0 \quad \forall Q \in \Max R$
  \end{enumerate}
  \begin{proof}
    (1) $\Rightarrow$ (2) and (2) $\Rightarrow$ (3) are clear.

    (3) $\Rightarrow$ (1):
    If $M \neq 0$, let $x \in M$ such that $x \neq 0$,
    then $\Ann(x) \subsetneq R$ since $1 \not\in \Ann(x)$.
    Let $\Ann(x) \subset Q \in \Max R$.
    By assumption, $M_Q = 0$ implies $\frac{x}{1} = \frac{0}{1}$.
    By the definition of equal in localization, $\exists r \not\in Q$
    such that $rx = 0$, thus $r \in \Ann(x)$ which leads to an contradiction.
  \end{proof}
\end{prop}

\begin{coro} \label{coro:modules-are-equal-iff-equal-in-localization}
  Let $N \subseteq M$, TFAE (consider $\quot{M}{N}$)
  \begin{enumerate}[(1)]
    \item $N = M$
    \item $N_P = M_P \quad \forall P \in \Spec R$
    \item $N_Q = M_Q \quad \forall Q \in \Max R$
  \end{enumerate}
\end{coro}

\begin{prop} \label{prop:localization-preserves-exactness}
  TFAE
  \begin{enumerate}[(1)]
    \item $0 \to M \xrightarrow{\phi} N \xrightarrow{\psi} L \to 0$ exact
    \item $0 \to M_P \xrightarrow{\phi_P} N_P \xrightarrow{\psi_P} L \to 0$
      exact $\forall P \in \Spec R$
    \item $0 \to M_Q \xrightarrow{\phi_Q} N_Q \xrightarrow{\psi_Q} L \to 0$
      exact $\forall Q \in \Max R$
  \end{enumerate}
  \begin{proof}
    (1) $\Rightarrow$ (2): By the fact that localization preserves exact sequence.

    (2) $\Rightarrow$ (3): Obvious.

    (3) $\Rightarrow$ (1): Let $K = \ker \phi$, then $0 \to K \to M \to N$
    exact. Since we just proved (1) $\Rightarrow$ (3),
    $0 \to K_Q \to M_Q \to N_Q$ exact, but $K_Q = 0$,
    by proposition~\ref{prop:module-is-zero-iff-zero-in-localization},
    $K = 0$.

    We could prove the other half similarly by letting $K$ to be the cokernel.
  \end{proof}
\end{prop}

\begin{definition} \mbox{}
  \begin{itemize}
    \item Let $R \subseteq S$. $\bar{R} = \Set{ x\in S \given x
      \text{ is integral over } R}$ is called the integral closure of $R$ in $S$.
    \item $R$ is integrally closed in $S$ if $R = \bar{R}$.
    \item An integral domain $R$ is called normal if $R$ is integrally closed
      in its field of fractions.
  \end{itemize}
\end{definition}

\begin{theorem}
  UFD is normal.
  \begin{proof}
    Let $R$ be a UFD and $K$ be its field of fractions.
    If $a \in K$ is integral over $R$ and $a^n + r_1a^{n-1} + \dots + r_n = 0$.
    Write $a = u/s$ with $\gcd(u, s) = 1$.
    Then $u^n + r_1 s u^{n-1} + \dots + r_n s^n = 0$.
    Now if $s$ is a non-unit, says $p \mid s$ with $p$ is a prime. Then
    $p \mid u$ obviously $\leadsto p \mid \gcd(u, s) = 1$, which is a contradiction.
    So $s$ is a unit $\implies a \in R$.
  \end{proof}
\end{theorem}

\begin{prop} \mbox{}
  \begin{itemize}
    \item Let $S/R$ is an integral extension and $T \subset R$ be a m.c. set
      with $1 \in T$.
      Then $S_T$ is also integral over $R_T$.

    \begin{proof}
      Let $a/t \in S_T$ with $a^n + r_1 a^{n-1} + \dotsb + r_n = 0$,
      then we have
      \[ \left(\frac{a}{t}\right)^n +
          \frac{r_1}{t}\left(\frac{a}{t}\right)^{n-1} +
          \dotsb +
          \frac{r_n}{t^n}\left(\frac{a}{t}\right)^n = 0 \]
        Thus $a/t$ is integral over $R_T$.
    \end{proof}
    \item Let $S/R$ be an arbitrary extension and $T \subset R$ be m.c.
      with $1 \in T$. Then $(\bar{R})_T = \ob{(R_T)}$ in $S_T$.

    \begin{proof}
      By 1., $(\overline{R})_T$ is integral over $R_T$.
      If $a/t \in S_T$ is integral over $R_T$, say
      \[ \left(\frac{a}{t}\right)^n +
          \frac{r_1}{t_1}\left(\frac{a}{t}\right)^n +
          \dotsb +
          \frac{r_n}{t_n}\left(\frac{a}{t}\right)^n = 0 \]
      If we let $v = t_1 t_2 \dotsm t_n$,
      multiply the equation by $(tv)^n$, we get
      \[ (va)^n +
          (r_1 t t_2 \dotsm t_n) (v a)^{n-1} +
          \dotsb = 0 \implies va \in \overline{R} \]
      So $a/t = va / (vt) \in \overline{R}_T$.
    \end{proof}
  \end{itemize}
\end{prop}

\begin{prop}
  ``Being normal'' is a local property. TFAE
  \begin{enumerate}[(1)]
    \item $R$ is normal
    \item $R_P$ is normal $\forall P \in \Spec R$
    \item $R_Q$ is normal $\forall Q \in \Max R$
  \end{enumerate}

  \begin{proof}
    The key is to realize that if $K$ is the field of fraction of $R$,
    then $K$ is also the field of fraction of any $R_P$. Then by
    lemma~\ref{prop:localization-preserves-exactness},
    \[ 0 \to R \to \overline{R} \to 0 \iff 0 \to R_P \to (\overline{R})_P \to 0, \, \forall P \]
    By the previous proposition, $(\overline{R})_P = \overline{R_P}$ in $S_P$, this proves all.
  \end{proof}
\end{prop}

\begin{definition}
  An $R$-module $F$ is \index{Module!flat module} flat if the functor
  ${-} \otimes_R M$ is exact (i.e., it preserves exact sequence).
\end{definition}

\begin{prop}
  Given an homomorphism $R_1 \to R_2$.
  If $M$ is a flat $R_1$-module, then $R_2 \otimes_{R_1} M$ is a flat $R_2$ module.

  \begin{proof}
    Notice that $N \otimes_{R_1} M \cong N \otimes_{R_2} (R_2 \otimes_{R_1} M)$,
    so
    \begin{align*}
      0 \to N \to N' \text{ exact }
      &\implies 0 \to N \otimes_{R_1} M \to N' \otimes_{R_1} M \text{ exact } \\
      &\implies 0 \to N \otimes_{R_2} (R_2 \otimes_{R_1} M)
      \to N' \otimes_{R_2} (R_2 \otimes_{R_1} M) \text{ exact }
    \end{align*}
    Which is to say that $R_2 \otimes_{R_1} M$ flat.
  \end{proof}
\end{prop}

\begin{prop}
  TFAE
  \begin{enumerate}[(1)]
    \item $M$ is a flat $R$-module
    \item $M_P$ is a flat $R$-module $\forall P \in \Spec R$
    \item $M_Q$ is a flat $R$-module $\forall Q \in \Max R$
  \end{enumerate}

  \begin{proof}
    (1) $\Rightarrow$ (2): By the previous proposition combined with the property
    of localization, $M_P \cong R_P \otimes_R M$ is a flat module.

    (2) $\Rightarrow$ (3): Obvious.

    (3) $\Rightarrow$ (1):
    If $0 \to N \to N'$ exact, then by prop~\ref{prop:localization-preserves-exactness},
    $0 \to N_Q \to N'_Q$ exact, so
    \[ 0 \to N_Q \otimes_{R_Q} M_Q \to N'_Q \otimes_{R_Q} M_Q \]
    is also exact. By the property of localization, $N_Q \otimes_{R_Q} M_Q \cong (N \otimes_R M)_Q$.
    Using prop~\ref{prop:localization-preserves-exactness}, $0 \to N \otimes_R M \to N' \otimes_R M$ exact.
  \end{proof}
\end{prop}

\subsection{Krull dimension}
\begin{definition} \hfill
  \begin{itemize}
    \item 
      The Krull dimension of a topological space $X$ is the supremum of the lengths $n$ of
      all chains $X_0 \subsetneq X_1 \subsetneq \dots \subsetneq X_n$, where $X_i$ are closed irreducible subset of $X$.
    \item
      The Krull dimension of a commutative ring $R$ with $1$ is the
      supremum of the lengths $n$ of all chains $P_0 \subsetneq \dots \subsetneq P_n$ where $P_i \in \Spec R$.
  \end{itemize}
\end{definition}

\begin{prop}
  Let $R \subseteq S$ be two integral domains and $S/R$ be integral. Then $S$ is a field if and only if $R$ is a field.

  \begin{proof}
    ``$\Rightarrow$'': For each $a \neq 0$ in $R$, $a^{-1} \in S$, so we could write
    \[ (a^{-1})^n + r_1 (a^{-1})^{n-1} + \dotsb + r_n = 0, \ r_i \in R \]
    Which implies
    \[ a^{-1}  = - (r_1 + \dotsb + r_n a^{n-1}) \in R \]

    ``$\Leftarrow$'': For each $a \neq 0$ is $S$, write
    \[ a^n + r_1 a^{n-1} + \dotsb + r_n = 0, \ r_i \in R \]
    Notice that we could assume $r_n \neq 0$, or else $a (a^{n-1} + r_1 a^{n-2} + \dotsb + r_{n-1}) = 0$
    and hence $a^{n-1} + r_1 a^{n-2} + \dotsb + r_{n-1} = 0$ because $R$ is an integral domain.
    Then
    \[ a^{-1} = - r_n^{-1} (a^{n-1} + r_1 a^{n-2} + \dotsb + r_{n-2}) \]
  \end{proof}
\end{prop}

\begin{prop} \label{prop:property-of-contraction-if-integral}
  Let $S/R$ be integral.
  \begin{enumerate}
    \item If $q \in \Spec S$ and $p = q \cap R \in \Spec R$, then
      $q \in \Max S \iff p \in \Max R$.

    \begin{proof}
      It is easy to see that $S/q$ is integral over $R/p$ by the identification
      \[
        \begin{tikzcd}[column sep=0.8cm,row sep=0ex]
          R/p \arrow[r, hookrightarrow] & S/p \\
          r+p \arrow[r, mapsto] & r+q
        \end{tikzcd}
      \]
      So \[ q \in \Max S \iff S/q \text{ is a field } \iff R/p \text{ is a field } \iff p \in \Max R \]
    \end{proof}
    \item If $q, q' \in \Spec S$ with $q \subseteq q'$ and $q \cap R = p = q' \cap R$.
      Then $q = q'$.
    \begin{proof}
      We know that $S_P \triangleq S_{R \setminus P}$ is integral over $R_P$.
      Since $q_P \subseteq q'_P$ and both $q_p \cap R_P$ and $q'_p \cap R_P$
      equal $P_P$ is maximal in $R_P$. Using 1., $q_P, q'_P$ are maximal
      in $S_P$, but $q_P \subseteq q'_P \implies q_P = q'_P$.
      By corollary~\ref{coro:modules-are-equal-iff-equal-in-localization},
      $q = q'$.
    \end{proof}
  \end{enumerate}

\end{prop}

\begin{theorem}[Going-up theorem] \label{thm:going-up}
  Let $S/R$ be integral, then
  \begin{itemize}
    \item If $p \in \Spec R$, then $\exists q \in \Spec S$ such that $q \cap R = p$.
      \begin{proof}
        We have the diagram:
        \[
          \begin{tikzcd}[cramped]
            p \ar[r, rightsquigarrow] & R \ar[r, hookrightarrow] \ar[d] & S \ar[d] \\
            P_p \ar[r, rightsquigarrow] & R_p \ar[r, hookrightarrow] & S_p \\
          \end{tikzcd}
        \]
        Pick $q_p = N \in \Max S_p$, then $N \cap R_p \in \Max R_p = \Set{P_p}$
        by 1.\ of proposition~\ref{prop:property-of-contraction-if-integral},
        so $N \cap R_p = P_p$, and $(q \cap R)_p = q_p \cap R_p = P_p$,
        thus $q \cap R = p$.
      \end{proof}
    \item If $p_1 \subset p_2$ in $\Spec R$ and $q_1 \in \Spec S$ with $q_1 \cap R = p_1$,
      then $\exists q_2 \in \Spec S$ with $q_1 \subset q_2$ and $q_2 \cap R = p_2$.
      \begin{proof}
        Let $R' = R/p_1$ and $S' = S/q_1$. Then again, $S'/R'$ is integral.
        By the previous statement, exists $q_2/q_1 \in \Spec S'$
        so that $q_2 / q_1 \cap R' = p_2 / p_1$, thus
        $q_2 \cap R = p_2$ and $q_2 \supseteq q_1$.
      \end{proof}
  \end{itemize}
\end{theorem}

\begin{theorem}
  If $S/R$ is integral, then $\dim S = \dim R$.

  \begin{proof}
    For any chain $q_0 \subsetneq q_1 \subsetneq \dotsm \subsetneq q_n$
    in $\Spec S$, by prop 2., $q_0 \cap R \subsetneq q_1 \cap R \subsetneq
    \dotsm \subsetneq a_n \cap R$.

    Conversely, given $p_0 \subsetneq p_1 \subsetneq \dotsm \subsetneq p_n$
    in $\Spec R$, there is $q_0 \subsetneq q_1 \subsetneq \dotsm \subsetneq q_n$
    by the going up theorem (\ref{thm:going-up}).
  \end{proof}
\end{theorem}

\begin{prop} \label{prop:integral-then-coefficients-in-radical}
  Let $S, R$ be integral domains and $S/R$ be integral, assume $R$ is normal
  with the field of fractions $K$. If $a \in S$ is integral over $I \subseteq R$, then
  $f = m_{\alpha, K} = x^n + r_1 x^{n-1} + \dotsm + r_n$ with $r_i \in \sqrt{I}$.

  \begin{proof}
    Assume $\deg f = n$ and $a_1, \dots, a_n \in \overline{K}$
    are the zeros of $f$. By assumption, $a^m + t_1 a^{m-1} + \dotsm + t_m = 0$
    with $t_i \in I \subset R \subset K$.
    For each $i$, exists $\varphi \in \Aut(\overline{K}/K)$ such that
    $\varphi(a) = a_i$.
    Then $0 = \varphi(a^m + t_1 a^{m-1} + \dots + t_m) = a_i^m + t_1 a_i^{m-1} + \dots
    + t_m$, so $a_i$ is integral over $I$. Moveover, the coefficients
    of $f$ are the elementary symmetry symmetric polynomial of $a_i$,
    hence they are integral over $I$ and lie in $\sqrt{I \overline{R}} = \sqrt{IR} = \sqrt{I}$.
  \end{proof}
\end{prop}

\begin{theorem}[Going-down theorem] \label{thm:going-down}
  Let $S, R$ be integral domains and $S/R$ be integral, assume $R$ is normal
  with the field of fractions $K$. If $p_1 \supset p_2$ in $\Spec R$ and $q_1 \in \Spec S$
  with $q_1 \cap R = p_1$,
  then $\exists q_2 \in \Spec S$ such that $q_1 \supset q_2$ and $q_2 \cap R = p_2$.

  \begin{proof}
    First we claim that $p_2 S_{q_1} \cap R = p_2$.

    ``$\supseteq$'': Obvious.

    ``$\subseteq$'': For $b/t \in p_2 S_{q_1} \cap R$, $b \in p_2 S \subset \sqrt{p_2 S}
    = \sqrt{p_2 \overline{R}}$, which means that $b$ is integral over $p_2$
    and $t \in S \setminus q_1$. By proposition~\ref{prop:integral-then-coefficients-in-radical},
    if $m_{b, K} = x^l + r_1 x^{l-1} + \dots + r_l$, then $r_i \in \sqrt{p_2} = p_2$.

    Now, $a = b/t \in R$, so $t = b/a \in S_{R \setminus \Set{0}} = SK$,
    so
    \[ \left( \frac{b}{a} \right)^l + 
      \left(\frac{r_1}{a}\right) \left( \frac{b}{a} \right)^{l-1} + \dots
      + \left(\frac{r_l}{a^l}\right) \leftrightarrow b^l + r_1 b^{l-1} + \dots + r_l = 0 \]
    is a correspondence. Thus we know that
    $m_{t, K} = x^l + (r_1/a) x^{l-1} + \dots + (r_l/a^l)$.

    Again by proposition~\ref{prop:integral-then-coefficients-in-radical},
    since $t$ is integral over $R$, $u_i \triangleq r_i / a^i \in R$,
    and $u_i a^i = r_i$ for each $i$.

    If $a \not\in p_2$, then $u_i a^i = r_i \in p_2$, so $u_i \in p_2$.
    But with $m_{t, K}$ we will find that $t^l \in p_2 S \subseteq p_1 S \subseteq q_1$,
    so $t \in q_1$, which leads to an contradiction. Thus $a \in p_2$.

    Now we've proved $p_2 S_{q_1} \cap R = p_2$,
    by exercise 12.4, $p_2 = Q \cap R$ for some $Q \in S_{q_1}$.
    Letting $q = Q \cap S$ and we're done.
  \end{proof}
\end{theorem}

\begin{theorem}
  All maximal chain in $\Spec K[x_1, \dots, x_n]$ have the same length $n$, and thus
  \[ \dim K[x_1, \dots, x_n] = n. \]

  \begin{proof}
    Let $P_0 \subset P_1 \subset \dots \subset P_m$ in $\Spec K[x_1, \dots, x_n]$
    We shall use induction on $n$ to prove $m = n$.

    $n = 0$: Then $\gen{0}$ is a max chain in $\Spec K$, so $m = 0 = n$.

    $n > 0$: Let $K[y_1, \dots, y_n] \hookrightarrow K[x_1, \dots, x_n]$
    be a strong Noether normalization with $P_1 \cap K[y_1, \dots, y_n] = \gen{y_{d+1}, \dots, y_n}$,
    then $h(p_1) = 1 \implies h(p_1 \cap k[y_1, \dots, y_n])$ by
    the going down theorem (\ref{thm:going-down}). (為什麼這樣就證完了???)
  \end{proof}
\end{theorem}
