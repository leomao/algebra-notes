%! TEX root=../main.tex
\subsection{The equivalence of algebra and geometry (week 10)}

In the following, $k$ will be an algebraically closed field.
\begin{definition}
  The category of affine algebraic sets $\mathcal{G}$, which its objects and morphisms are defined as following.

  \begin{description}[leftmargin=0cm]
    \item[objects:] The objects are affine algebraic sets in $k^n$.

    An {\bf affine algebraic set} is the common zero set of $\{ F_i \}_{i \in \Lambda} \subset k[x_1, \dots, x_n]$
    in $k^n$.
    We denote it by $V = \Vc(\Set{F_i}_{i \in \Lambda}) \subset k^n$.
    (In fact, $I = \langle F_i : i \in \Lambda \rangle$ is Noetherian, so
    $I = \langle F_1, \dots, F_n \rangle$ and $V = \Vc(I)$.)

    \item[morphisms:] The morphisms are the polynomial map from $k^n$ to $k^m$.

      A {\bf polynomial map} is a mapping as following:
      \[
        \begin{tikzcd}[column sep=1cm,row sep=0ex]
            k^n \arrow[r]& k^m \\
           \alpha \arrow[r, mapsto] & (F_1(\alpha), \dots, F_m(\alpha))
        \end{tikzcd}
      \]
      where each $F_i$ is a polynomial in $K[x_1, \dots, x_n]$.

      Given two affine algebraic sets $V \subset k^n$ and $W \subset k^m$, if a map $F: V \to W$ is
      the restriction of a polynomial map from $k^n$ to $k^m$, then $F$ is a morphism from $V$ to $W$.

      Moreover, if $F: V \to W$ and $G : W \to V$ satisfy $F \circ G = \Id$ and $G \circ F = \Id$,
      then we say $V \cong W$.
  \end{description}
\end{definition}

\begin{definition}
  The category of finitely generated reduced $k$-algebra $\mathcal{A}$,
  which its objects and morphisms are defined as following.

  \begin{description}
    \item[objects:] The objects are the reduced finitely generated $k$-algebra $R$.

    A finitely generated $k$-algebra $R$ is reduced if $R$ has no non-zero nilpotent elements.

    \item[morphisms:] The morphisms are the $k$-algebra homomorphisms.
  \end{description}
\end{definition}

\begin{example}
  It is easy to see that $\Vc(0) = k^n$ and $\Vc(1) = \varnothing$.
\end{example}

\subsubsection{One-one correspondence between affine algebraic sets and radical ideals}

\begin{definition}
  Define $\Ic(V) = \Set{ f \in k[x_1, \dots, x_n] \mid f(\alpha) = 0, \, \forall \alpha \in V}$.
\end{definition}

The one-one correspondence is given by
  \[
    \begin{tikzcd}[column sep=0.8cm,row sep=0ex]
      \Set{ \text{affine algebraic sets in } \Ab^n_k }
      \arrow[r, leftrightarrow] & \Set{ \text{ radical ideals in } k[x_1, \dots, x_n]} \\
      V \arrow[r, mapsto] & \Ic(V) \\
      \Vc(I) \arrow[r, mapsfrom] & I
    \end{tikzcd}
  \]

\begin{prop} \hfill
\begin{itemize}
  \item $\sqrt{\Ic(V)} = \Ic(V)$.
    \begin{proof}
      For all $f^n \in \Ic(V)$,
      $f^n(\alpha) = 0, \forall \alpha \in V \implies f(\alpha) = 0, \forall \alpha \in V$.
      Thus $f \in \Ic(V)$.
    \end{proof}
  \item If $V$ is an affine set, then $\Vc(\Ic(V)) = V$.
    \begin{proof}
      ``$\supset$'': $\forall \alpha \in V,\, f \in \Ic(V)$, $f(\alpha) = 0 \implies \alpha \in \Vc(\Ic(V))$.

      ``$\subset$'': Since $V$ is an affine set, $V = \Vc(I)$, then $I \subset \Ic(V)$,
      so $\Vc(\Ic(V)) \subset \Vc(I) = V$.
    \end{proof}
\end{itemize}
\end{prop}

\begin{lemma} \label{lemma:module-finite-ring-finite}
  Given $T/S/R$, a tower of rings.
  If $R$ is Noetherian, $T/S$ is a module finite and $T/R$ is a ring finite, then $S/R$ is a ring finite.

  \begin{proof}
    Let $T = R[a_1, \dots, a_n] = S \omega_1 + \dots + S \omega_m$.
    Then $a_i = \sum_j r_{i, k} \omega_k$ for some $r_{i, k}$
    and $\omega_{i, j} = \sum t_{i, j, k} w_k$ for some $t_{i, j, k}$.

    Let $S' = R\big[ \Set{r_{i, k}}, \Set{t_{i, j, k}} \big] \subseteq S$, which
    is Noetherian by the Hilbert basis theorem ($R$ Notherian $\implies$ $R[x]$ Notherian).
    Thus $T = S' \omega_1 + \dots + S' \omega_m$ is a Noetherian $S'$-module
    by the fact that finitely generated module over a Noetherian ring is a Noetherian module.

    Since $S \subset T$, $S$ is a finitely generated $S'$ submodule,
    so $S = S' v_1 + \dots + S' v_r = R\big[ \Set{r_{i, k}}, \Set{t_{i, j, k}}, \Set{v_i} \big]$.
  \end{proof}
\end{lemma}

\begin{lemma} \label{lemma:transcendental-is-not-ring-finite}
    If $S = k(z_1, \dots, z_p), \, p > 0$ with each $z_i$ transcendental, then $S/k$ is not ring finite.

  \begin{proof}
    If not, say $S = k[f_1, \dots, f_n]$ with $f_i = g_i / h_i$, $g_i, h_i \in k[z_1, \dots, z_p]$.
    Then for any irreducible polynomial $p$ such that $p \nmid h_i$ for each $h_i$
    (This polynomial exists since for each $h_i$ there are only finite degree $1$ factors).
    Then $1/p \notin k[f_1, \dots, f_n]$ by checking the divisibility of the denominator
    under addition and multiplication, which leads to a contradiction.
  \end{proof}
\end{lemma}

\begin{lemma} \label{lemma:ring-finite-is-alg}
  If $A/k$ is an extension of fields and ring finite, then $A/k$ is algebraic.

  \begin{proof}
    If $A/k$ is transcendental and let $\Set{z_1, \dots, z_t}$ be a transcendental base.
    Then $A/k(z_1, \dots, z_t)$ is algebraic, thus a module finite.
    By lemma~\ref{lemma:module-finite-ring-finite}, $k(z_1, \dots, z_t)$ is ring finite,
    which contradict with lemma~\ref{lemma:transcendental-is-not-ring-finite}.
  \end{proof}
\end{lemma}

\begin{theorem}[Weak form of Hilbert Nullstellensatz] \label{thm:weak-hilbert-null}
  \[ I \subsetneq k[x_1, \dots, x_n] \implies v(I) \neq \varnothing \]

  \begin{proof}
    Since $I$ proper, by lemma~\ref{lemma:max-ideal-exists},
    exists a maximal ideal $M$ such that $I \subseteq M$.
    Consider $K \triangleq k[x_1, \dots, x_n] / M = k[\bar{x}_1, \dots, \bar{x}_n]$.
    By proposition~\ref{prop:max-prime-to-field-int-domain}, $K$ is a field,
    and by lemma~\ref{lemma:ring-finite-is-alg}, $K/k$ is algebraic. Since $k$
    is already algebraically closed, $K = k$ and hence each $\bar{x}_i \in k$.
    Let $\alpha \triangleq (\bar{x}_1, \dots, \bar{x}_n) \in A_k^n$, then
    for any $f \in M$, $f(\alpha) = f(\bar{x}_1, \dots, \bar{x}_n) = \bar{f} = 0$,
    thus $\alpha \in \Vc(M) \subseteq \Vc(I)$.
  \end{proof}
\end{theorem}

\begin{theorem}[Strong form of Hilbert Nullstellensatz]
  $\Ic(\Vc(I)) = \sqrt{I}$

  \begin{proof}
    ``$\supset$'': $f \in \sqrt{I} \implies f^n \in I$, then
    $f^n(\alpha) = 0, \forall \alpha \in \Vc(I)
    \implies f(\alpha) = 0, \forall \alpha \in \Vc(I)$, thus $f \in \Ic(\Vc(I))$.

    ``$\subset$'': If $\Ic(\Vc(I)) = 0$, then $I \subseteq \sqrt{I} \subseteq \Ic(\Vc(I)) = 0$,
    thus $I = 0$.

    Otherwise, exists $0 \neq f \in \Ic(\Vc(I))$, Let $J = \langle I, ft-1 \rangle \subset k[x_1, \dots, x_n, t]$.
    If $(a_1, \dots, a_n, t_0)$ is a zero of $J$, then $ft-1 \in J \implies -1 = f(a_1, \dots, a_n) t_0 - 1 =
    0$, which is a contradiction, so by theorem~\ref{thm:weak-hilbert-null}, $J = k[x_1, \dots, x_n, t]$.

    Write $1 = \sum h_i f_i + s (ft-1)$, where each $f_i \in I$ and
    $h_i, s \in k[x_1, \dots, x_n, t]$.
    This is a equation of variables, so if we set $t = 1/f$, the equation still holds.
    Now each $h_i$ would be the form $\sum p_i / f^{k_i}$, so we could multiply each
    side by a suitable $f^\rho$ and get $f^\rho = \sum c_i f_i$ with each $c_i \in k[x_1, \dots, x_n]$.
    This implies $f^\rho \in I$, thus $f \in \sqrt{I}$.
  \end{proof}
\end{theorem}

\begin{definition}
  Let $V \in \mathcal{G}$, the coordinate ring of $V$ is $k[V] \triangleq k[x_1, \dots, x_n] / \mathcal{I}(V)$
\end{definition}

\subsubsection{Equivalence of $\mathcal{G}$ and $\mathcal{A}$}
We define a functor $F$ from $\mathcal{G}$ to $\mathcal{A}$ by
  \[
    \begin{tikzcd}[column sep=0.8cm,row sep=0ex]
      F: &[-0.7cm] \mathcal{G} \arrow[r] & \mathcal{A} \\
      & V \arrow[r, mapsto] & k[V]
    \end{tikzcd}
  \]
  And For a polynomial map $f : V \to W$, define
  \[
    \begin{tikzcd}[column sep=0.8cm,row sep=0ex]
      F(f) = f^*: &[-0.7cm] k[W] \arrow[r] & k[V] \\
      & g \arrow[r, mapsto] & g \circ f
    \end{tikzcd}
  \]

  Conversely, define a functor $G$ by

  \[
    \begin{tikzcd}[column sep=0.8cm,row sep=0ex]
      G: &[-0.7cm] \mathcal{A} \arrow[r] & \mathcal{G} \\
      & k[x_1, \dots, x_n]/I \arrow[r, mapsto] & \Vc(I)
    \end{tikzcd}
  \]

  Then if
  \[
    \begin{tikzcd}[column sep=0.8cm,row sep=0ex]
      \varphi: &[-0.7cm] k[\dots]/I \arrow[r] & k[\dots]/J \\
      & \bar{x}_i \arrow[r, mapsto] & \bar{f}_i
    \end{tikzcd}
  \]
  Define
  \[
    \begin{tikzcd}[column sep=0.8cm,row sep=0ex]
      G(\varphi) = \psi: &[-0.7cm] \Vc(J) \arrow[r] & \Vc(I) \\
      & \alpha = (a_1, \dots, a_m) \arrow[r, mapsto] & (f_1(\alpha), \dots, f_n(\alpha))
    \end{tikzcd}
  \]
