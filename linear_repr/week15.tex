%1 TEX root=../main.tex
\subsection{Week 15}
\subsubsection{Character Theory \RNum{2}}

\begin{prop}
  Let $\rho: G\to \text{GL}(V)$ and
  $\rho = \rho^{W_1} \oplus \dots \oplus \rho^{W_k}$ where
  $\rho_i = \rho^{W_i}$ is irr. $\forall i$.
  ($V \cong W_1 \oplus\dots\oplus W_k$)
  
  If $\tilde{\rho}: G\to \text{GL}(\tilde{W})$ is an irr. rep. then the number
  of $\rho_i$ isomorphic to $\tilde{\rho}$ is equal to
  $\inpd{\chi_\rho, \chi_{\tilde{\rho}}}$.
  \begin{proof}
    We know $\chi_\rho = \chi_{\rho_1} + \dots + \chi_{\rho_k}$, so
    \[
      \inpd{\chi_\rho, \chi_{\tilde{\rho}}}
      = \sum_{i=1}^k \inpd{\chi_{\rho_i}, \chi_{\tilde{\rho}}}
    \]
    Recall $\rho_i \cong \tilde{\rho} \implies
    \inpd{\chi_{\rho_i}, \chi_{\tilde{\rho}}} = 1$, otherwise 
    $\inpd{\chi_{\rho_i}, \chi_{\tilde{\rho}}} = 0$.
  \end{proof}
\end{prop}

\begin{remark} \mbox{}
  \begin{enumerate}
    \item The number of $W_i$ isomorphic to $\tilde{W}$ does not depend
      on the chosen decomposition. ($= \inpd{\chi_\rho, \chi_{\tilde{\rho}}}$)
    \item If $\chi_\rho = \chi_{\rho'}$, then $\rho \cong \rho'$:
      $\inpd{\chi_\rho, \chi_{\tilde{\rho}}} = \inpd{\chi_{\rho'}, \chi_{\tilde{\rho}}}$
      The type of irr. subrep of $\rho$ is the same as $\rho'$.

    \item If $\chi_1, \dots, \chi_l$ are distinct irr. characters of $G$, then
      since $x_1, \dots, x_l$ are orthonormal w.r.t. $\inpd{\cdot,\cdot}$ in
      $\Cc(G)$, $x_1, \dots, x_l$ are linearly indep. over $\Cb$ in $\Cc(G)$.

      But $\dim \Cc(G) = k =$ \# of conjugacy classes in $G$. So
      $l \le k$ i.e. we conclude that there are at most $k$ mutually
      non-equivalent irr. rep. of $G$, say $\rho_1, \dots, \rho_l, l \le k$.

      For any $\rho: G\to \text{GL}(V)$,
      $\rho \cong \rho_1^{\oplus m_1} \oplus\dots\oplus\rho_l^{\oplus m_l}$
      where $m_i = \inpd{\chi_\rho, \chi_{\rho_i}} \in \Zb^{\ge 0}$.
  \end{enumerate}
\end{remark}

\begin{theorem}[Orthogonality relations for $\chi$'s]
  The set of all irr. characters of $G$ forms an orthonormal {\bf basis}
  $\Cc(G)$ over $\Cb$. In particular, the number of irr. rep. of $G$ is equal
  to \# of conjugacy classes in $G$. (up to equivalence)

  \begin{proof}
    Let $\chi_i = \chi_{\rho_i}, i = 1, \dots, l$ be all irr.
    characters of $G$ and $\Dc = \gen{\chi_1, \dots, \chi_l}_\Cb
    \subseteq \Cc(G)$. Then $\Cc(G) = \Dc \oplus \Dc^\perp$.
    Claim: $\Dc^\perp = \{0\}$.

    Let $\varphi \in \Dc^\perp$, i.e. $\inpd{\varphi, \chi_i} = 0,
    \forall i = 1,\dots, l$.

    Write $\rho^{\text{reg}} \cong \rho_1^{\oplus m_1} \oplus\dots\oplus\rho_l^{\oplus m_l}
    \implies \chi^{\text{reg}} = m_1\chi_1 + \dots + m_k\chi_l$.
    By assumption, $\inpd{\varphi, \chi_\rho} = 0$.

    For each $i$, define ${\sf T}_{\rho_i} \in \Hom_\Cb(V, V)$ by
      \[
        {\sf T}_{\rho_i} \triangleq
        \frac{1}{\abs{G}} \sum_{g\in G} \ob{\varphi(g)} \rho_i(g)
      \] 
    Then we have
      \[
        \trace({\sf T}_{\rho_i}) =
        \frac{1}{\abs{G}} \sum_{g\in G} \ob{\varphi(g)} \chi_\rho(g)
        = \ob{\inpd{\varphi, \chi_\rho}} = 0
      \]

      Also, for all $h \in G$.
      \begin{align*}
        \rho(h)^{-1}\circ{\sf T}_{\rho_i} \circ\rho(h)
        &= \frac{1}{\abs{G}} \sum_{g\in G}
        \ob{\varphi(g)} \rho(h)^{-1}\circ\rho(g)\circ\rho(h) \\
        & \stackrel{*}{=} \frac{1}{\abs{G}} \sum_{g \in G} \ob{\varphi(h^{-1} g h)} \rho(h^{-1}gh) \\
        & = \frac{1}{\abs{G}} \sum_{g \in G} \ob{\varphi(g)} \rho(g) = \mathsf{T}_{\rho_i}
      \end{align*}
      Where $*$ is because $\varphi$ is a class function.  So $\mathsf{T}_{\rho_i}$ is $G$-equivariant.
      By Schur's lemma,
      ${\sf T}_{\rho_i} = \lambda_i 1_{W_i}$ where
      $\rho_i: G\to \text{GL}(W_i)$. \\
      But $\trace{{\sf T}_{\rho_i}} = 0 \implies \lambda_i = 0
      \implies {\sf T}_{\rho_i} = 0$.

      Also, because $\rho \cong \rho_1^{\oplus m_1} \oplus\dots\oplus\rho_l^{\oplus m_l}$,
      if we define
      \[
        {\sf T}_{\rho^{\text{reg}}} \triangleq
        \frac{1}{\abs{G}} \sum_{g\in G} \ob{\varphi(g)} \rho^{\text{reg}}(g)
        \implies \mathsf{T}_{\rho^{\text{reg}}}
        = \mathsf{T}^{\oplus m_1}_{\rho_1} \oplus \cdots \oplus
        \mathsf{T}^{\oplus m_k}_{\rho_k} = 0
      \] 

      Finally, let $\rho = \rho^{\text{reg}}: G\to \text{GL}(V)$ with
      $V = \bigoplus_{g\in G} \Cb e_g$.
      Then ${\sf T}_\rho = 0 \implies {\sf T}_\rho(e_1) = 0$ and
      \[
        0 = {\sf T}_\rho(e_1) = \frac{1}{\abs{G}} \sum_{g\in G}
        \ob{\varphi(g)} \rho(g)(e_1) = \frac{1}{\abs{G}} \sum_{g\in G}
        \ob{\varphi(g)} e_g
      \]
      Since $\{ e_g \}$ is a basis, $\ob{\varphi(g)} = 0 \quad \forall g$.
      That is, $\varphi \equiv 0$. \qedhere
  \end{proof}
\end{theorem}

\begin{prop}
  Each irr. rep. $\rho_i: G\to \text{GL}(W_i)$ is contained in
  $\rho^\text{reg}$ with multiplicity equal to $\dim W_i = m_i$,
  $i = 1,\dots, k$.

  In particular, $\bigoplus_{g\in G} \Cb e_g \cong
  \underbrace{W_1\oplus\dots\oplus W_1}_{m_1 \text{times}} \oplus \dots \oplus
  \underbrace{W_1\oplus\dots\oplus W_k}_{m_k \text{times}}$.
  So $\abs{G} = m_1^2 + \dots + m_k^2$.

  \begin{proof}
    Let $\chi^\text{reg} \defeq \chi_{\rho^\text{reg}}$ and
    $\chi_i = \chi_{\rho_i}, i = 1, \dots, k$. Then
    \[
      \inpd{\chi^\text{reg}, \chi_i} = \frac{1}{\abs{G}} \sum_{g\in G}
      \chi^\text{reg}(g)\chi_i(g^{-1})
      = \frac{1}{\abs{G}} \abs{G} \chi_i(1) = m_i
    \]
  \end{proof}
\end{prop}

\begin{theorem}[Divisibility]
  $\forall i = 1, \dots, k, \quad \chi_i(1) = m_i \Div \abs{G}$.

  \begin{proof} \mbox{}
    First, we shall proof that for each $\rho=\rho_i$, $\chi = \chi_i$ and $j$, we have
    \[
      \sT \triangleq \sum_{g\in C_j} \rho(g) = \frac{\abs{C_j}\chi(g_0)}{m_i}
      \sI_{m_i}, \quad \text{for any } g_0 \in C_j
    \]

    Observe that $\forall h \in G$,
    \[
      \rho(h)^{-1}\circ\sT\circ\rho(h) = \sum_{g\in C_j} \rho(h^{-1}gh)
      = \sum_{g'\in C_j} \rho(g') = \sT
    \]
    So $\sT$ is $G$-equivariant w.r.t. $\rho$.

    By Schur's lemma, $\sT = \lambda \sI_{m_i}$ for some $\lambda \in \Cb$.
    And $\lambda = \trace(\sT) / m_i = \sum_{g\in C_j} \chi(g) / m_i = \abs{C_j}\chi(g_0) / m_i$
    for any $g_0 \in C_j$, thus $\sum_{g \in C_j} \rho(g) = \frac{\abs{C_j}\chi(g_0)}{m_j} \sI$
    for any $g_0 \in C_j$.
    
    Define $\lambda_\mu(C_i) \triangleq \abs{C_i}\chi_\mu(g_i) / m_\mu$.
    Now, for a $g \in C_l$, define $a_{i,j,l} \triangleq \# \{\, (g_i, g_j) \in C_i \times C_j \mid
    g_ig_j = g \,\}$, which is indep. of the choice of $g$.

    We claim that $\lambda_\mu(C_i)\lambda_\mu(C_j) = \sum_{l=1}^k a_{i,j,l} \lambda_\mu(C_j)
    ,\forall i, j, \mu$. Then 
    \[ \lambda_\mu(C_i) 
      \begin{bmatrix}
        \lambda_\mu(C_1) \\
        \vdots \\
        \lambda_\mu(C_k) \\
      \end{bmatrix}
      = A
      \begin{bmatrix}
        \lambda_\mu(C_1) \\
        \vdots \\
        \lambda_\mu(C_k) \\
      \end{bmatrix}, \quad
      \text{where } A \triangleq
      \begin{bmatrix}
        a_{i, 1, 1} & \hdots & a_{i, 1, k} \\
        \vdots & \ddots & \vdots \\
        a_{i, k, 1} & \hdots & a_{i, 1, k} \\
      \end{bmatrix}
    \]
    So $\lambda_\mu(C_j)$ is an eigenvalue of
    $A$, i.e., $\lambda = \lambda_\mu(C_j)$ satisfies $\det(\lambda I - A) = 0$.
    And thus $\lambda_\mu(C_i)$ is an algebraic integer.

    We proof the claim by the following calculating.
    \begin{align*}
      \lambda_\mu(C_i)\lambda_\mu(C_j) I_{m_\mu}
      &= \left(\lambda_\mu(C_i) I_{m_\mu}\right)
         \left(\lambda_\mu(C_j) I_{m_\mu}\right)
      = \left(\sum_{g\in C_i} \rho(g)\right)
         \left(\sum_{g'\in C_j} \rho(g')\right) \\
      &= \sum_{\substack{g\in C_i\\ g'\in C_j}} \rho(gg')
      = \sum_{l=1}^k \sum_{\bar{g}\in C_l} a_{i,j,l}\rho(\bar{g}) \\
      &= \sum_{l=1}^k a_{i,j,l} \sum_{\bar{g}\in C_l}\rho(\bar{g}) \\
      &= \sum_{l=1}^k a_{i,j,l} \lambda_\mu(C_l) I_{m_\mu}
   \end{align*}

   Finally,
   \begin{align*}
     \frac{\abs{G}}{m_i}
     &= \frac{\abs{G}}{m_i} \inpd{\chi_i, \chi_i} \\
     &= \frac{\abs{G}}{m_i} \frac{1}{\abs{G}} \sum_{g\in G}
     \chi_i(g) \chi_i(g^{-1}) \\
     &= \sum_{g\in G} \frac{\chi_i(g)}{m_i} \chi_i(g^{-1}) \\
     &= \sum_{j=1}^k \sum_{g\in C_j} \frac{\chi_i(g)}{m_i} \chi_i(g^{-1}) \\
     &= \sum_{j=1}^k \frac{\abs{C_j}\chi_i(g_j)}{m_i} \chi_i(g_j^{-1}) \\
     &= \sum_{j=1}^k \lambda_i(C_j) \chi_i(g_j^{-1}) \\
   \end{align*}
   and thus is an algebraic integer. \\
   Also, $\abs{G} / m_i \in \mathbb{Q}$, so we conclude that
   $\abs{G} / m_i \in \mathbb{Z} \implies m_i \mid \abs{G}$.
  \end{proof}
\end{theorem}

\begin{exercise} \mbox{}
  \begin{enumerate}
    \item Show that if $g\in G$ and $g\ne 1$, then
      $\sum_{i=1}^k m_i\chi_i(g) = 0$.
    \item Show that each character $\chi$ of $G$ with $\chi(g) = 0 \quad
      \forall g \ne 1$ is an integral multiple of $\chi^\text{reg}$.
  \end{enumerate}
\end{exercise}

\begin{exercise} \mbox{}
  \begin{enumerate}
    \item Let $\abs{G} < \infty$. Then $G$ is abelian $\iff$ each irr. rep.
      of $G$ is of degree $1$.
    \item $\{ \text{the $\deg 1$ rep. of $G$} \} = \{
      \text{the irr. rep. of~} \quot{G}{[G,G]} \}$.
  \end{enumerate}
\end{exercise}

\subsubsection{Applications}

\begin{enumerate}
  \item $G = S_3 = D_3$, $6 = 1^2 + 1^2 + 2^2$.
    \begin{center}
      \begin{tabular}{cccc}
        Classes & $1$ & $\cycle{1,2}$ & $\cycle{1,2,3}$ \\
        size & $1$ & $3$ & $2$ \\
        \hline
        \tikz[anchor=base, baseline]{ \node(tb-chi-1) {$\chi_1$}; }
         & $1$ & $1$ & $1$ \\
        \tikz[anchor=base, baseline]{ \node(tb-chi-2) {$\chi_2$}; }
         & $1$ & $-1$ & $1$ \\
        \tikz[anchor=base, baseline]{ \node(tb-chi-3) {$\chi_3$}; }
         & $2$ & $0$ & $-1$
      \end{tabular}
      %\begin{tikzpicture}[overlay]
        %\node[inner sep=0pt,outer sep=0pt] (t) at ($(rho-equiv) + (2, -0.5)$) {
           %\footnotesize equivalent};
         %\path[->,shorten <= 3pt] (t.west) edge[bend left=20] 
           %($(rho-equiv.base) + (0, -0.05)$);
      %\end{tikzpicture}
    \end{center}
    The permutation representation

    $\deg 4$: $\tilde{\rho} = \rho^W \otimes \rho^W \leadsto
    \chi_{\tilde{\rho}} = \chi_3 \cdot \chi_3 = (4, 0, 1)$.

    By inner product with $\chi_1, \chi_2, \chi_3$, we can find
    $\chi_{\tilde{\rho}} = \chi_1 + \chi_2 + \chi_3 \leadsto
    \tilde{\rho} = \rho_1\oplus\rho_2\oplus\rho_3$.

  \item $G = D_4 = \gen{x, y \mid x^4=1, y^2=1, yxy^{-1} = x^{-1}}$.
    $\abs{G} = 8 = 1^2 + 1^2 + 1^2 + 1^2 + 2^2$.
    \begin{center}
      \begin{tabular}{cccccc}
        Classes & $1$ & $y$ & $x$ & $x^2$ & $xy$ \\
        size & $1$ & $2$ & $2$ & $1$ & $2$ \\
        \hline
        $\chi_1$ & $1$ & $1$ & $1$ & $1$ & $1$ \\
        $\chi_2$ & $1$ & $-1$ & $1$ & $1$ & $-1$ \\
        $\chi_3$ & $1$ & $1$ & $-1$ & $1$ & $-1$ \\
        $\chi_4$ & $1$ & $-1$ & $-1$ & $1$ & $1$ \\
        $\chi_5$ & $2$ & $0$ & $0$ & $-2$ & $0$
      \end{tabular}
    \end{center}
    $\chi^\text{reg} = (8, 0, 0, 0, 0) = \chi_1 + \chi_2 + \chi_3 + \chi_4
    + 2\chi_5$.
  \item $G = D_n$, ($n$ even)
    $[G, G] = H = \gen{x^2}$
  \item $G = D_n$, ($n$ odd)
    $[G, G] = H = \gen{x}$
  \item $G = S_4$.
    \begin{center}
      \begin{tabular}{cccccc}
        Classes & $1$ & $\cycle{1,2}$ & $\cycle{1,2,3}$ & $\cycle{1,2,3,4}$
                & $\cycle{1,2}\cycle{3,4}$ \\
        size & $1$ & $6$ & $8$ & $6$ & $3$ \\
        \hline
        $\chi_1$ & $1$ & $1$ & $1$ & $1$ & $1$ \\
        $\chi_2$ & $1$ & $-1$ & $1$ & $-1$ & $1$ \\
        $\chi_3$ & $2$ & $0$ & $-1$ & $0$ & $2$ \\
        $\chi_4$ & $3$ & $1$ & $0$ & $-1$ & $-1$ \\
        $\chi_5$ & $3$ & $-1$ & $0$ & $1$ & $-1$
      \end{tabular}
    \end{center}
  \item $G = A_4$, $[A_4, A_4] = V_4$.
    \begin{center}
      \begin{tabular}{ccccc}
        Classes & $1$ & $\cycle{1,2,3}$ & $\cycle{1,3,2}$
                & $\cycle{1,2}\cycle{3,4}$ \\
        size & $1$ & $4$ & $4$ & $3$ \\
        \hline
        $\chi_1$ & $1$ & $1$ & $1$ & $1$ \\
        $\chi_2$ & $1$ & $\omega$ & $\omega^2$ & $1$ \\
        $\chi_3$ & $1$ & $\omega^2$ & $\omega$ & $1$ \\
        $\chi_4$ & $3$ & $0$ & $0$ & $-1$
      \end{tabular}
    \end{center}
\end{enumerate}

\begin{theorem}[Product of groups]
For $\rho: G\to \text{GL}(V)$ and $\rho': G'\to \text{GL}(V')$, write
$\rho\otimes\rho' : G \times G' \to \text{GL}(V\otimes V')$.
If $\{ \rho_i \}$ are irreducible representations of $G$,
$\{ \rho'_j \}$ are irreducible representations of $G'$, then
$\{ \rho_i \otimes \rho'_j \}$ are exactly the irreducible
representations of $G \times G'$.

\begin{proof}
  It is evidence that $\rho_i \otimes \rho'_j$ is a homomorphism, and hence
  a representation.

  Notice that $\chi_{\rho \otimes \rho'} = \chi_\rho \odot \chi_{\rho'}$ where
  $\chi_\rho \odot \chi_{\rho'}(g, g') = \chi_\rho(g) \chi_{\rho'}(g')$

  Now we calculate 
  \begin{align*}
    \inpd{\chi_{\rho_1} \otimes \chi_{\rho'_1}, \chi_{\rho_2} \otimes \chi_{\rho'_2}}
    &= \frac{1}{\abs{G}\abs{G'}} \sum_{g, g'} \chi_{\rho_1}(g) \chi_{\rho'_1}(g') 
    \chi_{\rho_2}(g) \chi_{\rho'_2}(g')  \\
    &= \left( \frac{1}{\abs{G}} \sum_{g} \chi_{\rho_1}(g) \chi_{\rho_2}(g) \right)
    \left( \frac{1}{\abs{G'}} \sum_{g'} \chi_{\rho'_1}(g') \chi_{\rho'_2}(g') \right) \\
    &= \inpd{\chi_{\rho_1}, \chi_{\rho_2}} \inpd{\chi_{\rho'_1}, \chi_{\rho'_2}}
  \end{align*}

  So $\inpd{\chi_{\rho} \otimes \chi_{\rho'}, \chi_{\rho} \otimes \chi_{\rho'}} = 1$ hence
  each $\chi_{\rho} \otimes \chi_{\rho'}$ is irreducible.
  And $\inpd{\chi_{\rho_1} \otimes \chi_{\rho'_1}, \chi_{\rho_2} \otimes \chi_{\rho'_2}} = 0$ 
  if $\rho_1 \otimes \rho'_1 \neq \rho_2 \otimes \rho'_2$, and thus these representations are
  not isomorphic.

  Finally we proof that any irreducible representations of $G\times G'$ is isomorphic to some
  $\rho \otimes \rho'$.

  Let $\{\rho_1,\dots,\rho_k\}, \{\rho'_1,\dots,\rho'_{k'}\}$ be the sets
  of irreducible representations of $G, G'$ respectively.
  Write $\chi_i = \chi_{\rho_i}, \chi_i' = \chi_{\rho'_i}$.

  Let $\Dc \triangleq \Cc(G\times G') = \gen{\chi_i, \chi_j' \mid
    i = 1,\dots,k, j = 1,\dots,k'}_\Cb = $. We claim that $\Dc^\perp = \{0\}$.

  Let $f \in \Dc^\perp$. Then
  \begin{align*}
    0 &= \frac{1}{\abs{G\times G'}} \sum_{(g,g')\in G\times G'}
    f(g,g')\ob{\chi_i(g)\chi_j'(g')} \\
    &= \frac{1}{\abs{G'}} \sum_{g'} \left( \frac{1}{\abs{G}}\sum_g f(g, g') \ob{\chi_i(g)} \right) \chi'_j(g') \\
    &= \left\langle \frac{1}{\abs{G}} \sum_g f(g, \cdot) \ob{\chi_i(g)}, \chi'_j(g') \right\rangle
  \end{align*}
  Since $\rho'_j$ are othonogal basis of $\Cc(G')$, we have $\frac{1}{\abs{G}} \sum_g f(g, g') \ob{\chi_i(g)}
  = 0$ for all $g'$.
  Again, 
  \[ 0 = \frac{1}{\abs{G}} \sum_g f(g, g') \ob{\chi_i(g)} = \inpd{f(\cdot, g'), \chi_i(g)} \]
  Hence $f(g, g') = 0$ for all $g, g'$, which implies $f \equiv 0$.
\end{proof}

\end{theorem}

\begin{exercise}
  Determine all irr. rep. of $C_n$.
\end{exercise}

\begin{exercise}
  Calculate the character table of $Q_8$.
\end{exercise}

\begin{exercise}
  Calculate the character table of $\quot{\Zb}{2\Zb}\times S_4$ and
  $S_3\times S_4$.
\end{exercise}

To calculate $S_5$, $\abs{S_5} = 120 = 1^2 + 1^2 + 4^2 + 4^2 + 5^2 + 5^2 + 6^2$.
