%1 TEX root=../main.tex
\section{Introduction to the linear representation theory of finite groups}
\subsection{Week 14}
\subsubsection{Generatlities on linear representations}
\paragraph{Notation}

\begin{itemize}
  \item $G$: finite group
  \item $V$: vector space of finite dim over $\Cb$
  \item $\text{GL}(V)$: the group of all linear isom. $V \to V$
\end{itemize}

\begin{definition}
  A group homo. $\rho: G \to \text{GL}(V)$ is called a linear representation
  of $G$.
  $\dim V$ is called the degree of $\rho$.
  ($V$ is a representation space)

  For a fixed basis $\beta = \{\, e_i \,\}$,
  \[
    \begin{tikzcd}
      G \arrow{r}{\rho} \arrow[swap]{dr}{R} & \text{GL}(V) \arrow[d, "\beta"', "\rotatebox{90}{\(\sim\)}"] \\
                                            & \text{GL}_n(\Cb)
    \end{tikzcd}
  \]
  ($R$ is a matrix representation)
\end{definition}

\begin{example}
  A representation of degree $1$ of $G$ is $\rho: G \to \text{GL}(\Cb)
  \cong \Cb^*$.

  $\ord(g)$ is finite $\leadsto$ $\rho(g)^m = 1$ for some $m \in \Nb$
  $\leadsto$ $\rho(g)$ is a root of unity, i.e. $\abs{\rho(g)} = 1$.

  Note: So, $\rho: G \to S^1$, $S^1$ is the unit circle.

  \begin{enumerate}
    \item $G = \quot{\Zb}{p\Zb}$,
      $\rho: \ob{1} :: G \mapsto s_p :: S^1$ with $s_p^p = 1$.
    \item $G = S_3, V = \Cb e_1 \oplus \Cb e_2 \oplus \Cb e_3$.

      A permutation representation is
      $\rho: \tau :: S_3 \mapsto (\rho(\tau): e_i \mapsto e_{\tau(i)})
      :: \text{GL}(V)$.
      
    \item $G = S_3, V = \bigoplus_{\sigma \in S_3} \Cb e_{\sigma}$.
      The regular representation is
      \[ \rho^{\text{reg}}: \tau :: G \mapsto
      (\rho^{\text{reg}}(\tau): e_{\sigma} \mapsto e_{\tau \sigma})
      :: \text{GL}(V). \]
  \end{enumerate}
\end{example}

For general $G$, with $V = \bigoplus_{g\in G} \Cb e_g$,
\[ \rho^{\text{reg}}: h :: G \mapsto
  (\rho^{\text{reg}}(h): e_{g} \mapsto e_{hg})
:: \text{GL}(V). \]

\begin{definition} \mbox{}
  \begin{itemize}
    \item $\rho: g :: G \mapsto \text{id} :: \text{GL}(V)$:
      trivial representation.
    \item $\rho: G \toone \text{GL}(V)$: faithful representation.
    \item $\rho, \rho'$ are said to be equivalent if $\exists$ a linear isom.
      ${\sf T}: V \isoto V'$ s.t.
      \[
        \begin{tikzcd}
          V \arrow[d, "\rho(g)"'] \arrow[r, "{\sf T}"', "\sim"] & V' \arrow[d, "\rho'(g)"] \\
          V \arrow[r, "{\sf T}"', "\sim"] & V'
        \end{tikzcd}
      \]
  \end{itemize}
\end{definition}

\begin{remark}
  When we choose two bases $\beta, \beta'$ for $V$,
  \[
    \begin{tikzcd}
      G \arrow{r}{\rho} \arrow[swap]{dr}{R} & \text{GL}(V) \arrow[d, "\beta"', "\rotatebox{90}{\(\sim\)}"] \\
                                            & \text{GL}_n(\Cb)
    \end{tikzcd} \quad
    \begin{tikzcd}
      G \arrow{r}{\rho'} \arrow[swap]{dr}{R} & \text{GL}(V) \arrow[d, "\beta'"', "\rotatebox{90}{\(\sim\)}"] \\
                                            & \text{GL}_n(\Cb)
    \end{tikzcd}
  \]
  then $\rho, \rho'$ are equivalent.
\end{remark}

Let $T: e_i :: V \mapsto e_i' :: V$. For $g \in G, R(g) = \big(a_{ij}\big)$.

$T \circ \rho(g) = \rho'(g) \circ T$

\begin{definition}
  Let $\inpd{\cdot, \cdot}$ be a positive definite Hermitian form on $V$.

  Then $T: V \to V$ is called a unitary operator if
  $\inpd{T(x), T(y)} = \inpd{x, y} \quad \forall x, y \in V$.

  or $\forall \beta:$ orthonormal basis,
  $[T]_\beta^*[T]_\beta = [T]_\beta[T]_\beta^* = I_n$.
\end{definition}

\begin{theorem}
  $\forall \rho: G \to \text{GL}(V)$, $\exists$ a matrix representation
  $R: G \to U_n$.
  \begin{proof}
    We only need to $G$-invariant positive definite Hermitian form on $V$.
    ($\forall g \in G, \inpd{\rho(g)x, \rho(g)y} = \inpd{x, y} \quad
  \forall x, y \in V$)

  We start with an arbitrary positive definite Hermitian form
  $\inpd{\cdot, \cdot}'$ on $V$.

  Define a new form $\inpd{\cdot, \cdot}$ by
  \[
    \inpd{x, y} \defeq \frac{1}{\abs{G}} \sum_{g\in G}
    \inpd{\rho(g)(x), \rho(g)(y)}'
  \]
  which is a positive definite Hermitian form. (easy to check)
  \end{proof}
\end{theorem}

\begin{definition}
  Let $\rho: G \to \text{GL}(V)$, For $W \subset V$ (we use $\subset$ to denote
  subspace), if $\forall x \in W$, $\rho(g)(x) \in W, \forall g \in G$, then
  $W$ is said to be $G$-invariant and
  \[
    \arraycolsep=1pt
    \begin{array}{rcl}
      \rho^W: & G & \to \text{GL}(W) \\
              & g & \mapsto \rho(g) \big|_W
    \end{array}
  \]
  is called a subrepresentation of $\rho$.
\end{definition}

$W$ is $G$-invariant $\leadsto$ $\rho(g) \big|_W: W \isoto W$.

\begin{example}
  Let $\rho$ be the regular rep. of $S_3$.

  $W^\circ = \{\, \alpha_1e_1 + \dots + \alpha_6e_6 \mid \alpha_1 +\dots + \alpha_6 = 0 \,\}$
  is $G$-invariant.

  $W^1 = \gen{ e_1 + \dots + e_6}_\Cb$ is $G$-invariant.
\end{example}

\begin{theorem}
  Let $\rho: G \to \text{GL}(V)$ and $W \subset V$ be $G$-invariant.
  Then $\exists W^\circ \subset V$ is still $G$-invariant and
  $V = W \oplus W^\circ$.
  \begin{proof}
    We can pick an arbitrary $W'$ with $V = W \oplus W'$ and
    $\pi_1: V \to W$ is the projection to $W$. Then $W' = \ker \pi_1$.
    
    Now we need $\pi_1$ preserves the $G$ action ($G$-equivariant).
    Define
    \[
      \pi^\circ = \frac{1}{\abs{G}} \sum_{g\in G}
      \rho(g)^{-1} \circ \pi_1 \circ \rho(g) : V \to W
    \]
    \begin{itemize}
      \item well-defined: $\rho(g)(V) \subset V \leadsto
        \pi_1 \circ \rho(g)(V) \subset W \leadsto
        \rho(g)^{-1} \circ \pi_1 \circ \rho(g)(V) \subseteq W$.
      \item surjective: $\forall y \in W,
        \rho(g)^{-1}\circ \pi_1 \circ\rho(g)(y) = y$ since
        $\rho(g)(y) \in W$. Also, $(\pi^\circ)^2 = \pi^\circ$.
        So $V = \Image \pi^\circ \oplus \ker \pi^\circ$.
      \item $G$-equivariant: $\forall g' \in G$,
        \begin{align*}
          \pi^\circ \circ \rho(g')(x)
          &= \frac{1}{\abs{G}} \sum_{g\in G}
            \rho(g)^{-1}\circ\pi_1\circ\rho(g) (\rho(g')(x)) \\
          &= \rho(g') \frac{1}{\abs{G}} \sum_{gg'\in G}
            \rho(gg')^{-1}\circ\pi_1\circ\rho(gg')(x) \\
          &= \rho(g') \circ \pi^\circ(x)
        \end{align*}
      \item $W^\circ \defeq \ker \pi^\circ$ is $G$-invariant:
        $\forall x \in W^\circ$, $\pi^\circ(\rho(g)(x))
        = \rho(g)(\pi^\circ(x)) = \rho(g)(0) = 0$. So
        $\rho(g)(x) \in W^\circ$.
    \end{itemize}
  \end{proof}
\end{theorem}

\begin{remark}
  If $W \subset V$ is $G$-invariant, then $W^\perp$ is also $G$-invariant.
  (w.r.t. a $G$-invariant positive definite Hermitian form)
\end{remark}

\begin{definition}
  $\rho: G \to \text{GL}(V)$ is irreducible if $\rho$ has no proper notrivial
  subrepresentations.
\end{definition}

\begin{theorem}
  Each $\rho: G\to \text{GL}(V)$ is a direct sum of irreducible
  subrepresentations.
  \begin{proof}
    By induction on $\dim V$. For $\dim V = 1$, then $\rho$ is irr.

    For $\dim V > 1$, if $\rho$ is irr., then done.
    Otherwise, $\exists W, W^\circ$ are $G$-invariant s.t.
    $V = W \oplus W^\circ$ with $\dim W \ge 1, \dim W^\circ \ge 1$.
    By induction hypothesis, $\rho^W, \rho^{W^\circ}$ are direct sum
    of irr. subrep., and $\rho = \rho^W \oplus \rho^{W^\circ}$, done.
  \end{proof}
\end{theorem}

\begin{remark}
  Let $\rho: G \to \text{GL}(V)$ and $\rho': G \to \text{GL}(V')$.
  \begin{itemize}
    \item $\rho \oplus \rho': G \to \text{GL}(V\oplus V')$.
      矩陣是左上右下
    \item $\rho \otimes \rho': G \to \text{GL}(V\otimes V')$.
      矩陣是密密麻麻 ($\sum_{i,j} r_ip, r_{jq}' (e_i \otimes e_j')$)
  \end{itemize}
\end{remark}

\subsubsection{Character Theory \RNum{1}}

Main goal: To determine all equivalence classes of irreducible representations
of a finite group $G$.

\begin{definition}

\end{definition}

\begin{remark} \mbox{}
  \begin{enumerate}
    \item $\chi_\rho$ is independent of the choice of $\beta = \{ e_i \}$
      For another basis $\beta' = \{ e_i' \}$.
    \item $\rho \tikz[anchor=base, baseline]{ \node(rho-equiv) {$\cong$}; }
      \rho' \leadsto \chi_\rho = \chi_{\rho'}$.
      \begin{tikzpicture}[overlay]
        \node[inner sep=0pt,outer sep=0pt] (t) at ($(rho-equiv) + (2, -0.5)$) {
           \footnotesize equivalent};
         \path[->,shorten <= 3pt] (t.west) edge[bend left=20] 
           ($(rho-equiv.base) + (0, -0.05)$);
      \end{tikzpicture}
  \end{enumerate}
\end{remark}

\begin{definition}\mbox{}
  \begin{itemize}
    \item The degree of $\chi_\rho$ is defined to the degree of $\rho$
      ($ =\dim V$).
    \item $\chi_\rho$ is an irreducible character if $\rho$ is irr.
  \end{itemize}
\end{definition}

Basic facts:
\begin{enumerate}
  \item $\chi_\rho(1) = n$.
  \item $\chi_\rho$ is a class function, i.e., it is constant on each
    conjugacy class.
  \item $\chi_\rho(g^{-1}) = \ob{\chi_\rho(g)}$: Assume that the eigenvalues
    of $R(g)$ are $\lambda_1, \dots, \lambda_n$. Then the eigenvalues of
    $R(g^{-1})$ are $\lambda_1^{-1}, \dots, \lambda_n^{-1}$.
    \[
      0 = \det(\lambda I_n - A) =
      \det(\lambda I_n (A^{-1} - \lambda^{-1}I_n) A) =
      \det(\lambda I_n) \det(A^{-1} - \lambda^{-1} I_n) \det(A)
    \]
    So $\det(A^{-1} - \lambda^{-1} I_n) = 0$.
    Then $g^m = 1 \implies R(g)^m = I_n \implies \abs{\lambda_i} = 1
    \implies \lambda_i^{-1} = \ob{\lambda_i}$. Thus
    $\chi_\rho(g^{-1}) = \trace(R(g)^{-1}) =
    \ob{\lambda_1 + \dots + \lambda_n} = \ob{\chi_\rho(g)}$.
  \item $\chi_{\rho \oplus \rho'} = \chi_{\rho} + \chi_{\rho'}$.
  \item $\chi_{\rho \otimes \rho'} = \chi_{\rho} \chi_{\rho'}$.
\end{enumerate}


\begin{definition}
  $\Cc(G, \Cb)$ is the vector space of complex functions on $G$.

  $\chi_\rho \in \Cc(G) \subset \Cc(G, \Cb)$
  is the vector space of complex class functions of $G$.
\end{definition}


\begin{remark}
  Assume that $\{ C_1, \dots, C_k \}$ is the set of distinct conjugacy classes
  in $G$.
  Then $\{\, f_i(C_j) = \delta_{ij} \mid \forall i = 1, \dots, k \,\}$ forms
  a basis for $\Cc(G)$ over $\Cb$.
  \begin{itemize}
    \item $\forall f \in \Cc(G)$, let $f(C_i) = a_i$, then
      $f = \sum a_i f_i$.
    \item $\sum a_i f_i = 0$, pick $x_j \in C_j$, then
      $(\sum a_if_i)(x_j) = a_j = 0 \quad \forall j = 1, \dots k$.
  \end{itemize}
  So $\dim \Cc(G) = k$.
\end{remark}

\begin{definition}
  $\phi, \psi \in \Cc(G, \Cb)$, then
  \[ \inpd{\phi, \psi} \defeq \sum_{g\in G} \phi(g) \ob{\psi(g)} \]
  is a positive definite Hermitian form on $\Cc(G, \Cb)$.
\end{definition}

\begin{theorem}[Main theorem]
  The set of all irr. characters of $G$ forms an orthonormal basis for $\Cc(G)$
  over $\Cb$. So there are only $k$ irr. rep. up to equivalent.
\end{theorem}

\begin{lemma}[Schur's lemma]
  Let $\rho: G \to \text{GL}(V)$ and $\rho': G\to \text{GL}(V')$ be two irr.
  rep. of $G$.
  
  一個ㄛ圖

  Then
  \begin{enumerate}
    \item $\rho, \rho'$ are not equivalent $\implies$ ${\sf T} = 0$.
    \item $V = V', \rho = \rho' \implies {\sf T} = \lambda 1_V$ for some
      $\lambda \in \Cb$.
  \end{enumerate}

  \begin{proof}
    \begin{enumerate}
      \item Assume ${\sf T} \ne 0$. Since ${\sf T}$ is $G$-equivariant,
        $\ker {\sf T} \le V$ and $\Image {\sf T}\le V'$ are $G$-invariant.

        $\rho$ is irr $\leadsto$ $\ker {\sf T} = 0$ or $V$.

        $\rho'$ is irr $\leadsto$ $\Image {\sf T} = 0$ or $V$.

        ${\sf T}$ is an isom.

        $\rho, \rho'$ are equivalent.

      \item  Let $\lambda$ be an eigenvalue of ${\sf T}$, say
        ${\sf T}(v) = \lambda v$ with $v \ne 0$ in $V$.
        Put ${\sf T}' - {\sf T} - \lambda 1_V$.

        Also, ㄛ圖 since $\rho(g)$ is $\Cb$-linear.

        So ${\sf T}'$ is also $G$-equivariant. But $v \in \ker {\sf T}'$,
        i.e. ${\sf T}'$ is not 1-1. By 1., ${\sf T'} = 0$.
    \end{enumerate}
  \end{proof}
\end{lemma}

\begin{coro}
  $\rho, \rho'$ as above. Let ${\sf L}: V \to V'$ be a linear transformation.
  Define
  \[
    {\sf T} = \frac{1}{\abs{G}} \sum_{g\in G} \rho'(g)^{-1} {\sf L} \rho(g)
  \]
  is $G$-equivariant. Then
  \begin{enumerate}
    \item $\rho, \rho'$ are not equivalent $\implies {\sf T} = 0$.
    \item $V = V', \rho = \rho' \implies {\sf T} = \lambda 1_V,
      \lambda \frac{\trace({\sf L})}{\dim V}$.
  \end{enumerate}
\end{coro}

\begin{remark}
  Let $\rho \to_\beta R: G \to \text{GL}_n(\Cb)$ and $R(g) = (r_{ij}(g))$

  $\rho' \to_{\beta'} R': G \to \text{GL}_{n'}(\Cb)$ and $R'(g) = (r'_{ij}(g))$

  Let ${\sf L} .....> [{\sf L}]^{\beta'}_\beta =
  (x_{\mu\nu} \in M_{n' \times n}(\Cb)$

  Then ${\sf T} ...> [{\sf T}]^{\beta'}_\beta = (x^0_{tl})$ with
  \[
    x^0_{tl} = \frac{1}{\abs{G}}
    \sum_{\substack{g\in G \\ i=1,\dots, n \\ j=1,\dots, n'}}
    r'_{tj}(g^{-1})x_{ji}r_{il}(g)
  \]

  In case 1. of coro, $x^0_{tl} = 0 \quad \forall t, l$.


  In case 2. of coro, ${\sf T} = \lambda 1_V$, i.e.
  $x^0_{tl} = \lambda \delta_{tl}$.
  $\lambda = \frac{\trace({\sf L})}{n} = \frac{1}{n} \sum_{i=1}^n x_{ii}
  = \frac{1}{n} \sum_{i,j} \delta_{ji}x_{ji}$

  Hence,
  \[
    \frac{1}{\abs{G}} \sum_{g\in G}
    r_{tj}(g^{-1})r_{il}(g) = \frac{1}{n} \delta_{ji}\delta_{tl}
  \]
\end{remark}

\begin{prop} \mbox{}
  \begin{enumerate}
    \item If $\chi_\rho$ is irr., then $\inpd{\chi_\rho, \chi_\rho} = 1$.
    \item If two irr. rep. $\rho, \rho'$ are not equivalent, then
      $\inpd{\chi_\rho, \chi_{\rho'}} = 0$.
  \end{enumerate}

  \begin{proof}
    \begin{enumerate}
      \item
      \item 
    \end{enumerate}
  \end{proof}
\end{prop}

OMIMI above


\begin{remark}
  $\inpd{\chi_\rho,\chi_\rho} = 1 \implies \rho$ is irr.
  \begin{proof}
    We write $\rho = \rho_1^{\oplus m_1}\oplus\dots\oplus\rho^{\oplus m_l}$
    where $\rho_1, \dots, \rho_l$ are non-equivalent irr. rep.
    \[ \chi_\rho = \sum_{i=1}^l m_i \chi_{\rho_i} \]
    \[
      1 = \inpd{\chi_\rho, \chi_\rho} = \sum_{i=1}^l m_i^2
      \implies \exists m_i = 1 \text{~and~} m_j = 0 \text{~for~} j \ne i
    \]
    So $\rho \cong \rho_i$.
  \end{proof}
\end{remark}
